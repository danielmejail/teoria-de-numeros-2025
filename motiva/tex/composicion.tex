\theoremstyle{plain}
\newtheorem{teoMotivaComposicion}{\teoname}[section]
\newtheorem{lemaMotivaComposicion}[teoMotivaComposicion]{\lemaname}

\theoremstyle{definition}
\newtheorem{obsMotivaComposicion}[teoMotivaComposicion]{\obsname}

%-------------

En esta parte, veremos que el conjunto de clases de formas cuadr\'aticas
posee estructura adicional.

\begin{teoMotivaComposicion}\label{teo:motiva:composicion}
	Sea $D\equiv 0,1\tmodulo[4]$, $D\neq 0$.
	El conjunto $\Clases(D)$ de clases de equivalencia propia
	de formas cuadr\'aticas binarias primitivas de discriminante $D$
	admite una estructura de grupo abeliano (finito) con la siguiente
	propiedad: si $m_1,m_2\in\Enteros$ son representados por clases
	$C_1,C_2\in\Clases(D)$, respectivamente, entonces el producto
	$m_1m_2$ es representado por el producto de las clases, $C_1C_2$.
\end{teoMotivaComposicion}

Por otro lado, si $D\equiv 0,1\tmodulo[D]$, $D\neq 0$, el subconjunto
de $\Unidadesmod(D)$ conformado por aquellas clases de congruencia
m\'odulo $D$ que est\'an representadas por formas de discriminante
$D$ constituye un subgrupo. Precisamente, dicho subgrupo es
$\ker(\varkronecker[D])\subgrp\Unidadesmod(D)$.
El \teoname~\ref{teo:motiva:composicion} sugiere que existe una
relaci\'on entre el ahora \emph{grupo de clases} $\Clases(D)$ y
el subgrupo $\ker(\varkronecker[D])$.

Por \emph{otro} lado, dos formas primitivas de discriminante $D$
pertenecen al mismo g\'enero, si representan enteros en una misma
clase de congruencia m\'odulo $D$. Dicho de otra manera, si $f$ y $g$
son formas primitivas que \emph{no pertenecen} al mismo g\'enero,
entonces, ning\'un entero representado por $f$ es congruente
m\'odulo $D$ con un entero representado por $g$. Esta observaci\'on
nos permite asignarle inequ\'ivocamente, a cada clase de congruencia
m\'odulo $D$ repersentable, el g\'enero de formas (primitivas) que la
representan.

El siguiente diagrama sintetiza el estado de situaci\'on:
\begin{displaymath}
	\begin{tikzcd}
		& \ker(\varkronecker[D]) \arrow[r] \arrow[d] &
			\Unidadesmod(D) \arrow[r,"{\varkronecker[D]}"] &
			\cuadratico \\
		\Clases(D) \arrow[r] & \big\{\text{g\'eneros}\big\} & &
	\end{tikzcd}
	%
	\dispstop
\end{displaymath}
%
En particular, si cada g\'enero de formas primitivas de discriminante $D$
est\'a compuesto por una \'unica clase propia, entonces podemos decidir,
dado un primo (impar, que no divide a $D$), si es representable
por formas primitivas de discriminante $D$ y, en tal caso, exactamente
qu\'e formas lo representan, mirando solamente su clase de congruencia
m\'odulo $D$.

