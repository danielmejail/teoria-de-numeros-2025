\theoremstyle{plain}
\newtheorem{teoMotivaReciprocidad}{\teoname}[section]
\newtheorem{lemaMotivaReciprocidad}[teoMotivaReciprocidad]{\lemaname}

\theoremstyle{definition}
\newtheorem{obsMotivaReciprocidad}[teoMotivaReciprocidad]{\obsname}

%-------------

El objetivo de esta parte ser\'a describir, dado un entero $D$,
el conjunto de n\'umeros primos que son representados por alguna
forma cuadr\'atica de discriminante $D$.
El primer paso en esta direcci\'on se resume de la siguiente manera.

\begin{obsMotivaReciprocidad}\label{obs:motiva-reciprocidad}
	Sea $D\in\Enteros$, $D\equiv 0,1\tmodulo[4]$ y sea $p$
	un primo (positivo) impar. Las afirmaciones siguientes
	son equivalentes:
	\begin{enumerate}[(a)]
		\item\label{item:motiva-reciprocidad:representado}
			existe una forma cuadr\'atica de discriminante
			$D$ que representa $p$;
		\item\label{item:motiva-reciprocidad:residuo}
			la congruencia $x^2\equiv D\tmodulo[p]$ admite
			una soluci\'on.
	\end{enumerate}
	%
\end{obsMotivaReciprocidad}

El siguiente resultado importante, es la Ley de Reciprocidad
cuadr\'atica.

\begin{lemaMotivaReciprocidad}\label{lema:motiva-reciprocidad}
	Sean $p$ y $q$ primos (positivos) impares, distintos.
	Entonces, la congruencia $x^2\equiv q\tmodulo[p]$
	tiene soluci\'on, si y s\'olo si la congruencia
	$x^2\equiv p\tmodulo[q]$ tiene soluci\'on,
	salvo que $p\equiv q\equiv 3\tmodulo[4]$, en cuyo
	caso, exactamente una de las dos congruencias
	tiene soluci\'on.
\end{lemaMotivaReciprocidad}

En este contexto, podemos entender Reciprocidad como un lema
t\'ecnico clave para probar el resultado definitivo.

\begin{teoMotivaReciprocidad}\label{teo:motiva-reciprocidad}
	Sea $a$ un entero no cuadrado. Entonces, existe un
	morfismo sobreyectivo de grupos
	$\chi:\,\Unidadesmod(4a)\rightarrow\{\pm 1\}$ con la
	siguiente propiedad:
	si $p$ es un primo (positivo) impar que no divide a $a$,
	entonces la congruencia $x^2\equiv a\tmodulo[p]$ tiene
	soluci\'on, si y s\'olo si $\chi(p)=1$.
\end{teoMotivaReciprocidad}

Una de las caracter\'{\i}sticas m\'as importantes del \teoname~%
\ref{teo:motiva-reciprocidad} (o, mejor dicho, de la existencia de $\chi$)
es que la condici\'on sobre el primo $p$ de que
$x^2\equiv a\tmodulo[p]$ admita una soluci\'on se puede expresar como
una condici\'on sobre la clase de congruencia de $p$ m\'odulo $4a$.
Es decir, dados $p$ y $p'$ primos positivos impares que no dividen a $a$,
si $p\equiv p'\tmodulo[4a]$, entonces $x^2\equiv a\tmodulo[p]$
tiene soluci\'on, si y s\'olo si $x^2\equiv a\tmodulo[p']$ tiene soluci\'on.
Esto es as\'{\i}, porque, si $p\equiv p'$, entonces $\chi(p)=\chi(p')$.

