\theoremstyle{plain}
\newtheorem{teoMotivaGeneros}{\teoname}[section]
\newtheorem{coroMotivaGeneros}[teoMotivaGeneros]{\coroname}

\theoremstyle{definition}
\newtheorem{obsMotivaGeneros}[teoMotivaGeneros]{\obsname}

%-------------

% A cada forma $f$ podemos asociarle el subconjunto de enteros
% que ella representa:
% \begin{displaymath}
	% \Representados(f)\,=\,\big\{m\in\Enteros\,:\,
		% m\text{ est\'a representado por } f\big\}
	% \dispstop
% \end{displaymath}
% %
% El subconjunto $\Representados(f)\subseteq\Enteros$ es un
% invariante de la clase de equivalencia de $f$. Es decir,
% si $f$ y $g$ son equivalentes, entonces
% $\Representados(f)=\Representados(g)$.
% 
Dado $D\equiv 0,1\tmodulo[4]$, $D\neq 0$, un entero
$m$ (impar, coprimo con $D$) est\'a representado
(primitivamente) por alguna forma cuadr\'atica de
discriminante $D$, si y s\'olo $D$ es un residuo
cuadr\'atico m\'odulo $4m$. Si $m=p>0$ es primo,
entonces esto equivale tambi\'en a que
$\varkronecker[D](p)=\tlegendre D p=1$.

\begin{obsMotivaGeneros}\label{obs:motiva:generos}
	Si $p$ es un primo positivo impar que no divide
	a $D$ y $p$ es representable por alguna forma
	cuadr\'atica primitiva de discriminante $D$,
	entonces todo primo positivo impar
	$q\equiv p\tmodulo[D]$ tambi\'en es
	representable por alguna forma primitiva
	de discriminante $D$.
\end{obsMotivaGeneros}

\begin{teoMotivaGeneros}[Dirichlet]\label{teo:motiva:generos}
	Dados n\'umeros enteros coprimos $m$ y $D$, la
	sucesi\'on
	\begin{displaymath}
		m\dispcomma\quad
		m+D\dispcomma\quad
		m+2 D\dispcomma\quad\dots
	\end{displaymath}
	%
	contiene infinitos n\'umeros primos.
\end{teoMotivaGeneros}

% \'Este es el Teorema de Dirichlet sobre primos en
% progresiones aritm\'eticas.
En particular, el \teoname~\ref{teo:motiva:generos}
implica que toda clase de congruencia $c\in\Unidadesmod(D)$
contiene, al menos, un primo positivo impar. Esto
tiene la siguiente consecuencia.

\begin{coroMotivaGeneros}\label{coro:motiva:generos}
	Si $c\in\Unidadesmod(D)$ es tal que
	$\varkronecker[D](c)=1$, entonces $c$ es representable
	por alguna forma cuadr\'atica (primitiva) de
	discriminante $D$.
\end{coroMotivaGeneros}

Teniendo en cuenta el \coroname~\ref{coro:motiva:generos}
y la \obsname~\ref{obs:motiva:generos}, asociamos, a una
forma $f$ de discriminante $D$, el subconjunto de clase
de congruencia m\'odulo $D$, coprimas con $D$,%
\footnote{
	Si $m\equiv n\tmodulo[D]$, entonces
	$\mcd{m,D}=\mcd{n,d}$. En particular,
	la noci\'on de que una clase de congruencia
	m\'odulo $D$ sea coprima con $D$ est\'a bien
	definida.
}
representadas por $f$:
\begin{displaymath}
	\copRepresentados(f)\,=\,
		\big\{c\in\Unidadesmod(D)\,:\,
		c\text{ est\'a representada por } f\big\}
	\dispstop
\end{displaymath}
%

?`Es cierto que, si $c\in\Unidadesmod(D)$ est\'a representada
por alguna forma (primitiva) de discriminante $D$,
\emph{entonces} $\varkronecker[D](c)=1$?
La respuesta es que s\'{\i}.
En particular, dada una forma primitiva $f$ de discriminante
$D$, el subconjunto $\copRepresentados(f)$, en verdad, est\'a
contenido en
\begin{displaymath}
	\ker(\varkronecker[D])\,=\,
		\big\{c\in\Unidadesmod(D)\,:\,
		\varkronecker[D](c)=1\big\}
	\dispstop
\end{displaymath}
%
Fijado un discriminante $D$,
para cada $m\in\Enteros$, podemos preguntarnos qu\'e
formas cuadr\'aticas
% lo representan, qu\'e formas primitivas,
% qu\'e formas de discriminante $D$ y qu\'e formas
primitivas de discriminante $D$ lo representan.
A cada entero le corresponder\'a un subconjunto de formas y,
en realidad, un subconjunto de clases de formas.
Este subconjunto podr\'{\i}a ser vac\'{\i}o.
De la misma manera, fijado $D$, a cada clase de congruencia
$c\in\Unidadesmod(D)$, le asociamos el subconjunto de aquellas
clases de formas que la representan. Este subconjunto de
formas constituye un \emph{g\'enero de formas cuadr\'aticas}.

\begin{teoMotivaGeneros}\label{teo:motiva:generos:bis}
	Sea $D\equiv 0,1\tmodulo[4]$, $D\neq 0$ y sea
	$c\in\Unidadesmod(D)$. Si dos formas primitivas
	$f$ y $g$ de discriminante $D$ representan $c$,
	entonces $\copRepresentados(f)=\copRepresentados(g)$.
\end{teoMotivaGeneros}

Es decir, los subconjuntos $\copRepresentados(f)$ y
$\copRepresentados(g)$ son iguales o disjuntos.

