\theoremstyle{definition}
\newtheorem{ejerEuclidianos}{\ejername}[section]

%-------------

\begin{ejerEuclidianos}\label{ejer:euclidianos:eisenstein}
	Sea $\raizcubica\in\Complejos$ tal que $\raizcubica^3=1$, pero
	$\raizcubica\neq 1$.%
	\footnote{
		Por ejemplo, $\raizcubica=e^{2\pi i/3}=\frac{-1+\sqrt{-3}} 2$.
	}
	\begin{enumerate}[(i)]
		\item\label{ejer:euclidianos:eisenstein:pre:i}
			Probar que $\raizcubica^2+\raizcubica+1=0$.
		\item\label{ejer:euclidianos:eisenstein:pre:ii}
			Probar que $\conj\raizcubica\neq\raizcubica$ y que
			$\conj\raizcubica^2+\conj\raizcubica+1=0$, tambi\'en.
		\item\label{ejer:euclidianos:eisenstein:pre:iii}
			Concluir que
			$X^2+X+1=(X-\raizcubica)\,(X-\conj\raizcubica)$ y,
			en particular, que $\raizcubica+\conj\raizcubica=-1$
			y que $\raizcubica\conj\raizcubica=1$.
	\end{enumerate}
	%
	Dados $x,y\in\Racionales$, definimos
	\begin{displaymath}
		N(x+y\raizcubica):=x^2-xy+y^2
		\text{ .}
	\end{displaymath}
	%
	\begin{enumerate}[(i)]
		\item\label{item:ejer:euclidianos:eisenstein:norma:i}
			Como $\{1,\raizcubica\}$ es un conjunto l.i. sobre
			$\Racionales$, esta expresi\'on no es ambigua.
		\item\label{item:ejer:euclidianos:eisenstein:norma:i}
			Probar que $N(\beta\beta')=N(\beta)N(\beta')$ y
			que $N(x+y\raizcubica)\in\Enteros$, si
			$x,y\in\Enteros$.
	\end{enumerate}
	%
	Sea $\EnterosEisenstein\subset\Complejos$ el subconjunto
	\begin{displaymath}
		\EnterosEisenstein\,:=\,\big\{
			a+b\raizcubica\,:\,a,b\in\Enteros\big\}
		\text{ .}
	\end{displaymath}
	%
	\begin{enumerate}[(i)]
		\item\label{item:ejer:euclidianos:eisenstein:enteros:i}
			Probar que, si $\alpha,\beta\in\EnterosEisenstein$,
			entonces $\alpha+\beta$, $\alpha\beta$, $-\alpha$ y
			$\conj\alpha$ pertenecen a $\EnterosEisenstein$,
			tambi\'en.
		\item\label{item:ejer:euclidianos:eisenstein:enteros:ii}
			Probar que, $\Enteros\subset\EnterosEisenstein$.
	\end{enumerate}
	%
	Emular el argumento del \ejemname~\ref{ejem:euclidianos:gauss} para
	probar que $\EnterosEisenstein$ es un dominio euclidiano: dados
	$\alpha=a+b\raizcubica$ y $\beta=c+d\raizcubica$, $\alpha\neq 0$,
	\begin{enumerate}[(i)]
		\item\label{item:ejer:euclidianos:eisenstein:i}
			existen $u,v\in\Racionales$ tales que
			$(u+v\raizcubica)\alpha=\beta$;
		\item\label{item:ejer:euclidianos:eisenstein:ii}
			dados $u,v\in\Racionales$, existen $m,n\in\Enteros$
			tales que $|m-u|\leq 1/2$ y $|n-v|\leq 1/2$;
		\item\label{item:ejer:euclidianos:eisenstein:iii}
			si $q:=m+n\raizcubica$ y $r:=\beta-q\alpha$, entonces
			$N(r)<N(\alpha)$ (o $r=0$).
	\end{enumerate}
	%
	?`En qu\'e punto falla el argumento para el dominio
	$\polinomios[\sqrt{-6}]\Enteros$?
\end{ejerEuclidianos}

\begin{ejerEuclidianos}[Unidades de un anillo]\label{ejer:euclidianos:unidades}
	Dado $a\in A$, si existe $b\in A$ tal que $ab=ba=1$, se dice que
	$a$ es una \emph{unidad} del anillo o que es \emph{inversible} en el
	anillo. Por ejemplo, las unidades de $\Enteros$ son $\pm 1$.
	\begin{itemize}
		\item En $\EnterosGauss$, probar que
			$N(\alpha)=\alpha\conj\alpha$ ($=\conj\alpha\alpha$) y
			concluir que $\alpha\in\EnterosGauss$ es inversible,
			si y s\'olo si $N(\alpha)=\pm 1$ ?`Puede dar $-1$?
			Hallar todas las unidades, en este caso.
		\item En $\EnterosEisenstein$, probar que
			$N(\alpha)=\alpha\conj\alpha$ ($=\conj\alpha\alpha$) y
			concluir que $\alpha\in\EnterosEisenstein$ es
			inversible, si y s\'olo si $N(\alpha)=\pm 1$ ?`Puede
			dar $-1$? Hallar todas las unidades, en este caso.
		\item Hacer lo mismo con $\polinomios[\sqrt{-6}]\Enteros$
			y con $\polinomios[\sqrt{-2}]\Enteros$.%
			\footnote{
				Ver \ejername~%
				\ref{ejer:euclidianos:cuadratico:menos-dos}.
			}
	\end{itemize}
	%
\end{ejerEuclidianos}

\begin{ejerEuclidianos}\label{ejer:euclidianos:cuadratico:menos-dos}
	Repetir el argumento del \ejername~\ref{ejer:euclidianos:eisenstein}
	con $\polinomios[\delta]\Enteros\subset\Complejos$, el subconjunto
	de los complejos de la forma $a+b\delta$, donde $\delta\in\Complejos$
	es ra\'{\i}z de $X^2+2$.
\end{ejerEuclidianos}

\begin{ejerEuclidianos}
	Probar que $2$ es divisible por $(1+\raizcuarta)^2$ en $\EnterosGauss$.
\end{ejerEuclidianos}

\begin{ejerEuclidianos}
	Probar que $3$ es divisible por $(1-\raizcubica)^2$ en
	$\EnterosEisenstein$.
\end{ejerEuclidianos}

% \begin{ejerEuclidianos}\label{ejer:euclidianos:raiz-de-dos}
	% El objetivo de este ejercicio es resolver la ecuaci\'on
	% \begin{equation}
		% \label{eq:euclidianos:raiz-de-dos}
		% y^2\,-\,2x^2\,=\,1
		% \dispcomma
	% \end{equation}
	% %
	% con $x,y\in\Enteros$, darle sentido a las soluciones y describir
	% cierta estructura impl\'{\i}cita en el conjunto de soluciones.
	% \begin{enumerate}[(i)]
		% \item\label{item:euclidianos:raiz-de-dos:no-hay-racionales}
			% Probar que la \'unica soluci\'on a $y^2=2x^2$ con
			% $x,y\in\Enteros$ es $x=y=0$.
		% \item\label{item:euclidianos:raiz-de-dos:asintota}
			% Probar que la hip\'erbola $y^2=2x^2+1$ es
			% asint\'otica a la recta $y=\sqrt 2x$.
		% \item\label{item:euclidianos:raiz-de-dos:hay-racionales}
			% Probar que la hip\'erbola del \'{\i}tem~%
			% \eqref{item:euclidianos:raiz-de-dos:asintota}
			% tiene puntos con coordenadas enteras: $(0,\pm1)$
			% son dos; mostrar que hay otros (exhibir un ejemplo).
		% \item\label{item:euclidianos:raiz-de-dos:producto}
			% Mostrar que es posible ``multiplicar'' dos
			% soluciones a la ecuaci\'on~%
			% \eqref{eq:euclidianos:raiz-de-dos}:
			% si $(u,v)$ y $(U,V)$ son soluciones, entonces
			% \begin{equation}
				% \label{eq:euclidianos:raiz-de-dos:producto}
				% (u,v)\cdot(U,V)\,:=\,
					% (uV+vU,2uU+vV)
			% \end{equation}
			% %
			% tambi\'en es una soluci\'on;
			% comparar con el producto de los n\'umeros
			% \begin{displaymath}
				% \big(u\sqrt 2+v\big)\,\big(U\sqrt 2+V\big)
			% \end{displaymath}
			% %
		% \item\label{item:euclidianos:raiz-de-dos:producto:ejemplos}
			% ?`Qu\'e pasa si $(u,v)=(0,\pm 1)$?
			% ?`Qu\'e pasa si $(u,v)=(2,3)$ y $(U,V)=(-2,3)$?
		% \item\label{item:euclidianos:raiz-de-dos:inversos}
			% Mostrar que toda soluci\'on a la ecuaci\'on~%
			% \eqref{eq:euclidianos:raiz-de-dos} tiene un
			% ``inverso'': si $(u,v)$ es una soluci\'on, existe
			% una soluci\'on $(U,V)$ tal que
			% $(u,v)\cdot (U,V)=(0,1)$.
		% \item\label{item:euclidianos:raiz-de-dos:soluciones}
			% Sea $(x_n,y_n)=(2,3)^n$ ($n\geq 1$). Probar que,
			% si $(u,v)$ es soluci\'on de
			% \eqref{eq:euclidianos:raiz-de-dos} con
			% $u,v\in\Enteros$ positivos, entonces $(u,v)=(x_n,y_n)$
			% para alg\'un $n\geq 1$.%
			% \hint{
				% Hacer inducci\'on en $u$. El caso $u=1$
				% es trivialmente cierto, porque no hay
				% soluciones con $u=1$, y el caso $u=2$ es
				% cierto, porque corresponde a $(2,3)=(x_1,y_1)$.
				% Si $u>2$, considerar
				% $(u_1,v_1)=(-2,3)\cdot (u,v)$.
			% }
		% \item\label{item:euclidianos:raiz-de-dos:grupo}
			% Concluir de lo anterior que el conjunto
			% \begin{displaymath}
				% \big\{(x,y)\in\Enteros^2\,:\,y^2-2x^2=1\big\}
			% \end{displaymath}
			% %
			% es un grupo abeliano; probar que es isomorfo
			% a un subgrupo de
			% $\Unidades{\polinomios[\sqrt 2]\Enteros}$.
		% \item\label{item:euclidianos:raiz-de-dos:isomorfo}
			% Probar las siguientes f\'ormulas para la sucesi\'on
			% $(x_n,y_n)$ del \'{\i}tem~%
			% \eqref{item:euclidianos:raiz-de-dos:soluciones}:
			% para $n\geq 1$,
			% \begin{displaymath}
				% x_n\sqrt 2+y_n \,=\,
					% \big(2\sqrt 2+3\big)^n
					% \dispand
				% \begin{bmatrix}
					% x_n \\ y_n
				% \end{bmatrix} \,=\,
					% \begin{bmatrix}
						% 3 & 2 \\ 4 & 3
					% \end{bmatrix}^n\,
					% \begin{bmatrix}
						% 0 \\ 1
					% \end{bmatrix}
				% \dispstop
			% \end{displaymath}
			% %
		% \item\label{item:euclidianos:raiz-de-dos:crecimiento}
			% Probar que, si $n>1$,
			% \begin{displaymath}
				% x_n\,>\,\big(2\sqrt 2+3)^{n-1}\,x_1
			% \end{displaymath}
			% %
			% y concluir que $x_n\to\infty$.
		% \item\label{item:euclidianos:raiz-de-dos:aproxima}
			% Probar que, si $(x,y)$ es soluci\'on de
			% \eqref{eq:euclidianos:raiz-de-dos} con $x,y>0$,
			% entonces
			% \begin{displaymath}
				% \left|\frac y x-\sqrt 2\right|\,<\,
					% \frac 1{2\sqrt 2 x^2}
				% \dispstop
			% \end{displaymath}
			% %
		% \item\label{item:euclidianos:raiz-de-dos:aproxima:bis}
			% Probar que, si $x,y\in\Enteros$, $x,y>0$ y se
			% cumple $\big|y/x-\sqrt 2\big|<1/2x^2$, entonces
			% $y^-2x^2=\pm 1$.%
			% \hint{
				% Usar las cotas:
				% \begin{displaymath}
					% \left|y-\sqrt 2x\right|\,<\,\frac 1{2x}
					% \dispand
					% \left|y+\sqrt 2x\right|\,<\,2y+\frac 1{2x}
				% \end{displaymath}
				% %
				% y multiplicar.
			% }
			% Rec\'{\i}procamente, probar que, si
			% $\big|y^2-2x^2\big|=1$ con $x,y\in\Enteros$, $x,y>0$,
			% entocnes $\big|y/x-\sqrt 2\big|<1/2x^2$.
		% \item\label{item:euclidianos:raiz-de-dos:todas}
			% Probar que $(1,1)$ es soluci\'on de
			% $y^2-2x^2=-1$, que $(1,1)^2=(2,3)$ y que, por lo tanto,
			% toda soluci\'on de
			% \begin{displaymath}
				% y^2\,-\,2x^2\,=\,\pm 1
			% \end{displaymath}
			% %
			% con $x,y\in\Enteros$, $x,y>0$, es de la forma
			% $(1,1)^n$ para alg\'un $n\geq 1$.
	% \end{enumerate}
	% %
	% En definitiva, tenemos cuatro interpretaciones distintas
	% del mismo objeto:
	% \begin{itemize}
		% \item soluciones enteras a la ecuaci\'on $y^2-2x^2=1$;
		% \item subgrupo de la unidades de
			% $\polinomios[\sqrt 2]\Enteros$;
		% \item subgrupo de matrices $\sbmatrix{ 3 & 2 \\ 4 & 3 }^n$,
			% $n\in\Enteros$;
		% \item aproximaciones racionales a $\sqrt 2$.
	% \end{itemize}
	% %
	% % \begin{displaymath}
	% % \begin{tikzcd}
		% % \big\{(x,y)\in\Enteros^2\,:\,y^2-2x^2=1\big\}
			% % \arrow[r] &
			% % \Unidades{\polinomios[\sqrt 2]\Enteros}
				% % \arrow[d] \\
		% % \left\{\begin{bmatrix} 3 & 2 \\ 4 & 3 \end{bmatrix}^n\,:\,
				% % n\in\Enteros\right\} \arrow[u] &
			% % \big\{\text{aproximaciones racionales a }\sqrt 2\big\}
	% % \end{tikzcd}
	% % %
	% % \dispstop
	% % \end{displaymath}
	% % %
% \end{ejerEuclidianos}


