\theoremstyle{definition}
\newtheorem{ejerEuclidianos}{\ejername}[section]

%-------------

\begin{ejerEuclidianos}\label{ejer:euclidianos:eisenstein}
	Sea $\raizcubica\in\Complejos$ tal que $\raizcubica^3=1$, pero
	$\raizcubica\neq 1$.%
	\footnote{
		Por ejemplo, $\raizcubica=e^{2\pi i/3}=\frac{-1+\sqrt{-3}} 2$.
	}
	\begin{enumerate}[(i)]
		\item\label{ejer:euclidianos:eisenstein:pre:i}
			Probar que $\raizcubica^2+\raizcubica+1=0$.
		\item\label{ejer:euclidianos:eisenstein:pre:ii}
			Probar que $\conj\raizcubica\neq\raizcubica$ y que
			$\conj\raizcubica^2+\conj\raizcubica+1=0$, tambi\'en.
		\item\label{ejer:euclidianos:eisenstein:pre:iii}
			Concluir que
			$X^2+X+1=(X-\raizcubica)\,(X-\conj\raizcubica)$ y,
			en particular, que $\raizcubica+\conj\raizcubica=-1$
			y que $\raizcubica\conj\raizcubica=1$.
	\end{enumerate}
	%
	Dados $x,y\in\Racionales$, definimos
	\begin{displaymath}
		N(x+y\raizcubica):=x^2-xy+y^2
		\text{ .}
	\end{displaymath}
	%
	\begin{enumerate}[(i)]
		\item\label{item:ejer:euclidianos:eisenstein:norma:i}
			Como $\{1,\raizcubica\}$ es un conjunto l.i. sobre
			$\Racionales$, esta expresi\'on no es ambigua.
		\item\label{item:ejer:euclidianos:eisenstein:norma:i}
			Probar que $N(\beta\beta')=N(\beta)N(\beta')$ y
			que $N(x+y\raizcubica)\in\Enteros$, si
			$x,y\in\Enteros$.
	\end{enumerate}
	%
	Sea $\EnterosEisenstein\subset\Complejos$ el subconjunto
	\begin{displaymath}
		\EnterosEisenstein\,:=\,\big\{
			a+b\raizcubica\,:\,a,b\in\Enteros\big\}
		\text{ .}
	\end{displaymath}
	%
	\begin{enumerate}[(i)]
		\item\label{item:ejer:euclidianos:eisenstein:enteros:i}
			Probar que, si $\alpha,\beta\in\EnterosEisenstein$,
			entonces $\alpha+\beta$, $\alpha\beta$, $-\alpha$ y
			$\conj\alpha$ pertenecen a $\EnterosEisenstein$,
			tambi\'en.
		\item\label{item:ejer:euclidianos:eisenstein:enteros:ii}
			Probar que, $\Enteros\subset\EnterosEisenstein$.
	\end{enumerate}
	%
	Emular el argumento del \ejemname~\ref{ejem:euclidianos:gauss} para
	probar que $\EnterosEisenstein$ es un dominio euclidiano: dados
	$\alpha=a+b\raizcubica$ y $\beta=c+d\raizcubica$, $\alpha\neq 0$,
	\begin{enumerate}[(i)]
		\item\label{item:ejer:euclidianos:eisenstein:i}
			existen $u,v\in\Racionales$ tales que
			$(u+v\raizcubica)\alpha=\beta$;
		\item\label{item:ejer:euclidianos:eisenstein:ii}
			dados $u,v\in\Racionales$, existen $m,n\in\Enteros$
			tales que $|m-u|\leq 1/2$ y $|n-v|\leq 1/2$;
		\item\label{item:ejer:euclidianos:eisenstein:iii}
			si $q:=m+n\raizcubica$ y $r:=\beta-q\alpha$, entonces
			$N(r)<N(\alpha)$ (o $r=0$).
	\end{enumerate}
	%
	?`En qu\'e punto falla el argumento para el dominio
	$\polinomios[\sqrt{-6}]\Enteros$?
\end{ejerEuclidianos}

\begin{ejerEuclidianos}[Unidades de un anillo]\label{ejer:euclidianos:unidades}
	Dado $a\in A$, si existe $b\in A$ tal que $ab=ba=1$, se dice que
	$a$ es una \emph{unidad} del anillo o que es \emph{inversible} en el
	anillo. Por ejemplo, las unidades de $\Enteros$ son $\pm 1$.
	\begin{itemize}
		\item En $\EnterosGauss$, probar que
			$N(\alpha)=\alpha\conj\alpha$ ($=\conj\alpha\alpha$) y
			concluir que $\alpha\in\EnterosGauss$ es inversible,
			si y s\'olo si $N(\alpha)=\pm 1$ ?`Puede dar $-1$?
			Hallar todas las unidades, en este caso.
		\item En $\EnterosEisenstein$, probar que
			$N(\alpha)=\alpha\conj\alpha$ ($=\conj\alpha\alpha$) y
			concluir que $\alpha\in\EnterosEisenstein$ es
			inversible, si y s\'olo si $N(\alpha)=\pm 1$ ?`Puede
			dar $-1$? Hallar todas las unidades, en este caso.
		\item Hacer lo mismo con $\polinomios[\sqrt{-6}]\Enteros$
			y con $\polinomios[\sqrt{-2}]\Enteros$.%
			\footnote{
				Ver \ejername~%
				\ref{ejer:euclidianos:cuadratico:menos-dos}.
			}
	\end{itemize}
	%
\end{ejerEuclidianos}

\begin{ejerEuclidianos}\label{ejer:euclidianos:cuadratico:menos-dos}
	Repetir el argumento del \ejername~\ref{ejer:euclidianos:eisenstein}
	con $\polinomios[\delta]\Enteros\subset\Complejos$, el subconjunto
	de los complejos de la forma $a+b\delta$, donde $\delta\in\Complejos$
	es ra\'{\i}z de $X^2+2$.
\end{ejerEuclidianos}

\begin{ejerEuclidianos}
	Probar que $2$ es divisible por $(1+\raizcuarta)^2$ en $\EnterosGauss$.
\end{ejerEuclidianos}

\begin{ejerEuclidianos}
	Probar que $3$ es divisible por $(1-\raizcubica)^2$ en
	$\EnterosEisenstein$.
\end{ejerEuclidianos}

