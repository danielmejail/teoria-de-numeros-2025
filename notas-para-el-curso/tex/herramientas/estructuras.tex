\theoremstyle{plain}
\newtheorem{teoEuclidianos}{\teoname}[section]
\newtheorem{coroEuclidianos}[teoEuclidianos]{\coroname}

\theoremstyle{definition}
\newtheorem{defEuclidianos}[teoEuclidianos]{\defname}
\newtheorem{obsEuclidianos}[teoEuclidianos]{\obsname}
\newtheorem{ejemEuclidianos}[teoEuclidianos]{\ejemname}

%-------------

% \subsection{Anillos}%
	% \label{subsec:anillos}
\begin{defEuclidianos}\label{def:anillo}
	Llamaremos \emph{anillo} a un conjunto $A$ dotado de:
	\begin{itemize}
		\item operaciones binarias $+,\cdot:\,A\times A\rightarrow A$,
			que llamamos \emph{suma} y \emph{producto},
		\item elementos distinguidos $0,1\in A$, que llamamos
			\emph{cero} y \emph{uno},
		\item una funci\'on $-:\,A\rightarrow A$ (operaci\'on
			unitaria), que escribimos $a\mapsto -a$ y cuya imagen
			llamamos ``menos $a$''
	\end{itemize}
	%
	que cumplen:
	\begin{itemize}
		\item con respecto a la suma,
			\begin{displaymath}
				\begin{aligned}
					(a+b)+c & \,=\,a+(b+c)\text{ ,}\quad
					0+a \,=\,a+0\,=\,a\text{ ,} \\
					a+(-a) & \,=\,(-a)+a\,=\,0
						\quad\text{y}\quad
					a+b \,=\,b+a
					\text{ ,}
				\end{aligned}
				%
			\end{displaymath}
			%
		\item con respecto al producto,
			\begin{displaymath}
				(a\cdot b)\cdot c \,=\, a\cdot (b\cdot c)
					\quad\text{y}\quad
					1\cdot a \,=\,a\cdot 1\,=\,a
				\text{ ,}
			\end{displaymath}
			%
		\item con respecto a ambas,
			\begin{displaymath}
				(a+b)\cdot c\,=\,a\cdot c+b\cdot c
					\quad\text{y}\quad
					c\cdot (a+b)\,=\,c\cdot a+c\cdot b
				\text{ .}
			\end{displaymath}
			%
	\end{itemize}
	%
\end{defEuclidianos}

\begin{defEuclidianos}\label{def:euclidianos:conmutativo}
	Si $a\cdot b=b\cdot a$, se dice que \emph{$a$ y $b$ conmutan};
	si todo par de elementos conmuta, se dice que el anillo es
	\emph{conmutativo}.
\end{defEuclidianos}

\begin{obsEuclidianos}\label{obs:euclidianos}
	En general, esta estructura se denomina \emph{anillo con unidad}
	(los anillos \emph{sin} un $1$ tambi\'en son importantes).
\end{obsEuclidianos}

\begin{ejemEuclidianos}\label{ejem:euclidianos}
	Los n\'umeros enteros $\Enteros$ constituyen un anillo con la suma,
	el producto, el cero, el uno y el menos usuales. Los n\'umeros
	racionales $\Racionales$, reales $\Reales$ y complejos $\Complejos$,
	tambi\'en. Todos \'estos son anillos conmutativos.
	Los n\'umeros naturales $\Naturales$, en cambio, no forman un anillo
	con la suma y el producto usuales (dependiendo de la convenci\'on,
	no hay cero, pero, en general, $-n$ no es natural, si $n\geq 1$ es
	natural).
\end{ejemEuclidianos}

% \begin{ejemEuclidianos}\label{ejem:euclidianos:polinomios}
	% Los polinomios en una indeterminada constituyen anillos:
	% $\polinomios\Racionales{X}$, $\polinomios\Reales{X}$ y
	% $\polinomios\Complejos{X}$ son anillos conmutativos con la suma
	% y el producto usuales. M\'as en general, si $k$ es un cuerpo,
	% $\polinomios k{X}$ es un anillo conmutativo. M\'as en general aun,
	% si $A$ es un anillo conmutativo, $\polinomios A{X}$ es un anillo
	% conmutativo. Un caso particular de esto \'ultimo es
	% $\polinomios\Enteros{X}$. Los polinomios en varias indeterminadas
	% tambi\'en son anillos:
	% $\polinomios\Enteros{X,Y}=\polinomios{\polinomios\Enteros{X}}{Y}$
	% es un anillo conmutativo.%
	% \footnote{
		% T\'ecnicamente, estos $\polinomios\Enteros{X,Y}$ y
		% $\polinomios{\polinomios\Enteros{X}}{Y}$ no son lo mismo,
		% pero son iguales en cierto sentido.
	% }
% \end{ejemEuclidianos}

\begin{ejemEuclidianos}\label{ejem:euclidianos:cuadratico}
	El subconjunto $A\subset\Complejos$ del \ejername~%
	\ref{ejer:primos:factorizacion:cuadratico},
	\begin{displaymath}
		A\,=\,\big\{x+y\sqrt{-6}\,:\,x,y\in\Enteros\big\}
		\text{ ,}
	\end{displaymath}
	%
	es un anillo. La suma, el producto, el cero, el uno y el menos son
	los heredados de $\Complejos$. En particular, $A$ es un anillo
	conmutativo (porque $\Complejos$ lo es). Este anillo se suele denotar
	por $\polinomios\Enteros{\sqrt{-6}}$.
\end{ejemEuclidianos}

\begin{ejemEuclidianos}\label{ejem:euclidianos:matrices}
	Si $A$ es un anillo, las matrices \emph{cuadradas} con coeficientes
	en $A$ constituyen un anillo. Por ejemplo, las matrices de tama\~no
	$2\times 2$ con coeficientes enteros, $\MM(2\times 2,\Enteros)$,
	son un anillo con las operaciones usuales. Pero, en general, estos
	anillos no son conmutativos, aunque $A$ lo sea.
\end{ejemEuclidianos}

Nos concentraremos en anillos conmutativos (con uno).

% \subsection{Dominios \'{\i}ntegro}%
	% \label{subsec:dominio}
\begin{defEuclidianos}\label{def:dominio}
	Un \emph{dominio \'{\i}ntegro} es un anillo conmutativo (con uno)
	que tiene la siguiente propiedad:%
	\footnote{
		C.f. la propiedad de la \obsname~\ref{obs:primos:irreducibles}.
	}
	\begin{center}
		para todo par de elementos $a$ y $b$, $ab=0$ implica
			$a=0$ o $b=0$.
	\end{center}
	%
\end{defEuclidianos}

\begin{ejemEuclidianos}\label{ejem:euclidianos:dominio}
	Los anillos $\Enteros$, $\Racionales$, $\Reales$, $\Complejos$,
	sus anillos de polinomios y $\polinomios\Enteros{\sqrt{-6}}$ son
	dominios \'{\i}ntegros.
\end{ejemEuclidianos}

% \subsection{Dominios euclidianos}%
	% \label{subsec:dominios-euclidianos}
\begin{defEuclidianos}\label{def:euclidianos}
	Un \emph{dominio euclidiano} es un dominio \'{\i}ntegro $D$
	que admite una funci\'on $N:\,D\rightarrow\Enteros$ con las siguientes
	propiedades:
	\begin{itemize}
		\item $N(x)\geq 0$ para todo $x\in D$ y,
		\item dados $a,b\in D$, $a\neq 0$, existen $q,r\in D$ tales
			que $b=qa+r$ y $r=0$ o bien $N(r)<N(a)$.
			% (decrece).
	\end{itemize}
	%
\end{defEuclidianos}

\begin{ejemEuclidianos}\label{ejem:euclidianos}
	Los enteros $\Enteros$ son un dominio euclidiano: la funci\'on
	valor absoluto $N(x)=|x|$ tiene la propiedad que los caracteriza.
	Los anillos de polinomios $\polinomios\Racionales{X}$,
	$\polinomios\Reales{X}$ y $\polinomios\Complejos{X}$ (o, m\'as en
	general, polinomios en una indeterminada, sobre un cuerpo) son
	dominios euclidianos. Sin embargo, $\polinomios\Enteros{X}$ no lo es.
	El dominio $\polinomios\Enteros{\sqrt{-6}}$ tampoco es un dominio
	euclidiano.
\end{ejemEuclidianos}

\begin{ejemEuclidianos}\label{ejem:euclidianos:gauss}
	El anillo $\EnterosGauss$ es, como conjunto, el subconjunto
	de $\Complejos$ de elementos de la forma $x+yi$, $x,y\in\Enteros$.
	Si $\alpha=a+bi$ y $\beta=c+di$ son elementos de
	$\EnterosGauss$ ($a,b,c,d\in\Enteros$), entonces
	\begin{displaymath}
		\begin{aligned}
			\alpha+\beta & \,=\,(a+bi)+(c+di)\,=\,(a+c)+(b+d)\,i
				\,\in\,\EnterosGauss\quad\text{y} \\
			\alpha\beta & \,=\,(a+bi)\,(c+di)\,=\,
				(ac-bd)+(ad+bc)\,i
				\,\in\,\EnterosGauss
			\text{ .}
		\end{aligned}
		%
	\end{displaymath}
	%
	Adem\'as, $0,1\in\EnterosGauss$ (m\'as aun,
	$\Enteros\subset\EnterosGauss$). Tambi\'en se cumple que
	$-\alpha\in\EnterosGauss$, si $\alpha\in\EnterosGauss$.
	De esto se deduce que $\EnterosGauss$ es un dominio \'{\i}ntegro
	\quedacomoejercicio.%
	\hint{
		Todo ocurre dentro de $\Complejos$.
	}

	Veamos que $\EnterosGauss$ es un dominio euclidiano.%
	\footnote{
		Esto s\'{\i} depende de $\EnterosGauss$, no es algo
		``heredado''.
	}
	Dados $x,y\in\Racionales$, definimos la \emph{norma} de $x+yi$
	como
	\begin{displaymath}
		N(x+yi)\,=\,x^2+y^2
		\text{ .}
	\end{displaymath}
	%
	Esta funci\'on cumple $N(\beta\beta')=N(\beta)N(\beta')$ y
	$N(x+yi)\in\Enteros$, si $x,y\in\Enteros$.
	Dados elementos $\alpha=a+bi$ y $\beta=c+di$ de $\EnterosGauss$,
	$\alpha\neq 0$,
	\begin{enumerate}[(i)]
		\item\label{item:ejem:euclidianos:gauss:i}
			existen $u,v\in\Racionales$ tales que
			$(u+vi)\alpha=\beta$;
		\item\label{item:ejem:euclidianos:gauss:ii}
			dados $u,v\in\Racionales$, existen $m,n\in\Enteros$
			tales que $|m-u|\leq 1/2$ y $|n-v|\leq 1/2$;
		\item\label{item:ejem:euclidianos:gauss:iii}
			si $q:=m+ni$ y $r:=\beta-q\alpha$, entonces
			$N(r)<N(\alpha)$ (o $r=0$).
	\end{enumerate}
	%
	Con respecto a \eqref{item:ejem:euclidianos:gauss:iii},
	$N(r)=N((u+vi)-q)N(\alpha)$, pero $N((u+vi)-q)\leq 1/2$. La funci\'on
	norma tiene la propiedad de la \defname~\ref{def:euclidianos}
\end{ejemEuclidianos}

% \subsection{Cuerpos}%
	% \label{subsec:cuerpos}
\begin{defEuclidianos}\label{def:euclidianos:unidades}
	Dado $a\in A$, si existe $b\in A$ tal que $ab=ba=1$, se dice que $a$
	es una \emph{unidad} del anillo o que es \emph{inversible} en el
	anillo.
\end{defEuclidianos}

\begin{ejemEuclidianos}\label{ejem:euclidianos:unidades}
	Las unidades de $\Enteros$ son $\pm 1$.
\end{ejemEuclidianos}

\begin{defEuclidianos}\label{def:euclidianos:cuerpo}
	Un anillo (conmutativo, con $1$) con la propiedad de que todos sus
	elementos distintos de cero son inversibles se denomina \emph{cuerpo}.
\end{defEuclidianos}

\begin{ejemEuclidianos}\label{ejem:euclidianos:cuerpo}
	% Los anillos $\Racionales$, $\Reales$, $\Complejos$ y $\Enterosmod[p]$
	% ($p$ primo) no son cuerpos.
	% Los anillos $\Enteros$ y $\Enterosmod[m]$ ($m$ compuesto) no son
	% cuerpos.
	Los anillos $\Racionales$, $\Reales$ y $\Complejos$ son cuerpos.
	El anillo $\Enteros$ no es un cuerpo.
\end{ejemEuclidianos}

\subsection*{Ejercicios}
\theoremstyle{definition}
\newtheorem{ejerEuclidianos}{\ejername}[section]

%-------------

\begin{ejerEuclidianos}\label{ejer:euclidianos:eisenstein}
	Sea $\raizcubica\in\Complejos$ tal que $\raizcubica^3=1$, pero
	$\raizcubica\neq 1$.%
	\footnote{
		Por ejemplo, $\raizcubica=e^{2\pi i/3}=\frac{-1+\sqrt{-3}} 2$.
	}
	\begin{enumerate}[(i)]
		\item\label{ejer:euclidianos:eisenstein:pre:i}
			Probar que $\raizcubica^2+\raizcubica+1=0$.
		\item\label{ejer:euclidianos:eisenstein:pre:ii}
			Probar que $\conj\raizcubica\neq\raizcubica$ y que
			$\conj\raizcubica^2+\conj\raizcubica+1=0$, tambi\'en.
		\item\label{ejer:euclidianos:eisenstein:pre:iii}
			Concluir que
			$X^2+X+1=(X-\raizcubica)\,(X-\conj\raizcubica)$ y,
			en particular, que $\raizcubica+\conj\raizcubica=-1$
			y que $\raizcubica\conj\raizcubica=1$.
	\end{enumerate}
	%
	Dados $x,y\in\Racionales$, definimos
	\begin{displaymath}
		N(x+y\raizcubica):=x^2-xy+y^2
		\text{ .}
	\end{displaymath}
	%
	\begin{enumerate}[(i)]
		\item\label{item:ejer:euclidianos:eisenstein:norma:i}
			Como $\{1,\raizcubica\}$ es un conjunto l.i. sobre
			$\Racionales$, esta expresi\'on no es ambigua.
		\item\label{item:ejer:euclidianos:eisenstein:norma:i}
			Probar que $N(\beta\beta')=N(\beta)N(\beta')$ y
			que $N(x+y\raizcubica)\in\Enteros$, si
			$x,y\in\Enteros$.
	\end{enumerate}
	%
	Sea $\EnterosEisenstein\subset\Complejos$ el subconjunto
	\begin{displaymath}
		\EnterosEisenstein\,:=\,\big\{
			a+b\raizcubica\,:\,a,b\in\Enteros\big\}
		\text{ .}
	\end{displaymath}
	%
	\begin{enumerate}[(i)]
		\item\label{item:ejer:euclidianos:eisenstein:enteros:i}
			Probar que, si $\alpha,\beta\in\EnterosEisenstein$,
			entonces $\alpha+\beta$, $\alpha\beta$, $-\alpha$ y
			$\conj\alpha$ pertenecen a $\EnterosEisenstein$,
			tambi\'en.
		\item\label{item:ejer:euclidianos:eisenstein:enteros:ii}
			Probar que, $\Enteros\subset\EnterosEisenstein$.
	\end{enumerate}
	%
	Emular el argumento del \ejemname~\ref{ejem:euclidianos:gauss} para
	probar que $\EnterosEisenstein$ es un dominio euclidiano: dados
	$\alpha=a+b\raizcubica$ y $\beta=c+d\raizcubica$, $\alpha\neq 0$,
	\begin{enumerate}[(i)]
		\item\label{item:ejer:euclidianos:eisenstein:i}
			existen $u,v\in\Racionales$ tales que
			$(u+v\raizcubica)\alpha=\beta$;
		\item\label{item:ejer:euclidianos:eisenstein:ii}
			dados $u,v\in\Racionales$, existen $m,n\in\Enteros$
			tales que $|m-u|\leq 1/2$ y $|n-v|\leq 1/2$;
		\item\label{item:ejer:euclidianos:eisenstein:iii}
			si $q:=m+n\raizcubica$ y $r:=\beta-q\alpha$, entonces
			$N(r)<N(\alpha)$ (o $r=0$).
	\end{enumerate}
	%
	?`En qu\'e punto falla el argumento para el dominio
	$\polinomios[\sqrt{-6}]\Enteros$?
\end{ejerEuclidianos}

\begin{ejerEuclidianos}[Unidades de un anillo]\label{ejer:euclidianos:unidades}
	Dado $a\in A$, si existe $b\in A$ tal que $ab=ba=1$, se dice que
	$a$ es una \emph{unidad} del anillo o que es \emph{inversible} en el
	anillo. Por ejemplo, las unidades de $\Enteros$ son $\pm 1$.
	\begin{itemize}
		\item En $\EnterosGauss$, probar que
			$N(\alpha)=\alpha\conj\alpha$ ($=\conj\alpha\alpha$) y
			concluir que $\alpha\in\EnterosGauss$ es inversible,
			si y s\'olo si $N(\alpha)=\pm 1$ ?`Puede dar $-1$?
			Hallar todas las unidades, en este caso.
		\item En $\EnterosEisenstein$, probar que
			$N(\alpha)=\alpha\conj\alpha$ ($=\conj\alpha\alpha$) y
			concluir que $\alpha\in\EnterosEisenstein$ es
			inversible, si y s\'olo si $N(\alpha)=\pm 1$ ?`Puede
			dar $-1$? Hallar todas las unidades, en este caso.
		\item Hacer lo mismo con $\polinomios[\sqrt{-6}]\Enteros$
			y con $\polinomios[\sqrt{-2}]\Enteros$.%
			\footnote{
				Ver \ejername~%
				\ref{ejer:euclidianos:cuadratico:menos-dos}.
			}
	\end{itemize}
	%
\end{ejerEuclidianos}

\begin{ejerEuclidianos}\label{ejer:euclidianos:cuadratico:menos-dos}
	Repetir el argumento del \ejername~\ref{ejer:euclidianos:eisenstein}
	con $\polinomios[\delta]\Enteros\subset\Complejos$, el subconjunto
	de los complejos de la forma $a+b\delta$, donde $\delta\in\Complejos$
	es ra\'{\i}z de $X^2+2$.
\end{ejerEuclidianos}

\begin{ejerEuclidianos}
	Probar que $2$ es divisible por $(1+\raizcuarta)^2$ en $\EnterosGauss$.
\end{ejerEuclidianos}

\begin{ejerEuclidianos}
	Probar que $3$ es divisible por $(1-\raizcubica)^2$ en
	$\EnterosEisenstein$.
\end{ejerEuclidianos}



