\theoremstyle{definition}
\newtheorem{ejerPolinomios}{\ejername}[section]

%-------------

\begin{ejerPolinomios}
	Probar que $\EnterosGauss$ y $\EnterosEisenstein$ son iguales a los
	subanillos $\Complejos$ de expresiones polinomiales en $\raizcuarta$
	y, respectivamente, $\raizcubica$ con coefcientes en $\Enteros$.
\end{ejerPolinomios}

\begin{ejerPolinomios}
	Probar que el subanillo $\polinomios \Enteros{\sqrt 2}\subset\Reales$
	de expresiones polinomiales en $\sqrt 2$ con coeficientes en
	$\Enteros$ es, como conjunto, igual al subconjunto de elementos
	de la forma $a+b\sqrt 2$ donde $a,b\in\Enteros$.
	Hacer lo an\'alogo con $\polinomios \Enteros{\sqrt 3}$.
\end{ejerPolinomios}

\begin{ejerPolinomios}
	Mostrar que no existe ninguna funci\'on
	\begin{math}
		f:\,\polinomios\Enteros{\sqrt 2}\rightarrow
			\polinomios\Enteros{\sqrt 3}
	\end{math}
	% tal que $f(1)=1$ y que \emph{respete la suma y el producto}, es decir,
	% tal que
	% \begin{displaymath}
		% f(u+v)\,=\,f(u)\,+\,f(v)\quad\text{y}\quad
			% f(uv)\,=\,f(u)\,f(v)
		% \text{ ,}
	% \end{displaymath}
	% %
	% siempre que $u,v\in\polinomios\Enteros{\sqrt 2}$.
	tal que $f(1)=1$ y que respete la suma y el producto.
\end{ejerPolinomios}

\begin{ejerPolinomios}
	Describir los elementos del anillo de expresiones polinomiales
	$\polinomios\Enteros{\frac 1{2}}$ ?`Es $\frac 1{2}$ trascendente
	sobre $\Enteros$? ?`Existe un polinomio m\'onico (coeficiente
	principal igual a $1$) $p\in\polinomios\Enteros x$ tal que
	$p\big(\frac 1{2}\big)=0$?
\end{ejerPolinomios}

\begin{ejerPolinomios}\label{ejer:polinomios:grado}
	Sean $p,q\in\polinomios\Enteros x$ los polinomios
	$p=3x^5-2x^3+x^2-5x-1$ y $q=2x^4-3x^2-x+5$. Determinar:
	\begin{enumerate}[(i)]
		\item\label{item:ejer:polinomios:grado:i}
			$\grado(p^2-q^3)$;
		\item\label{item:ejer:polinomios:grado:ii}
			el coeficiente de $x^6$ en $pq$;
		\item\label{item:ejer:polinomios:grado:iii}
			$\grado(p+q^3)$;
		\item\label{item:ejer:polinomios:grado:iv}
			el coeficiente de $x^{10}$ en $pq$;
		\item\label{item:ejer:polinomios:grado:v}
			si existe un polinomio $t\in\polinomios\Enteros x$
			tal que $p=tq$;
		\item\label{item:ejer:polinomios:grado:vi}
			si existen enteros $m,n$ tales que $p^n=q^m$.
	\end{enumerate}
	%
\end{ejerPolinomios}

\begin{ejerPolinomios}
	Sean $p,q\in\polinomios\Enteros x$ los polinomios
	$p=x^3-2x+3$ y $q=2x^5-5x^4+3x^3+2x^2$. Hallar polinomios
	$t,r\in\polinomios\Enteros x$ tales que $q=tp+r$ con $r=0$ o
	$\grado(r)<3$.
\end{ejerPolinomios}

\begin{ejerPolinomios}\label{ejer:polinomios:grado:bis}
	Sea $\simEnterosmod[4]$ el anillo de enteros m\'odulo $4$.
	\begin{enumerate}[(i)]
		\item\label{item:ejer:polinomios:grado:bis:i}
			Calcular los grados de $p^2$, $pq$ y $q-h$,
			donde $p,q,h\in\polinomios{\simEnterosmod[4]} x$ son
			los polinomios $p=1+2x$, $q=1+2x+3x^2$ y $h=2-x+x^2$.
		\item\label{item:ejer:polinomios:grado:bis:i}
			?`Existen polinomios
			$t,s\in\polinomios{\simEnterosmod[4]} x$, ambos
			no nulos, tales que $ts=0$?
		\item\label{item:ejer:polinomios:grado:bis:i}
			?`Existen polinomios
			$t\in\polinomios{\simEnterosmod[4]} x$ tales que
			$t^n=0$ para alg\'un $n\geq 0$, pero $t\neq 0$?
		\item\label{item:ejer:polinomios:grado:bis:i}
			?`Existen en $\polinomios{\simEnterosmod[4]} x$
			polinomios inversibles de grado mayor que $0$?
	\end{enumerate}
	%
\end{ejerPolinomios}

\begin{ejerPolinomios}
	Si $A$ es un dominio \'{\i}ntegro y $p,q\in\polinomios A x$ son
	polinomios distintos del polinomio nulo, entonces
	$\grado(pq)=\grado(p)+\grado(q)$. Probar que $A$ es un dominio
	\'{\i}ntegro, si y s\'olo si $\polinomios A x$ lo es.
\end{ejerPolinomios}

\begin{ejerPolinomios}
	Si $A$ es un dominio \'{\i}ntegro, entonces los polinomios
	inversibles en $\polinomios A x$ son exactamente los elementos
	inversibles en $A$, es decir, los polinomios de grado $0$ que son
	unidades de $A$.
\end{ejerPolinomios}

\begin{ejerPolinomios}
	Determinar los polinomios inversibles en
	$\polinomios{\simEnterosmod[4]} x$ y en
	$\polinomios{\simEnterosmod[5]} x$.
\end{ejerPolinomios}

\begin{ejerPolinomios}\label{ejer:polinomios:divisibilidad}
	Probar propiedades an\'alogas a las de la relaci\'on de divisibilidad
	en $\Enteros$ para la relaci\'on de divisibilidad en polinomios.
\end{ejerPolinomios}

\begin{ejerPolinomios}\label{ejer:polinomios:irreducibles}
	Si $k$ es un cuerpo, dar una definici\'on de primo (polinomios
	irreducibles) en $\polinomios k x$ y enunciar y demostrar algunas
	de sus propiedades.
\end{ejerPolinomios}

\begin{ejerPolinomios}\label{ejer:polinomios;mcd}
	El \emph{m\'aximo com\'un divisor} entre dos polinomios no nulos
	$f,g\in\polinomios k x$ con coeficientes en un cuerpo $k$ se define
	como el polinomio \emph{m\'onico} (coeficiente principal igual a $1$)
	de grado m\'aximo que divide a ambos.
	\begin{enumerate}[(i)]
		\item\label{ejer:polinomios:mcd:bezout}
			Probar la identidad de B\'ezout: que, si $h=\mcd{f,g}$
			es el m\'aximo com\'un divisor de $f$ y $g$, entonces
			existen polinomios $p$ y $q$ tales que $h=fp+gq$.
		\item\label{ejer:polinomios:mcd:equivalencias}
			Probar que las siguientes afirmaciones sobre un
			polinomio $h\in\polinomios k x$ son equivalentes:
			\begin{enumerate}[(a)]
				\item\label{ejer:polinomios:mcd:a}
					$h$ es \emph{el} polinomio m\'onico
					de grado m\'{\i}nimo de la forma
					$h=fp+gq$ con $p,q\in\polinomios k x$;
				\item\label{ejer:polinomios:mcd:b}
					$h$ es m\'onico, es divisor com\'un
					de $f$ y de $g$ y es divisible por
					cualquier otro divisor com\'un;
				\item\label{ejer:polinomios:mcd:c}
					$h$ es el m\'aximo com\'un divior
					de $f$ y $g$.
			\end{enumerate}
			%
	\end{enumerate}
	%
\end{ejerPolinomios}

\begin{ejerPolinomios}\label{ejer:polinomios:enteros}
	?`C\'omo se adapta la noci\'on de primo (ver \ejername~%
	\ref{ejer:polinomios:irreducibles}) al anillo de polinomios con
	coeficientes enteros $\polinomios\Enteros x$?
	?`Tiene sentido hablar de m\'aximo com\'un divisor en
	$\polinomios\Enteros x$? ?`Se verifica la identidad de B\'ezout?
\end{ejerPolinomios}

