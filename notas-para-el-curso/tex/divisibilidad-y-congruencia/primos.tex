\theoremstyle{plain}
\newtheorem{teoPrimos}{\teoname}[section]
\newtheorem{coroPrimos}[teoPrimos]{\coroname}
\newtheorem{lemaPrimos}[teoPrimos]{\lemaname}

\theoremstyle{definition}
\newtheorem{defPrimos}[teoPrimos]{\defname}
\newtheorem{obsPrimos}[teoPrimos]{\obsname}

%-------------

% \subsection{N\'umeros primos}%
	% \label{subsec:numeros-primos}
\begin{defPrimos}\label{def:primos}
	Un n\'umero natural $n>1$ se dice \emph{primo}, si no posee
	\emph{divisores propios}, es decir, no existe n\'umero natural $d$
	que cumpla simult\'aneamente que $d\mid n$ y que $1<d<n$.
\end{defPrimos}

% \subsection{Factorizaci\'on en primos}%
	% \label{subsec:factorizacion-en-primos}
\begin{teoPrimos}\label{teo:factorizacion}
	Todo n\'umero natural $n>1$ es producto de n\'umeros primos.
\end{teoPrimos}

\begin{proof}
	No existen naturales entre $1$ y $2$. En particular, $2$ es primo.
	Si $n>2$, o bien $n$ es primo, o bien posee divisores propios.
	Si $d\mid n$ y $1<d<n$ es un divisor propio, el natural $m=n/d$
	satisface $1<m<n$ y $md=n$.
	% Esto \'ultimo es la \emph{definici\'on} de $m$.
	Inductivamente, $m$ y $d$ son productos de primos y, por lo tanto,
	tambi\'en lo es $n$.
\end{proof}

\begin{defPrimos}\label{def:compuesto}
	Si $n>1$ no es primo, se dice que es \emph{compuesto} (pues es
	producto de m\'as de un factor primo).
	% Incluso sin factorizaci\'on \'unica, no puede un n\'umero ser
	% primo (irreducible) y compuesto, a la vez.
\end{defPrimos}

% \subsection{El Teorema fundamental de la Aritm\'etica}%
	% \labe{subsec:el-teorema-fundamental-de-la-aritmetica}
\begin{obsPrimos}\label{obs:primos:mcd}
	Si $a$ y $p$ son n\'umeros naturales y $p$ es primo, entonces,
	o bien $p\mid a$, o bien $\mcd{a,p}=1$, pues $g=\mcd{a,p}$, siendo
	un divisor de $p$ (que es primo), es, o bien $g=1$, o bien $g=p$.
\end{obsPrimos}

\begin{lemaPrimos}\label{lema:primos}
	Si $a$, $b$ y $p$ son n\'umeros naturales, $p$ es primo y $p\mid ab$,
	entonces $p\mid a$, o bien $p\mid b$. M\'as en general, si
	$p\mid a_1\cdots a_k$, entonces $p\mid a_i$ para alg\'un $i$.
\end{lemaPrimos}

\begin{proof}
	\quedacomoejercicio.%
	\hint{
		Si $p\nmid a$, entonces existen $x,y\in\Enteros$ tales
		que $ax+py=1$%.
		% Primero, habr\'{\i}a que extender la noci\'on de
		% divisibilidad a $\Enteros$.
		% (ver \ejername~\ref{ejer:divisibilidad:enteros})
		.
	}
\end{proof}

\begin{obsPrimos}\label{obs:primos:irreducibles}
	Rec\'{\i}procamente, si un n\'umero natural $p>1$ posee la propiedad
	siguiente:
	\begin{center}
		para todo par de n\'umeros naturales $a$ y $b$,
		$p\mid ab$ imlpica que $p\mid a$ o $p\mid b$,
	\end{center}
	%
	entonces $p$ es primo, pues, si $d\mid p$ es un divisor, se cumple
	que $1\leq d\leq p$, luego $p=dm$, con lo que, aplicando la propiedad,
	o bien $p\mid d$, o bien $p\mid m$ y, por lo tanto,
	\begin{itemize}
		\item $p\mid d$ y $p=d$, por
			\eqref{item:teo:divisibilidad:propiedades:iv} del
			\teoname~\ref{teo:divisibilidad:propiedades}
			(y, as\'{\i}, $m=1$), o bien
		\item $p\mid m$ y $m=np$ y, as\'{\i}, $p=dnp$, con lo que,
			cancelando, $dn=1$ y $d=n=1$.
	\end{itemize}
	%
\end{obsPrimos}

\begin{defPrimos}\label{def:factorizacion}
	Dado un natural $n>1$, llamamos \emph{factorizaci\'on de $n$ como %
	producto de primos} a toda escritura de $n$ como producto de factores
	primos \'unicamente (o, por abuso, primos a potencias).
\end{defPrimos}

\begin{teoPrimos}[Teorema fundamental de la Aritm\'etica]%
	\label{teo:fundamental}
	Si $n>1$ es un n\'umero natural, existe un \'unica factorizaci\'on de
	$n$ como producto de primos, salvo por el orden de los factores.
\end{teoPrimos}

\begin{proof}
	Que todo natural $n>1$ se puede expresar como producto de primos es
	la conclusi\'on del \teoname~\ref{teo:factorizacion}. Lo que resta
	probar es que dicha factorizaci\'on es \'unica, a menos de
	intercambiar el orden de los factores. Precisamente, si
	\begin{displaymath}
		% \label{eq:teo:fundamental}
		n\,=\,p_1\,\cdots\,p_r
		\quad\text{y}\quad n\,=\,q_1\,\cdots\,q_s
	\end{displaymath}
	%
	% ($r,s\geq 1$)
	son dos factorizaciones de $n$ como producto de primos,
	entonces $r=s$ y
	% existe una permutaci\'on
	% $\sigma:\,\{1,\,\dots,\,r\}\rightarrow\{1,\,\dots,\,r\}$ tal que
	% $p_{\sigma i}=q_i$ para todo $i$.
	los factores primos distintos que aparecen son los mismos y la
	cantidad de veces que cada uno de ellos aparece es igual en ambas
	factorizaciones.

	Igualando $p_1\cdots p_r=q_1\cdots q_s$ y cancelando factores que
	aparecen a ambos lados de la igualdad, podemos suponer que ning\'un
	$p_i$ est\'a entre los $q_j$ y viceversa.
	% (pero, ahora, $r,s\geq 0$, es decir, podr\'{\i}an ser iguales a $0$)
	Pero, si $t$ es un primo que divide a $n$, entonces $t\mid p_{i_0}$
	para alg\'un $i_0$, con lo que $t=p_{i_0}$. An\'alogamente,
	$t=q_{j_0}$ para alg\'un $j_0$. Como consecuencia, $p_{i_0}=q_{j_0}$,
	contradiciendo la suposici\'on.
\end{proof}

\subsection*{Ejercicios}
\theoremstyle{definition}
\newtheorem{ejerPrimos}{\ejername}[section]

%-------------

\begin{ejerPrimos}\label{ejer:factorizacion:pares}
	Consideremos el subconjunto $\cal P\subset\Naturales$ conformado por
	los n\'umeros naturales pares. Dados $a,b\in\cal P$, el producto
	$ab$ tambi\'en pertenece a $\cal P$. Podemos, entonces, definir una
	noci\'on de divisibilidad en $\cal P$: decimos que \emph{$a$ divide %
	a $b$ en $\cal P$}, si existe $c\in\cal P$ talque $ac=b$.
	Podemos, por lo tanto, hablar de primos en $\cal P$: un elemento
	$p\in\cal P$ es \emph{primo en $\cal P$}, si no posee
	\emph{divisores propios en $\cal P$}, es decir, si no existe
	$d\in\cal P$ tal que $d$ divide a $p$ en $\cal P$ y $d<p$.
	Con estas definiciones, probar que
	\begin{enumerate}[(i)]
		\item\label{item:ejer:factorizacion:pares:casos}
			los elementos $2$, $6$, $10$, $14$, \dots y $30$ son
			primos en $\cal P$, pero que
			$4$, $8$, $12$, $16$, \dots y $28$ no lo son, son
			\emph{compuestos en $\cal P$};
		\item\label{item:ejer:factorizacion:pares:existe}
			todo elemento de $\cal P$ se puede escribir como
			producto de primos en $\cal P$;
		\item\label{item:ejer:factorizacion:pares:no-es-unica}
			existe un elemento de $\cal P$ que admite m\'as de
			una factorizaci\'on en primos de $\cal P$ (exhibir
			un ejemplo).
	\end{enumerate}
	%
\end{ejerPrimos}

\begin{ejerPrimos}\label{ejer:primos:factorizacion:cuadratico}
	Sea $A\subset\Complejos$ el siguiente subconjunto de los n\'umeros
	complejos:
	\begin{displaymath}
		A \,=\,\big\{x+y\sqrt{-6}\,:\,x,y\in\Enteros\big\}
		\text{ .}
	\end{displaymath}
	%
	Probar que
	\begin{enumerate}[(i)]
		\item\label{item:ejer:primos:cuadratico:enteros}
			$\Enteros\subset A$;
		\item\label{item:ejer:primos:cuadratico:anillo}
			si $\alpha,\beta\in A$, entonces
			$\alpha+\beta\in A$ y $\alpha\beta\in A$;
		\item\label{item:ejer:primos:cuadratico:conjugado}
			si $\alpha\in A$, su conjugado, $\conj\alpha$,
			tambi\'en pertenece a $A$.
	\end{enumerate}
	%
	Podemos, entonces, hablar de divisibilidad en $A$:
	\emph{$\alpha$ divide a $\beta$ en $A$}, si existe $\gamma\in A$ tal
	que $\alpha\gamma=\beta$.

	Dado $\alpha=x+y\sqrt{-6}\in A$, definimos su \emph{norma} como
	\begin{displaymath}
		N(\alpha)\,=\,x^2+6y^2
		\text{ .}
	\end{displaymath}
	%
	Probar que
	\begin{enumerate}[(i)]
		\item\label{item:ejer:primos:cuadratico:norma:entera}
			si $\alpha\in A$, entonces $N(\alpha)\in\Enteros$;
		\item\label{item:ejer:primos:norma:no-negativa-cero-y-unidades}
			para todo $\alpha\in A$, $N(\alpha)\geq 0$,
			$N(\alpha)=0$, si y s\'olo si $\alpha=0$ y
			$N(\alpha)=1$, si y s\'olo si $\alpha=\pm 1$;%
			\footnote{
				En cualquier otro caso, $N(\alpha)>1$.
			}
		\item\label{item:ejer:primos:cuadratico:norma:bis}
			si $\alpha\in A$, entonces
			$N(\alpha)=\alpha\conj\alpha$;
		\item\label{item:ejer:primos:cuadratico:norma:multiplicativa}
			si $\alpha,\beta\in A$, entonces
			$N(\alpha\beta)=N(\alpha)N(\beta)$.
	\end{enumerate}
	%
	Con esto podemos definir una noci\'on de divisor propio $A$:
	decimos que \emph{$\alpha$ divide propiamente a $\beta$ en $A$},
	si $\alpha$ divide a $\beta$ en $A$ y $N(\alpha)>1$.

	Un elemento $\gamma\in A$ es \emph{primo en $A$}, si $N(\gamma)>1$ y
	$\gamma$ no posee divisores propios en $A$. Probar que
	\begin{enumerate}[(i)]
		\item\label{item:ejer:primos:cuadratico:casos}
			$2$ y $5$ son primos en $A$;
		\item\label{item:ejer:primos:cuadratico:factorizacion:existe}
			todo elemento de $A$ se puede escribir como producto
			de primos en $A$;
		\item\label{item:ejer:primos:cuadratico:factorizacion:no-es-unica}
			existe un elemento de $A$ que admite m\'as de una
			factorizaci\'on en primos de $A$ (exhibir un ejemplo);
		\item\label{item:ejer:primos:cuadratico:otros-casos}
			$2+\sqrt{-6}$ y $2-\sqrt{-6}$ son primos en $A$.
	\end{enumerate}
	%
\end{ejerPrimos}

\begin{ejerPrimos}\label{ejer:primos:fundamental:otra-demo}
	En este ejercicio veremos otra demostraci\'on del Teorema fundamental
	de la Aritm\'etica (\teoname~\ref{teo:fundamental}). Supongamos que el
	resultado es falso y sea $n$ el menor natural para el cual existen dos
	factorizaciones distintas: $n=p_1\cdots p_r=q_1\cdots q_s$, donde los
	factores primos $p_i$ no son los mismos que los $q_j$. Mostrar que
	\begin{enumerate}[(i)]
		\item\label{item:ejer:primos:fundamental:i}
			$r$ y $s$ son ambos $>1$ y \emph{ning\'un} $p_i$ es
			igual a un $q_j$;%
			% (y viceversa);
			\hint{
				Para esto \'ultimo, usar la minimalidad de $n$.
			}
		\item\label{item:ejer:primos:fundamental:ii}
			si $p_1<q_1$, entonces $N:=(q_1-p_1)q_2\cdots q_r$
			es $<n$;
		\item\label{item:ejer:primos:fundamental:iii}
			el n\'umero natural $N$ definido en
			\eqref{item:ejer:primos:fundamental:ii} satisface
			$N=p_1(p_2\cdots p_r-q_2\cdots q_s)$;
		\item\label{item:ejer:primos:fundamental:iv}
			$N$ admite dos factorizaciones distintas,
			contradiciendo la minimalidad de $n$.
	\end{enumerate}
	%
\end{ejerPrimos}

\begin{ejerPrimos}\label{ejer:primos:euclides}
	La cantidad de n\'umeros primos es infinita.%
	\hint{
		Talvez la demostraci\'on m\'as com\'un de esto sea: asumir
		que la cantidad de primos es finita, $\lista p{k}$ y
		considerar $n:=1+p_1\cdots p_k$.
	}
\end{ejerPrimos}

\begin{ejerPrimos}
	Fijado un entero positivo $k$, existen $k$ n\'umeros naturales
	compuestos consecutivos.%
	\hint{
		Si $2\leq j\leq k+1$, entonces $j\mid (k+1)!+j$.
	}
\end{ejerPrimos}

\begin{ejerPrimos}\label{ejer:primos:suma-de-reciprocos-diverge}
	La sumatoria sobre los rec\'{\i}procos de los primos diverge.
	Espec\'{\i}ficamente, dado $y\geq 2$, la sumatoria
	\begin{equation}
		\label{eq:ejer:primos:suma-de-reciprocos}
		\sum_{p\leq y}\,\frac 1{p}\,>\,\llog\,y\,-\,1
		\text{ .}
	\end{equation}
	%
	En particular, hay infinitos primos.
	Separamos la demostraci\'on en distintos pasos:
	\begin{enumerate}[(i)]
		\item\label{item:ejer:primos:suma-de-reciprocos:i}
			probar que, si $\cal N$ denota el conjunto de
			naturales $n$ en cuya factorizaci\'on s\'olo
			aparecen primos $p\leq y$, entonces%
			\footnote{
				La suma de las potencias de $1/p$, $p$ primo
				converge absolutamente, con lo que no hay
				problema en la definici\'on del producto
				(finito) sobre los primos $p\leq y$.
			}
			\begin{displaymath}
				\prod_{p\leq y}\,\bigg(
					1+\frac 1{p}+\frac 1{p^2}+\cdots
					\bigg)
					\,=\,\sum_{n\in\cal N}\,\frac 1{n}
				\text{ ;}
			\end{displaymath}
			%
		\item\label{item:ejer:primos:suma-de-reciprocos:ii}
			probar que, si $n\leq y$, entonces $n\in\cal N$ y
			los sumandos de $\sum_{n\leq y}\frac 1{n}$ son
			sumandos de $\sum_{n\in\cal N}\frac 1{n}$;
		\item\label{item:ejer:primos:suma-de-reciprocos:iii}
			probar que, si $N$ es el mayor entero $\leq y$,
			entonces
			\begin{displaymath}
				\sum_{n=1}^N\,\frac 1{n}\,\geq\,
					\int_1^{N+1}\,\frac{\de x} x\,=\,
					\log(N+1)\,>\,\log\,y
				\text{ ;}
			\end{displaymath}
			%
		\item\label{item:ejer:primos:suma-de-reciprocos:iv}
			deducir de
			\eqref{item:ejer:primos:suma-de-reciprocos:iii} que
			\begin{displaymath}
				\prod_{p\leq y}\,\bigg(1-\frac 1{p}\bigg)^{-1}
					\,>\,\log\,y
				\text{ ;}
			\end{displaymath}
			%
		\item\label{item:ejer:primos:suma-de-reciprocos:v}
			usando la desigualdad $e^{v+v^2}\geq (1-v)^{-1}$,
			probar que
			\begin{displaymath}
				\sum_{p\leq y}\,\frac 1{p}\,+\,
					\sum_{p\leq y}\,\frac 1{p^2}\,>\,
					\llog\,y
				\text{ ;}
			\end{displaymath}
			%
		\item\label{item:ejer:primos:suma-de-reciprocos:vi}
			notar que $\sum_{p\leq y}\,\frac 1{p^2}\,<\,1$;%
			\hint{
				Usando el criterio de comparaci\'on
				($\sum_{n\geq 2}\,\frac 1{n}$) y el criterio
				integral
				($\int_1^\infty\,\frac{\de x}{x^2}=1$),
				por ejemplo.
			}
		\item\label{item:ejer:primos:suma-de-reciprocos:vii}
			deducir la desigualdad
			\eqref{eq:ejer:primos:suma-de-reciprocos}.
	\end{enumerate}
	%
\end{ejerPrimos}

\begin{ejerPrimos}
	Probar que
	\begin{itemize}
		\item todo entero de la forma $3k+2$ tiene un factor primo
			del mismo tipo;
		\item todo entero de la forma $4k+3$ tiene un factor primo
			del mismo tipo;
		\item todo entero de la forma $6k+5$ tiene un factor primo
			del mismo tipo.
	\end{itemize}
	%
\end{ejerPrimos}

\begin{ejerPrimos}
	Dados $a,b,c,d,m,n,u,v\in\Enteros$ tales que
	$ad-bc=\pm1$, $u=am+bn$ y $v=cm+dn$, se cumple que
	$\mcd{m,n}=\mcd{u,v}$.
\end{ejerPrimos}

\begin{ejerPrimos}
	Si $n>4$ es compuesto, entonces $n\mid (n-1)!$.
\end{ejerPrimos}

\begin{ejerPrimos}
	Supongamos que $u,v\in\Enteros$ son coprimos ($\mcd{u,v}=1$).
	Para todo $n\in\Enteros$, si $u\mid n$ y $v\mid n$, entonces
	$uv\mid n$. Esto no es cierto (existe un $n$ para el cual no es
	cierto), si $\mcd{u,v}>1$.
\end{ejerPrimos}

\begin{ejerPrimos}
	Si $\mcd{u,v}=1$, entonces $\mcd{u+v,u-v}\in\{1,2\}$.
\end{ejerPrimos}

% \begin{ejerPrimos}
	% $\mcd{a,a+k}\mid k$.
% \end{ejerPrimos}

% \begin{ejerPrimos}
	% Supongamos que encastramos varias copias de un pol\'{\i}gono regular
	% de manera que tengan un v\'ertice en com\'un y que no quede sector
	% libre alrededor de dicho punto del plano. Entonces, o bien
	% son seis tri\'angulos, o bien cuatro cuadrados, o bien tres
	% hex\'agonos.
% \end{ejerPrimos}

\begin{ejerPrimos}\label{ejer:primos:orden}
	Sea $p>1$ un primo. Dado $n\in\Enteros$, $n\neq 0$, el
	\emph{orden de $n$ en $p$} es el n\'umero entero no negativo
	$\valuacion[p](n)=j$ que cumple $p^j\mid n$ pero $p^{j+1}\nmid n$.
	Probar que, si $a,b\in\Enteros$,
	\begin{itemize}
		\item
			\begin{math}
				\valuacion[p](a+b)\geq\min\{
					\valuacion[p](a),\valuacion[p](b)\}
			\end{math}
			y que, si
			\begin{math}
				\valuacion[p](a)\neq \valuacion[p](b)
			\end{math},
			entonces vale $=$;
		\item
			\begin{math}
				\valuacion[p](ab)=\valuacion[p](a)+
					\valuacion[p](b)
			\end{math}
			y la funci\'on $\valuacion[p]$ se extiende a
			$\Racionales\setmin\{0\}$ de una \'unica manera
			posible de forma que esta igualdad valga para
			$a,b\in\Racionales\setmin\{0\}$.
	\end{itemize}
	%
\end{ejerPrimos}

\begin{ejerPrimos}
	Sean $a,b,c,d\in\Enteros$ tales que $\mcd{a,b}=\mcd{c,d}=1$.
	Si $\frac a{b}+\frac c{d}\in\Enteros$, entonces $b=\pm d$.
\end{ejerPrimos}


