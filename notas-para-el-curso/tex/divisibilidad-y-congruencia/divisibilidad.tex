\theoremstyle{plain}
\newtheorem{teoDivisibilidad}{\teoname}[section]
\newtheorem{coroDivisibilidad}[teoDivisibilidad]{\coroname}

\theoremstyle{definition}
\newtheorem{defDivisibilidad}[teoDivisibilidad]{\defname}
\newtheorem{obsDivisibilidad}[teoDivisibilidad]{\obsname}
\newtheorem{ejemDivisibilidad}[teoDivisibilidad]{\ejemname}

%-------------

En el fondo, Teor\'{\i}a de n\'umeros es el estudio de los n\'umeros naturales
--los n\'umeros $1$, $2$, $3$, \dots-- y sus propiedades.

% \subsection{La relaci\'on de divisibilidad}%
	% \label{subsec:la-relacion-de-divisibilidad}
\begin{defDivisibilidad}\label{def:divisibilidad}
	Un n\'umero natural $a$ \emph{divide} a un n\'umero natural $b$, si
	existe un n\'umero natural $c$ tal que $ac=b$; en tal caso,
	escribimos $a\mid b$.
\end{defDivisibilidad}

\begin{teoDivisibilidad}[Propiedades b\'asicas de la divisi\'on]%
	\label{teo:divisibilidad:propiedades}
	Sean $a$, $b$ y $c$ n\'umeros naturales
	(en particular $a,b,c>0$). Entonces,
	\begin{enumerate}[(i)]
		\item\label{item:teo:divisibilidad:propiedades:i}
			$a\mid b$ implica $a\mid bc$;
		\item\label{item:teo:divisibilidad:propiedades:ii}
			$a\mid b$ y $b\mid c$ implican $a\mid c$;
		\item\label{item:teo:divisibilidad:propiedades:iii}
			$a\mid b$ y $a\mid c$ implican $a\mid bx+cy$,
			para todo par de n\'umeros naturales $x$ e $y$;
		\item\label{item:teo:divisibilidad:propiedades:iv}
			$a\mid b$ y $b\mid a$ implican $a=b$;
		\item\label{item:teo:divisibilidad:propiedades:v}
			$a\mid b$ implica $a\leq b$;
		\item\label{item:teo:divisibilidad:propiedades:vi}
			$a\mid b$ es equivalente a $ma\mid mb$,
			para todo natural $m$.
	\end{enumerate}
	%
\end{teoDivisibilidad}

\begin{coroDivisibilidad}\label{coro:divisibilidad:propiedades}
	Si $b$ es un n\'umero natural ($\neq 0$), existe otro natural $x$ tal
	que $x\nmid b$; si $a$ es un natural y $a\neq 1$, existe otro natural
	$x$ tal que $a\nmid x$.
\end{coroDivisibilidad}

% \subsection{El Algoritmo de divisi\'on}%
	% \label{subsec:el-algoritmo-de-division}
\begin{teoDivisibilidad}[Algoritmo de divisi\'on]%
	\label{teo:divisibilidad:algoritmo}
	Dados n\'umeros naturales $a$ y $b$, si $b\geq a$, entonces
	\begin{enumerate}[(A)]
		\item\label{item:teo:divisibilidad:algoritmo:1}
			$a\mid b$ y existe un \'unico natural $q$ tal que
			$b=qa$, o bien
		\item\label{item:teo:divisibilidad:algoritmo:2}
			existen \'unicos naturales $q$ y $r$ tales que
			$b=qa+r$ y $1\leq r<a$.
	\end{enumerate}
	%
\end{teoDivisibilidad}

\begin{proof}
	El conjunto $\{b-ta\,:\,t\text{ natural, }ta\leq b\}$ es finito.
	Si $a\nmid b$, este conjunto est\'a contenido en los naturales y, por
	lo tanto, tiene un primer elemento, $r$ ($\neq 0$). Se cumple
	$r<a$ y $r=b-qa$ para cierto natural $q$. Si $b=qa+r=q_1a+r_1$
	con $1\leq r,r_1<a$, asumiendo $r<r_1$, se ve que
	$1\leq r_1-r<a$ y que $r_1-r=(q_1-q)a$ es divisible por $a$,
	contradiciendo \eqref{item:teo:divisibilidad:propiedades:v} del
	\teoname~\ref{teo:divisibilidad:propiedades}.
\end{proof}

% \subsection{El m\'aximo com\'un divisor}%
	% \label{subsec:el-maximo-comun-divisor}
\begin{defDivisibilidad}\label{def:mcd}
	Dados naturales $a$, $b$ y $c$, decimos que $a$ es un
	\emph{divisor com\'un} de $b$ y de $c$, si $a\mid b$ y $a\mid c$.
	El \emph{m\'aximo com\'un divisor} de $b$ y $c$, es el supremo de
	los divisores comunes de $b$ y de $c$; escribimos $\mcd{b,c}$ para
	denotar el m\'aximo com\'un divisor de $b$ y $c$.
\end{defDivisibilidad}

\begin{obsDivisibilidad}\label{obs:mcd}
	El conjunto de divisores de un natural es un conjunto finito.
	En particular, es finito el conjunto de divisores comunes de dos
	naturales y, por lo tanto, debe existir un elemento de valor
	absoluto m\'aximo dentro de este conjunto; dicho elemento es \'unico
	y $\geq 1$.
\end{obsDivisibilidad}

% \subsection{La identidad de B\'ezout}%
	% \label{subsec:la-identidad-de-bezout}
\begin{teoDivisibilidad}[Identidad de B\'ezout]\label{teo:mcd:bezout}
	Si $g=\mcd{b,c}$, existen \emph{enteros} $x$ e $y$ tales que
	\begin{displaymath}
		g\,=\,bx+cy
		\text{ .}
	\end{displaymath}
	%
\end{teoDivisibilidad}

\begin{proof}
	Sea $\cal C$ el conjunto de n\'umeros enteros de la forma $bx+cy$,
	donde $x,y\in\Enteros$. Se cumple que $0\in\cal C$ y que existe
	$l\in\cal C$ \emph{positivo} y m\'{\i}nimo entre los elementos
	positivos de $\cal C$ \quedacomoejercicio.
	% \footnote{
		% Los elementos positivos de $\cal C$ forman un subconjunto no
		% vac\'{\i}o de los naturales, pues, por ejemplo, con $x=y=1$,
		% $b+c\in\cal C$ y es positivo, ya que $b$ y $c$ lo son;
		% $b$ y $c$ tambi\'en sirven.
	% }
	Este elemento es un n\'umero natural, se escribe como $l=bx+cy$ para
	ciertos $x,y\in\Enteros$ y es $l\leq b,c$.

	Veamos que $l\mid b$  que $l\mid c$. Si $l\nmid b$, por Algoritmo de
	divisi\'on (\teoname~\ref{teo:divisibilidad:algoritmo}),
	existir\'{\i}an (\'unicos) naturales $q$ y $r$ tales que $b=lq+r$,
	$1\leq r<l$. Pero, entonces
	\begin{displaymath}
		r\,=\,b-lq\,=\,b-(bx+cy)q\,=\,b(1-qx)+c(-yq)
	\end{displaymath}
	%
	pertenecer\'{\i}a a $\cal C$, ser\'{\i}a positivo y estrictamente
	menor que $l$, lo que es absurdo. Por lo tanto, $l\mid b$.
	An\'alogamente, $l\mid c$.

	Por otro lado, si $d$ es un divisor com\'un de $b$ y de $c$, entonces
	$d\mid bx+cy=l$ y $d\leq l$. En consecuencia, el m\'aximo com\'un
	divisor de $b$ y $c$ debe ser $g=l$.
\end{proof}

\begin{coroDivisibilidad}\label{coro:mcd:caracterizacion}
	Sean $b$ y $c$ n\'umeros naturales. Las siguientes propiedades sobre
	un n\'umero natural $g$ son equivalentes:
	\begin{enumerate}[(a)]
		\item\label{item:coro:mcd:caracterizacion:a}
			$g$ es el menor natural de la forma $bx+cy$ con
			$x,y\in\Enteros$;
		\item\label{item:coro:mcd:caracterizacion:b}
			$g$ es un divisor com\'un de $b$ y de $c$ y es
			divisible por cualquier otro divisor com\'un;
		\item\label{item:coro:mcd:caracterizacion:c}
			$g$ es el m\'aximo com\'un divisor de $b$ y $c$.
	\end{enumerate}
	%
\end{coroDivisibilidad}

% \subsection{Propiedades del m\'aximo com\'un divisor}%
	% \label{subsec:propiedades-del-maximo-comun-divisor}
\begin{teoDivisibilidad}[Propiedades del m\'aximo com\'un divior]%
	\label{teo:mcd:propiedades}
	Sean $a$, $b$, $c$ y $d$ n\'umeros naturales. Entonces,
	\begin{enumerate}[(i)]
		\item\label{item:mcd:propiedades:i}
			$\mcd{ma,mb}=m\mcd{a,b}$, para todo natural $m$;
		\item\label{item:mcd:propiedades:ii}
			si $d\mid a$ y $d\mid b$, entonces
			$\mcd{a/d,b/d}=\mcd{a,b}/d$;
		\item\label{item:mcd:propiedades:iii}
			$\mcd{a,b}=\mcd{a,b-ax}$, para todo entero $x$ tal
			que $ax<b$;
		\item\label{item:mcd:propiedades:iv}
			$\mcd{a,b}=\mcd{b,a}$;
		\item\label{item:mcd:propiedades:v}
			si $c\mid ab$ y $\mcd{b,c}=1$, entonces $c\mid a$.
	\end{enumerate}
	%
\end{teoDivisibilidad}

\subsection*{Ejercicios}
\theoremstyle{definition}
\newtheorem{ejerDivisibilidad}{\ejername}[section]

%-------------

\begin{ejerDivisibilidad}\label{ejer:divisibilidad:enteros}
	Extender la relaci\'on de divisibilidad a los enteros y demostrar
	propiedades an\'alogas. En particular, mostrar que existe un
	algoritmo de divisi\'on correspondiente;
	probar que $a\mid b$, si y s\'olo si $r=0$.%
	\hint{
		Si $a,b\in\Enteros$ y $a\neq 0$, existe un \'unico par
		$q,r\in\Enteros$ tal que $b=qa+r$ y $0\leq r<|a|$.
		% Describir un algoritmo para dividir dos enteros, a partir
		% de esta propiedad.
	}
\end{ejerDivisibilidad}

\begin{ejerDivisibilidad}\label{ejer:divisibilidad:resto}
	Dados naturales $a$ y $b$, sea $\resto[a](b)$ el resto de la
	divisi\'on de $b$ por $a$, es decir, $a$ divide a $b-\resto[a](b)$ y
	$0\leq \resto[a](b)<a$. Probar las siguientes propiedades de
	$\resto=\resto[a]$:
	\begin{enumerate}[(i)]
		\item\label{item:ejer:resto:i}
			para todo natural $b$,
			$\resto(\resto(b))=\resto(b)$;
		\item\label{item:ejer:resto:ii}
			para todo par de naturales $b$ y $c$,
			$\resto(b+c)=\resto(\resto(b)+\resto(c))$;
		\item\label{item:ejer:resto:iii}
			para todo par de naturales $b$ y $c$,
			$\resto(bc)=\resto(\resto(b)\resto(c))$.
	\end{enumerate}
	%
\end{ejerDivisibilidad}

\begin{ejerDivisibilidad}\label{ejer:divisibilidad:mcd}
	Extender la noci\'on de m\'aximo com\'un divisor a los enteros
	?`Es cierto que todo par de enteros admite un m\'aximo divisor
	com\'un? Probar propiedades an\'alogas. En particular, demostrar
	\begin{enumerate}[(i)]
		\item\label{item:ejer:mcd:bezout}
			que, si $b,c\in\Enteros$ y $g=\mcd{b,c}$ denota el
			m\'aximo com\'un divisor, existen
			$x,y\in\Enteros$ tales que $g=bx+cy$;
		\item\label{item:ejer:mcd:caracterizacion}
			que las caracterizaciones del \coroname~%
			\ref{coro:mcd:caracterizacion} siguen siendo
			v\'alidas;
		\item\label{item:ejer:mcd:propiedades}
			que, dados $b,c\in\Enteros$, siempre que exista,
			\begin{math}
				\mcd{b,c}=\mcd{b,-c}=\mcd{b,c+bx}
			\end{math},
			para todo $x\in\Enteros$.
	\end{enumerate}
	%
\end{ejerDivisibilidad}

% \begin{ejerDivisibilidad}
	% Hallar $q,r\in\Enteros$ tales que $963=q428+r$ y $0\leq r<428$.
% \end{ejerDivisibilidad}
% 
% \begin{ejerDivisibilidad}
	% Hallar $\mcd{963,657}$.
% \end{ejerDivisibilidad}
% 
\begin{ejerDivisibilidad}
	Hallar el m\'aximo com\'un divisor de los siguientes pares de
	enteros y expresarlo como combinaci\'on lineal entera de ellos:
	\begin{itemize}
		\item $7469$ y $2464$,
		\item $2689$ y $4001$,
		\item $2947$ y $3997$,
		\item $1109$ y $4999$,
		\item $1819$ y $3587$.
	\end{itemize}
	%
\end{ejerDivisibilidad}

\begin{ejerDivisibilidad}[Algoritmo de Euclides para hallar el m\'aximo %
	com\'un divisor]\label{ejer:divisibilidad:euclides}
	Dados $b,c\in\Enteros$, $c>0$, se definen las sucesiones siguientes:
	\begin{displaymath}
		r_{-1}\,=\,b\text{ ,}\quad r_0\,=\,c\text{ ,}\quad
			r_i\,=\,r_{i-2}\,-\,q_ir_{i-1}\text{ ,}
	\end{displaymath}
	%
	si $i\geq 1$, de manera que $r_{i-1}=0$ o bien $0\leq r_i<r_{i-1}$,
	y, luego,
	\begin{displaymath}
		\begin{aligned}
			x_{-1} & \,=\,1\text{ ,}\quad
				x_0\,=\,0\text{ ,}\quad
				x_i\,=\,x_{i-2}\,-\,q_ix_{i-1}
				\quad\text{(\phantom)}i\geq 1
					\text{\phantom(),} \\
			y_{-1} & \,=\,0\text{ ,}\quad
				y_0\,=\,1\text{ ,}\quad
				y_i\,=\,y_{i-2}\,-\,q_iy_{i-1}
				\quad\text{(\phantom)}i\geq 1
					\text{\phantom().}
		\end{aligned}
		%
	\end{displaymath}
	%
	Entonces, si $j$ es tal que $r_{j+1}=0$ y $r_j\neq 0$ (\'utimo
	resto no nulo), se cumple que
	\begin{displaymath}
		\mcd{b,c}\,=\,r_j\,=\,bx_j+cy_j
		\text{ .}
	\end{displaymath}
	%
\end{ejerDivisibilidad}

\begin{ejerDivisibilidad}
	Hallar, si existen, $x,y\in\Enteros$ tales que
	\begin{itemize}
		\item $423x+198y=9$,
		\item $71x-50y=1$,
		\item $43x+64y=1$,
		\item $93x-81y=3$,
		\item $6x+10y+15z=1$.
	\end{itemize}
	%
\end{ejerDivisibilidad}

\begin{ejerDivisibilidad}
	Probar que, dados $b,c\in\Enteros$, no ambos nulos, la ecuaci\'on
	$bx+cy=k$ tiene soluci\'on, si y s\'olo si $\mcd{b,c}\mid k$.
	Describir el conjunto soluci\'on.%
	\hint{
		si $(x,y)$ y $(x_1,y_1)$ son soluciones, entonces
		$b'(x-x_1)+c'(y-y_1)=0$, donde
		$b'=b/\mcd{b,c}$ y $c'=c/\mcd{b,c}$; notar que $b'$ y $c'$
		son coprimos.
	}
	Generalizar a una cantidad arbitraria de enteros
	(ver el \ejername~\ref{ejer:divisibilidad:mcd:iterado}).
\end{ejerDivisibilidad}

\begin{ejerDivisibilidad}\label{ejer:divisibilidad:mcd:iterado}
	Extender la noci\'on de m\'aximo com\'un divisor a enteros
	$\lista b{n}$, $n\geq 1$. Probar:
	\begin{enumerate}[(i)]
		\item\label{item:ejer:mcd:bezout:bis}
			el an\'alogo del \teoname~\ref{teo:mcd:bezout};
		\item\label{item:ejer:mcd:propiedades:bis}
			propiedades an\'alogas a las del \teoname~%
			\ref{teo:mcd:propiedades} (en particular, probar que
			\begin{math}
				\mcd{\lista b{n}}=\mcd{b_{\sigma 1},%
					\,\dots,\,b_{\sigma n}}
			\end{math},
			para toda permutaci\'on $\sigma$ de
			$\{1,\,\dots,\,n\}$);
		\item\label{item:ejer:mcd:iterado}
			que
			\begin{math}
				\mcd{\lista b{n}}=\mcd{%
					\mcd{\lista b{n-1}},b_n}=
				\mcd{\mcd{\lista b{j}},\mcd{\lista[j+1] b{n}}}
			\end{math}.
	\end{enumerate}
	%
\end{ejerDivisibilidad}

\begin{ejerDivisibilidad}
	En el Algoritmo de Euclides (\ejername~%
	\ref{ejer:divisibilidad:euclides}), si $j\geq 0$ es tal que
	$r_{j+1}=0$ y $r_j\neq 0$, probar que
	\begin{enumerate}[(i)]
		\item\label{ejer:euclides:propiedades:i}
			si $-1\leq i\leq j+1$, entonces
			$(-1)^ix_i\leq 0$ y $(-1)^iy_i\geq 0$;
		\item\label{ejer:euclides:propiedades:ii}
			si $0\leq i\leq j$, entonces
			$|x_{i+1}|=|x_{i-1}|+q_{i+1}|x_i|$ y que
			$|y_{i+1}|=|y_{i-1}|+q_{i+1}|y_i|$;
		\item\label{ejer:euclides:propiedades:iii}
			si $0\leq i\leq j+1$, entonces
			$x_{i-1}y_i-x_iy_{i-1}=(-1)^i$;
		\item\label{ejer:euclides:propiedades:iv}
			si $-1\leq i\leq j+1$, entonces
			$\mcd{x_i,y_i}=1$;
		\item\label{ejer:euclides:propiedades:v}
			si $g=\mcd{b,g}$, entonces
			$|x_{j+1}|=c/g$ y $|y_{j+1}|=b/g$;
		\item\label{ejer:euclides:propiedades:vi}
			$|x_j|\leq c/(2g)$ y que se cumple
			$|x_j|=c/(2g)$, si y s\'olo si $q_{j+1}=2$ y
			$x_{j-1}=0$; de manera similar,
			$|y_j|\leq b/(2g)$.
	\end{enumerate}
	%
\end{ejerDivisibilidad}


