\theoremstyle{plain}
\newtheorem{teoLineales}{\teoname}[section]
\newtheorem{coroLineales}[teoLineales]{\coroname}

\theoremstyle{definition}
\newtheorem{defLineales}[teoLineales]{\defname}
\newtheorem{ejemLineales}[teoLineales]{\ejemname}
\newtheorem{obsLineales}[teoLineales]{\obsname}

%-------------

En esta secci\'on estudiamos la resolubilidad de ecuaciones del estilo:
\begin{equation}
	\label{eq:lineales}
	a_1x_1+\,\cdots\,+a_rx_r\,=\,n
	\dispstop
\end{equation}
%
En general supondremos que $n$ y los coeficientes $\lista a r\in\Enteros$.
Empezamos generalizando la noci\'on de m\'aximo com\'un divisor
(\defname~\ref{def:mcd}).

\begin{defLineales}\label{def:lineales:mcd}
	Dado un subconjunto $S\subset\Enteros$ tal que
	$S\setmin\{0\}\neq\varnothing$, el \emph{m\'aximo com\'un %
	divisor de $S$} es el m\'as grande de todos los divisores
	comunes de los elementos de $S$.
	Lo denotamos $\mcd{S}$.
\end{defLineales}

\begin{obsLineales}\label{obs:lineales:mcd:existe}
	Si $S$ contiene alg\'un entero no nulo, cualquier divisor
	com\'un de los elementos de $S$ es, en particular, un divisor
	de dicho entero y, por lo tanto, est\'a acotado. Existe,
	en ese caso un m\'aximo com\'un divisor.
\end{obsLineales}

\begin{obsLineales}\label{obs:lineales:mcd}
	Si $S=\{b,c\}$, entonces el m\'aximo com\'un divisor de $S$
	coincide con el m\'aximo com\'un divisor de $b$ y de $c$,
	seg\'un la \defname~\ref{def:mcd}.
\end{obsLineales}

\begin{teoLineales}\label{teo:lineales:bezout}
	Si $S=\{\lista a r\}\subset\Enteros$, $a_1\neq 0$, y $g=\mcd{S}$,
	entonces existen $\lista x r\in\Enteros$ tales que
	\begin{displaymath}
		a_1x_1+\,\cdots\,+a_rx_r\,=\,g
		\dispstop
	\end{displaymath}
	%
	Adem\'as, si definimos inductivamente
	$c_1=a_1$ y $c_i=\mcd{c_{i-1},a_i}$, entonces $g=c_r$.
\end{teoLineales}

% La demostraci\'on consiste en, primero, probar que $c_r$ es el MCD
% y, luego, probar la expresi\'on como combinaci\'on (inducci\'on y caso
% $r=2$.

\begin{ejemLineales}\label{ejem:lineales:bezout}
	Con $S=\{42,105,15,11\}$, $\mcd{S}=1$.
	En este caso,
	\begin{displaymath}
		\begin{aligned}
			c_1 & \,=\, 42 \dispcomma \\
			c_2 & \,=\,\mcd{c_1,105}\,=\,21
				\,=\,42\cdot 3\,+\,105\cdot (-1) \dispcomma \\
			c_3 & \,=\,\mcd{c_2,15}\,=\,3
				\,=\,21\cdot 3\,+\,15\cdot (-4) \dispand \\
			c_4 & \,=\,\mcd{c_3,11}\,=\,1
				\,=\,3\cdot 4\,+\,11\cdot (-1) \dispstop
		\end{aligned}
		%
	\end{displaymath}
	%
	Entonces,
	\begin{displaymath}
		1\,=\,42\cdot 36+105\cdot(-12)+15\cdot (-16)+11\cdot(-1)
		\dispstop
	\end{displaymath}
	%
\end{ejemLineales}

% \begin{proof}
	% Si $r=1$, no hay nada que probar. Si $r=2$, la afirmaci\'on
	% es, esencialmente, el \teoname~\ref{teo:mcd:bezout}.
	% Inductivamente, existe $B\in\Mat((r-1)\times(r-1),\Enteros)$
	% tal que
	% \begin{displaymath}
		% (\lista a r)\,
		% \begin{bmatrix}
			% B & \phantom 0 \\
			% \phantom 0 & 1
		% \end{bmatrix} \,=\, (0,\,\dots,\,0,\,c,\,a_r)
		% \dispcomma
	% \end{displaymath}
	% %
	% donde $c=\mcd{\lista a{r-1}}$. Resta acomodar los \'ultimos dos
	% coeficientes. Por el caso $r=2$, sabemos que existe
	% $B'\in\Mat(2\times 2,\Enteros)$ tal que
	% \begin{displaymath}
		% (0,\,\dots,\,0,\,c,\,a_r)\,
		% \begin{bmatrix}
			% \Id[r-2] & \phantom 0 \\
			% \phantom 0 & B'
		% \end{bmatrix} \,=\, (0,\,\dots,\,0,\,0,\,d)
		% \dispcomma
	% \end{displaymath}
	% %
	% donde $d=\mcd{c,a_r}$.
% \end{proof}

\begin{coroLineales}\label{coro:lineales:caracterizacion}
	Sea $S\subset\Enteros$, $S\setmin\{0\}\neq\varnothing$.
	Las siguientes propiedades sobre un natural $g$ son equivalentes:
	\begin{enumerate}[(a)]
		\item\label{item:coro:lineales:caracterizacion:a}
			$g$ es el menor natrural de la forma
			$\sum_a\,ax_a$ con $a\in S$, $x_a\in\Enteros$
			y $x_a=0$ para todos salvo finitos $a$;
		\item\label{item:coro:lineales:caracterizacion:b}
			$g$ es divisor com\'un de los elementos de $S$
			y es divisible por cualquier otro divisor com\'un;
		\item\label{item:coro:lineales:caracterizacion:c}
			$g$ es el m\'aximo com\'un divisor de $S$.
	\end{enumerate}
	%
\end{coroLineales}

\begin{coroLineales}\label{coro:lineales:bezout:resolubilidad}
	Si $n,\,\lista a r\in\Enteros$, no todos nulos la ecuaci\'on
	\eqref{eq:lineales} tiene soluci\'on con $\lista x r\in\Enteros$,
	si y s\'olo si $\mcd{\lista a r}\mid n$.
\end{coroLineales}

\begin{teoLineales}\label{teo:lineales:resolubilidad}
	Sean $\lista a r\in\Enteros$, no todos nulos, y sea
	$g=\mcd{\lista a r}$. Entonces,
	\begin{enumerate}[(i)]
		\item\label{item:lineales:resolubilidad:matriz}
			existe una matriz $C\in\Mat(r\times r,\Enteros)$
			tal que $\det(C)=\pm 1$ y
			\begin{displaymath}
				(\lista a r)\,C\,=\,(0,\,\dots,\,0,\,g)
				\quad\text{y ,}
			\end{displaymath}
			%
		\item\label{item:lineales:resolubilidad:soluciones}
			si $c_i$ denota la $i$-\'esima columna de $C$ y
			$N\in\Enteros$, el vector
			$x=\trnsp{(\lista x r)}\in\Enteros^r$ es soluci\'on
			de $\sum_i\,a_ix_i=Ng$, si y s\'olo si existen
			$\lista m{r-1}\in\Enteros$ tales que
			\begin{displaymath}
				x=\sum_{i=1}^{r-1}\,m_ic_i\,+\,Nc_r
				\dispstop
			\end{displaymath}
			%
	\end{enumerate}
	%
\end{teoLineales}

% \begin{proof}
	% \dots
% \end{proof}

\begin{ejemLineales}\label{ejem:lineales:resolubilidad:matriz}
	Siguiendo con el \ejemname~\ref{ejem:lineales:bezout},
	para hallar la matriz $C$ correspondiente, procedemos
	recursivamente. Empezamos con una matriz $2\times 2$
	que haga que $(42,105)$ pase a ser $(0,21)$:
	\begin{displaymath}
		(42,105,15,11)\,
		\begin{bmatrix}
			-5 & 3 & & \\
			2 & -1 & & \\
			& & 1 & \\
			& & & 1
		\end{bmatrix} \,=\, (0,21,15,11)
		\dispstop
	\end{displaymath}
	%
	Seguimos con una que transforme $(21,15)$ en $(0,3)$:
	\begin{displaymath}
		(0,21,15,11)\,
		\begin{bmatrix}
			1 & & & \\
			& -5 & 3 & \\
			& 7 & -4 & \\
			& & & 1
		\end{bmatrix} \,=\, (0,0,3,11)
		\dispstop
	\end{displaymath}
	%
	Finalmente, buscamos una que lleve $(3,11)$ en $(0,1)$:
	\begin{displaymath}
		(0,0,3,11)\,
		\begin{bmatrix}
			1 & & & \\
			& 1 & & \\
			& & -11 & 4 \\
			& & 3 & -1
		\end{bmatrix} \,=\, (0,0,0,1)
		\dispstop
	\end{displaymath}
	%
	En definitiva,
	\begin{displaymath}
		(42,105,15,11)\,
		\begin{bmatrix}
			-5 & -15 & -99 & 36 \\
			2 & 5 & 33 & -12 \\
			& 7 & 44 & -16 \\
			& & 3 & -1
		\end{bmatrix} \,=\, (0,0,0,1)
		\dispstop
	\end{displaymath}
	%
\end{ejemLineales}

\begin{ejemLineales}\label{ejem:lineales:resolubilidad:soluciones}
	Para terminar con el \ejemname~%
	\ref{ejem:lineales:resolubilidad:matriz},
	dado que $\mcd{42,105,15,11}=1$, dado $N\in\Enteros$,
	las soluciones a la ecuaci\'on
	\begin{displaymath}
		42\,x+105\,y+15\,z+11\,w\,=\,N
	\end{displaymath}
	%
	son los vectores
	\begin{displaymath}
		\begin{bmatrix} x \\ y \\ z \\ w \end{bmatrix} \,=\,
			\begin{bmatrix}
				-5 & -15 & -99 & 36 \\
				2 & 5 & 33 & -12 \\
				& 7 & 44 & -16 \\
				& & 3 & -1
			\end{bmatrix}\,
			\begin{bmatrix} m \\ n \\ \tilde n \\ N \end{bmatrix}
		\dispcomma
	\end{displaymath}
	%
	con $m,n,\tilde n\in\Enteros$.
\end{ejemLineales}

\subsection*{Ejercicios}
\theoremstyle{definition}
\newtheorem{ejerLineales}{\ejername}[section]

%-------------

\begin{ejerLineales}
	Hallar las soluciones a las ecuaciones siguientes:
	\begin{enumerate}[(i)]
		\item $5x+7y+11z=2$,
		\item $2x+3y+5z=1$,
		\item $6x+15y+35z=1$,
		\item $2x^2-xy-3y^2=8$.
	\end{enumerate}
	%
\end{ejerLineales}

\begin{ejerLineales}
	Hallar todos los enteros $a,b,c$ tales que
	\begin{displaymath}
		\frac 1{8\cdot 27\cdot 125}\,=\,
			\frac a 8\,+\,\frac b{27} \,+\,\frac c{125}
		\dispstop
	\end{displaymath}
	%
\end{ejerLineales}

\begin{ejerLineales}
	Hallar todas las soluciones a ambas ecuaciones
	\begin{displaymath}
		2x+3y+5z\,=\,0\dispand
			3x+5y+7z\,=\,0
		\dispstop
	\end{displaymath}
	%
\end{ejerLineales}

\begin{ejerLineales}
	Probar que la cantidad de soluciones a la ecuaci\'on
	$x+2y+3z=n$, con $x,y,z$ enteros no negativos es igual
	al coeficiente que acompa\~na $x^n$ en el desarrollo en serie de
	$(1-x)^{-1}(1-x^2)^{-1}(1-x^3)^{-1}$.
\end{ejerLineales}



