\theoremstyle{plain}
\newtheorem{teoHensel}{\teoname}[section]

\theoremstyle{definition}
\newtheorem{ejemHensel}[teoHensel]{\ejemname}

%-------------

El \teorema~\ref{teo:chino} nos permite reducir el
problema de resolver una congruencia al caso en el
que el m\'odulo de la congruencia es una potencia de
un n\'umero primo. En esta secci\'on el Lema de Hensel
nos dar\'a condiciones suficientes para ``levantar
soluciones'', es decir, empezando con una soluci\'on
a una congruencia $f(x)\equiv 0\tmodulo[p^k]$,
$f\in\polinomios\Enteros$, conseguir una soluci\'on a
la congruencia $f(x)\equiv 0\tmodulo[p^{k+1}]$.

\begin{ejemHensel}\label{ejem:hensel:cinco}
	La congruencia $x^2-x+3\equiv 0\tmodulo[5]$ tiene dos
	soluciones: $x\equiv 2$ y $x\equiv 4\tmodulo[5]$.
	Busquemos soluciones m\'odulo $25$. Sea $f(x)=x^2-x+3$.
	Si $x\in\Enteros$ cumple $f(x)\equiv 0\tmodulo[25]$,
	en particular, $f(x)\equiv 0\tmodulo[5]$, con lo cual,
	debe ser $x\equiv 2$ o bien $x\equiv 4\tmodulo[5]$.
	Es decir, las posibles soluciones a la congruencia
	m\'odulo $25$ son:
	\begin{displaymath}
		x\,\equiv\,2,\,7,\,12,\,17,\,22\modulo[25]
		\dispcomma
	\end{displaymath}
	%
	correspondientes a $x\equiv 2\tmodulo[5]$, y
	\begin{displaymath}
		x\,\equiv\,4,\,9,\,14,\,19,\,24\modulo[25]
		\dispcomma
	\end{displaymath}
	%
	correspondientes a $x\equiv 4\tmodulo[5]$.
	No todas estas clases son soluci\'on.
	Lo que necesitamos es calcular la clase de $f(x)$
	m\'odulo $25$, para cada una de las diez posibilidades.
	Para simplificar la cuenta, observamos lo siguiente:
	entre las operaciones que tenemos que realizar est\'a
	$x^2$ para distintos valores de $x$ que difieren en
	m\'ultiplos de $5$. Ahora, si $a,t\in\Enteros$,
	\begin{displaymath}
		\begin{aligned}
			(a+5t)^2 & \,=\,a^2+2\cdot a5t+25t^2
				\,\equiv\,a^2+10at\tmodulo[25]
				\dispand \\
			f(a+5t) & \,\equiv\,
				a^2+10xt-(a+5t)+3 \,=\,
				(a^2-a+3) + 5t\,(2a-1)
			\dispstop
		\end{aligned}
		%
	\end{displaymath}
	%
	En particular, tenemos que calcular $f(x+5t)$ para
	$a=2,4$ y $t=0,1,2,3,4$.
	Para estos valores obtenemos la \tablename~%
	\ref{tab:hensel:cinco}.
	\begin{table}
		\centering
		\begin{tabular}{c|ccccc|ccccc}
			$a$ & $2$ & & & &
				& $4$ & & & & \\
			$t$ & $0$ & $1$ & $2$ & $3$ & $4$
				& $0$ & $1$ & $2$ & $3$ & $4$ \\
			$f(a+5t)\tmodulo[25]$ & $5$ & $20$ & $10$ & $0$ & $15$
				& $15$ & $0$ & $10$ & $20$ & $5$
		\end{tabular}
		%
		\caption{
			Valores de $f(x)$ m\'odulo $25$ con
			$x\equiv 2,4\tmodulo[5]$.
		}\label{tab:hensel:cinco}
	\end{table}
	%
	Las soluciones a la congruencia $f(x)\equiv 0\tmodulo[25]$
	son, entonces $x\equiv 17,9\tmodulo[25]$.

	Si queremos las soluciones a la congruencia
	$f(x)\equiv 0\tmodulo[125]$, las posibilidades van a ser
	\begin{displaymath}
		x\,\equiv\,17,\,42,\,67,\,92,\,117\modulo[125]
		\dispcomma
	\end{displaymath}
	%
	correspondientes a $x\equiv 17\tmodulo[25]$, y
	\begin{displaymath}
		x\,\equiv\,9,\,34,\,59,\,84,\,109\modulo[125]
		\dispcomma
	\end{displaymath}
	%
	correspondientes a $x\equiv 9\tmodulo[25]$.
	Nuevamente, necesitamos calcular $f(a+25t)$ con
	$a=17,9$ y $t=0,1,2,3,4$. Pero%
	\footnote{
		Podr\'{\i}amos decir m\'odulo $625$,
		pero, en ese caso, $t$ deber\'{\i}a recorrer
		los valores de $0$ a $24$ para considerar
		todas las clases m\'odulo $625$
		correspondientes a cada clase m\'odulo $25$.
	}
	\begin{displaymath}
		f(a+25t)\,=\,(a+25t)^2-(a+25t)+3\,=\,
			(a^2-a+3)+25t\,(2a-1)+25^2t^2\,\equiv\,
			(a^2-a+3)+25t\,(2a-1)\tmodulo[125]
	\end{displaymath}
	%
	Las soluciones m\'odulo $125$ se obtienen con los pares
	$(a,t)\in\{(17,3),(9,1)\}$, o sea,
	$x\equiv 92,34\tmodulo[125]$.
	Para obtener las soluciones m\'odulo $625$, repetimos el
	procedimiento y notamos que
	\begin{displaymath}
		f(a+125t)\,\equiv\,f(a)+125t\,f'(a)\modulo[625]
		\dispstop
	\end{displaymath}
	%
	Calculando las clases m\'odulo $625$ de $f(a+125t)$
	para $a=92,34$ y $t=0,1,2,3,4$, encontramos que las
	soluciones se consiguen con los pares
	$(a,t)\in\{(92,1),(34,3)\}$, o sea,
	$x\equiv 217,409\tmodulo[625]$.
\end{ejemHensel}

\subsection*{Ejercicios}
\theoremstyle{definition}
\newtheorem{ejerHensel}{\ejername}[section]

%-------------

\begin{ejerHensel}
	Resolver las siguientes congruencias:
	\begin{enumerate}[(i)]
		\item $x^2+x+47\equiv 0\tmodulo[441]$,
		\item $x^2+x+7\equiv 0\tmodulo[81]$,
		\item $x^2+x+223\equiv 0\tmodulo[3^j]$
			para disitintos valores de $j$,
		\item $x^5+x^4+1\equiv 0\tmodulo[81]$,
		\item $x^3+x+57\equiv 0\tmodulo[125]$,
		\item $x^2+5x+24\equiv 0\tmodulo[36]$,
		\item $x^3+10x^2+x+3\equiv 0\tmodulo[27]$,
		\item $x^3+x^2-4\equiv 0\tmodulo[441]$,
		\item $x^3+x^2-5\equiv 0\tmodulo[441]$,
		\item $x^2+2x+2\equiv 0\tmodulo[5^j]$
			para $j\geq 1$.
	\end{enumerate}
	%
\end{ejerHensel}

\begin{ejerHensel}
	Discutir la resolubilidad de $x^2\equiv a\tmodulo[p^j]$,
	donde $p>2$ es un primo impar y $a\not\equiv 0\tmodulo[p]$
	?`Qu\'e se puede decir del caso $p=2$?
\end{ejerHensel}

\begin{ejerHensel}
	Estudiar las siguientes congruencias: con $f(x)=x^3+x^2-4$,
	\begin{enumerate}[(i)]
		\item $f(x)\equiv 0\tmodulo[143]$,
		\item $f(x)\equiv 0\tmodulo[169]$,
		\item $f(x)\equiv 0\tmodulo[121]$,
		\item $f(x)\equiv 0\tmodulo[1573]$,
		\item $f(x)\equiv 0\tmodulo[11^a 13^b]$,
			con $a,b\geq 0$.
	\end{enumerate}
	%
\end{ejerHensel}



