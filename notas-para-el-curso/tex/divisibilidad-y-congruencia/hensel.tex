\theoremstyle{plain}
\newtheorem{teoHensel}{\teoname}[section]

\theoremstyle{definition}
\newtheorem{defHensel}[teoHensel]{\defname}
\newtheorem{obsHensel}[teoHensel]{\obsname}
\newtheorem{ejemHensel}[teoHensel]{\ejemname}

%-------------

El \teoname~\ref{teo:chino} nos permite reducir el
problema de resolver una congruencia al caso en el
que el m\'odulo de la congruencia es una potencia de
un n\'umero primo. En esta secci\'on el Lema de Hensel
nos dar\'a condiciones suficientes para ``levantar
soluciones'', es decir, empezando con una soluci\'on
a una congruencia $f(x)\equiv 0\tmodulo[p^k]$,
$f\in\polinomios\Enteros$, conseguir una soluci\'on a
la congruencia $f(x)\equiv 0\tmodulo[p^{k+1}]$.
En esta secci\'on $p$ denota n\'umeros primos.

\begin{ejemHensel}\label{ejem:hensel:cinco}
	La congruencia $x^2-x+3\equiv 0\tmodulo[5]$ tiene dos
	soluciones: $x\equiv 2$ y $x\equiv 4\tmodulo[5]$.
	Busquemos soluciones m\'odulo $25$. Sea $f(x)=x^2-x+3$.
	Si $x\in\Enteros$ cumple $f(x)\equiv 0\tmodulo[25]$,
	en particular, $f(x)\equiv 0\tmodulo[5]$, con lo cual,
	debe ser $x\equiv 2$ o bien $x\equiv 4\tmodulo[5]$.
	Es decir, las posibles soluciones a la congruencia
	m\'odulo $25$ son:
	\begin{displaymath}
		x\,\equiv\,2,\,7,\,12,\,17,\,22\modulo[25]
		\dispcomma
	\end{displaymath}
	%
	correspondientes a $x\equiv 2\tmodulo[5]$, y
	\begin{displaymath}
		x\,\equiv\,4,\,9,\,14,\,19,\,24\modulo[25]
		\dispcomma
	\end{displaymath}
	%
	correspondientes a $x\equiv 4\tmodulo[5]$.
	No todas estas clases son soluci\'on.
	Lo que necesitamos es calcular la clase de $f(x)$
	m\'odulo $25$, para cada una de las diez posibilidades.
	Para simplificar la cuenta, observamos lo siguiente:
	entre las operaciones que tenemos que realizar est\'a
	$x^2$ para distintos valores de $x$ que difieren en
	m\'ultiplos de $5$. Ahora, si $a,t\in\Enteros$,
	\begin{displaymath}
		\begin{aligned}
			(a+5t)^2 & \,=\,a^2+2\cdot a5t+25t^2
				\,\equiv\,a^2+10at\tmodulo[25]
				\dispand \\
			f(a+5t) & \,\equiv\,
				a^2+10at-(a+5t)+3 \,=\,
				(a^2-a+3) + 5t\,(2a-1)
			\dispstop
		\end{aligned}
		%
	\end{displaymath}
	%
	En particular, tenemos que calcular $f(a+5t)$ para
	$a=2,4$ y $0\leq t\leq 4$.
	Para estos valores obtenemos la \tablename~%
	\ref{tab:hensel:cinco}.
	\begin{table}
		\centering
		\begin{tabular}{c|ccccc|ccccc}
			$a$ & $2$ & & & &
				& $4$ & & & & \\
			$t$ & $0$ & $1$ & $2$ & $3$ & $4$
				& $0$ & $1$ & $2$ & $3$ & $4$ \\
			$f(a+5t)\tmodulo[25]$ & $5$ & $20$ & $10$ & $0$ & $15$
				& $15$ & $0$ & $10$ & $20$ & $5$
		\end{tabular}
		%
		\caption{
			Valores de $f(x)$ m\'odulo $25$ con
			$x\equiv 2,4\tmodulo[5]$.
		}\label{tab:hensel:cinco}
	\end{table}
	%
	Las soluciones a la congruencia $f(x)\equiv 0\tmodulo[25]$
	son, entonces $x\equiv 17,9\tmodulo[25]$.
	Si queremos las soluciones a la congruencia
	$f(x)\equiv 0\tmodulo[125]$, las posibilidades van a ser
	\begin{displaymath}
		x\,\equiv\,17,\,42,\,67,\,92,\,117\modulo[125]
		\dispcomma
	\end{displaymath}
	%
	correspondientes a $x\equiv 17\tmodulo[25]$, y
	\begin{displaymath}
		x\,\equiv\,9,\,34,\,59,\,84,\,109\modulo[125]
		\dispcomma
	\end{displaymath}
	%
	correspondientes a $x\equiv 9\tmodulo[25]$.
	Nuevamente, necesitamos calcular $f(a+25t)$ con
	$a=17,9$ y $0\leq t\leq 4$. Pero%
	\footnote{
		Podr\'{\i}amos decir m\'odulo $625$,
		pero, en ese caso, $t$ deber\'{\i}a recorrer
		los valores de $0$ a $24$ para considerar
		todas las clases m\'odulo $625$
		correspondientes a cada clase m\'odulo $25$.
	}
	\begin{displaymath}
		\begin{aligned}
			f(a+25t) & \,=\,(a+25t)^2-(a+25t)+3\,=\,
				(a^2-a+3)+25t\,(2a-1)+25^2t^2 \\
			& \,\equiv\, (a^2-a+3)+25t\,(2a-1)\tmodulo[125]
		\end{aligned}
		%
	\end{displaymath}
	%
	Las soluciones m\'odulo $125$ se obtienen con los pares
	$(a,t)\in\{(17,3),(9,1)\}$, o sea,
	$x\equiv 92,34\tmodulo[125]$.
	Para obtener las soluciones m\'odulo $625$, repetimos el
	procedimiento y notamos que
	\begin{displaymath}
		f(a+125t)\,\equiv\,f(a)+125t\,f'(a)\modulo[625]
		\dispstop
	\end{displaymath}
	%
	Calculando las clases m\'odulo $625$ de $f(a+125t)$
	para $a=92,34$ y $0\leq t\leq 4$, encontramos que las
	soluciones se consiguen con los pares
	$(a,t)\in\{(92,1),(34,3)\}$, o sea,
	$x\equiv 217,409\tmodulo[625]$.
\end{ejemHensel}

\begin{obsHensel}\label{obs:hensel:taylor}
	El procedimiento empleado en el \ejemname~%
	\ref{ejem:hensel:cinco} para conseguir las soluciones
	m\'odulo potencias de $5$ a partir de las soluciones
	m\'odulo $5$ se puede resumir de la siguiente manera.
	Dado $f\in\polinomios\Enteros$,
	\begin{equation}
		\label{eq:hensel:taylor}
		f(a+p^jt)\,\equiv\,f(a)\,+\,p^jt\,f'(a)
			\modulo[p^{j+1}]
		\dispcomma
	\end{equation}
	%
	donde $f'$ denota el polinomio derivado de $f$
	\quedacomoejercicio.%
	\hint{
		Aplicando el desarrollo de Taylor:
		\begin{displaymath}
			f(a+p^jt)\,=\,f(a)+p^jt\,f'(a)
				+p^{2j}t^2\,f''(a)/2
				+p^{3j}t^3\,f'''(a)/3!
				+\,\cdots\,+
				p^{nj}t^n\,f^{(n)}(a)/n!
			\dispcomma
		\end{displaymath}
		%
		si $n=\grado(f)$. Lo \'unico que hay que
		ver es que cada t\'ermino $f^{(k)}(a)/k!$
		es un n\'umero entero.
		Pero, en el caso particular en que $f(x)=cx^n$,
		derivando $k$ veces, se ve que
		\begin{math}
			f^{(k)}(a)=c\,\tfrac{n!}{(n-k)!}\,a^{n-k}
		\end{math}.
	}
	Luego, dada una soluci\'on m\'odulo $p^j$, buscamos
	$t\in\{0,\,1,\,\dots,\,p-1\}$ que haga que se verifique
	\eqref{eq:hensel:taylor}.
	Ahora bien, en lugar de buscar a fuerza bruta los valores
	de $t$ que hacen esto, podemos intentar despejarlo de la
	congruencia; con respecto a la variable $t$,
	\eqref{eq:hensel:taylor} es una ecuaci\'on lineal de
	congruencia. Asumiendo que $f(a)\equiv 0\tmodulo[p^j]$
	--es decir, que $x\equiv a$ es una soluci\'on m\'odulo $p^j$--
	la condici\'on sobre $t$ es equivalente a
	\begin{equation}
		\label{eq:hensel:pre}
		t\,f'(a)\,\equiv\,-\frac{f(a)}{p^j}\modulo[p]
		\dispstop
	\end{equation}
	%
	La fracci\'on $f(a)/p^j\in\Enteros$, pues $p^j\mid f(a)$,
	por hip\'otesis ?`Qu\'e condiciones garantizan que
	podemos hallar $t$ que resuelva \eqref{eq:hensel:pre}?
	?`Cu\'antas posibilidades hay?
\end{obsHensel}

\begin{teoHensel}[Lema de Hensel]\label{teo:hensel}
	Sea $f\in\polinomios\Enteros$ y sea $a\in\Enteros$ tal
	que $f(a)\equiv 0\tmodulo[p^j]$.
	Si $f'(a)\neq 0\tmodulo[p]$, entonces existe un \'unico
	% $0\leq t\leq p-1$ tal que $f(a+p^jt)\equiv 0\tmodulo[p^{j+1}]$.
	% En particular, existe uno \'unico
	% \begin{enumerate}[(i)]
		% \item\label{item:hensel:mejora}
			% $f(x)\equiv 0\tmodulo[p^{j+1}]$ y
		% \item\label{item:hensel:consistencia}
			% $x\equiv a\tmodulo[p^j]$.
	% \end{enumerate}
	% %
	$x\equiv a\tmodulo[p^j]$ tal que $f(x)\equiv 0\tmodulo[p^{j+1}]$.
\end{teoHensel}

O sea, en este caso, cada soluci\'on m\'odulo $p^j$ da lugar a
(``se levanta a'') una \'unica soluci\'on m\'odulo $p^{j+1}$
?`Qu\'e pasa si $f'(a)\equiv 0\tmodulo[p]$?

\begin{proof}
	Dado que $p\nmid f'(a)$, podemos hallar $u\in\Enteros$
	tal que $f'(a)\,u\equiv 1\tmodulo[p]$ y, entonces,
	el valor de $t$ que verifica \eqref{eq:hensel:pre} est\'a
	un\'{\i}vocamente determinado: es
	$t\equiv -\frac{f(a)}{p^j}\,u\tmodulo[p]$.
	Esto determina $t$ en el rango $0\leq t\leq p-1$ y
	$x=a+p^jt$.
\end{proof}

\begin{ejemHensel}\label{ejem:hensel:cinco:singular}
	Siguiendo con el \ejemname~\ref{ejem:hensel:cinco},
	sea $g(x)=x^3+2x+3=(x+1)\,f(x)$.
	Si buscamos ra\'{\i}ces m\'odulo $5$, es decir,
	soluciones a $g(x)\equiv 0\tmodulo[5]$, encontramos
	las mismas que para $f$: $x\equiv 2,4\tmodulo[5]$ son
	las \'unicas soluciones m\'odulo $5$. Pero, en cierto
	sentido, $x\equiv 4\tmodulo[5]$ es una ra\'{\i}z doble,
	porque es ra\'{\i}z de $f$ m\'odulo $5$ y tambi\'en es
	ra\'{\i}z del factor $x+1$ m\'odulo $5$.
	Esto concuerda con que, en este caso, $g'(4)\equiv 0\tmodulo[5]$.%
	\footnote{
		Notemos que, por la regla del producto,
		\begin{displaymath}
			g'(x)\,=\,(x+1)\,f'(x)\,+\,f(x)
			\dispstop
		\end{displaymath}
		%
	}
	De hecho, $g'(4)=50\equiv 0\tmodulo[25],\,\not\equiv 0\tmodulo[125]$.
	Por otro lado, como $g'(2)\not\equiv 0\tmodulo[5]$,
	podemos aplicar el \teoname~\ref{teo:hensel} para levantar (y de
	manera \'unica) la soluci\'on $x=2$ m\'odulo $5$ a una soluci\'on
	para cada potencia de $5$. El problema est\'a en $x=4$.
	Si hay soluciones m\'odulo $25$, $125$, \dots, entonces ellas
	deber\'an seguir siendo, igual que antes, congruentes con
	$2$ o con $4$ m\'odulo $5$. De acuerdo con el comentario
	anterior, s\'olo tenemos que concentramos en buscar
	soluciones congruentes ``arriba de $4$'', es decir,
	soluciones m\'odulo $25$, $125$, \dots que sean congruentes
	con $4$ m\'odulo $5$.
	El siguiente diagrama muestra las soluciones a $g(x)\equiv 0$
	m\'odulo algunas potencias de $5$.
	\begin{displaymath}
		\begin{tikzcd}[column sep=.01pt, row sep=small]
			& 15409 & 31034 & 46659 & 62284 & 77909 & &
				& 15624 & 31249 & 46874 & 62499 & 78124 \\
			& 2909 & 6034 & 9159 & 12284 & 15409
				\arrow[ullll,-]
				\arrow[ulll,-]
				\arrow[ull,-]
				\arrow[ul,-]
				\arrow[u,-] & &
				& 3124 & 6249 & 9374 & 12499 & 15624
				\arrow[ullll,-]
				\arrow[ulll,-]
				\arrow[ull,-]
				\arrow[ul,-]
				\arrow[u,-] \\
			& 409 & 1034 & 1659 & 2284 & 2909
				\arrow[ullll,-]
				\arrow[ulll,-]
				\arrow[ull,-]
				\arrow[ul,-]
				\arrow[u,-] & &
				& 624 & 1249 & 1874 & 2499 & 3124
				\arrow[ullll,-]
				\arrow[ulll,-]
				\arrow[ull,-]
				\arrow[ul,-]
				\arrow[u,-] \\
			& 34 & 159 & 284 & 409
				\arrow[ulll,-]
				\arrow[ull,-]
				\arrow[ul,-]
				\arrow[u,-]
				\arrow[ur,-] & 534 & &
				& 124 & 249 & 374 & 499 & 624
				\arrow[ullll,-]
				\arrow[ulll,-]
				\arrow[ull,-]
				\arrow[ul,-]
				\arrow[u,-] \\
			& 9 & 34
				\arrow[ul,-]
				\arrow[u,-]
				\arrow[ur,-]
				\arrow[urr,-]
				\arrow[urrr,-]
				& 59 & 84 & 109 & &
				& 24 & 49 & 74 & 99 & 124
				\arrow[ullll,-]
				\arrow[ulll,-]
				\arrow[ull,-]
				\arrow[ul,-]
				\arrow[u,-] \\
			4 & & & 9
				\arrow[ull,-]
				\arrow[ul,-]
				\arrow[u,-]
				\arrow[ur,-]
				\arrow[urr,-]
				& & & 14 & 19 & & & 24 
				\arrow[ull,-]
				\arrow[ul,-]
				\arrow[u,-]
				\arrow[ur,-]
				\arrow[urr,-] \\
			& & & & & & 4
				\arrow[ullllll,-]
				\arrow[ulll,-]
				\arrow[u,-]
				\arrow[ur,-]
				\arrow[urrrr,-]
		\end{tikzcd}
		%
				%
		% \caption{
			% Soluciones a $g(x)\equiv 0\tmodulo[5^j]$
			% para $1\leq j\leq ***$, $x\equiv 4\tmodulo[5]$,
			% agrupadas por clase de congruencia.
		% }\label{tab:hensel:cinco:singular}
	\end{displaymath}
	%
	Que la ra\'{\i}z sea ``doble'' se puede ver reflejado
	en que hay dos ramas por donde se levantan de la ra\'{\i}z
	$4$ m\'odulo $5$. Notemos que, cada vez que una soluci\'on
	m\'odulo $5^j$ se levanta a una soluci\'on m\'odulo $5^{j+1}$,
	se levanta de exactamente cinco maneras distintas:
	son las clases congruentes m\'odulo $5^j$ con la soluci\'on.
	Esto sucede a partir de $j=3$. Por ejemplo, $9$ es soluci\'on
	$5^3$ y la soluci\'on $9$ m\'odulo $5^2$ se levanta de cinco
	maneras a soluciones m\'odulo $5^3$. Otro ejemplo, $34$ es
	soluci\'on m\'odulo $5^3$ y m\'odulo $5^4$ y se levanta a
	cinco soluciones m\'odulo $5^4$ congruentes con $34$ m\'odulo
	$5^3$. Para $9$ y $24$ (y todas las soluciones construidas
	a partir de ellas) $5\mid g'(9)$ pero $25\nmid g'(9)$, y
	$5\mid g'(24)$ pero $25\nmid g'(24)$.
	Para ver el contraste, las soluciones congruentes con $2$
	m\'odulo $5$ progresan de la siguiente manera:
	\begin{displaymath}
		\begin{tikzcd}[row sep=small]
			62717 \\
			217 \arrow[u,-] \\
			217 \arrow[u,-] \\
			217 \arrow[u,-] \\
			92 \arrow[u,-] \\
			17 \arrow[u,-] \\
			2 \arrow[u,-]
		\end{tikzcd}
		%
		\dispstop
	\end{displaymath}
	%
\end{ejemHensel}

Para poder enunciar el resultado que generaliza el \teoname~\ref{teo:hensel},
necesitamos introducir algo de notaci\'on.

\begin{defHensel}\label{def:valuacion}
	Si $p$ es un n\'umero primo y $x\in\Enteros\setmin\{0\}$,
	la \emph{valuaci\'on de $x$ en $p$}
	(o \emph{valuaci\'on $p$-\'adica de $x$}) es la m\'axima potencia
	de $p$ que divide a $x$.
\end{defHensel}

\begin{obsHensel}\label{obs:valuacion}
	La valuaci\'on en $p$ define una funci\'on
	\begin{math}
		\valuacion=\valuacion[p]:\,
			\Enteros\setmin\{0\}\rightarrow\Enteros
	\end{math}.%
	\footnote{
		De hecho, es posible extender esta funci\'on
		a $\Unidades\Racionales$.
	}
	Que $\valuacion(x)=k$ quiere decir que $x\equiv 0\tmodulo[p^k]$,
	pero $x\not\equiv 0\tmodulo[p^{k+1}]$.
\end{obsHensel}

\begin{teoHensel}\label{teo:hensel:plus}
	Sea $f\in\polinomios\Enteros$ y sea $a\in\Enteros$ tal que
	$f(a)\equiv 0\tmodulo[p^j]$.
	Si $k=\valuacion[p](f'(a))<j/2$, entonces
	\begin{enumerate}[(i)]
		\item\label{item:hensel:plus:mejora}
			existe un \'unico $x\equiv a\tmodulo[p^{j-k}]$
			tal que $f(x)\equiv 0\tmodulo[p^{j+1}]$,
			sin embargo,
		\item\label{item:hensel:plus:perdida}
			todo $b\equiv a\tmodulo[p^{j-k}]$ cumple
			$f(b)\equiv 0\tmodulo[p^j]$ y
			$\valuacion[p](f'(b))=k$.
	\end{enumerate}
	%
\end{teoHensel}

\begin{ejemHensel}\label{ejem:hensel:cinco:singular:recap}
	En el \ejemname~\ref{ejem:hensel:cinco:singular},
	con $a=9$, $\valuacion(g'(9))=1$ y $j=3$. Entonces,
	todos los enteros $\equiv 9\tmodulo[5^2]$ son
	soluci\'on m\'odulo $5^3$ tambi\'en
	(\eqref{item:hensel:plus:perdida}) y s\'olo uno de ellos,
	$34$, es tal que $g(34)\equiv 0\tmodulo[5^4]$.
\end{ejemHensel}

\begin{proof}
	Si $b\equiv a\tmodulo[p^{j-k}]$, entonces $b=a+p^{j-k}t$.
	Aplicando el desarrollo de Taylor a $f$,
	\begin{displaymath}
		f(b)\,\equiv\,f(a)\,+\,p^{j-k}t\,f'(a)\modulo[p^{2\,(j-k)}]
		\dispstop
	\end{displaymath}
	%
	Por hip\'otesis, $2\,(j-k)\geq j+1$, con lo cual,
	deducimos que, si $f(a)\equiv 0\tmodulo[p^j]$, entonces
	$f(b)$ es divisible por $p^j$ tambi\'en.
	Podemos, en ese caso, dividir por $p^j$ y obtener la congruencia
	\begin{displaymath}
		\frac{f(b)}{p^j}\,\equiv\,\frac{f(a)}{p^j}\,+\,
			\frac{f'(a)}{p^k}\,t\modulo[p]
		\dispstop
	\end{displaymath}
	%
	Dado que $p\nmid\frac{f'(a)}{p^k}$, existe
	$u\in\Enteros$ tal que $\frac{f'(a)}{p^k}\,u\equiv 1\tmodulo[p]$
	y, por lo tanto, $t$ queda un\'{\i}vocamente determinado
	de manera que $0\leq t\leq p-1$ y $b=a+p^{j-k}t$ sea soluci\'on
	m\'odulo $p^{j+1}$. En cuanto a la valuaci\'on de $f'(b)$,
	como $f'$ tambi\'en es un polinomio con coeficientes enteros,
	$b\equiv a\tmodulo[p^{j-k}]$ implica
	$f'(b)\equiv f'(a)\tmodulo[p^{j-k}]$.
	Como $j-k\geq k+1$, tambi\'en se deduce que
	$f'(b)\equiv f'(a)\tmodulo[p^{k+1}]$ y, as\'{\i},
	$\valuacion(f'(b))=\valuacion(f'(a))$.
\end{proof}

\subsection*{Ejercicios}
\theoremstyle{definition}
\newtheorem{ejerHensel}{\ejername}[section]

%-------------



