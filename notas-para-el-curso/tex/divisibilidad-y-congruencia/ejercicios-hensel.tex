\theoremstyle{definition}
\newtheorem{ejerHensel}{\ejername}[section]

%-------------

\begin{ejerHensel}
	Resolver las siguientes congruencias:
	\begin{enumerate}[(i)]
		\item $x^2+x+47\equiv 0\tmodulo[441]$,
		\item $x^2+x+7\equiv 0\tmodulo[81]$,
		\item $x^2+x+223\equiv 0\tmodulo[3^j]$
			para disitintos valores de $j$,
		\item $x^5+x^4+1\equiv 0\tmodulo[81]$,
		\item $x^3+x+57\equiv 0\tmodulo[125]$,
		\item $x^2+5x+24\equiv 0\tmodulo[36]$,
		\item $x^3+10x^2+x+3\equiv 0\tmodulo[27]$,
		\item $x^3+x^2-4\equiv 0\tmodulo[441]$,
		\item $x^3+x^2-5\equiv 0\tmodulo[441]$,
		\item $x^2+2x+2\equiv 0\tmodulo[5^j]$
			para $j\geq 1$.
	\end{enumerate}
	%
\end{ejerHensel}

\begin{ejerHensel}
	Discutir la resolubilidad de $x^2\equiv a\tmodulo[p^j]$,
	donde $p>2$ es un primo impar y $a\not\equiv 0\tmodulo[p]$
	?`Qu\'e se puede decir del caso $p=2$?
\end{ejerHensel}

\begin{ejerHensel}
	Estudiar las siguientes congruencias: con $f(x)=x^3+x^2-4$,
	\begin{enumerate}[(i)]
		\item $f(x)\equiv 0\tmodulo[143]$,
		\item $f(x)\equiv 0\tmodulo[169]$,
		\item $f(x)\equiv 0\tmodulo[121]$,
		\item $f(x)\equiv 0\tmodulo[1573]$,
		\item $f(x)\equiv 0\tmodulo[11^a 13^b]$,
			con $a,b\geq 0$.
	\end{enumerate}
	%
\end{ejerHensel}

