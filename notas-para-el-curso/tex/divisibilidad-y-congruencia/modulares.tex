\theoremstyle{plain}
\newtheorem{teoModulares}{\teoname}[section]
\newtheorem{coroModulares}[teoModulares]{\coroname}
\newtheorem{lemaModulares}[teoModulares]{\lemaname}

\theoremstyle{definition}
\newtheorem{defModulares}[teoModulares]{\defname}
\newtheorem{obsModulares}[teoModulares]{\obsname}
\newtheorem{ejemModulares}[teoModulares]{\ejemname}

%-------------

Denotaremos el conjunto de clases de congruencia m\'odulo $m$
($m\in\Enteros$, $m\neq 0$) por $\Enterosmod[m]$, $\simEnterosmod[m]$ o por
$\varEnterosmod[m]$ (talvez): un conjunto finito que podemos representar con
los enteros $a$ en el rango $0\leq a\leq |m|-1$, o bien $1\leq a\leq |m|$,
por ejemplo. Estudiando la resolubilidad de una ecuaci\'on m\'odulo $m$,
obtenemos informaci\'on acerca de su esolubilidad en $\Enteros$; la ventaja
que esto tiene es que la b\'usqueda de soluciones se lleva a cabo en un
conjunto finito. En otras ocasiones, ser\'a relevante entender las soluciones
a una ecuaci\'on de congruencia, sin que esto tenga por objetivo conocer
las soluciones de la misma ecuaci\'on en $\Enteros$. Por esta raz\'on es
relevante entender qu\'e estructura tiene el conjunto de clases
$\Enterosmod[m]$.

% \subsection{El anillo de enteros modulares}%
	% \label{subsec:el-anillo-de-enteros-modulares}
\begin{teoModulares}\label{teo:modulares}
	El conjunto $\Enterosmod[m]$, junto con las clases
	$\clase 0,\clase 1\in\Enterosmod[m]$ y las operaciones
	\begin{displaymath}
		\clase a+\clase c\,:=\,\clase{a+c}
		\quad\text{y}\quad
		\clase a\,\clase c\,:=\,\clase{ac}
		\text{ ,}
	\end{displaymath}
	%
	constituye un anillo conmutativo con unidad; el cero es $\clase 0$ y
	el uno es $\clase 1$.
\end{teoModulares}

\begin{proof}
	\quedacomoejercicio. Que las operaciones est\'an bien definidas,
	es decir, que no dependen de los representantes elegidos, es
	consecuencia del \teoname~\ref{teo:congruencias}.
\end{proof}

Simplificaremos la notaci\'on y escribiremos ``$a$'', en lugar de
``$\clase a$'' cuando creamos que no hay riesgo de confundir una expresi\'on
en $\Enterosmod[m]$ por una en $\Enteros$. En ocasiones, hablaremos de
``buscar soluciones en $\Enterosmod[m]$''. Con esto querremos decir buscar
soluciones a una ecuaci\'on de congruencia, es decir, buscar soluciones
m\'odulo $m$.

\begin{ejemModulares}\label{ejem:modulares:motiva:congruencias}
	La ecuaci\'on $x^2-117x+31=0$ no tiene soluciones en $\Enterosmod[2]$,
	es decir, no tiene soluciones m\'odulo $2$.
\end{ejemModulares}

\begin{ejemModulares}\label{ejem:modulares:lineal}
	Dados $a,m\in\Enteros$, $m\neq 0$, si $\mcd{a,m}=1$, entonces la
	ecuaci\'on $ax=b$ tiene soluciones en $\Enterosmod[m]$. De hecho,
	existe una \'unica soluci\'on m\'odulo $m$. Si $b=1$, podemos
	hacer lo siguiente: por la Identidad de B\'ezout (\teoname~%
	\ref{teo:mcd:bezout}), existen $x,y\in\Enteros$ tales que $ax+my=1$.
	Pero, entonces, $ax\equiv 1\tmodulo[m]$, o sea, $ax=1$ en
	$\Enterosmod[m]$. Todo esto es lo mismo que decir que $m\mid ax-1$.
\end{ejemModulares}

\begin{coroModulares}\label{coro:modulares:primo}
	Si $p\in\Enteros$ es primo, entonces $\Enterosmod[p]$ es un dominio
	\'{\i}ntegro. Rec\'{\i}procamente, si $m\neq 0$ y $\Enterosmod[m]$
	es un dominio \'{\i}ntegro, entonces $m$ es primo. En particular,
	$\Enterosmod[m]$ es un cuerpo, si y s\'olo si $m$ es primo.
\end{coroModulares}

\begin{proof}
	Supongamos, primero, que $p$ es primo.
	Sean $\clase a,\clase b\in\Enterosmod[p]$ tales que
	$\clase a\clase b=0$. Como, por definici\'on,
	$\clase a\clase b=\clase{ab}$, esta suposici\'on implica que
	$ab\equiv 0\tmodulo[p]$, o sea, $p\mid ab$. Como $p$ es primo,
	por el \lemaname~\ref{lema:primos}, o bien $p\mid a$ o bien $p\mid b$.
	Pero esto implica que $a\equiv 0$ o $b\equiv 0\tmodulo[p]$.
	En t\'erminos de clases, esto se traduce en que, o bien
	$\clase a=0$, o bien $\clase b=0$ en $\Enterosmod[p]$. O sea,
	$\Enterosmod[p]$ es un dominio \'{\i}ntergro.
	Rec\'{\i}procamente, si $m\neq 0$ es tal que $\Enterosmod[m]$ es
	dominio \'{\i}ntegro, entonces se verifica que $m$ tiene la propiedad
	de la \obsname~\ref{obs:primos:irreducibles} y, por lo tanto, que
	$m$ es primo.
\end{proof}

% \subsection{Unidades modulares}%
	% \label{subsec:unidades-modulares}
\begin{defModulares}\label{def:modulares:unidades:anillo}
	Si $A$ es un anillo conmutativo y $a\in A$, decimos que $a$ es una
	\emph{unidad de $A$} (o \emph{en $A$}), si existe $x\in A$ tal que
	$ax=1$ ($=xa$).
\end{defModulares}

\begin{coroModulares}\label{coro:modulares:unidades}
	Dados $a,m\in\Enteros$, $m\neq 0$, la clase $\clase a\in\Enterosmod[m]$
	es una unidad en $\Enterosmod[m]$, si y s\'olo si $\mcd{a,m}=1$.
	En particular, las unidades de $\Enterosmod[m]$ est\'an representadas
	por los elementos de un sistema reducido.
\end{coroModulares}

\begin{defModulares}\label{def:modulares:unidades}
	Decimos que \emph{$a$ es una unidad m\'odulo $m$}, si
	$\clase a\in\Enterosmod[m]$ es una unidad de $\Enterosmod[m]$.
\end{defModulares}

\begin{obsModulares}\label{obs:modulares:unidades}
	Si $a\in A$ es una unidad, existe un \'unico $x\in A$ tal que
	$ax=xa=1$, al que denotamos $a^{-1}$. Las unidades de un anillo
	forman un grupo, el \emph{grupo de unidades} del anillo.
	Escribimos $\Unidades A$ o bien $\Unidadesmod[A]$ para referirnos
	al grupo de unidades de $A$. En el caso de $A=\Enterosmod[m]$,
	a veces escribiremos $\Unidadesmod[m]$.
\end{obsModulares}

\begin{coroModulares}\label{coro:modulares:unidades:phi}
	\begin{math}
		\indice{\Unidadesmod[m]}=\eulerphi(m)=
			\cardinal{\{k\in\Enteros\,:\,
				0\leq k\leq |m|-1,\,\mcd{k,m}=1\}}
	\end{math}.
	En particular, si $\mcd{a,m}=1$, entonces $a^{\eulerphi(m)}=1$ en
	$\Enterosmod[m]$.
\end{coroModulares}

\begin{proof}
	Con respecto a la \'ultima afirmaci\'on, $\Unidadesmod[m]$ es un
	grupo de orden $\eulerphi(m)$. Por lo tanto, $a^{\eulerphi(m)}=1$
	en $\Unidadesmod[m]$. Pero esto quiere decir que
	$a^{\eulerphi(m)}\equiv 1\tmodulo[m]$, o sea que
	$a^{\eulerphi(m)}=1$ en $\Enterosmod[m]$.
\end{proof}

\subsection*{Ejercicios}
\theoremstyle{definition}
\newtheorem{ejerModulares}{\ejername}[section]

%-------------

\begin{ejerModulares}\label{ejer:modulares:lineal}
	Probar que, si $\mcd{a,m}=1$, entonces $ax=b$ tiene una \'unica
	soluci\'on en $\Enterosmod[m]$, cualquiera sea $b\in\Enteros$.
	Describir las soluciones a la ecuaci\'on de congruencia
	$ax\equiv b\tmodulo[m]$, es decir, el conjunto de $x\in\Enteros$
	tales que $ax\equiv b\tmodulo[m]$.
\end{ejerModulares}

\begin{ejerModulares}
	Hacer una tabla de multiplicaci\'on y suma para los anillos
	$\Enterosmod[5]$, $\Enterosmod[8]$ y $\Enterosmod[10]$.
\end{ejerModulares}

\begin{ejerModulares}
	Sea $p$ un primo impar y sea $k\in\{1,\,\dots,\,p-1\}$. Entonces
	existe un\'unico $b_k\in\{1,\,\dots,\,p-1\}$ tal que
	$kb_k\equiv 1\tmodulo[p]$. Adem\'as, $k\neq b_k$, excepto en los
	casos $k=1$ y $k=p-1$ ?`Son ciertas estas afirmaciones en el caso
	de un m\'odulo $m$ cualquiera, no necesariamente primo?
\end{ejerModulares}

\begin{ejerModulares}
	Si $p$ es primo $(p-1)!\equiv -1\tmodulo[p]$. Si $n=4$, entonces
	$(4-1)!=3!=6\equiv 2\tmodulo[4]$. Si $n>4$ \emph{no es} primo,
	entonces $(n-1)!\equiv 0\tmodulo[n]$.
\end{ejerModulares}

\begin{ejerModulares}
	Sea $R=\{\lista r{\eulerphi(m)}\}$ un sistema reducido de
	representantes de las clases m\'odulo $m$ y sea $N\geq 0$ la
	cantidad de soluciones a la ecuaci\'on de congruencia
	$x^2\equiv 1\tmodulo[m]$. Probar que%
	\hint{
		Si $r\in R$, entonces, por un lado, existe $r'\in R$
		tal que $rr'\equiv 1$. Por otro, $\mcd{-r,m}=1$ y, si
		$m>2$, entonces tambi\'en $-r\not\equiv r$.
	}
	\begin{displaymath}
		\prod_{i=1}^{\eulerphi(m)}\,r_i\,\equiv\,(-1)^{N/2}\modulo[m]
		\text{ .}
	\end{displaymath}
	%
\end{ejerModulares}

% \begin{ejerModulares}
	% Sea $\binomial p{k}=\frac{p!}{k!(p-k)!}$ el coeficiente binomial.
	% Si $p$ es primo y $1\leq k\leq p-1$, entonces $p$ divide a
	% $\binomial p{k}$. Deducir que
	% \begin{displaymath}
		% (a+b)^p\,\equiv\,a^p+b^p\modulo[p]
		% \text{ .}
	% \end{displaymath}
% \end{ejerModulares}

\begin{ejerModulares}
	Sean $p$ y $q$ primos impares distintos y supongamos, adem\'as,
	que $p-1\mid q-1$. Si $\mcd{n,pq}=1$, entonces
	\begin{math}
		n^{q-1}\,\equiv\,1\tmodulo[pq]
	\end{math}.%
	\hint{
		% Ver \ejername~\ref{ejer:congruencias:chino}.
		Probar que $n^{q-1}\equiv 1\tmodulo[p]$ y que
		$n^{q-1}\equiv 1\tmodulo[q]$.
	}
\end{ejerModulares}

\begin{ejerModulares}
	Probar que, si $p$ es primo, entonces $p$ divide al numerador de
	$1+\frac 1{2}+\frac 1{3}+\cdots+\frac 1{p-1}$.
\end{ejerModulares}

\begin{ejerModulares}\label{ejer:modulares:residuos}
	Probar que
	\begin{enumerate}[(i)]
		\item\label{item:ejer:modulares:residuos:primo-impar}
			si $p$ es un primo impar y $a\geq 1$,
			las \'unicas soluciones $x^2=1$ en $\Enterosmod[p^a]$
			son $\pm 1$;
		\item\label{item:ejer:modulares:residuos:dos}
			$x^2=1$ tiene una \'unica soluci\'on en
			$\Enterosmod[2]$, dos soluciones en $\Enterosmod[4]$
			y cuatro soluciones en $\Enterosmod[2^b]$, si
			$b\geq 3$.
	\end{enumerate}
	%
	Determinar la cantidad de soluciones a $x^2\equiv 1\tmodulo[m]$
	para $m\in\Enteros$, $m\neq 0$.%
	\hint{
		% Ver \ejername~\ref{ejer:congruencias:chino}.
		Teorema chino del resto.
	}
\end{ejerModulares}

\begin{ejerModulares}
	Si $R=\{\lista r{p-1}\}$ es un sistema reducido m\'odulo $p$, primo,
	entonces
	\begin{displaymath}
		\prod_{i=1}^{p-1}\,r_i\,\equiv\,-1\modulo[p]
		\text{ .}
	\end{displaymath}
	%
\end{ejerModulares}

\begin{ejerModulares}
	Si $R=\{r_1,\,\dots,\,r_p\}$ y $R'=\{r_1',\,\dots,\,r_p'\}$ son
	dos sistemas completos de representantes m\'odulo un primo $p>2$,
	entonces $r_1r_1',\,\dots,\,r_pr_p'$ no pueden formar un sistema
	completo de representantes m\'odulo $p$.
\end{ejerModulares}



