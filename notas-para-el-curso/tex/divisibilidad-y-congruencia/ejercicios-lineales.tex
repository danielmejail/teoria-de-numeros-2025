\theoremstyle{definition}
\newtheorem{ejerLineales}{\ejername}[section]

%-------------

\begin{ejerLineales}
	Hallar las soluciones a las ecuaciones siguientes:
	\begin{enumerate}[(i)]
		\item $5x+7y+11z=2$,
		\item $2x+3y+5z=1$,
		\item $6x+15y+35z=1$,
		\item $2x^2-xy-3y^2=8$.
	\end{enumerate}
	%
\end{ejerLineales}

\begin{ejerLineales}
	Hallar todos los enteros $a,b,c$ tales que
	\begin{displaymath}
		\frac 1{8\cdot 27\cdot 125}\,=\,
			\frac a 8\,+\,\frac b{27} \,+\,\frac c{125}
		\dispstop
	\end{displaymath}
	%
\end{ejerLineales}

\begin{ejerLineales}
	Hallar todas las soluciones a ambas ecuaciones
	\begin{displaymath}
		2x+3y+5z\,=\,0\dispand
			3x+5y+7z\,=\,0
		\dispstop
	\end{displaymath}
	%
\end{ejerLineales}

\begin{ejerLineales}
	Probar que la cantidad de soluciones a la ecuaci\'on
	$x+2y+3z=n$, con $x,y,z$ enteros no negativos es igual
	al coeficiente que acompa\~na $x^n$ en el desarrollo en serie de
	$(1-x)^{-1}(1-x^2)^{-1}(1-x^3)^{-1}$.
\end{ejerLineales}

