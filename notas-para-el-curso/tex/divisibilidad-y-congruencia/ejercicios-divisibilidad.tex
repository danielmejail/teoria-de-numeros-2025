\theoremstyle{definition}
\newtheorem{ejerDivisibilidad}{\ejername}[section]

%-------------

\begin{ejerDivisibilidad}\label{ejer:divisibilidad:enteros}
	Extender la relaci\'on de divisibilidad a los enteros y demostrar
	propiedades an\'alogas. En particular, mostrar que existe un
	algoritmo de divisi\'on correspondiente;
	probar que $a\mid b$, si y s\'olo si $r=0$.%
	\hint{
		Si $a,b\in\Enteros$ y $a\neq 0$, existe un \'unico par
		$q,r\in\Enteros$ tal que $b=qa+r$ y $0\leq r<|a|$.
		% Describir un algoritmo para dividir dos enteros, a partir
		% de esta propiedad.
	}
\end{ejerDivisibilidad}

\begin{ejerDivisibilidad}\label{ejer:divisibilidad:resto}
	Dados naturales $a$ y $b$, sea $\resto[a](b)$ el resto de la
	divisi\'on de $b$ por $a$, es decir, $a$ divide a $b-\resto[a](b)$ y
	$0\leq \resto[a](b)<a$. Probar las siguientes propiedades de
	$\resto=\resto[a]$:
	\begin{enumerate}[(i)]
		\item\label{item:ejer:resto:i}
			para todo natural $b$,
			$\resto(\resto(b))=\resto(b)$;
		\item\label{item:ejer:resto:ii}
			para todo par de naturales $b$ y $c$,
			$\resto(b+c)=\resto(\resto(b)+\resto(c))$;
		\item\label{item:ejer:resto:iii}
			para todo par de naturales $b$ y $c$,
			$\resto(bc)=\resto(\resto(b)\resto(c))$.
	\end{enumerate}
	%
\end{ejerDivisibilidad}

\begin{ejerDivisibilidad}\label{ejer:divisibilidad:mcd}
	Extender la noci\'on de m\'aximo com\'un divisor a los enteros
	?`Es cierto que todo par de enteros admite un m\'aximo divisor
	com\'un? Probar propiedades an\'alogas. En particular, demostrar
	\begin{enumerate}[(i)]
		\item\label{item:ejer:mcd:bezout}
			que, si $b,c\in\Enteros$ y $g=\mcd{b,c}$ denota el
			m\'aximo com\'un divisor, existen
			$x,y\in\Enteros$ tales que $g=bx+cy$;
		\item\label{item:ejer:mcd:caracterizacion}
			que las caracterizaciones del \coroname~%
			\ref{coro:mcd:caracterizacion} siguen siendo
			v\'alidas;
		\item\label{item:ejer:mcd:propiedades}
			que, dados $b,c\in\Enteros$, siempre que exista,
			\begin{math}
				\mcd{b,c}=\mcd{b,-c}=\mcd{b,c+bx}
			\end{math},
			para todo $x\in\Enteros$.
	\end{enumerate}
	%
\end{ejerDivisibilidad}

% \begin{ejerDivisibilidad}
	% Hallar $q,r\in\Enteros$ tales que $963=q428+r$ y $0\leq r<428$.
% \end{ejerDivisibilidad}
% 
% \begin{ejerDivisibilidad}
	% Hallar $\mcd{963,657}$.
% \end{ejerDivisibilidad}
% 
\begin{ejerDivisibilidad}
	Hallar el m\'aximo com\'un divisor de los siguientes pares de
	enteros y expresarlo como combinaci\'on lineal entera de ellos:
	\begin{itemize}
		\item $7469$ y $2464$,
		\item $2689$ y $4001$,
		\item $2947$ y $3997$,
		\item $1109$ y $4999$,
		\item $1819$ y $3587$.
	\end{itemize}
	%
\end{ejerDivisibilidad}

\begin{ejerDivisibilidad}[Algoritmo de Euclides para hallar el m\'aximo %
	com\'un divisor]\label{ejer:divisibilidad:euclides}
	Dados $b,c\in\Enteros$, $c>0$, se definen las sucesiones siguientes:
	\begin{displaymath}
		r_{-1}\,=\,b\text{ ,}\quad r_0\,=\,c\text{ ,}\quad
			r_i\,=\,r_{i-2}\,-\,q_ir_{i-1}\text{ ,}
	\end{displaymath}
	%
	si $i\geq 1$, de manera que $r_{i-1}=0$ o bien $0\leq r_i<r_{i-1}$,
	y, luego,
	\begin{displaymath}
		\begin{aligned}
			x_{-1} & \,=\,1\text{ ,}\quad
				x_0\,=\,0\text{ ,}\quad
				x_i\,=\,x_{i-2}\,-\,q_ix_{i-1}
				\quad\text{(\phantom)}i\geq 1
					\text{\phantom(),} \\
			y_{-1} & \,=\,0\text{ ,}\quad
				y_0\,=\,1\text{ ,}\quad
				y_i\,=\,y_{i-2}\,-\,q_iy_{i-1}
				\quad\text{(\phantom)}i\geq 1
					\text{\phantom().}
		\end{aligned}
		%
	\end{displaymath}
	%
	Entonces, si $j$ es tal que $r_{j+1}=0$ y $r_j\neq 0$ (\'utimo
	resto no nulo), se cumple que
	\begin{displaymath}
		\mcd{b,c}\,=\,r_j\,=\,bx_j+cy_j
		\text{ .}
	\end{displaymath}
	%
\end{ejerDivisibilidad}

\begin{ejerDivisibilidad}
	Hallar, si existen, $x,y\in\Enteros$ tales que
	\begin{itemize}
		\item $423x+198y=9$,
		\item $71x-50y=1$,
		\item $43x+64y=1$,
		\item $93x-81y=3$,
		\item $6x+10y+15z=1$.
	\end{itemize}
	%
\end{ejerDivisibilidad}

\begin{ejerDivisibilidad}
	Probar que, dados $b,c\in\Enteros$, no ambos nulos, la ecuaci\'on
	$bx+cy=k$ tiene soluci\'on, si y s\'olo si $\mcd{b,c}\mid k$.
	Describir el conjunto soluci\'on.%
	\hint{
		si $(x,y)$ y $(x_1,y_1)$ son soluciones, entonces
		$b'(x-x_1)+c'(y-y_1)=0$, donde
		$b'=b/\mcd{b,c}$ y $c'=c/\mcd{b,c}$; notar que $b'$ y $c'$
		son coprimos.
	}
	Generalizar a una cantidad arbitraria de enteros
	(ver el \ejername~\ref{ejer:divisibilidad:mcd:iterado}).
\end{ejerDivisibilidad}

\begin{ejerDivisibilidad}\label{ejer:divisibilidad:mcd:iterado}
	Extender la noci\'on de m\'aximo com\'un divisor a enteros
	$\lista b{n}$, $n\geq 1$. Probar:
	\begin{enumerate}[(i)]
		\item\label{item:ejer:mcd:bezout:bis}
			el an\'alogo del \teoname~\ref{teo:mcd:bezout};
		\item\label{item:ejer:mcd:propiedades:bis}
			propiedades an\'alogas a las del \teoname~%
			\ref{teo:mcd:propiedades} (en particular, probar que
			\begin{math}
				\mcd{\lista b{n}}=\mcd{b_{\sigma 1},%
					\,\dots,\,b_{\sigma n}}
			\end{math},
			para toda permutaci\'on $\sigma$ de
			$\{1,\,\dots,\,n\}$);
		\item\label{item:ejer:mcd:iterado}
			que
			\begin{math}
				\mcd{\lista b{n}}=\mcd{%
					\mcd{\lista b{n-1}},b_n}=
				\mcd{\mcd{\lista b{j}},\mcd{\lista[j+1] b{n}}}
			\end{math}.
	\end{enumerate}
	%
\end{ejerDivisibilidad}

\begin{ejerDivisibilidad}
	En el Algoritmo de Euclides (\ejername~%
	\ref{ejer:divisibilidad:euclides}), si $j\geq 0$ es tal que
	$r_{j+1}=0$ y $r_j\neq 0$, probar que
	\begin{enumerate}[(i)]
		\item\label{ejer:euclides:propiedades:i}
			si $-1\leq i\leq j+1$, entonces
			$(-1)^ix_i\leq 0$ y $(-1)^iy_i\geq 0$;
		\item\label{ejer:euclides:propiedades:ii}
			si $0\leq i\leq j$, entonces
			$|x_{i+1}|=|x_{i-1}|+q_{i+1}|x_i|$ y que
			$|y_{i+1}|=|y_{i-1}|+q_{i+1}|y_i|$;
		\item\label{ejer:euclides:propiedades:iii}
			si $0\leq i\leq j+1$, entonces
			$x_{i-1}y_i-x_iy_{i-1}=(-1)^i$;
		\item\label{ejer:euclides:propiedades:iv}
			si $-1\leq i\leq j+1$, entonces
			$\mcd{x_i,y_i}=1$;
		\item\label{ejer:euclides:propiedades:v}
			si $g=\mcd{b,g}$, entonces
			$|x_{j+1}|=c/g$ y $|y_{j+1}|=b/g$;
		\item\label{ejer:euclides:propiedades:vi}
			$|x_j|\leq c/(2g)$ y que se cumple
			$|x_j|=c/(2g)$, si y s\'olo si $q_{j+1}=2$ y
			$x_{j-1}=0$; de manera similar,
			$|y_j|\leq b/(2g)$.
	\end{enumerate}
	%
\end{ejerDivisibilidad}
