\theoremstyle{definition}
\newtheorem{ejerCongruencias}{\ejername}[section]

%-------------

\begin{ejerCongruencias}
	Si $f\in\polinomios\Enteros{X}$ y $f(0)\equiv f(1)\equiv 1\tmodulo[2]$,
	entonces $f$ no tiene ra\'{\i}ces enteras. Generalizar.
\end{ejerCongruencias}

\begin{ejerCongruencias}
	Probar que la ecuaci\'on $3x^2+2=y^2$ no tiene soluciones con
	$x,y\in\Enteros$.
\end{ejerCongruencias}

\begin{ejerCongruencias}
	Probar que la ecuaci\'on $7x^3+2=y^3$ no tiene soluciones con
	$x,y\in\Enteros$.
\end{ejerCongruencias}

\begin{ejerCongruencias}
	Probar que la ecuaci\'on $x^2-117x+31=0$ no tiene soluciones m\'odulo
	$117$. Investigar la ecuaci\'on m\'odulo $m$ para valores de $m$
	hasta \dots%
	%$400$, por ejemplo
	; confeccionar una tabla. Mirar m\'odulo $m\leq 30$, al menos,
	m\'odulo $85$, $115$, $391$ y $2713$.
\end{ejerCongruencias}

\begin{ejerCongruencias}
	Si $f\in\polinomios\Enteros{X}$ y $a\equiv b\tmodulo[m]$, entonces
	$f(a)\equiv f(b)\tmodulo[m]$.
\end{ejerCongruencias}

\begin{ejerCongruencias}
	Si $d\mid m$ y $a\equiv b\tmodulo[m]$, entonces $a\equiv b\tmodulo[d]$.
\end{ejerCongruencias}

\begin{ejerCongruencias}
	Sea $g=\mcd{a,m}$. La congruencia $ax\equiv b\tmodulo[m]$ tiene
	soluci\'on $x\in\Enteros$, si y s\'olo si $g\mid b$. En tal caso,
	hay exactamente $g$ soluciones (m\'odulo $m$):
	si $x$ e $y$ son soluciones, entonces $x\equiv y\tmodulo[m/g]$.
\end{ejerCongruencias}

\begin{ejerCongruencias}
	Dados $a,m\in\Enteros$, $m\neq 0$, se cumple $ax\equiv ay\tmodulo[m]$,
	si y s\'olo si $x\equiv y\tmodulo[\tfrac m{\mcd{a,m}}]$.
\end{ejerCongruencias}

\begin{ejerCongruencias}
	Si $d\mid m$ y si $a\equiv b\tmodulo[d]$, entonces
	$a\equiv b+kd\tmodulo[m]$, para un \'unico $k$ en el rango
	$1\leq k\leq m/d$.
	% (existe $k$ y es \'unico, en el rango).
\end{ejerCongruencias}

\begin{ejerCongruencias}
	Si $p$ es primo, $\mcd{a,p}=1$ equivale a $a\not\equiv 0\tmodulo[p]$.
\end{ejerCongruencias}

\begin{ejerCongruencias}
	?`Existen $x\in\Enteros$ tales que $x^2\equiv -1\tmodulo[17]$?
	?`$x^3\equiv 2\tmodulo[17]$? Hallar un sistema reducido m\'odulo $17$
	cuyos elementos sean m\'ultiplos de $3$.
\end{ejerCongruencias}

\begin{ejerCongruencias}
	Determinar $\eulerphi(26)$. Para cada $x$ en el rango
	$0\leq x\leq 25$ (un sistema completo), determinar el menor
	natural $k$ ($k\geq 1$) tal que $x^k\equiv 1\tmodulo[26]$.
	Repetir con $36$ en lugar de $26$ (y $35$ en lugar de $25$)
	?`Qu\'e relaci\'on hay entre estos exponentes y los valores de
	$\eulerphi(26)$ y de $\eulerphi(36)$, respectivamente?
\end{ejerCongruencias}

\begin{ejerCongruencias}
	Probar que $19\nmid 4n^2+n$ para ning\'un $n$ natural.
\end{ejerCongruencias}

% \begin{ejerCongruencias}[Teorema chino del resto]%
	% \label{ejer:congruencias:chino}
	% Este ejercicio trata la resolubilidad de ciertos sistemas de
	% ecuaciones de congruencia.
	% \begin{enumerate}[(i)]
		% \item\label{item:ejer:chino:i}
			% Sean $a_1,a_2,m_1,m_2\in\Enteros$, $m_1m_2\neq 0$.
			% Las ecuaciones de congruencia
			% \begin{displaymath}
				% \begin{aligned}
					% x & \,\equiv\,a_1\modulo[m_1]
						% \quad\text{y} \\
					% x & \,\equiv\,a_2\modulo[m_2]
				% \end{aligned}
				% %
			% \end{displaymath}
			% %
			% admiten una soluci\'on com\'un, si y s\'olo si
			% $\mcd{m_1,m_2}\mid a_1-a_2$. En tal caso, si $m$
			% denota el menor m\'ultiplo com\'un positivo de $m_1$
			% y $m_2$, entonces las soluciones pertenecen a una
			% misma clase m\'odulo $m$.
		% \item\label{item:ejer:chino:ii}
			% Si $\lista m{r}\in\Enteros$ son distintos de cero y
			% coprimos de a pares, entonces las ecuaciones
			% \begin{displaymath}
				% x\,\equiv\,a_i\modulo[m_i]
				% \text{ ,}
			% \end{displaymath}
			% %
			% $1\leq i\leq r$, son consistentes y las soluciones en
			% com\'un pertenecen a una misma clase m\'odulo el
			% producto $m_1\cdots m_r$.
		% \item\label{item:ejer:chino:iii}
			% Dado $f\in\polinomios\Enteros{X}$ y dados enteros
			% $\lista m{r}\neq 0$ coprimos de a pares, la cantidad
			% de soluciones a la ecuaci\'on
			% \begin{displaymath}
				% f(x)\,\equiv\,0\modulo[m_1\cdots m_r]
			% \end{displaymath}
			% %
			% es igual al producto de las cantidades de soluciones
			% a las ecuaciones
			% \begin{displaymath}
				% f(x)\,\equiv\,0\modulo[m_i]
				% \text{ ,}
			% \end{displaymath}
			% %
			% para cada $i$.
	% \end{enumerate}
	% %
% \end{ejerCongruencias}

\begin{ejerCongruencias}\label{ejer:congruencias:chino:representantes}
	Hallar sistemas de representantes de las clases m\'odulo $26$
	cuyos elementos sean:
	\begin{enumerate}[(i)]
		\item\label{item:ejer:chino:representantes:i}
			todos divisibles por $7$;
		\item\label{item:ejer:chino:representantes:ii}
			congruentes con $1$ m\'odulo $3$ y congruentes con $1$
			m\'odulo $5$
	\end{enumerate}
	%
	?`Es posible hallar un sistema de representantes que cumpla con las
	condiciones \eqref{item:ejer:chino:representantes:i} y
	\eqref{item:ejer:chino:representantes:ii}?
\end{ejerCongruencias}

\begin{ejerCongruencias}
	Sean $a$ y $b$ enteros positivos coprimos y sean $A$ y $B$ sistemas
	completos de representantes m\'odulo $a$ y m\'odulo $b$,
	respectivamente. Probar que el conjunto
	\begin{displaymath}
		R\,:=\,\big\{ax+by\,:\,x\in B,\,y\in A\big\}
	\end{displaymath}
	%
	es un sistema completo de representates de las clases m\'odulo $ab$.
	Probar que, si $A$ y $B$ son, en cambio, sistemas reducidos,
	entonces el conjunto $R$ an\'alogo es un sistema reducido de
	representantes m\'odulo $ab$.
\end{ejerCongruencias}
