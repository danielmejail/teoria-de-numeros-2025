\theoremstyle{plain}
\newtheorem{teoChino}{\teoname}[section]
\newtheorem{coroChino}[teoChino]{\coroname}

\theoremstyle{definition}
\newtheorem{obsChino}[teoChino]{\obsname}
\newtheorem{ejemChino}[teoChino]{\ejemname}

%-------------

De la secci\'on \S~\ref{sec:congruencias}, sabemos que una ecuaci\'on
$ax\equiv b\tmodulo[m]$ tiene soluci\'on, si y s\'olo si
$\mcd{a,m}\mid b$. Tales ecuaciones imponen una restricci\'on a la clase
de congruencia de $x$ m\'odulo $m$ ?`Qu\'e podemos decir si tenemos varias
de tales restricciones en simult\'aneo?

\begin{ejemChino}\label{ejem:chino:setenta-y-siete}
	?`Existe alg\'un $x\in\Enteros$ tal que
	$x\equiv 5\tmodulo[7]$ y $x\equiv 7\tmodulo[11]$?
	El conjunto de enteros que son soluciones a ambas ecuaciones
	de manera simult\'anea el de enteros de resto $5$ al dividir por
	$7$ y de resto $7$ al dividir por $11$.
	Para encontrar una soluci\'on, podemos empezar por las
	soluciones a una de las dos ecuaciones, $x\equiv 5\tmodulo[7]$,
	por ejemplo, y buscar, dentro de este conjunto, un entero
	que tambi\'en satisfaga la otra ecuaci\'on, $x\equiv 7\tmodulo[11]$:
	\begin{displaymath}
		\big\{x\in\Enteros\,:\,x\equiv 5\tmodulo[7]\big\}
			\,\supset\,
			\big\{x\in\Enteros\,:\,
				x\equiv 5\tmodulo[7]\text{ y }
				x\equiv 7\tmodulo[11]\big\}
		\dispstop
	\end{displaymath}
	%
	Las soluciones a la ecuaci\'on m\'odulo $7$ son exactamente
	los enteros que tienen resto $5$ al dividir por $7$:
	$x=5+7u$, $u\in\Enteros$. Calculemos los primeros con $u\geq 0$
	y su resto al dividir por $11$. Podemos ver los resultados
	en la \tablename~\ref{tab:ejem:chino:setenta-y-siete}.
	\begin{table}
		\centering
		\begin{tabular}{c|cccccccccccc}
			$u$ & $0$ & $1$ & $2$ & $3$ & $4$ & $5$
				& $6$ & $7$ & $8$ & $9$ & $10$ & $11$ \\
			\hline
			$x=5+7u$ & $5$ & $12$ & $19$ & $26$ & $33$ & $40$
				& $47$ & $54$ & $61$ & $68$ & $75$ & $82$ \\
			\hline
			$\resto[11](x)$ & $5$ & $1$ & $8$ & $4$ & $0$ & $7$
				& $3$ & $10$ & $6$ & $2$ & $9$ & $5$
		\end{tabular}
		%
		\caption{
			Algunos enteros $5+7u$, $u\in\Enteros$, y sus
			restos de dividir por $11$.
		}\label{tab:ejem:chino:setenta-y-siete}
	\end{table}
	%
	A partir de $u=11$, los restos se empezar\'an a repetir.
	El menor entero positivo que es soluci\'on es $x=40$.
	Toda otra soluci\'on es congruente con $40$ m\'odulo $77$.
\end{ejemChino}

\begin{ejemChino}\label{ejem:chino:setenta-y-siete:bis}
	Veamos otra manera de llegar a la soluci\'on del
	\ejemname~\ref{ejem:chino:setenta-y-siete}.
	Sabemos que todo $x\in\Enteros$ que cumpla simult\'aneamente
	$x\equiv 5\tmodulo[7]$ y $x\equiv 7\tmodulo[11]$ debe ser,
	por la primera condici\'on, de la forma
	$x=5+7u$, $u\in\Enteros$. Reemplazando esta expresi\'on en la
	segunda congruencia, obtenemos una condici\'on para $u$:
	\begin{displaymath}
		5+7u\,\equiv\,7\modulo[11]
		\dispstop
	\end{displaymath}
	%
	Podemos intentar despejar $u$, o, mejor, su clase de congruencia.
	Por un lado, la congruencia anterior es equivalente a
	\begin{displaymath}
		7u\,\equiv\,2\modulo[11]
		\dispstop
	\end{displaymath}
	%
	Falta despejar el $7$. Ahora,
	\begin{displaymath}
		7\cdot 3\,=\,21\,\equiv\,-1\tmodulo[11]\dispand
		7\,(-3)\,\equiv\,1\tmodulo[11]
		\dispstop
	\end{displaymath}
	%
	Pero entonces,
	\begin{displaymath}
		u\,\equiv\,(-3)\,2\,=\,-6
		% \,\equiv\,5
		\modulo[11]
		\dispstop
	\end{displaymath}
	%
	En definitiva, las soluciones simult\'aneas a las congruencias
	m\'odulo $7$ y m\'odulo $11$ son
	$x=5+7u$ con $u\equiv -6\tmodulo[11]$, o sea, $u=-6+11k$,
	$k\in\Enteros$. Es decir, el conjunto soluci\'on es:
	\begin{displaymath}
		\big\{x=40+77k\,:\,k\in\Enteros\big\}
		\dispstop
	\end{displaymath}
	%
	Notamos que $-6\equiv 5\tmodulo[11]$, con lo cual, esto
	coincide con lo que dedujimos de la \tablename~%
	\ref{tab:ejem:chino:setenta-y-siete}.
\end{ejemChino}

\begin{obsChino}\label{obs:chino:setenta-y-siete}
	Para poder resolver el sistema de congruencias con el
	m\'etodo del \ejemname~\ref{ejem:chino:setenta-y-siete:bis},
	usamos el hecho de que $7\,(-3)\equiv 1\tmodulo[11]$,
	es decir, que la ecuaci\'on $7b\equiv 1\tmodulo[11]$ admite
	soluci\'on. Esto es coherente con que $\mcd{7,11}=1$.
	Por otro lado, podr\'{\i}amos preguntarnos qu\'e nos
	garantizaba que, con el m\'etodo del \ejemname~%
	\ref{ejem:chino:setenta-y-siete}, \'{\i}bamos a encontrar
	una soluci\'on
	(?`Y qu\'e nos garantiza que los posibles restos que
	aparecieron en la tabla iban a ser una cantidad finita?)
\end{obsChino}

\begin{teoChino}[Teorema chino del resto]
	\label{teo:chino}
	Sean $m_1,\,m_2\in\Enteros$ tales que $\mcd{m_1,m_2}=1$.
	Entonces, para todo par $a_1,\,a_2\in\Enteros$, existe una
	soluci\'on com\'un $x$ al sistema de ecuaciones de congruencia
	\begin{displaymath}
		% \begin{aligned}
			x \,\equiv\,a_1\tmodulo[m_1] \dispand
			x \,\equiv\,a_2\tmodulo[m_2] \dispstop
		% \end{aligned}
		% %
	\end{displaymath}
	%
	Adem\'as, el conjunto de todas las soluciones comunes a ambas
	ecuaciones es igual a una clase de congruencia m\'odulo el
	producto $m_1m_2$.
\end{teoChino}

En el \ejername~\ref{ejer:chino} se puede encontrar una variante de
este enunciado.

\begin{obsChino}\label{obs:chino}
	Dicho de otra manera, seg\'un el \teoname~\ref{teo:chino},
	si $\mcd{m_1,m_2}=1$, entonces:
	\begin{itemize}
		\item existe $x\in\Enteros$ que cumple
			$x\equiv a_1\tmodulo[m_1]$ y
			$x\equiv a_2\tmodulo[m_2]$,
		\item toda la clase de congruencia de $x$ m\'odulo
			$m_1m_2$ es soluci\'on:
			si $y\equiv x\tmodulo[m_1m_2]$, entonces
			$y$ tambi\'en es soluci\'on (satisface ambas
			condiciones) y
		\item dicho entero es \'unico m\'odulo $m_1m_2$:
			si $y$ es soluci\'on, entonces
			$y\equiv x\tmodulo[m_1m_2]$.
	\end{itemize}
	%
	En una sola frase, existe una soluci\'on y dicha
	soluci\'on es \'unica m\'odulo $m=m_1m_2$.
\end{obsChino}

\begin{proof}
	Las soluciones a la congruencia $x\equiv a_1\tmodulo[m_1]$
	son los enteros de la forma $x=a_1+m_1u$, $u\in\Enteros$.
	Reemplazando esta expresi\'on en la congruencia
	$x\equiv a_2\tmodulo[m_2]$ obtenemos que $x$ es una soluci\'on
	simult\'anea, si y s\'olo si
	\begin{itemize}
		\item $x$ es de la forma $x=a_1+m_1u$, $u\in\Enteros$, y
		\item $a_1+m_1u\equiv a_2\tmodulo[m_2]$.
	\end{itemize}
	%
	Ahora, la congruencia $a_1+m_1u\equiv a_2\tmodulo[m_2]$ admite
	soluci\'on $u$; las soluciones son exactamente los enteros $u$
	que cumplen
	\begin{displaymath}
		m_1u\,\equiv\,a_2-a_1\modulo[m_2]
		\dispstop
	\end{displaymath}
	%
	Pero esta congruencia admite soluci\'on, dado que estamos
	asumiendo $\mcd{m_1,m_2}=1$. Para describir las soluciones,
	notamos que, si $b\in\Enteros$ es tal que
	$bm_1\equiv 1\tmodulo[m_2]$ --que tal entero exista est\'a
	garantizado justamente por la condic\'on de coprimalidad--,
	entonces la condici\'on sobre $u$ es equivalente a:
	\begin{displaymath}
		u\,\equiv\,b\,(a_2-a_1)\modulo[m_2]
		\dispcomma
	\end{displaymath}
	%
	o sea, $u=b\,(a_2-a_1)+m_2k$, $k\in\Enteros$.
	En definitiva, las soluciones simult\'aneas a ambas congruencias
	son los enteros de la forma
	\begin{displaymath}
		x\,=\,a_1+m_1b\,(a_2-a_1)+m_1m_2k
	\end{displaymath}
	%
	con $k\in\Enteros$. En particular, $x=a_1+m_1b\,(a_2-a_1)$ es
	soluci\'on simult\'anea.

	Si $y\equiv x\tmodulo[m]$, entonces $y\equiv x\tmodulo[m_1]$ y
	tambi\'en $y\equiv x\tmodulo[m_2]$. Con lo cual, toda
	la clase de congruencia de $x$ m\'odulo $m=m_1m_2$ es soluci\'on
	simult\'anea. Si, por otro lado, $y\in\Enteros$ es soluci\'on
	simult\'anea, entonces
	\begin{displaymath}
		y\,\equiv\,a_1\,\equiv\, x\tmodulo[m_1]
		\quad\text{e}\quad
		y\,\equiv\,a_2\,\equiv\, x\tmodulo[m_2]
		\dispcomma
	\end{displaymath}
	%
	es decir, $m_1\mid y-x$ y tambi\'en $m_2\mid y-x$. Como
	$\mcd{m_1,m_2}=1$, se deduce que $m=m_1m_2\mid y-x$ y que
	$y\equiv x\tmodulo[m]$. O sea que toda soluci\'on debe pertenecer
	a la clase de $x$ m\'odulo $m$.
\end{proof}

Si los m\'odulos no son coprimos \emph{podr\'{\i}a no} existir soluci\'on.

\begin{ejemChino}\label{ejem:chino:no}
	Las congruencias $x\equiv 29\tmodulo[52]$ y
	$x\equiv 19\tmodulo[72]$ no tienen soluciones en com\'un.
	En este caso, $\mcd{52,72}=4$ y las toda soluci\'on com\'un
	a estas congruencias debe satisfacer
	\begin{displaymath}
		x\,\equiv\,29\tmodulo[4]\dispand
		x\,\equiv\,19\tmodulo[4]
		\dispstop
	\end{displaymath}
	%
	Pero $29\equiv 1\tmodulo[4]$, mientras que $19\equiv 3\tmodulo[4]$.
	En particular, las condiciones sobre $x$ son inconsistentes.
\end{ejemChino}

\begin{coroChino}\label{coro:chino:euler}
	Sean $m$ y $n$ enteros coprimos. Entonces,
	$\eulerphi(mn)=\eulerphi(m)\eulerphi(n)$.
\end{coroChino}

\begin{proof}
	Sean $R$, $S$ y $T$ sistemas completos de representantes de las
	clases m\'odulo $m$, $n$ y $mn$, respectivamente.
	Podemos tomar, por ejemplo, los conjuntos $R=\{1,\,2,\,\dots,\,m\}$,
	etc.
	Por el \teoname~\ref{teo:chino}, hay una biyecci\'on
	\begin{equation}
		\label{eq:chino:euler}
		T\,\simeq\,R\,\times\,S
	\end{equation}
	%
	dada por $t\mapsto (r,s)$, donde $r\in R$ y $s\in S$ son los
	representantes tales que $t\equiv r\tmodulo[m]$ y
	$t\equiv s\tmodulo[n]$: la sobreyectividad se deduce de que,
	como $\mcd{m,n}=1$, las congruencias
	$x\equiv r\tmodulo[m]$ y $x\equiv s\tmodulo[n]$ admiten una
	soluci\'on com\'un, mientras que la inyectividad se deduce de
	que toda soluci\'on simult\'anea pertenece a la misma clase
	m\'odulo $mn$.

	Ahora, sean $R'$, $S'$ y $T'$ los sistemas reducidos de
	representantes obtenidos a partir de $R$, de $S$ y de $T$,
	respectivamente.
	Vamos a ver que la biyecci\'on \eqref{eq:chino:euler} determina
	una biyecci\'on
	\begin{displaymath}
		T'\,\simeq\,R'\,\times\,S'
		\dispstop
	\end{displaymath}
	%
	Como $\mcd{m,n}=1$, se cumple que
	\begin{equation}
		\label{eq:chino:euler:mcd}
		\mcd{x,mn}\,=\,\mcd{x,m}\,\mcd{x,n}
	\end{equation}
	%
	para todo $x\in\Enteros$.
	En particular, $\mcd{x,mn}=1$, si y s\'olo si
	$\mcd{x,m}=\mcd{x,n}=1$.
	Sea, ahora, $t\in T$ y sea $(r,s)\in R\times S$ el par
	determinado por $t$ seg\'un \eqref{eq:chino:euler}.
	Dado que $t\equiv r\tmodulo[m]$, deducimos que
	$\mcd{t,m}=\mcd{r,m}$. An\'alogamente,
	$\mcd{t,n}=\mcd{s,n}$. As\'{\i}, si $t\in T'$, entonces
	$\mcd{t,mn}=1$ y concluimos que $r\in R'$ y que $s\in S'$.
	Rec\'{\i}procamente, si $r\in R'$ y $s\in S'$, entonces
	$\mcd{r,m}=1$ y $\mcd{s,n}=1$, de lo que deducimos que
	$\mcd{t,mn}=1$ y concluimos que $t\in T'$.

	La igualdad $\eulerphi(mn)=\eulerphi(m)\eulerphi(n)$ es
	consecuencia de que el valor de $\eulerphi$ es igual
	al cardinal de cualquier sistema reducido de representantes.
\end{proof}

En particular, si tuvi\'esemos una f\'ormula para $\eulerphi(p^r)$,
$p$ primo, podr\'{\i}amos obtener una f\'ormula para $\eulerphi(m)$
en funci\'on de la factorizaci\'on de $m$ como producto de primos
a potencias.%
\footnote{
	Ver el \ejername~\ref{ejer:chino:euler}.
}

\begin{coroChino}\label{coro:chino:polinomial}
	Sea $f\in\polinomios\Enteros$ y, para cada $m\in\Enteros$,
	sea $N(m)$ la cantidad de soluciones a $f(x)\equiv 0\tmodulo[m]$.
	Si $m=m_1m_2$ es una factorizaci\'on de $m$ con $\mcd{m_1,m_2}=1$,
	entonces $N(m)=N(m_1)N(m_2)$.
	% la ecuaci\'on de congruencia
	% \begin{equation}
		% \label{eq:chino:una}
		% f(x)\,\equiv\,a\modulo[m]
	% \end{equation}
	% %
	% es equivalente al sistema de ecuaciones
	% \begin{equation}
		% \label{eq:chino:dos}
		% % \begin{aligned}
			% f(x) \,\equiv\,a\tmodulo[m_1]\dispand
			% f(x) \,\equiv\,a\tmodulo[m_2]\dispstop
		% % \end{aligned}
		% % %
	% \end{equation}
	% %
	% Adem\'as, si $\cal R$, $\cal R_1$ y $\cal R_2$ son, respectivamente,
	% sistemas completos de representantes de las clases m\'odulo $m$,
	% $m_1$ y $m_2$, entonces hay una biyecci\'on
	% \begin{displaymath}
		% \begin{aligned}
			% & \big\{x\in\cal R\,:\,f(x)\equiv a\tmodulo[m]\big\}
				% \,\simeq\, \\
			% & \qquad
			% \big\{x_1\in\cal R_1\,:\,
				% f(x_1)\equiv a\tmodulo[m_1]\big\}
			% \,\times\,
			% \big\{x_2\in\cal R_2\,:\,
				% f(x_2)\equiv a\tmodulo[m_2]\big\}
		% \end{aligned}
		% %
	% \end{displaymath}
	% %
	% dada por $x\mapsto (x_1,x_2)$, donde $x_1$ y $x_2$ est\'an
	% determinados por:
	% \begin{displaymath}
		% % \begin{aligned}
			% x \,\equiv\,x_1\tmodulo[m_1]\dispand
			% x \,\equiv\,x_2\tmodulo[m_2]\dispstop
		% % \end{aligned}
		% % %
	% \end{displaymath}
	% %
\end{coroChino}

\begin{proof}
	Cada soluci\'on $x\in\Enteros$ de $f(x)\equiv 0\tmodulo[m]$
	da lugar a una soluci\'on de $f(x)\equiv 0\tmodulo[m_1]$ y a
	una soluci\'on de $f(x)\equiv 0\tmodulo[m_2]$ (el mismo entero sirve).
	En la otra direcci\'on,
	cada par de soluciones $x_1,x_2\in\Enteros$ m\'odulo $m_1$ y
	m\'odulo $m_2$, respectivamente, es decir, enteros que satisfacen
	$f(x_1)\equiv 0\tmodulo[m_1]$ y $f(x_2)\equiv 0\tmodulo[m_2]$,
	dan lugar a un $x\in\Enteros$ que verifica
	$x\equiv x_1\tmodulo[m_1]$ y $x\equiv x_2\tmodulo[m_2]$.
	Dicho entero es \'unico m\'odulo $m$ (\teoname~\ref{teo:chino})
	y verifica $f(x)\equiv 0\tmodulo[m]$ (tambi\'en por
	\teoname~\ref{teo:chino}, aplicado al entero $f(x)$).
\end{proof}

% \begin{obsChino}\label{obs:chino:coro}
	% Dado que, en un sistema de representantes, los representantes se
	% corresponden con las clases que representan, podemos expresar el
	% resultado del \coroname~\ref{coro:chino:polinomial} diciendo que
	% las soluciones m\'odulo $m$ est\'an en biyecci\'on
	% con el producto cartesiano de las soluciones m\'odulo $m_1$ por
	% las soluciones m\'odulo $m_2$.
% \end{obsChino}

\begin{ejemChino}\label{ejem:chino:polinomial:quince}
	La ecuaci\'on $x^2+x+7\equiv 0\tmodulo[15]$ no tiene soluci\'on.
	Por el Teorema chino del resto, la congruencia es equivalente al
	sistema
	\begin{displaymath}
		x^2+x+7\,\equiv\,0\tmodulo[3]\dispand
		x^2+x+7\,\equiv\,0\tmodulo[5]
		\dispstop
	\end{displaymath}
	%
	Ahora, $7\equiv 1\tmodulo[3]$ y $7\equiv 2\tmodulo[5]$.
	Si bien $x^2+x+1\equiv 0\tmodulo[3]$ tiene como (\'unica)
	soluci\'on m\'odulo $3$ a $x=1$, la
	congruencia $x^2+x+2\equiv 0\tmodulo[5]$ no tiene soluci\'on
	con $x\in\Enteros$.
	En consecuencia, $N(15)=N(3)N(5)=1\cdot 0=0$.
\end{ejemChino}

\begin{ejemChino}\label{ejem:chino:polinomial:veinticiete}
	Vimos que $x^2+x+7\equiv 0\tmodulo[3]$ tiene una \'unica
	soluci\'on m\'odulo $3$ ?`Cu\'antas soluciones m\'odulo $63$ hay?
	En primer lugar, factorizamos $63=9\cdot 7$ y resolvemos
	las congruencias m\'odulo $9$ y m\'odulo $7$ por separado.
	Dado que hay soluciones m\'odulo $3$, \emph{es posible que haya}
	soluciones m\'odulo $9$.

	Sea $f(x)=x^2+x+7$. Sabemos que los $x\in\Enteros$ tales que
	$f(x)\equiv 0\tmodulo[9]$ deben ser $x\equiv 1\tmodulo[3]$.
	Mirando m\'odulo $9$, las distintas posibilidades son
	$x=1$, $x=4$ y $x=7$. Efectivamente,
	$f(1)\equiv f(4)\equiv f(7)\equiv 0\tmodulo[9]$.
	Con lo cual, $N(9)=3$.
	Por otro lado, si $x\in\Enteros$, se cumple que
	$f(x)\equiv x^2+x\tmodulo[7]$. Entonces, como $7$ es primo,
	$f(x)\equiv 0\tmodulo[7]$ tiene exactamente dos soluciones
	m\'odulo $7$: $0$ y $6\equiv -1$. O sea, $N(7)=2$.
	As\'{\i}, $N(63)=3\cdot 2=6$. Las seis soluciones m\'odulo $63$
	vienen dadas por resolver los seis sistemas siguientes:
	\begin{displaymath}
		\begin{aligned}
			\left\{
			\begin{array}{r@{\,\equiv\,}l}
				x & 1 \tmodulo[9] \dispand \\[5pt]
				x & 0 \tmodulo[7] \dispcomma
			\end{array}
			\right.
			\left\{
			\begin{array}{r@{\,\equiv\,}l}
				x & 4 \tmodulo[9] \dispand \\[5pt]
				x & 0 \tmodulo[7] \dispcomma
			\end{array}
			\right.
			\left\{
			\begin{array}{r@{\,\equiv\,}l}
				x & 7 \tmodulo[9] \dispand \\[5pt]
				x & 0 \tmodulo[7] \dispcomma
			\end{array}
			\right. \\
			\left\{
			\begin{array}{r@{\,\equiv\,}l}
				x & 1 \tmodulo[9] \dispand \\[5pt]
				x & 6 \tmodulo[7] \dispcomma
			\end{array}
			\right.
			\left\{
			\begin{array}{r@{\,\equiv\,}l}
				x & 4 \tmodulo[9] \dispand \\[5pt]
				x & 6 \tmodulo[7] \dispcomma
			\end{array}
			\right.
			\left\{
			\begin{array}{r@{\,\equiv\,}l}
				x & 7 \tmodulo[9] \dispand \\[5pt]
				x & 6 \tmodulo[7] \dispstop
			\end{array}
			\right.
		\end{aligned}
		%
	\end{displaymath}
	%
	Las soluciones son: $28$, $49$, $7$, $55$, $13$ y $34$.
\end{ejemChino}

\subsection*{Ejercicios}
\theoremstyle{definition}
\newtheorem{ejerChino}{\ejername}[section]

%-------------

\begin{ejerChino}\label{ejer:chino}
	El sistema de congruencias
	\begin{displaymath}
		x\,\equiv\,a_1\tmodulo[m_1]\dispand
		x\,\equiv\,a_2\tmodulo[m_2]
	\end{displaymath}
	%
	admite una soluci\'on, si y s\'olo si
	$\mcd{m_1,m_2}\mid a_1-a_2$.
	En ese caso, si $x\in\Enteros$ es una soluci\'on, entonces
	$y\in\Enteros$ es soluci\'on, si y s\'olo si
	$y\equiv x\tmodulo[m]$, donde $m=\mcm{m_1,m_2}$ es el
	m\'{\i}nimo com\'un m\'ultiplo de $m_1$ y $m_2$.
\end{ejerChino}

\begin{ejerChino}\label{ejer:chino:varios}
	Si $\lista m r\in\Enteros$ son coprimos de a pares y
	$\lista a r\in\Enteros$, entonces existe una soluci\'on
	com\'un a las congruencias
	\begin{displaymath}
		x\,\equiv\, a_i\modulo[m_i]
	\end{displaymath}
	%
	y, dada una soluci\'on $x\in\Enteros$, un $y\in\Enteros$
	es soluci\'on, si y s\'olo si
	$y\equiv x\tmodulo[m]$, donde $m=m_1\cdots m_r$.
	Este resultado se puede demostrar aplicando inductivamente
	el \teoname~\ref{teo:chino}.
	La siguiente es una demostraci\'on alternativa.
	\begin{enumerate}[(i)]
		\item\label{item:ejer:chino:varios:i}
			Probar esta afirmaci\'on en los casos
			con $a_1=1,a_2=0,\,\dots,\,a_r=0$, etc.
			(considerar $x_1=(m/m_1)\,b_1$, donde
			$b_1$ es soluci\'on de la ecuaci\'on
			$(m/m_1)\,b_1\equiv 1\tmodulo[m_1]$).
		\item\label{item:ejer:chino:varios:ii}
			Probar que, si $x_j$ es soluci\'on
			del sistema con $a_j=1$ como en
			\eqref{item:ejer:chino:varios:i}, entonces,
			en el sistema con $a_j$ arbitrarios,
			$x=\sum_j\,x_ja_j$ es soluci\'on.
		\item\label{item:ejer:chino:varios:iii}
			Probar que, $y$ es otra soluci\'on,
			entonces $y\equiv x\tmodulo[m_i]$ para cada $i$
			(o sea, $m_i\mid y-x$)
			y concluir que $y\equiv x\tmodulo[m]$.
	\end{enumerate}
	%
\end{ejerChino}

\begin{ejerChino}\label{ejer:chino:varios:coro}
	Si $f\in\polinomios\Enteros$, $\lista m r\in\Enteros$ son
	coprimos de a pares y $m=m_1\cdots m_r$,
	entonces la ecuaci\'on de congruencia
	\begin{displaymath}
		f(x)\,\equiv\,a\modulo[m]
	\end{displaymath}
	es equivalente al sistema de ecuaciones
	\begin{displaymath}
		f(x)\,\equiv\,a\modulo[m_i]
		\dispstop
	\end{displaymath}
	%
	Adem\'as, las soluciones m\'odulo $m$ est\'an en biyecci\'on
	con el producto cartesiano de las soluciones m\'odulo
	$m_i$, para cada $1\leq i\leq r$.
\end{ejerChino}

\begin{ejerChino}\label{ejer:chino:euler}
	Si $p$ es un n\'umero primo, entonces
	$\eulerphi(p^r)=p^{r-1}\,(p-1)=p^r\,(1-1/p)$.
	Dar una f\'ormula para $\eulerphi(m)$, conociendo la
	factorizaci\'on de $m$ como potencia de primos.
\end{ejerChino}

\begin{ejerChino}
	Hallar el menor entero positivo $\neq 1$ que es soluci\'on de
	\begin{displaymath}
		x\,\equiv\,1\tmodulo[3]\dispcomma\quad
		x\,\equiv\,1\tmodulo[5]\dispand
		x\,\equiv\,1\tmodulo[7]
		\dispstop
	\end{displaymath}
	%
\end{ejerChino}

\begin{ejerChino}
	Hallar todos los enteros que satisfacen los siguientes sistemas:
	\begin{enumerate}[(i)]
		\item
			\begin{displaymath}
				x\,\equiv\,2\tmodulo[3]\dispcomma\quad
				x\,\equiv\,3\tmodulo[5]\dispand
				x\,\equiv\,5\tmodulo[2]
				\text{ ;}
			\end{displaymath}
			%
		\item
			\begin{displaymath}
				x\,\equiv\,1\tmodulo[4]\dispcomma\quad
				x\,\equiv\,0\tmodulo[3]\dispand
				x\,\equiv\,5\tmodulo[7]
				\text{ ;}
			\end{displaymath}
			%
		\item
			\begin{displaymath}
				5x\,\equiv\,1\tmodulo[6]\dispcomma\quad
				4x\,\equiv\,13\tmodulo[15]
				\dispstop
			\end{displaymath}
		%
	\end{enumerate}
	%
\end{ejerChino}

\begin{ejerChino}
	Resolver las siguientes ecuaciones de congruencia:
	\begin{enumerate}[(i)]
		\item $x^3+2x-3\equiv 0\tmodulo[9]$;
		\item $x^3+2x-3\equiv 0\tmodulo[5]$;
		\item $x^3+2x-3\equiv 0\tmodulo[45]$;
		\item $x^3+4x+8\equiv 0\tmodulo[15]$;
		\item $x^3-9x^2+23x-15\equiv 0\tmodulo[503]$;%
			\hint{
				$503$ es primo y
				$x^3-9x^2+23-15=(x-1)\,(x-3)\,(x-5)$.
			}
		\item $x^3-9x^2+23x-15\equiv 0\tmodulo[143]$.
	\end{enumerate}
	%
\end{ejerChino}

\begin{ejerChino}\label{ejer:chino:idempotentes}
	Si $N(m)$ es la cantidad de soluciones a $x^2\equiv x\tmodulo[m]$,
	hallar una f\'ormula para $N(p^r)$, con $p$ primo.%
	\hint{
		Hacer el caso $r=1$.
	}
	Deducir una f\'ormula para $N(m)$, $m$ arbitrario.
\end{ejerChino}

\begin{ejerChino}\label{ejer:chino:idempotentes:bis}
	Para $m\geq 1$, entero, sea
	\begin{math}
		\psi(m)=\cardinal{%
			\{1\leq t\leq m\,:\,
				\mcd{t,m}=1,\,\mcd{t+1,m}=1\}
			}
	\end{math}.
	Probar las siguientes afirmaciones:
	\begin{enumerate}[(i)]
		\item si $p$ es primo, $\psi(p)=p-2$;
		\item si $p$ es primo y $r\geq 1$,
			$\psi(p^r)=p^{r-1}\,(p-2)=p^r\,(1-2/p)$;
		\item si $\mcd{m,n}=1$, entonces $\psi(mn)=\psi(m)\psi(n)$.
	\end{enumerate}
	%
	Deducir una f\'ormula para $\psi(m)$.
\end{ejerChino}

\begin{ejerChino}\label{ejer:chino:polinomial}
	Sea $f\in\polinomios\Enteros$ y sean
	\begin{itemize}
		\item $N(m)$ la cantidad de soluciones de
			$f(x)\equiv 0\tmodulo[m]$ y
		\item
			\begin{math}
				\phi_f(m)=\cardinal{
					\{1\leq t\leq m\,:\,\mcd{f(t),m}=1\}
					}
			\end{math}.
	\end{itemize}
	%
	Probar las siguientes afirmaciones:
	\begin{enumerate}[(i)]
		\item si $p$ es primo, $\phi_f(p)=p-N(p)$;
		\item si $p$ es primo y $r\geq 1$,
			$\phi_f(p^r)=p^{r-1}\phi_f(p)=p^r\,(1-N(p)/p)$;
		\item si $\mcd{m,n}=1$, entonces
			$\phi_f(mn)=\phi_f(m)\phi_f(n)$.
	\end{enumerate}
	%
	Concluir que, si $m\in\Enteros$, vale
	\begin{displaymath}
		\phi_f(m)\,=\,m\,\prod_{p\mid m}\,\big(1-N(p)/p\big)
		\dispstop
	\end{displaymath}
	%
	Comparar con la f\'ormula para $\eulerphi$ y la f\'ormula para
	la funci\'on $\psi$ del \ejername~\ref{ejer:chino:idempotentes}.
	Deducir nuevamente las f\'ormulas para estas funciones con el
	esquema de este ejercicio ?`Cu\'ales son los polinomios en cada
	caso?
\end{ejerChino}


