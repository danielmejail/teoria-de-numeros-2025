\theoremstyle{definition}
\newtheorem{ejerModulares}{\ejername}[section]

%-------------

\begin{ejerModulares}\label{ejer:modulares:lineal}
	Probar que, si $\mcd{a,m}=1$, entonces $ax=b$ tiene una \'unica
	soluci\'on en $\Enterosmod[m]$, cualquiera sea $b\in\Enteros$.
	Describir las soluciones a la ecuaci\'on de congruencia
	$ax\equiv b\tmodulo[m]$, es decir, el conjunto de $x\in\Enteros$
	tales que $ax\equiv b\tmodulo[m]$.
\end{ejerModulares}

\begin{ejerModulares}
	Hacer una tabla de multiplicaci\'on y suma para los anillos
	$\Enterosmod[5]$, $\Enterosmod[8]$ y $\Enterosmod[10]$.
\end{ejerModulares}

\begin{ejerModulares}
	Sea $p$ un primo impar y sea $k\in\{1,\,\dots,\,p-1\}$. Entonces
	existe un\'unico $b_k\in\{1,\,\dots,\,p-1\}$ tal que
	$kb_k\equiv 1\tmodulo[p]$. Adem\'as, $k\neq b_k$, excepto en los
	casos $k=1$ y $k=p-1$ ?`Son ciertas estas afirmaciones en el caso
	de un m\'odulo $m$ cualquiera, no necesariamente primo?
\end{ejerModulares}

\begin{ejerModulares}
	Si $p$ es primo $(p-1)!\equiv -1\tmodulo[p]$. Si $n=4$, entonces
	$(4-1)!=3!=6\equiv 2\tmodulo[4]$. Si $n>4$ \emph{no es} primo,
	entonces $(n-1)!\equiv 0\tmodulo[n]$.
\end{ejerModulares}

\begin{ejerModulares}
	Sea $R=\{\lista r{\eulerphi(m)}\}$ un sistema reducido de
	representantes de las clases m\'odulo $m$ y sea $N\geq 0$ la
	cantidad de soluciones a la ecuaci\'on de congruencia
	$x^2\equiv 1\tmodulo[m]$. Probar que%
	\hint{
		Si $r\in R$, entonces, por un lado, existe $r'\in R$
		tal que $rr'\equiv 1$. Por otro, $\mcd{-r,m}=1$ y, si
		$m>2$, entonces tambi\'en $-r\not\equiv r$.
	}
	\begin{displaymath}
		\prod_{i=1}^{\eulerphi(m)}\,r_i\,\equiv\,(-1)^{N/2}\modulo[m]
		\text{ .}
	\end{displaymath}
	%
\end{ejerModulares}

% \begin{ejerModulares}
	% Sea $\binomial p{k}=\frac{p!}{k!(p-k)!}$ el coeficiente binomial.
	% Si $p$ es primo y $1\leq k\leq p-1$, entonces $p$ divide a
	% $\binomial p{k}$. Deducir que
	% \begin{displaymath}
		% (a+b)^p\,\equiv\,a^p+b^p\modulo[p]
		% \text{ .}
	% \end{displaymath}
% \end{ejerModulares}

\begin{ejerModulares}
	Sean $p$ y $q$ primos impares distintos y supongamos, adem\'as,
	que $p-1\mid q-1$. Si $\mcd{n,pq}=1$, entonces
	\begin{math}
		n^{q-1}\,\equiv\,1\tmodulo[pq]
	\end{math}.%
	\hint{
		% Ver \ejername~\ref{ejer:congruencias:chino}.
		Probar que $n^{q-1}\equiv 1\tmodulo[p]$ y que
		$n^{q-1}\equiv 1\tmodulo[q]$.
	}
\end{ejerModulares}

\begin{ejerModulares}
	Probar que, si $p$ es primo, entonces $p$ divide al numerador de
	$1+\frac 1{2}+\frac 1{3}+\cdots+\frac 1{p-1}$.
\end{ejerModulares}

\begin{ejerModulares}\label{ejer:modulares:residuos}
	Probar que
	\begin{enumerate}[(i)]
		\item\label{item:ejer:modulares:residuos:primo-impar}
			si $p$ es un primo impar y $a\geq 1$,
			las \'unicas soluciones $x^2=1$ en $\Enterosmod[p^a]$
			son $\pm 1$;
		\item\label{item:ejer:modulares:residuos:dos}
			$x^2=1$ tiene una \'unica soluci\'on en
			$\Enterosmod[2]$, dos soluciones en $\Enterosmod[4]$
			y cuatro soluciones en $\Enterosmod[2^b]$, si
			$b\geq 3$.
	\end{enumerate}
	%
	Determinar la cantidad de soluciones a $x^2\equiv 1\tmodulo[m]$
	para $m\in\Enteros$, $m\neq 0$.%
	\hint{
		% Ver \ejername~\ref{ejer:congruencias:chino}.
		Teorema chino del resto.
	}
\end{ejerModulares}

\begin{ejerModulares}
	Si $R=\{\lista r{p-1}\}$ es un sistema reducido m\'odulo $p$, primo,
	entonces
	\begin{displaymath}
		\prod_{i=1}^{p-1}\,r_i\,\equiv\,-1\modulo[p]
		\text{ .}
	\end{displaymath}
	%
\end{ejerModulares}

\begin{ejerModulares}
	Si $R=\{r_1,\,\dots,\,r_p\}$ y $R'=\{r_1',\,\dots,\,r_p'\}$ son
	dos sistemas completos de representantes m\'odulo un primo $p>2$,
	entonces $r_1r_1',\,\dots,\,r_pr_p'$ no pueden formar un sistema
	completo de representantes m\'odulo $p$.
\end{ejerModulares}

