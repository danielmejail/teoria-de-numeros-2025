\theoremstyle{definition}
\newtheorem{ejerChino}{\ejername}[section]

%-------------

\begin{ejerChino}\label{ejer:chino}
	El sistema de congruencias
	\begin{displaymath}
		x\,\equiv\,a_1\tmodulo[m_1]\dispand
		x\,\equiv\,a_2\tmodulo[m_2]
	\end{displaymath}
	%
	admite una soluci\'on, si y s\'olo si
	$\mcd{m_1,m_2}\mid a_1-a_2$.
	En ese caso, si $x\in\Enteros$ es una soluci\'on, entonces
	$y\in\Enteros$ es soluci\'on, si y s\'olo si
	$y\equiv x\tmodulo[m]$, donde $m=\mcm{m_1,m_2}$ es el
	m\'{\i}nimo com\'un m\'ultiplo de $m_1$ y $m_2$.
\end{ejerChino}

\begin{ejerChino}\label{ejer:chino:varios}
	Si $\lista m r\in\Enteros$ son coprimos de a pares y
	$\lista a r\in\Enteros$, entonces existe una soluci\'on
	com\'un a las congruencias
	\begin{displaymath}
		x\,\equiv\, a_i\modulo[m_i]
	\end{displaymath}
	%
	y, dada una soluci\'on $x\in\Enteros$, un $y\in\Enteros$
	es soluci\'on, si y s\'olo si
	$y\equiv x\tmodulo[m]$, donde $m=m_1\cdots m_r$.
	Este resultado se puede demostrar aplicando inductivamente
	el \teoname~\ref{teo:chino}.
	La siguiente es una demostraci\'on alternativa.
	\begin{enumerate}[(i)]
		\item\label{item:ejer:chino:varios:i}
			Probar esta afirmaci\'on en los casos
			con $a_1=1,a_2=0,\,\dots,\,a_r=0$, etc.
			(considerar $x_1=(m/m_1)\,b_1$, donde
			$b_1$ es soluci\'on de la ecuaci\'on
			$(m/m_1)\,b_1\equiv 1\tmodulo[m_1]$).
		\item\label{item:ejer:chino:varios:ii}
			Probar que, si $x_j$ es soluci\'on
			del sistema con $a_j=1$ como en
			\eqref{item:ejer:chino:varios:i}, entonces,
			en el sistema con $a_j$ arbitrarios,
			$x=\sum_j\,x_ja_j$ es soluci\'on.
		\item\label{item:ejer:chino:varios:iii}
			Probar que, $y$ es otra soluci\'on,
			entonces $y\equiv x\tmodulo[m_i]$ para cada $i$
			(o sea, $m_i\mid y-x$)
			y concluir que $y\equiv x\tmodulo[m]$.
	\end{enumerate}
	%
\end{ejerChino}

\begin{ejerChino}\label{ejer:chino:varios:coro}
	Si $f\in\polinomios\Enteros$, $\lista m r\in\Enteros$ son
	coprimos de a pares y $m=m_1\cdots m_r$,
	entonces la ecuaci\'on de congruencia
	\begin{displaymath}
		f(x)\,\equiv\,a\modulo[m]
	\end{displaymath}
	es equivalente al sistema de ecuaciones
	\begin{displaymath}
		f(x)\,\equiv\,a\modulo[m_i]
		\dispstop
	\end{displaymath}
	%
	Adem\'as, las soluciones m\'odulo $m$ est\'an en biyecci\'on
	con el producto cartesiano de las soluciones m\'odulo
	$m_i$, para cada $1\leq i\leq r$.
\end{ejerChino}

\begin{ejerChino}\label{ejer:chino:euler}
	Si $p$ es un n\'umero primo, entonces
	$\eulerphi(p^r)=p^{r-1}\,(p-1)=p^r\,(1-1/p)$.
	Dar una f\'ormula para $\eulerphi(m)$, conociendo la
	factorizaci\'on de $m$ como potencia de primos.
\end{ejerChino}

\begin{ejerChino}
	Hallar el menor entero positivo $\neq 1$ que es soluci\'on de
	\begin{displaymath}
		x\,\equiv\,1\tmodulo[3]\dispcomma\quad
		x\,\equiv\,1\tmodulo[5]\dispand
		x\,\equiv\,1\tmodulo[7]
		\dispstop
	\end{displaymath}
	%
\end{ejerChino}

\begin{ejerChino}
	Hallar todos los enteros que satisfacen los siguientes sistemas:
	\begin{enumerate}[(i)]
		\item
			\begin{displaymath}
				x\,\equiv\,2\tmodulo[3]\dispcomma\quad
				x\,\equiv\,3\tmodulo[5]\dispand
				x\,\equiv\,5\tmodulo[2]
				\text{ ;}
			\end{displaymath}
			%
		\item
			\begin{displaymath}
				x\,\equiv\,1\tmodulo[4]\dispcomma\quad
				x\,\equiv\,0\tmodulo[3]\dispand
				x\,\equiv\,5\tmodulo[7]
				\text{ ;}
			\end{displaymath}
			%
		\item
			\begin{displaymath}
				5x\,\equiv\,1\tmodulo[6]\dispcomma\quad
				4x\,\equiv\,13\tmodulo[15]
				\dispstop
			\end{displaymath}
		%
	\end{enumerate}
	%
\end{ejerChino}

\begin{ejerChino}
	Resolver las siguientes ecuaciones de congruencia:
	\begin{enumerate}[(i)]
		\item $x^3+2x-3\equiv 0\tmodulo[9]$;
		\item $x^3+2x-3\equiv 0\tmodulo[5]$;
		\item $x^3+2x-3\equiv 0\tmodulo[45]$;
		\item $x^3+4x+8\equiv 0\tmodulo[15]$;
		\item $x^3-9x^2+23x-15\equiv 0\tmodulo[503]$;%
			\hint{
				$503$ es primo y
				$x^3-9x^2+23-15=(x-1)\,(x-3)\,(x-5)$.
			}
		\item $x^3-9x^2+23x-15\equiv 0\tmodulo[143]$.
	\end{enumerate}
	%
\end{ejerChino}

\begin{ejerChino}\label{ejer:chino:idempotentes}
	Si $N(m)$ es la cantidad de soluciones a $x^2\equiv x\tmodulo[m]$,
	hallar una f\'ormula para $N(p^r)$, con $p$ primo.%
	\hint{
		Hacer el caso $r=1$.
	}
	Deducir una f\'ormula para $N(m)$, $m$ arbitrario.
\end{ejerChino}

\begin{ejerChino}\label{ejer:chino:idempotentes:bis}
	Para $m\geq 1$, entero, sea
	\begin{math}
		\psi(m)=\cardinal{%
			\{1\leq t\leq m\,:\,
				\mcd{t,m}=1,\,\mcd{t+1,m}=1\}
			}
	\end{math}.
	Probar las siguientes afirmaciones:
	\begin{enumerate}[(i)]
		\item si $p$ es primo, $\psi(p)=p-2$;
		\item si $p$ es primo y $r\geq 1$,
			$\psi(p^r)=p^{r-1}\,(p-2)=p^r\,(1-2/p)$;
		\item si $\mcd{m,n}=1$, entonces $\psi(mn)=\psi(m)\psi(n)$.
	\end{enumerate}
	%
	Deducir una f\'ormula para $\psi(m)$.
\end{ejerChino}

\begin{ejerChino}\label{ejer:chino:polinomial}
	Sea $f\in\polinomios\Enteros$ y sean
	\begin{itemize}
		\item $N(m)$ la cantidad de soluciones de
			$f(x)\equiv 0\tmodulo[m]$ y
		\item
			\begin{math}
				\phi_f(m)=\cardinal{
					\{1\leq t\leq m\,:\,\mcd{f(t),m}=1\}
					}
			\end{math}.
	\end{itemize}
	%
	Probar las siguientes afirmaciones:
	\begin{enumerate}[(i)]
		\item si $p$ es primo, $\phi_f(p)=p-N(p)$;
		\item si $p$ es primo y $r\geq 1$,
			$\phi_f(p^r)=p^{r-1}\phi_f(p)=p^r\,(1-N(p)/p)$;
		\item si $\mcd{m,n}=1$, entonces
			$\phi_f(mn)=\phi_f(m)\phi_f(n)$.
	\end{enumerate}
	%
	Concluir que, si $m\in\Enteros$, vale
	\begin{displaymath}
		\phi_f(m)\,=\,m\,\prod_{p\mid m}\,\big(1-N(p)/p\big)
		\dispstop
	\end{displaymath}
	%
	Comparar con la f\'ormula para $\eulerphi$ y la f\'ormula para
	la funci\'on $\psi$ del \ejername~\ref{ejer:chino:idempotentes}.
	Deducir nuevamente las f\'ormulas para estas funciones con el
	esquema de este ejercicio ?`Cu\'ales son los polinomios en cada
	caso?
\end{ejerChino}

\begin{ejerChino}
	Sean $a,b,n\in\Enteros$. Probar que, si $\mcd{a,n}=1$, entonces
	existe $t\in\Enteros$ tal que
	\begin{math}
		\mcd{at+n,b}=1
	\end{math}.
\end{ejerChino}

