\theoremstyle{plain}
\newtheorem{teoPolinomios}{\teoname}[section]
\newtheorem{coroPolinomios}[teoPolinomios]{\coroname}

\theoremstyle{definition}
\newtheorem{defPolinomios}[teoPolinomios]{\defname}
\newtheorem{obsPolinomios}[teoPolinomios]{\obsname}
\newtheorem{ejemPolinomios}[teoPolinomios]{\ejemname}

%-------------

% \subsection{Subanillos}%
	% \label{subsec:subanillos}
En esta secci\'on, salvo aclaraci\'on, ``anillo'' querr\'a decir
``anillo conmutativo con unidad''.

\begin{defPolinomios}\label{def:polinomios:subanillo}
	Si $A$ es un anillo y $B\subset A$ es un subconjunto con las
	propiedades: $0\in B$, $B+B\subset B$, $B\cdot B\subset B$ y
	$1\in B$, decimos que $B$ es \emph{subanillo} de $A$.
\end{defPolinomios}

\begin{ejemPolinomios}\label{ejem:polinomios:subanillo}
	El subconjunto $\Enteros\subset\Racionales$ es un subanillo del
	anillo (cuerpo) de n\'umeros racionales; como subconjuntos de los
	n\'umeros complejos, $\Enteros$, $\Racionales$ y $\Reales$ son
	subanillos que satisfacen:
	$\Enteros\subset\Racionales\subset\Reales\subset\Complejos$.
	Los subconjuntos $\EnterosGauss$, $\EnterosEisenstein$ y
	$\polinomios\Enteros{\sqrt{-6}}$ tambi\'en son subanillos de
	$\Complejos$, pero ninguno de ellos est\'a contenido en $\Reales$.
	El anillo $\polinomios\Enteros{\sqrt 6}$ --definido de manera
	an\'aloga a $\polinomios\Enteros{\sqrt{-6}}$-- es un subanillo de
	$\Reales$ (pero no de $\Racionales$).
\end{ejemPolinomios}

% \subsection{Expresiones polinomiales}%
	% \label{subsec:expresiones}
\begin{obsPolinomios}\label{obs:polinomios:expresiones}
	Si $a\in A$ y $B\subset A$ es un subanillo, entonces
	$a,\,a^2,\,\dots$ son elementos de $A$ y tambi\'en lo son
	$ba,\,ba^2,\,\dots$, si $b\in B$. M\'as en general, la expresi\'on
	\begin{equation}
		\label{eq:polinomios:expresiones}
		b_0\,+\,b_1a\,+\,\cdots\,+\,b_ma^m
		\text{ ,}
	\end{equation}
	%
	donde $m$ es natural o $0$ y $\lista[0] b{m}\in B$,
	define un elemento de $A$.
	El elemento $0$ es de este tipo (eligiendo $m=0$ y $b_0=0$,
	por ejemplo), como tambi\'en lo son $1$ ($m=0$ y $b_0=1$) y
	$a$ ($m=1$, $b_0=0$ y $b_1=1$).
	M\'as aun, dados dos elementos del tipo
	\eqref{eq:polinomios:expresiones}, $p$ y $q$, su suma, $p+q$, y su
	producto, $p\cdot q$, % (elementos tambi\'en de $A$)
	son tambi\'en del tipo \eqref{eq:polinomios:expresiones}
	\quedacomoejercicio.
	En particular, el subconjunto de $A$,
	\begin{displaymath}
		\polinomios B a\,=\,\big\{b_0+b_1a+\,\cdots\,+b_ma^m\,:\,
			m\text{ natural o } 0\text{ y }
			\lista[0] b{m}\in B\big\}
	\end{displaymath}
	%
	es un subanillo de $A$.
\end{obsPolinomios}

\begin{defPolinomios}\label{def:polinomios:expresiones}
	Sean $A$ un anillo, $a\in A$ un elemento y $B\subset A$ un subanillo.
	Una \emph{expresi\'on polinomial en $a$ con coeficientes en $B$}
	es un elemento de $A$ del tipo \eqref{eq:polinomios:expresiones}.
	El \emph{anillo de expresiones polinomiales en $a$ con coeficientes %
	en $B$} es el subanillo $\polinomios B a\subset A$.
\end{defPolinomios}

\begin{ejemPolinomios}\label{ejem:polinomios:expresiones}
	% Si $\sqrt{-6}\in\Complejos$ es un n\'umero complejo que cumple que
	% $\sqrt{-6}^2=-6$, entonces $\sqrt{-6}^n=(-6)^{n/2}$, si $n$ es par y
	% $\sqrt{-6}^n=(-6)^{(n-1)/2}\sqrt{-6}$, si $n$ es impar.
	% El anillo $\polinomios \Enteros{\sqrt{-6}}$ coincide con el
	% subanillo de $\Complejos$ de expresiones polinomiales en
	% $\sqrt{-6}$ con coeficientes en $\Enteros$
	Si $\sqrt{-6}\in\Complejos$ es un n\'umero complejo que cumple que
	$\sqrt{-6}^2=-6$, entonces el subanillo de expresiones polinomiales
	en $\sqrt{-6}$ con coeficientes en $\Enteros$ coincide con el
	anillo $\polinomios \Enteros{\sqrt{-6}}$
	\quedacomoejercicio.
\end{ejemPolinomios}

% \subsection{Elementos trascendentes}%
	% \label{subsec:trascendentes}
\begin{obsPolinomios}\label{obs:polinomios:trascendente}
	En general, no tiene por qu\'e ser cierto \emph{a priori} que
	\begin{math}
		b_0+b_1a+\,\cdots\,+b_ma^m= c_0+c_1a+\,\cdots\,+c_na^n
	\end{math}
	implique $m=n$ y $b_i=c_i$ para cada $i$.
	Siguiendo con el \ejemname~\ref{ejem:polinomios:expresiones},
	la igualdad $\sqrt{-6}^2=-6$ muestra un ejemplo de esto.
	En particular, podr\'{\i}a ocurrir que
	\begin{math}
		b_0+b_1a+\,\cdots\,+b_ma^m=0
	\end{math},
	pero que $b_m\neq 0$.
\end{obsPolinomios}

\begin{defPolinomios}\label{def:polinomios:trascendente}
	Sean $B$ un anillo, $A$ un anillo del cual $B$ es subanillo y
	$a\in A$ un elemento.
	Decimos que \emph{$a$ es trascendente sobre $B$}, si las \'unicas
	expresiones polinomiales en $a$ con coeficientes en $B$ iguales a cero
	son las \emph{expresiones triviales}, es decir, aquellas cuyos
	coeficientes son todos iguales a cero.
	Expresado de otra manera, $a\in A$ es trascendente sobre $B$, si
	\begin{displaymath}
		b_0\,+\,b_1a\,+\,\cdots\,+\,b_ma^m\,=\,0
		\quad\text{implica}\quad b_0\,=\,b_1\,=\,\cdots\,=\,b_m\,=\,0
		\text{ .}
	\end{displaymath}
	%
\end{defPolinomios}

% \begin{teoPolinomios}\label{teo:polinomios:trascendente}
	% Sean $B$ un anillo, $A$ un anillo del cual $B$ es subanillo y
	% $x,a\in A$ elementos.
	% % Sean $A$ un anillo, $x,a\in A$ elementos y $B\subset A$ un subanillo.
	% Si $x$ es trascendente sobre $B$, existe una \'unica funci\'on
	% $f:\,\polinomios B x\rightarrow A$ tal que
	% \begin{displaymath}
		% f(b)\,=\,b \text{ para todo }b\in B
		% \quad\text{y}\quad f(x)\,=\,a
	% \end{displaymath}
	% %
	% y tal que
	% \begin{displaymath}
		% f(u+v)\,=\,f(u)+f(v)\quad\text{y}\quad
			% f(uv)\,=\,f(u)f(v)
		% \text{ ,}
	% \end{displaymath}
	% %
	% siempre que $u,v\in\polinomios B x$.%
	% \footnote{
		% En este caso, decimos que
		% $f:\polinomios B x\rightarrow A$
		% respeta la suma y el producto (de $A$).
	% }
	% La imagen de esta funci\'on es igual a $\polinomios B a$.
% \end{teoPolinomios}
% 
% \begin{proof}
	% Veamos, primero, que puede existir a lo sumo una funci\'on con
	% estas propiedades. Todo elemento de $\polinomios B x$ se puede
	% expresar como $b_0+b_1x+\,\cdots\,+b_mx^m$, donde $m$ es natural o
	% $0$ y $b_i\in B$ para cada $i$. Entonces, si
	% $f:\,\polinomios B x\rightarrow A$ tiene las
	% propiedades del enunciado y $p=b_0+b_1x+\,\cdots\,+b_mx^m$ es una
	% expresi\'on polinomial en $x$ con coeficientes en $B$, $f(p)$ es
	% igual a
	% \begin{equation}
		% \label{eq:polinomios:trascendente}
		% \begin{aligned}
			% f\big(b_0+b_1x+\,\cdots\,+b_mx^m\big) & \,=\,
				% f(b_0)\,+\,f(b_1)\,f(x)\,+\,\cdots\,+\,
					% f(b_m)\,f(x)^m \\
			% & \,=\, b_0\,+\,b_1a\,+\,\cdots\,+\,b_ma^m
			% \text{ .}
		% \end{aligned}
		% %
	% \end{equation}
	% %
	% En palabras, una funci\'on
	% $f:\,\polinomios B x\rightarrow A$ que respeta la
	% suma y el producto y es la identidad en $B$ est\'a determinada
	% por su valor en $x$. Aun no hemos usado que $x$ es trascendente.
	% El hecho de que $x\in A$ sea trascendente sobre $B$ es equivalente
	% a que cada elemento de $\polinomios B x$ se puede expresar de a lo
	% sumo una \'unica manera como expresi\'on polinomial en $x$ con
	% coeficientes en $B$. Con esto y la ecuaci\'on~%
	% \eqref{eq:polinomios:trascendente} en mente, definimos la siguiente
	% funci\'on: si $p=b_0+b_1x+\,\cdots\,+b_mx^m\in\polinomios B x$,
	% \begin{displaymath}
		% % f\big(b_0+b_1x+\,\cdots\,+b_mx^m\big) \,=\,
		% f(p)\,:=\,
			% b_0\,+\,b_1a\,+\,\cdots\,+\,b_ma^m
		% \text{ .}
	% \end{displaymath}
	% %
	% La unicidad de la expresi\'on para $p$ garantiza que la
	% identidad anterior \emph{define una funci\'on}.
	% S\'olo resta verificar que esta funci\'on tiene las propiedades
	% deseadas \quedacomoejercicio.
% \end{proof}

% % \subsection{El anillo de polinomios}%
	% % \label{subsec:polinomios}
% \begin{obsPolinomios}\label{obs:polinomios:anillo}
	% Sean $A$ un anillo, $x,y\in A$ elementos y $B\subset A$ un subanillo.
	% Si $x$ e $y$ son trascendentes sobre $B$, existen, por
	% el \teoname~\ref{teo:polinomios:trascendente},
	% \'unicas funciones $f:\,\polinomios B x\rightarrow\polinomios B y$ y
	% $g:\,\polinomios B y\rightarrow\polinomios B x$ que respetan la suma
	% y el producto de $A$, $f(b)=g(b)=b$ para todo $b\in B$ y
	% $f(x)=y$ y $g(y)=x$. En particular, $fg=\id[{\polinomios B y}]$ y
	% $gf=\id[{\polinomios B x}]$. Es decir, desde un punto de vista
	% algebraico, los anillos $\polinomios B x$ y $\polinomios B y$ son
	% indistinguibles.
% \end{obsPolinomios}

\begin{teoPolinomios}\label{teo:polinomios:trascendente}
	Sean $B$ un anillo, $A,C$ anillos del cual $B$ es subanillo y
	$x\in A$ y $c\in C$ elementos.
	Si $x$ es trascendente sobre $B$, existe una \'unica funci\'on
	$f:\,\polinomios B x\rightarrow C$ tal que
	\begin{displaymath}
		f(b)\,=\,b\text{ para todo }b\in B
			\quad\text{y}\quad f(x)\,=\,c
	\end{displaymath}
	%
	y tal que 
	\begin{displaymath}
		f(u+v)\,=\,f(u)+f(v)\quad\text{y}\quad f(uv)\,=\,f(u)f(v)
		\text{ ,}
	\end{displaymath}
	%
	siempre que $u,v\in\polinomios B x$. La imagen de esta funci\'on
	es igual a $\polinomios B c$.
\end{teoPolinomios}

% \begin{proof}
	% Es an\'aloga a la del \teoname~\ref{teo:polinomios:trascendente}.
% \end{proof}

\begin{proof}
	Veamos, primero, que puede existir a lo sumo una funci\'on con
	estas propiedades. Todo elemento de $\polinomios B x$ se puede
	expresar como $b_0+b_1x+\,\cdots\,+b_mx^m$, donde $m$ es natural o
	$0$ y $b_i\in B$ para cada $i$. Entonces, si
	$f:\,\polinomios B x\rightarrow C$ tiene las
	propiedades del enunciado y $p=b_0+b_1x+\,\cdots\,+b_mx^m$ es una
	expresi\'on polinomial en $x$ con coeficientes en $B$, $f(p)$ es
	igual a
	\begin{equation}
		\label{eq:polinomios:trascendente}
		\begin{aligned}
			f\big(b_0+b_1x+\,\cdots\,+b_mx^m\big) & \,=\,
				f(b_0)\,+\,f(b_1)\,f(x)\,+\,\cdots\,+\,
					f(b_m)\,f(x)^m \\
			& \,=\, b_0\,+\,b_1c\,+\,\cdots\,+\,b_mc^m
			\text{ .}
		\end{aligned}
		%
	\end{equation}
	%
	En palabras, una funci\'on
	$f:\,\polinomios B x\rightarrow C$ que respeta la
	suma y el producto y es la identidad en $B$ est\'a determinada
	por su valor en $x$. Aun no hemos usado que $x$ es trascendente.
	El hecho de que $x\in A$ sea trascendente sobre $B$ es equivalente
	a que cada elemento de $\polinomios B x$ se puede expresar de a lo
	sumo una \'unica manera como expresi\'on polinomial en $x$ con
	coeficientes en $B$. Con esto y la ecuaci\'on~%
	\eqref{eq:polinomios:trascendente} en mente, definimos la siguiente
	funci\'on: si $p=b_0+b_1x+\,\cdots\,+b_mx^m\in\polinomios B x$,
	\begin{displaymath}
		% f\big(b_0+b_1x+\,\cdots\,+b_mx^m\big) \,=\,
		f(p)\,:=\,
			b_0\,+\,b_1c\,+\,\cdots\,+\,b_mc^m
		\text{ .}
	\end{displaymath}
	%
	La unicidad de la expresi\'on para $p$ garantiza que la
	identidad anterior \emph{define una funci\'on}.
	S\'olo resta verificar que esta funci\'on tiene las propiedades
	deseadas \quedacomoejercicio.
\end{proof}

\begin{obsPolinomios}\label{obs:polinomios:trascendente}
	Sean $B$ un anillo, $A,C$ anillos del cual $B$ es subanillo y
	$x\in A$ e $y\in C$ elementos trascendentes. Por
	% Sean $A$ y $C$ anillos, $x\in A$ e $y\in C$ elementos y
	% $B\subset A,C$ un anillo que es subanillo de $A$ y de $C$.
	% Si $x$ e $y$ son trascendentes sobre $B$, existen, por
	el \teoname~\ref{teo:polinomios:trascendente},
	existen
	\'unicas funciones $f:\,\polinomios B x\rightarrow\polinomios B y$ y
	$g:\,\polinomios B y\rightarrow\polinomios B x$ que respetan la suma
	y el producto y tales que $f(b)=g(b)=b$ para todo $b\in B$ y
	$f(x)=y$ y $g(y)=x$. En particular, $fg=\id[{\polinomios B y}]$ y
	$gf=\id[{\polinomios B x}]$. Es decir, desde un punto de vista
	algebraico, los anillos $\polinomios B x$ y $\polinomios B y$ son
	indistinguibles.
	Esta estructura com\'un es la % del anillo
	de polinomios en una indeterminada con coeficientes en $B$.
\end{obsPolinomios}

% \subsection{Polinomios con coefcientes en un anillo}%
	% \label{subsec:anillo}
% \begin{defPolinomios}\label{def:polinomios:anillo}
	% Sea $B$ un anillo. Un 
	% \emph{anillo de polinomios en una indeterminada con coeficientes %
	% en $B$} es un anillo $\tilde B$ del cual $B$ es subanillo, junto
	% con un elemento $x\in\tilde B$, que posee la propiedad siguiente:
	% \begin{quote}
		% si $C$ es un anillo del cual $B$ es subanillo y $c\in C$ es
		% un elemento arbitrario, existe una \'unica funci\'on
		% $f:\,\tilde B\rightarrow C$ tal que $f(b)=b$ para todo
		% $b\in B$, que respeta la suma y el producto y cuya imagen
		% es $\polinomios B c$.
	% \end{quote}
% \end{defPolinomios}

\begin{teoPolinomios}\label{teo:polinomios:anillo}
	Sea $B$ un anillo conmutativo con unidad $1\neq 0$. Existe un
	anillo $A$ del cual $B$ es subanillo y al cual pertenece un
	elemento $x\in A$ trascendente sobre $B$.
\end{teoPolinomios}

\begin{proof}
	Sea $S$ el conjunto de los enteros no negativos y sea $A$ el
	conjunto de funciones $S\rightarrow B$ que toman el valor
	$0\in B$ en todos los elementos de $S$ salvo, talvez, en una
	cantidad finita de ellos. Sea $X^k$ la funci\'on $l\mapsto 0$,
	si $l\neq k$ y $l\mapsto 1$, si $k=l$. Todos los elementos de $A$
	se pueden expresar como combinaci\'on de una cantidad finita de
	las funciones $X^k$ con coeficientes en $B$: si $p\in A$,
	\begin{displaymath}
		p\,=\,b_iX^i\,=\,b_0X^0+b_1X^1+b_2X^2+\,\cdots\,+b_mX^m
		\text{ ;}
	\end{displaymath}
	%
	dicha escritura es \'unica, si suponemos que los coeficientes
	son distintos de $0$.
	Definimos la suma en $A$ como la suma coordenada a coordenada:
	\begin{displaymath}
		b_iX^i\,+\,b_j'X^j\,=\,(b_k+b_k')X^k
	\end{displaymath}
	%
	y el producto como el producto de convoluci\'on:
	\begin{displaymath}
		\big(b_iX^i\big)\cdot\big(b_j'X^j\big)\,=\,
			(b_ib_j')X^{i+j}
		\text{ .}
	\end{displaymath}
	%
	Estas operaciones, juntas con la funci\'on constante cero como cero,
	la funci\'on $X^0$ como uno y $-(b_iX^i)=(-b_i)X^i$ dan estructura
	de anillo conmutativo con unidad a $A$. El anillo $B$ se puede ver
	como subanillo de $A$ identificando $b\in B$ con $bX^0$.
	El elemento $X^1$ es trascendente sobre $B$.
\end{proof}

\begin{defPolinomios}\label{def:polinomios:anillo}
	Sea $B$ un anillo conmutativo con unidad $1\neq 0$. Las
	expresiones polinomiales en un elemento trascendente,
	perteneciente a un anillo del cual $B$ es subanillo ser\'an
	\emph{polinomios en una indetermianda con coeficientes en $B$}.
\end{defPolinomios}

\begin{obsPolinomios}\label{obs:polinomios:anillo}
	El \teoname~\ref{teo:polinomios:anillo} exhibe un modelo de polinomios
	en una indeterminada con coeficientes en $B$. El \teoname~%
	\ref{teo:polinomios:trascendente} garantiza que,
	no importa cu\'al sea el anillo $A$ del cual $B$ sea subanillo, ni
	tampoco el elemento trascendente $x\in A$ elegido,
	la estructura algebraica es esencialmente la misma.
\end{obsPolinomios}

\subsection*{Ejercicios}
\theoremstyle{definition}
\newtheorem{ejerPolinomios}{\ejername}[section]

%-------------

\begin{ejerPolinomios}
	Probar que $\EnterosGauss$ y $\EnterosEisenstein$ son iguales a los
	subanillos $\Complejos$ de expresiones polinomiales en $\raizcuarta$
	y, respectivamente, $\raizcubica$ con coefcientes en $\Enteros$.
\end{ejerPolinomios}

\begin{ejerPolinomios}
	Probar que el subanillo $\polinomios[\sqrt 2]\Enteros\subset\Reales$
	de expresiones polinomiales en $\sqrt 2$ con coeficientes en
	$\Enteros$ es, como conjunto, igual al subconjunto de elementos
	de la forma $a+b\sqrt 2$ donde $a,b\in\Enteros$.
	Hacer lo an\'alogo con $\polinomios[\sqrt 3]\Enteros$.
\end{ejerPolinomios}

\begin{ejerPolinomios}
	Mostrar que no existe ninguna funci\'on
	\begin{math}
		f:\,\polinomios[\sqrt 2]\Enteros\rightarrow
			\polinomios[\sqrt 3]\Enteros
	\end{math}
	% tal que $f(1)=1$ y que \emph{respete la suma y el producto}, es decir,
	% tal que
	% \begin{displaymath}
		% f(u+v)\,=\,f(u)\,+\,f(v)\quad\text{y}\quad
			% f(uv)\,=\,f(u)\,f(v)
		% \text{ ,}
	% \end{displaymath}
	% %
	% siempre que $u,v\in\polinomios[\sqrt 2]\Enteros$.
	tal que $f(1)=1$ y que respete la suma y el producto.
\end{ejerPolinomios}

\begin{ejerPolinomios}
	Describir los elementos del anillo de expresiones polinomiales
	$\polinomios[\frac 1{2}]\Enteros$ ?`Es $\frac 1{2}$ trascendente
	sobre $\Enteros$? ?`Existe un polinomio m\'onico (coeficiente
	principal igual a $1$) $p\in\polinomios[x]\Enteros$ tal que
	$p\big(\frac 1{2}\big)=0$?
\end{ejerPolinomios}

\begin{ejerPolinomios}\label{ejer:polinomios:grado}
	Sean $p,q\in\polinomios[x]\Enteros$ los polinomios
	$p=3x^5-2x^3+x^2-5x-1$ y $q=2x^4-3x^2-x+5$. Determinar:
	\begin{enumerate}[(i)]
		\item\label{item:ejer:polinomios:grado:i}
			$\grado(p^2-q^3)$;
		\item\label{item:ejer:polinomios:grado:ii}
			el coeficiente de $x^6$ en $pq$;
		\item\label{item:ejer:polinomios:grado:iii}
			$\grado(p+q^3)$;
		\item\label{item:ejer:polinomios:grado:iv}
			el coeficiente de $x^{10}$ en $pq$;
		\item\label{item:ejer:polinomios:grado:v}
			si existe un polinomio $t\in\polinomios[x]\Enteros$
			tal que $p=tq$;
		\item\label{item:ejer:polinomios:grado:vi}
			si existen enteros $m,n$ tales que $p^n=q^m$.
	\end{enumerate}
	%
\end{ejerPolinomios}

\begin{ejerPolinomios}
	Sean $p,q\in\polinomios[x]\Enteros$ los polinomios
	$p=x^3-2x+3$ y $q=2x^5-5x^4+3x^3+2x^2$. Hallar polinomios
	$t,r\in\polinomios[x]\Enteros$ tales que $q=tp+r$ con $r=0$ o
	$\grado(r)<3$.
\end{ejerPolinomios}

\begin{ejerPolinomios}\label{ejer:polinomios:grado:bis}
	Sea $\simEnterosmod[4]$ el anillo de enteros m\'odulo $4$.
	\begin{enumerate}[(i)]
		\item\label{item:ejer:polinomios:grado:bis:i}
			Calcular los grados de $p^2$, $pq$ y $q-h$,
			donde $p,q,h\in\polinomios[x]{\simEnterosmod[4]}$ son
			los polinomios $p=1+2x$, $q=1+2x+3x^2$ y $h=2-x+x^2$.
		\item\label{item:ejer:polinomios:grado:bis:i}
			?`Existen polinomios
			$t,s\in\polinomios[x]{\simEnterosmod[4]}$, ambos
			no nulos, tales que $ts=0$?
		\item\label{item:ejer:polinomios:grado:bis:i}
			?`Existen polinomios
			$t\in\polinomios[x]{\simEnterosmod[4]}$ tales que
			$t^n=0$ para alg\'un $n\geq 0$, pero $t\neq 0$?
		\item\label{item:ejer:polinomios:grado:bis:i}
			?`Existen en $\polinomios[x]{\simEnterosmod[4]}$
			polinomios inversibles de grado mayor que $0$?
	\end{enumerate}
	%
\end{ejerPolinomios}

\begin{ejerPolinomios}
	Si $A$ es un dominio \'{\i}ntegro y $p,q\in\polinomios[x] A$ son
	polinomios distintos del polinomio nulo, entonces
	$\grado(pq)=\grado(p)+\grado(q)$. Probar que $A$ es un dominio
	\'{\i}ntegro, si y s\'olo si $\polinomios[x] A$ lo es.
\end{ejerPolinomios}

\begin{ejerPolinomios}
	Si $A$ es un dominio \'{\i}ntegro, entonces los polinomios
	inversibles en $\polinomios[x] A$ son exactamente los elementos
	inversibles en $A$, es decir, los polinomios de grado $0$ que son
	unidades de $A$.
\end{ejerPolinomios}

\begin{ejerPolinomios}
	Determinar los polinomios inversibles en
	$\polinomios[x]{\simEnterosmod[4]}$ y en
	$\polinomios[x]{\simEnterosmod[5]}$.
\end{ejerPolinomios}

\begin{ejerPolinomios}\label{ejer:polinomios:divisibilidad}
	Probar propiedades an\'alogas a las de la relaci\'on de divisibilidad
	en $\Enteros$ para la relaci\'on de divisibilidad en polinomios.
\end{ejerPolinomios}

\begin{ejerPolinomios}\label{ejer:polinomios:irreducibles}
	Si $k$ es un cuerpo, dar una definici\'on de primo (polinomios
	irreducibles) en $\polinomios[x] k$ y enunciar y demostrar algunas
	de sus propiedades.
\end{ejerPolinomios}

\begin{ejerPolinomios}\label{ejer:polinomios;mcd}
	El \emph{m\'aximo com\'un divisor} entre dos polinomios no nulos
	$f,g\in\polinomios[x] k$ con coeficientes en un cuerpo $k$ se define
	como el polinomio \emph{m\'onico} (coeficiente principal igual a $1$)
	de grado m\'aximo que divide a ambos.
	\begin{enumerate}[(i)]
		\item\label{ejer:polinomios:mcd:bezout}
			Probar la identidad de B\'ezout: que, si $h=\mcd{f,g}$
			es el m\'aximo com\'un divisor de $f$ y $g$, entonces
			existen polinomios $p$ y $q$ tales que $h=fp+gq$.
		\item\label{ejer:polinomios:mcd:equivalencias}
			Probar que las siguientes afirmaciones sobre un
			polinomio $h\in\polinomios[x] k$ son equivalentes:
			\begin{enumerate}[(a)]
				\item\label{ejer:polinomios:mcd:a}
					$h$ es \emph{el} polinomio m\'onico
					de grado m\'{\i}nimo de la forma
					$h=fp+gq$ con $p,q\in\polinomios[x] k$;
				\item\label{ejer:polinomios:mcd:b}
					$h$ es m\'onico, es divisor com\'un
					de $f$ y de $g$ y es divisible por
					cualquier otro divisor com\'un;
				\item\label{ejer:polinomios:mcd:c}
					$h$ es el m\'aximo com\'un divior
					de $f$ y $g$.
			\end{enumerate}
			%
	\end{enumerate}
	%
\end{ejerPolinomios}

\begin{ejerPolinomios}\label{ejer:polinomios:enteros}
	?`C\'omo se adapta la noci\'on de primo (ver \ejername~%
	\ref{ejer:polinomios:irreducibles}) al anillo de polinomios con
	coeficientes enteros $\polinomios[x]\Enteros$?
	?`Tiene sentido hablar de m\'aximo com\'un divisor en
	$\polinomios[x] \Enteros$? ?`Se verifica la identidad de B\'ezout?
\end{ejerPolinomios}



