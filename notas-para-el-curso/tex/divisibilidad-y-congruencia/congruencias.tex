\theoremstyle{plain}
\newtheorem{teoCongruencias}{\teoname}[section]
\newtheorem{coroCongruencias}[teoCongruencias]{\coroname}
\newtheorem{lemaCongruencias}[teoCongruencias]{\lemaname}

\theoremstyle{definition}
\newtheorem{defCongruencias}[teoCongruencias]{\defname}
\newtheorem{obsCongruencias}[teoCongruencias]{\obsname}
\newtheorem{ejemCongruencias}[teoCongruencias]{\ejemname}
\newtheorem{pregCongruencias}[teoCongruencias]{\pregname}

%-------------

% \subsection{La relaci\'on de congruencia}%
	% \label{subsec:la-relacion-de-congruencia}
\begin{ejemCongruencias}\label{ejem:congruencias:motiva}
	La ecuaci\'on $x^2-117x+31=0$ no tiene soluciones enteras.
	Deben haber varias maneras de probar esto. Una de ellas es notar
	que $31$ es primo y que, en consecuencia, si $x$ fuese soluci\'on,
	entonces $x\in\{1,\,31\}$, pero ninguno de estos n\'umeros es una
	soluci\'on ?`Qu\'e hubiese pasado con $x^2-117x+2^{136.279.841}-1$?
	?`Es $2^{136.279.841}-1$ primo? Si lo es, parece razonable que la
	ecuaci\'on no tenga soluci\'on en $\Enteros$ ?`Si no lo fuese?
	Veamos una prueba alternativa de que no tiene soluci\'on. Calculando
	algunos valores del polinomio $f(x)=x^2-117x+31$, se puede ver que
	todos los resultados que se obtienen son impares. Probemos, entonces,
	que $f(x)$ siempre es impar, si $x\in\Enteros$ (en particular, $f(x)$
	nunca ser\'a cero, en ese caso). Para que $f(x)$ sea par,
	$x^2-117x=(x-117)\,x$ deber\'{\i}a ser impar, porque $31$ es impar.
	En particular, $x$ deber\'{\i}a ser impar y $x-117$ tambi\'en. Pero,
	si $x$ es impar, $x-117$ es par. Absurdo.
\end{ejemCongruencias}

\begin{defCongruencias}\label{def:congruencias}
	Dados enteros $a$, $b$ y $m$, $m\neq 0$, se dice que
	\emph{$a$ es congruente a $b$ m\'odulo $m$}, si $m$ divide a $b-a$;
	expresamos esta condici\'on por $a\equiv b\tmodulo[m]$.
\end{defCongruencias}

\begin{ejemCongruencias}\label{ejem:congruencias}
	$-17$ y $5$ son congruentes m\'odulo $11$, pues $-17-5=-22$, que
	es m\'ultiplo de $11$. Es decir, $-17\equiv 5\tmodulo[11]$.
\end{ejemCongruencias}

\begin{ejemCongruencias}\label{ejem:congruencias:bis}
	Se cumple que $-117\equiv 1\tmodulo[2]$, $31\equiv 1\tmodulo[2]$ y
	$-117\equiv 31\tmodulo[2]$. Tambi\'en es cierto que
	$-117\not\equiv 0\tmodulo[2]$, $31\not\equiv 0\tmodulo[2]$ y que
	$1\not\equiv 0\tmodulo[2]$. Por otro lado,
	$-117 + 31\equiv 0\tmodulo[2]$ y $(-117)\,31\equiv 1\tmodulo[2]$.
\end{ejemCongruencias}

\begin{teoCongruencias}\label{teo:congruencias}
	La condici\'on $a\equiv b\tmodulo[m]$ determina, fijado $m$, una
	relaci\'on de equivalencia en $\Enteros$. Esta relaci\'on cumple,
	adem\'as, que
	\begin{displaymath}
		a\equiv c\tmodulo[m]\quad\text{y}\quad
		b\equiv d\tmodulo[m]
	\end{displaymath}
	%
	implican
	\begin{displaymath}
		ab\equiv cd\tmodulo[m] \quad\text{y, tambi\'en,}\quad
		a+b\equiv c+d\tmodulo[m]
		\text{ .}
	\end{displaymath}
	%
\end{teoCongruencias}

% \subsection{Clases de congruencia}%
	% \label{subsec:clases-de-congruencia}
\begin{defCongruencias}\label{def:congruencias:clases}
	Dado $m\in\Enteros$, $m\neq 0$, las clases de equivalencia en
	$\Enteros$ determinadas por $a\equiv b\tmodulo[m]$ se denominan
	\emph{clases de congruencia m\'odulo $m$} y, a la relaci\'on,
	\emph{relaci\'on de congruencia (m\'odulo $m$)}.
\end{defCongruencias}

\begin{ejemCongruencias}\label{ejem:congruencias:clases}
	Las clases de congruencia m\'odulo $2$ separan a los enteros entre
	pares, $a\equiv 0\tmodulo[2]$, e impares, $a\equiv 1\tmodulo[2]$.
	Los enteros $0$ y $1$ representan cada una de las clases de
	congruencia m\'odulo $2$; los enteros $2$ y $3$, tambi\'en
	representan ambas clases de congruencia. Mientras que $0$ y $3$
	representan clases distintas (y, por lo tanto, disjuntas),
	$0$ y $2$ representan la misma clase.
\end{ejemCongruencias}

\begin{ejemCongruencias}\label{ejem:congruencias:clases:bis}
	Las clases de congruencia m\'odulo $3$ separan a los enteros entre
	aquellos cuyo resto de dividir por $3$ es $0$, $1$ o bien $2$:
	si $a\equiv b\tmodulo[3]$, entonces $3$ divide a la diferencia
	$a-b$; si $a=3k+r$ y $b=3l+s$, entonces $a-b=3\,(k-l)+(r-s)$, o sea,
	$3$ divide a $a-b$, si y s\'olo si $3$ divide a $r-s$. Entonces,
	las clases est\'an representandas por los enteros
	$0$, $1$ y $2$, que adem\'as representan clases distintas;
	tambi\'en est\'an representadas por $-1$, $0$ y $1$, es decir, todo
	entero es congruente m\'odulo $3$ a alguno de ellos (y, en este caso
	tambi\'en, a exactamente uno de ellos).
\end{ejemCongruencias}

\begin{defCongruencias}\label{def:congruencias:sistema-completo}
	Un \emph{sistema de representantes m\'odulo $m$}
	(o, tambi\'en, \emph{sistema completo de representantes}) es un
	subconjunto $R\subset\Enteros$ tal que todo entero sea congruente
	a un \'unico elemento de $R$, es decir,
	\begin{enumerate}[(i)]
		\item\label{item:def:sistema-completo:existe}
			si $x\in\Enteros$, existe $r\in R$ tal que
			$x\equiv r\tmodulo[m]$ y
		\item\label{item:def:sistema-completo:unico}
			si $r,r'\in R$ y $r\neq r'$, entonces
			$r\not\equiv r'\tmodulo[m]$.
	\end{enumerate}
	%
\end{defCongruencias}

\begin{ejemCongruencias}\label{ejem:congruencias:sistema-completo}
	Con $m=7$, un sistema completo de representantes es el de restos
	de la divisi\'on por $7$: $\{0,1,2,3,4,5,6\}$. Otro sistema completo
	de representantes es $\{0,1,2,3,-3,-2,-1\}$.
	Incluso otro puede ser $\{-28,50,16,-4,18,-30,-8\}$. Notemos que
	en este \'ultimo ejemplo, todos los enteros del sistema de
	representantes son pares.%
	\footnote{
		Ver \ejername~%
		\ref{ejer:congruencias:chino:representantes}.
	}
\end{ejemCongruencias}

\begin{teoCongruencias}\label{teo:congruencias:sistema-completo}
	Dado $m\in\Enteros$, $m\neq 0$, la relaci\'on de congruencia
	m\'odulo $m$ divide a $\Enteros$ en $|m|$ clases. Un sistema de
	representantes est\'a dado por los $|m|$ restos de la divisi\'on
	por $m$. En particular, todos los sistemas completos de
	representantes tienen el mismo cardinal, $|m|$.
\end{teoCongruencias}

\begin{proof}
	$a\equiv b\tmodulo[m]$, si y s\'olo si $\resto[m](a)=\resto[m](b)$.
\end{proof}

\begin{ejemCongruencias}\label{ejem:congruencias:motiva:bis}
	Si $x\in\Enteros$,
	\begin{displaymath}
		x^2-117x+31\,\equiv\,x^2+x+1\modulo[2]
		\text{ ,}
	\end{displaymath}
	%
	que es impar, es decir, congruente con $1$ m\'odulo $2$, tanto si
	$x$ es par ($x\equiv 0\tmodulo[2]$), como si $x$ es impar
	($x\equiv 1\tmodulo[2]$). Los polinomios $3x^2+3x+1$ y
	$3x^3+x^2+3x+4$ tampoco tienen ra\'{\i}ces enteras. En el primer
	caso, para todo $x$,
	\begin{displaymath}
		3x^2+3x+1\,\equiv\,1\modulo[3]
		\text{ .}
	\end{displaymath}
	%
	En el segundo,
	\begin{displaymath}
		3x^3+x^2+3x+4\,\equiv\,x^2+1\modulo[3]
		\text{ .}
	\end{displaymath}
	%
	Si $x\equiv 0\tmodulo[3]$, entonces $x^2+1\equiv 1\tmodulo[3]$.
	Si, en cambio, $x\equiv 1\tmodulo[3]$, entonces
	$x^2+1\equiv 2\tmodulo[3]$. Y, si $x\equiv 2\tmodulo[3]$, entonces
	$x^2+1\equiv 5\equiv 2\tmodulo[3]$. Con lo que el resultado siempre
	es congruente a $1$ o a $2$ m\'odulo $3$; nunca es congruente a $0$.
\end{ejemCongruencias}

% \subsection{Soluciones a una ecuaci\'on de congruencia}%
	% \label{subsec:soluciones-a-una-congruencia}
% Si $f$ es un polinomio con coeficientes enteros y $a$ y $b$ son enteros
% congruentes m\'odulo $m$, entonces $f(a)$ y $f(b)$ son congruentes
% m\'odulo $m$.%
% \footnote{
	% Ver \ejername~\ref{ejer:congruencias:}
% }
% Esto dice, en particular, que el resto de dividir al valor de $f(x)$ en un
% entero $x\in\Enteros$ por $m$ s\'olo depende del resto de dividir $x$ por $m$.
% En particular, $f(x)$ y $f(\resto[m](x))$ tienen igual resto de dividir por
% $m$. El siguiente diagrama resume esto:
% \begin{displaymath}
	% \begin{tikzcd}
		% \Enteros \arrow[r] \arrow[d] & \Enteros \arrow[d] \\
		% % \bigg\{\begin{array}{c} \text{clases} \\
		% % \text{m\'odulo } m\end{array}\bigg\} \arrow[r] &
		% \big\{0,\,1,\,\dots,\,m-1\big\} \arrow[r] &
		% % \bigg\{\begin{array}{c} \text{clases} \\
		% % \text{m\'odulo } m\end{array}\bigg\}
		% \big\{0,\,1,\,\dots,\,m-1\big\}
	% \end{tikzcd}
	% \quad
	% \begin{tikzcd}
		% x \arrow[r,mapsto] \arrow[d,mapsto] & f(x) \arrow[d,mapsto] \\
		% \resto[m](x) \arrow[r,mapsto] &
			% \boxed{\resto[m](f(\resto[m](x)))=\resto[m](f(x))}
	% \end{tikzcd}
	% \dispstop
% \end{displaymath}
% %
% Si pensamos en t\'erminos de clases de congruencia m\'odulo $m$, esto
% significa que, a cada clase de congruencia m\'odulo $m$ le podemos asignar
% un valor bien definido y que tiene que ver con $f$.
% Si una clase est\'a representada por un entero $x$,
% este ``valor'' es la clase de congruencia m\'odulo $m$ del valor de $f(x)$.
% Que este valor est\'e bien definido quiere decir que el ``valor m\'odulo $m$''
% es el mismo para todo entero en la clase de $x$.
% Resumimos esto en el siguiente diagrama:
% \begin{displaymath}
	% \begin{tikzcd}
		% \Enteros \arrow[r] \arrow[d] & \Enteros \arrow[d] \\
		% \bigg\{\begin{array}{c} \text{clases} \\
		% \text{m\'odulo } m\end{array}\bigg\} \arrow[r, dashed] &
		% \bigg\{\begin{array}{c} \text{clases} \\
		% \text{m\'odulo } m\end{array}\bigg\}
	% \end{tikzcd}
	% \quad
	% \begin{tikzcd}
		% x \arrow[r,mapsto] \arrow[d,mapsto] & f(x) \arrow[d,mapsto] \\
		% \big(\text{clase de }x\big) \arrow[r,mapsto] &
			% \big(\text{clase de }f(x)\big)
			% % \boxed{\resto[m](f(\resto[m](x)))=\resto[m](f(x))}
	% \end{tikzcd}
	% \dispstop
% \end{displaymath}
% %

\begin{defCongruencias}\label{def:congruencias:soluciones}
	La \emph{cantidad de soluciones m\'odulo $m$} a una ecuaci\'on es la
	cantidad de clases de congruencia m\'odulo $m$ que hacen que la
	ecuaci\'on se verifique m\'odulo $m$.
\end{defCongruencias}

\begin{ejemCongruencias}\label{ejem:congruencias:soluciones}
	La ecuaci\'on $f(x):=x^2-117x+31=0$ no tiene soluciones m\'odulo $2$;
	tampoco tiene soluciones m\'odulo $3$. Pero tiene una soluci\'on
	m\'odulo $5$: $f(1)=-115\equiv 0\tmodulo[5]$. De hecho,
	\begin{displaymath}
		x^2-117x+31\,\equiv\,x^2-2x+1\tmodulo[5]
		\text{ ,}
	\end{displaymath}
	%
	de donde se puede verificar que $x^2-2x+1\equiv 0\tmodulo[5]$,
	s\'olo si $x\equiv 1\tmodulo[5]$.%
	\footnote{
		$x^2-2x+1=(x-1)^2$.
	}
	Esto quiere decir que hay una \'unica soluci\'on m\'odulo $5$.
\end{ejemCongruencias}

% \subsection{Sistemas reducidos de representantes}%
	% \label{subsec:sistemas-reducidos}
\begin{defCongruencias}\label{def:congruencias:sistema-reducido}
	Un \emph{sistema reducido de representantes m\'odulo $m$} es un
	subconjunto $R\subset\Enteros$ cuyos elementos son coprimos con $m$
	y tal que todo entero \emph{coprimo con $m$} sea congruente a un
	\'unico elemento de $R$, es decir,
	\begin{enumerate}[(i)]
		\item\label{item:def:sistema-reducido:coprimo}
			si $r\in R$, entonces $\mcd{r,m}=1$,
		\item\label{item:def:sistema-reducido:existe}
			si $x\in\Enteros$ y $\mcd{x,m}=1$, existe $r\in R$
			tal que $x\equiv r\tmodulo[m]$ y
		\item\label{item:def:sistema-reducido:unico}
			si $r,r'\in R$ y $r\neq r'$, entonces
			$r\not\equiv r'\tmodulo[m]$.
	\end{enumerate}
	%
\end{defCongruencias}

\begin{ejemCongruencias}\label{ejem:congruencias:sistema-reducido}
	En el \ejemname~\ref{ejem:congruencias:sistema-completo} vimos que
	$\{0,1,2,3,4,5,6\}$, $\{0,1,2,3,-3,-2,-1\}$ y
	$\{-28,50,16,-4,18,-30,-8\}$ son sistemas completos de representantes
	de las clases m\'odulo $7$. A partir de ellos, podemos conseguir
	sistemas reducidos de representantes de las clases quitando aquellos
	elementos divisibles por $7$: los conjuntos
	$\{1,2,3,4,5,6\}$, $\{1,2,3,-3,-2,-1\}$ y $\{50,16,-4,18,-30,-8\}$
	son sistemas reducidos de representantes m\'odulo $7$.
\end{ejemCongruencias}

\begin{ejemCongruencias}\label{ejem:congruencias:sistema-reducido:bis}
	Con $m=21$, un sistema completo de representantes es el de los
	restos:
	\begin{displaymath}
		\{0,1,2,3,4,5,6,7,8,9,10,11,12,13,14,15,16,17,18,19,20\}
		\dispstop
	\end{displaymath}
	%
	Un sistema reducido de representantes se obtiene quitando aquellos
	enteros que no son coprimos con $21$, es decir, divisibles por $3$
	o por $7$:
	\begin{displaymath}
		\{1,2,4,5,8,10,11,13,16,17,19,20\}
		\dispstop
	\end{displaymath}
	%
	Otro sistema reducido de representantes es:
	\begin{displaymath}
		\{1,2,4,5,8,10,-10,-8,-5,-4,-2,-1\}
		\dispstop
	\end{displaymath}
	%
	El siguiente conjunto tambi\'en es un sistema reducido de
	representantes y cumple que todos sus elementos son pares:
	\begin{displaymath}
		\{22,2,4,26,8,10,-10,-8,16,-4,-2,20\}
		\dispstop
	\end{displaymath}
	%
	Tambi\'en el conjunto
	\begin{displaymath}
		\{1,-19,46,26,-34,31,11,-29,16,-4,61,41\}
	\end{displaymath}
	%
	es un sistema reducido de representantes de las clases m\'odulo $21$
	y cumple que todos sus elementos son $\equiv 1\tmodulo[5]$.%
	\footnote{
		Ver \ejername~\ref{ejer:congruencias:chino:representantes}.
	}
\end{ejemCongruencias}

\begin{teoCongruencias}\label{teo:congruencias:sistema-reducido}
	Todos los sistemas reducidos de representantes tienen el mismo
	cardinal.
\end{teoCongruencias}

\begin{proof}
	Este resultado es consecuencia de las dos propiedades siguientes
	\quedacomoejercicio:
	\begin{itemize}
		\item si $b\equiv c\tmodulo[m]$, entonces
			$\mcd{b,m}=\mcd{c,m}$ y
		\item todo sistema reducido se puede completar a un sistema
			completo de representantes.
	\end{itemize}
	%
\end{proof}

\begin{defCongruencias}\label{def:congruencias:sistema-reducido:phi}
	El cardinal de un sistema reducido de representantes se denota por
	$\eulerphi(m)$; la funci\'on
	$m\in\Enteros\setmin\{0\}\mapsto\eulerphi(m)\in\Naturales$ se llama
	\emph{funci\'on de Euler}.
\end{defCongruencias}

\begin{coroCongruencias}\label{coro:congruencias:sistema-reducido}
	$\eulerphi(m)=\cardinal{\{1\leq t\leq m\,:\,\mcd{t,m}=1\}}$.
\end{coroCongruencias}

\begin{ejemCongruencias}\label{ejem:congruencias:phi}
	Seg\'un lo visto en el \ejemname~%
	\ref{ejem:congruencias:sistema-reducido}, $\eulerphi(7)=6$.
\end{ejemCongruencias}

\begin{ejemCongruencias}\label{ejem:congruencias:phi:bis}
	Seg\'un lo visto en el \ejemname~%
	\ref{ejem:congruencias:sistema-reducido:bis}, $\eulerphi(21)=12$.
\end{ejemCongruencias}

% \subsection{El Fermatito}%
	% \label{subsec:el-fermatito}
\begin{teoCongruencias}\label{teo:congruencias:fermatito}
	Si $\mcd{a,m}=1$, entonces $a^{\eulerphi(m)}\equiv 1\tmodulo[m]$.
\end{teoCongruencias}

\begin{proof}
	Sea $R=\{\lista r{\eulerphi(m)}\}$ un sistema reducido m\'odulo $m$.
	Entonces, si $\mcd{a,m}=1$, el conjunto
	$\{ar_1,\,\dots,\,ar_{\eulerphi(m)}\}$ \emph{tambi\'en} es un sistema
	reducido (*). As\'{\i}, para cada $i$, $1\leq i\leq\eulerphi(m)$,
	existe un \'unico $j=j(i)$, $1\leq j\leq\eulerphi(m)$, tal que
	\begin{math}
		r_i\equiv ar_j\tmodulo[m]
	\end{math}.
	Adem\'as, si $i\neq i'$, entonces $j(i)\neq j(i')$ (*).
	En consecuencia,
	\begin{equation}
		\label{eq:congruencias:fermatito:i}
		\prod_{j=1}^{\eulerphi(m)}\,ar_j\,\equiv\,
			\prod_{i=1}^{\eulerphi(m)}\,r_i\modulo[m]
		\text{ .}
	\end{equation}
	%
	Pero
	\begin{equation}
		\label{eq:congruencias:fermatito:ii}
		\prod_{j=1}^{\eulerphi(m)}\,ar_j\,=\,
			a^{\eulerphi(m)}\,\prod_{j=1}^{\eulerphi(m)}\,r_j
		\text{ .}
	\end{equation}
	%
	De \eqref{eq:congruencias:fermatito:i} y de
	\eqref{eq:congruencias:fermatito:ii} se deduce que
	$a^{\eulerphi(m)}\equiv 1\tmodulo[m]$ (*).
\end{proof}

En la demostraci\'on del \teoname~\ref{teo:congruencias:fermatito}, las
afirmaciones marcadas con (*) son consecuencias del siguiente resultado.

\begin{lemaCongruencias}\label{lema:congruencias:cancelable}
	Si $\mcd{b,m}=1$ y $bx\equiv by\tmodulo[m]$, entonces
	$x\equiv y\tmodulo[m]$.
\end{lemaCongruencias}

\begin{proof}
	\quedacomoejercicio.
\end{proof}

% \subsection*{\appendixname: El Teorema chino del resto}
% % \begin{obsCongruencias}[Teorema chino del resto]\label{obs:congruencias:phi}
	% Un sistema completo para las clases m\'odulo $21$ es, como hemos
	% mencionado, el conjunto de restos
	% $\cal R_{21}=\{0,1,2,\,\dots,20\}$
	% de dividir por $21$.
	% Si a los enteros de este conjunto les tomamos los restos de dividir
	% por $7$, obtenemos el conjunto $\cal R_7=\{0,1,2,3,4,5,6\}$, que es el
	% conjunto de todos los restos posibles de dividir por $7$ y, por lo
	% tanto, un sistema conmpleto de representantes de las clases m\'odulo
	% $7$. Lo mismo sucede, si tomamos resto de dividir por $3$: obtenemos
	% $\cal R_3=\{0,1,2\}$.
	% Notemos que
	% \begin{equation}
		% \label{eq:congruencias:chino:cardinales}
		% \indice{\cal R_{21}}\,=\,\indice{\cal R_7}\,\indice{\cal R_3}
		% \dispstop
	% \end{equation}
	% %
	% % Notemos, tambi\'en, que podemos describir el conjunto de restos
	% % $\cal R_{21}$ de las dos siguientes maneras:
	% % \begin{displaymath}
		% % \cal R_{21}\,=\,
			% % \big\{ r+7\,s\,:\,r\in\cal R_7,\,s\in\cal R_3\big\}
			% % \,=\,
			% % \big\{ s+3\,r\,:\,r\in\cal R_7,\,s\in\cal R_3\big\}
		% % \dispstop
	% % \end{displaymath}
	% %
	% Si $a\in\Enteros$ y conocemos su clase m\'odulo $21$, podemos decir
	% cu\'al es su clase m\'odulo $3$ y m\'odulo $7$, tambi\'en: si
	% $a=21k+r$ ($k,r\in\Enteros$), entonces $a\equiv r\tmodulo[3]$ y
	% $a\equiv r\tmodulo[7]$, tambi\'en, pues $a-r=3\cdot 7k=7\cdot 3k$.
	% El siguiente diagrama ilustra este fen\'omeno:
	% \begin{equation}
		% \label{eq:congruencias:chino}
		% \begin{tikzcd}% [column sep=small, row sep=small]
			% & \cal R_3 \\
			% \cal R_{21} \arrow[ur] \arrow[dr] \arrow[r,dashed] &
				% \cal R_3\times\cal R_7 \arrow[u] \arrow[d] \\
			% & \cal R_7
		% \end{tikzcd}
		% \dispcomma
		% \quad
		% a\,\mapsto\,\big(\resto[3](a),\resto[7](a)\big)
		% \dispstop
	% \end{equation}
	% %
	% Rec\'{\i}procamente, conociendo las clases de $a$ m\'odulo $3$ y
	% m\'odulo $7$, podemos determinar su clase m\'odulo $21$: si
	% $a\equiv u\tmodulo[3]$ y $a\equiv v\tmodulo[7]$, entonces
	% $u=a-3k$ y $v=a-7l$ ($k,l\in\Enteros$) y
	% \begin{displaymath}
		% 7u+15v\,=\,7\,(a-3k)\,+\,15\,(a-7l)\,=\,22a\,-\,21k\,-\,105\,l
		% \,\equiv\,a\modulo[21]
		% \dispstop
	% \end{displaymath}
	% %
	% O sea, si $a$ es congruente a $u$ m\'odulo $3$ y a $v$ m\'odulo $7$,
	% entonces $a$ es congruente a $7+15v$ m\'odulo $21$. 
	% Esto dice que la flecha punteda de \eqref{eq:congruencias:chino}
	% tiene una flecha en la direcci\'on opuesta que es su inversa:
	% \begin{displaymath}
		% \cal R_3\times\cal R_7\,\rightarrow\,\cal R_{21}
		% \dispcomma\quad (u,v)\,\mapsto\,\resto[21](7u+15v)
		% \dispstop
	% \end{displaymath}
	% %
	% En conclusi\'on, el conjunto de restos $\cal R_{21}$ est\'a en
	% biyecci\'on con el producto cartesiano $\cal R_3\times\cal R_7$.
	% Esto justifica la observaci\'on acerca de los cardinales de los
	% conjuntos de restos \eqref{eq:congruencias:chino:cardinales}.
	% Notamos, adem\'as, que, si $u$ es coprimo con $3$, entonces
	% $7u+15v$ tambi\'en lo es, y que, si $v$ es coprimo con $7$, entonces
	% $7u+15v$ tambi\'en lo es. En particular, si
	% $\tilde{\cal R}_{21}\subset\cal R_{21}$ es el subconjunto
	% de restos coprimos con $21$ y, an\'alogamente,
	% $\tilde{\cal R}_3=\{1,2\}$ y $\tilde{\cal R}_7=\{1,2,3,4,5,6\}$,
	% entonces la biyecci\'on $\cal R_3\times\cal R_7\simeq\cal R_{21}$
	% se restringe a una biyecci\'on
	% \begin{displaymath}
		% \tilde{\cal R}_{21}\,\simeq\,
			% \tilde{\cal R}_3\times\tilde{\cal R}_7
		% \dispstop
	% \end{displaymath}
	% %
	% Esto se corresponde con el hecho de que
	% \begin{displaymath}
		% \eulerphi(21)\,=\,\eulerphi(3)\,\eulerphi(7)
		% \dispstop
	% \end{displaymath}
	% %
% % \end{obsCongruencias}
% \begin{pregCongruencias}\label{preg:congruencias:chino}
	% Si s\'olo conocemos el resto de $a$ al dividirlo por $21$, podemos
	% decir cu\'al es el resto de $a$ al dividirlo por $3$ o por $7$
	% ?`Podemos hacder lo mismo con el resto de $a$ al dividirlo por,
	% por ejemplo, $2$ (o por cualquier otro entero que no divida a $21$)?
% \end{pregCongruencias}

\subsection*{Ejercicios}
\theoremstyle{definition}
\newtheorem{ejerCongruencias}{\ejername}[section]

%-------------

\begin{ejerCongruencias}
	Si $f\in\polinomios\Enteros{X}$ y $f(0)\equiv f(1)\equiv 1\tmodulo[2]$,
	entonces $f$ no tiene ra\'{\i}ces enteras. Generalizar.
\end{ejerCongruencias}

\begin{ejerCongruencias}
	Probar que la ecuaci\'on $3x^2+2=y^2$ no tiene soluciones con
	$x,y\in\Enteros$.
\end{ejerCongruencias}

\begin{ejerCongruencias}
	Probar que la ecuaci\'on $7x^3+2=y^3$ no tiene soluciones con
	$x,y\in\Enteros$.
\end{ejerCongruencias}

\begin{ejerCongruencias}
	Probar que la ecuaci\'on $x^2-117x+31=0$ no tiene soluciones m\'odulo
	$117$. Investigar la ecuaci\'on m\'odulo $m$ para valores de $m$
	hasta \dots%
	%$400$, por ejemplo
	; confeccionar una tabla. Mirar m\'odulo $m\leq 30$, al menos,
	m\'odulo $85$, $115$, $391$ y $2713$.
\end{ejerCongruencias}

\begin{ejerCongruencias}
	Si $f\in\polinomios\Enteros{X}$ y $a\equiv b\tmodulo[m]$, entonces
	$f(a)\equiv f(b)\tmodulo[m]$.
\end{ejerCongruencias}

\begin{ejerCongruencias}
	Si $d\mid m$ y $a\equiv b\tmodulo[m]$, entonces $a\equiv b\tmodulo[d]$.
\end{ejerCongruencias}

\begin{ejerCongruencias}
	Sea $g=\mcd{a,m}$. La congruencia $ax\equiv b\tmodulo[m]$ tiene
	soluci\'on $x\in\Enteros$, si y s\'olo si $g\mid b$. En tal caso,
	hay exactamente $g$ soluciones (m\'odulo $m$):
	si $x$ e $y$ son soluciones, entonces $x\equiv y\tmodulo[m/g]$.
\end{ejerCongruencias}

\begin{ejerCongruencias}
	Dados $a,m\in\Enteros$, $m\neq 0$, se cumple $ax\equiv ay\tmodulo[m]$,
	si y s\'olo si $x\equiv y\tmodulo[\tfrac m{\mcd{a,m}}]$.
\end{ejerCongruencias}

\begin{ejerCongruencias}
	Si $d\mid m$ y si $a\equiv b\tmodulo[d]$, entonces
	$a\equiv b+kd\tmodulo[m]$, para un \'unico $k$ en el rango
	$1\leq k\leq m/d$.
	% (existe $k$ y es \'unico, en el rango).
\end{ejerCongruencias}

\begin{ejerCongruencias}
	Si $p$ es primo, $\mcd{a,p}=1$ equivale a $a\not\equiv 0\tmodulo[p]$.
\end{ejerCongruencias}

\begin{ejerCongruencias}
	?`Existen $x\in\Enteros$ tales que $x^2\equiv -1\tmodulo[17]$?
	?`$x^3\equiv 2\tmodulo[17]$? Hallar un sistema reducido m\'odulo $17$
	cuyos elementos sean m\'ultiplos de $3$.
\end{ejerCongruencias}

\begin{ejerCongruencias}
	Determinar $\eulerphi(26)$. Para cada $x$ en el rango
	$0\leq x\leq 25$ (un sistema completo), determinar el menor
	natural $k$ ($k\geq 1$) tal que $x^k\equiv 1\tmodulo[26]$.
	Repetir con $36$ en lugar de $26$ (y $35$ en lugar de $25$)
	?`Qu\'e relaci\'on hay entre estos exponentes y los valores de
	$\eulerphi(26)$ y de $\eulerphi(36)$, respectivamente?
\end{ejerCongruencias}

\begin{ejerCongruencias}
	Probar que $19\nmid 4n^2+n$ para ning\'un $n$ natural.
\end{ejerCongruencias}

% \begin{ejerCongruencias}[Teorema chino del resto]%
	% \label{ejer:congruencias:chino}
	% Este ejercicio trata la resolubilidad de ciertos sistemas de
	% ecuaciones de congruencia.
	% \begin{enumerate}[(i)]
		% \item\label{item:ejer:chino:i}
			% Sean $a_1,a_2,m_1,m_2\in\Enteros$, $m_1m_2\neq 0$.
			% Las ecuaciones de congruencia
			% \begin{displaymath}
				% \begin{aligned}
					% x & \,\equiv\,a_1\modulo[m_1]
						% \quad\text{y} \\
					% x & \,\equiv\,a_2\modulo[m_2]
				% \end{aligned}
				% %
			% \end{displaymath}
			% %
			% admiten una soluci\'on com\'un, si y s\'olo si
			% $\mcd{m_1,m_2}\mid a_1-a_2$. En tal caso, si $m$
			% denota el menor m\'ultiplo com\'un positivo de $m_1$
			% y $m_2$, entonces las soluciones pertenecen a una
			% misma clase m\'odulo $m$.
		% \item\label{item:ejer:chino:ii}
			% Si $\lista m{r}\in\Enteros$ son distintos de cero y
			% coprimos de a pares, entonces las ecuaciones
			% \begin{displaymath}
				% x\,\equiv\,a_i\modulo[m_i]
				% \text{ ,}
			% \end{displaymath}
			% %
			% $1\leq i\leq r$, son consistentes y las soluciones en
			% com\'un pertenecen a una misma clase m\'odulo el
			% producto $m_1\cdots m_r$.
		% \item\label{item:ejer:chino:iii}
			% Dado $f\in\polinomios\Enteros{X}$ y dados enteros
			% $\lista m{r}\neq 0$ coprimos de a pares, la cantidad
			% de soluciones a la ecuaci\'on
			% \begin{displaymath}
				% f(x)\,\equiv\,0\modulo[m_1\cdots m_r]
			% \end{displaymath}
			% %
			% es igual al producto de las cantidades de soluciones
			% a las ecuaciones
			% \begin{displaymath}
				% f(x)\,\equiv\,0\modulo[m_i]
				% \text{ ,}
			% \end{displaymath}
			% %
			% para cada $i$.
	% \end{enumerate}
	% %
% \end{ejerCongruencias}

\begin{ejerCongruencias}\label{ejer:congruencias:chino:representantes}
	Hallar sistemas de representantes de las clases m\'odulo $26$
	cuyos elementos sean:
	\begin{enumerate}[(i)]
		\item\label{item:ejer:chino:representantes:i}
			todos divisibles por $7$;
		\item\label{item:ejer:chino:representantes:ii}
			congruentes con $1$ m\'odulo $3$ y congruentes con $1$
			m\'odulo $5$
	\end{enumerate}
	%
	?`Es posible hallar un sistema de representantes que cumpla con las
	condiciones \eqref{item:ejer:chino:representantes:i} y
	\eqref{item:ejer:chino:representantes:ii}?
\end{ejerCongruencias}

\begin{ejerCongruencias}
	Sean $a$ y $b$ enteros positivos coprimos y sean $A$ y $B$ sistemas
	completos de representantes m\'odulo $a$ y m\'odulo $b$,
	respectivamente. Probar que el conjunto
	\begin{displaymath}
		R\,:=\,\big\{ax+by\,:\,x\in B,\,y\in A\big\}
	\end{displaymath}
	%
	es un sistema completo de representates de las clases m\'odulo $ab$.
	Probar que, si $A$ y $B$ son, en cambio, sistemas reducidos,
	entonces el conjunto $R$ an\'alogo es un sistema reducido de
	representantes m\'odulo $ab$.
\end{ejerCongruencias}


