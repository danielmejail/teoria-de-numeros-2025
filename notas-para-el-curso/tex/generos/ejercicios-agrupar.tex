\theoremstyle{definition}
\newtheorem{ejerAgrupar}{\ejername}[section]

%-------------

\begin{ejerAgrupar}\label{ejer:agrupar:uno}
	Probar que, para cada discriminante $D$, s\'olo las formas de
	una \'unica clase de equivalencia propia representan $1$ (el
	n\'umero entero).
\end{ejerAgrupar}

\begin{ejerAgrupar}\label{ejer:agrupar}
	Para cada discriminante
	\begin{displaymath}
		D\,\in\,\{-3,-15,-19,-25,5,13\}\,\cup\,
			\{-4,-8,-16,-24,-28,-56,8,12\}
	\end{displaymath}
	%
	\begin{itemize}
		\item calcular $\eulerphi(D)$ y hallar las clases en
			$\Unidadesmod[D]$;
		\item hallar las clases de equivalencia propia de formas
			gaussianas de discriminante $D$ y agruparlas por
			g\'eneros.
	\end{itemize}
	%
\end{ejerAgrupar}

