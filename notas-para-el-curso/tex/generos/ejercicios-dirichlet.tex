\theoremstyle{definition}
\newtheorem{ejerDirichlet}{\ejername}[section]

%-------------

\begin{ejerDirichlet}\label{ejer:dirichlet:sobreyectividad}
	Sea $D\equiv 0,1\tmodulo[4]$. Probar que existe $m\in\Enteros$
	tal que $\varkronecker[D](m)=-1$, es decir, la morfismo de grupos
	$\varkronecker[D]:\,\Unidadesmod[D]\rightarrow\cuadratico$
	es sobreyectivo.
\end{ejerDirichlet}

\begin{ejerDirichlet}\label{ejer:dirichlet:primos-representables}
	Sea $D\equiv 0,1\tmodulo[4]$ y sea $c\in\Unidadesmod[D]$.
	Probar que, si $p,q\in c$ son primos positivos impares,
	entonces $p$ es representable por alguna forma de discriminante
	$D$, si y s\'olo si $q$ es representable por alguna forma de
	discriminante $D$.%
	\hint{
		Recordar que, para primos (positivos, impares, que no
		dividen al discriminante $D$), ser representable equivale
		a $\varkronecker[D](p)=1$.
	}
\end{ejerDirichlet}

\begin{ejerDirichlet}\label{ejer:dirichlet:generos}
	Para $D\in\{-24,-31,-52\}$, describir los g\'eneros de formas
	cuadr\'aticas de discriminante $D$.
	Para cada valor de $D$, hallar
	\begin{itemize}
		\item el subconjunto $\Unidadesmod[D]\subset\Enterosmod[D]$,
		\item el subgrupo $\ker(\chi)\subgrp\Unidadesmod[D]$,
		\item las formas reducidas,
			% (\ejername~\ref{ejer:reducidas:varios}),
		\item las clases de congruencia m\'odulo $D$ representadas
			por la \emph{forma principal}, es decir, por
			$\binaria{1,0,-D/4}$, si $D\equiv 0\tmodulo[4]$,
			y por $\binaria{1,1,(1-D)/4}$, si
			$D\equiv 1\tmodulo[4]$
	\end{itemize}
	%
	y agrupar
	\begin{itemize}
		\item las formas reducidas por g\'enero y
		\item las clases de congruencia m\'odulo $D$ de acuerdo
			con el g\'enero que las representa.
	\end{itemize}
	%
	Verificar, en cada caso, que
	\begin{displaymath}
		H\,=\,\big\{c\in\Unidadesmod[D]\,:\,
			c\text{ es representada por la forma principal}
			\big\}
	\end{displaymath}
	%
	es un subgrupo de $\ker(\chi)$ (de $\Unidadesmod[D]$).
\end{ejerDirichlet}

