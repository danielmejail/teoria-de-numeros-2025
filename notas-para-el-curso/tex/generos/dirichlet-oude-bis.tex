\theoremstyle{plain}
\newtheorem{teoDirichlet}{\teoname}[section]
\newtheorem{coroDirichlet}[teoDirichlet]{\coroname}

\theoremstyle{definition}
\newtheorem{defDirichlet}[teoDirichlet]{\defname}
\newtheorem{ejemDirichlet}[teoDirichlet]{\ejemname}

%-------------

En esta secci\'on hacemos una disgresi\'on por el
Teorema de Dirichlet sobre primos en progresiones aritm\'eticas y
lo relacionaremos con el estudio de la representabilidad por formas
cuadr\'aticas. Esto nos ser\'a \'util en la \S~\ref{sec:generos},
cuando hablemos de \emph{g\'enero} de una forma cuadr\'atica.

\begin{teoDirichlet}[Dirichlet]\label{teo:dirichlet}
	Dados n\'umeros enteros coprimos $m$ y $D$, la sucesi\'on
	\begin{displaymath}
		m\dispcomma\quad
		m+D\dispcomma\quad
		m+2 D\dispcomma\quad\dots
	\end{displaymath}
	%
	contiene infinitos n\'umeros primos.
\end{teoDirichlet}

Una consecuencia del \teoname~\ref{teo:dirichlet} es que toda clase
de congruencia m\'odulo $D$ contiene alg\'un n\'umero primo.
Veamos c\'omo se relaciona esto con el problema de representabilidad
por formas cuadr\'aticas.

Por el \lemaname~\ref{lema:representaciones:primitiva}, fijado un entero
$m$ (impar), podemos decidir f\'acilmente qu\'e discriminantes (coprimos
con $m$) lo representan:
son aquellos que pertenecen al subgrupo de cuadrados,
\begin{displaymath}
	\big\{ x^2\,:\,x\in\Unidadesmod[m] \big\}\,\leq\,\Unidadesmod[m]
	\dispstop
\end{displaymath}
%
Si $m=p$ es primo (impar), entonces
\begin{displaymath}
	\big\{ x^2\,:\,x\in\Unidadesmod[p] \big\}\,=\,
		\big\{ y\in\Unidadesmod[p]\,:\,\tlegendre y p=1\big\}
	\dispstop
\end{displaymath}
%
\emph{Rec\'{\i}procamente}, fijado un discriminante $D$,
?`existen condiciones sobre las clases de congruencia m\'odulo $D$
para determinar qu\'e enteros $m$ se pueden representar por formas de
discriminante $D$?
El \teoname~\ref{teo:representaciones} nos da un criterio sencillo
para determinar qu\'e \emph{primos} podemos representar por formas
de discriminante $D$:
son aquellos que pertenecen al n\'ucleo de la funci\'on $\chi$ del
\lemaname~\ref{lema:ecuacion:kronecker},
\begin{displaymath}
	\ker(\chi)\,=\,\big\{c\in\Unidadesmod[D]\,:\,\chi(c)=1\big\}
		\,\leq\,\Unidadesmod[D]
	\dispstop
\end{displaymath}
%
% En particular, si $p$ y $q$ son primos (impares) que no dividen a $D$,
% y si $p\equiv q\tmodulo[D]$, entonces la condici\'on para que $p$ est\'e
% representado por una forma de discriminante $D$ es la misma que para $q$.
% Dicho de otra manera, si nos interesa saber si podemos representar un
% primo particular $p$ por una forma de discriminante $D$, nos bastar\'a
% con decidir si podemos representar \emph{alg\'un} primo perteneciente a
% la clase de congruencia de $p$ m\'odulo $D$.
% Cuando restringimos el problema de representaci\'on de representar
% enteros a representar primos, lo que importa es la clase de congruencia
% del primo.
% 

Si $D\equiv 0,1\tmodulo[4]$, $D<0$, es un discriminante y
$c\in\Unidadesmod[D]$, existe, por el \teoname~\ref{teo:dirichlet},
al menos un primo $p\in c$ ($\clase p=c$).
Ahora, por el \teoname~\ref{teo:representaciones}, si $c\in\ker(\chi)$,
entonces $\chi(p)=1$ y $p$ est\'a representado por alguna forma
de discriminante $D$. En particular, deducimos lo siguiente:
si $c\in\ker(\chi)$, entonces alg\'un elemento $m\in c$ de la clase
es representado (propiamente) por alguna forma (primitiva) de
discriminante $D$.
% Para que sea m\'as claro, introducimos la siguiente definici\'on
% (que tambi\'en ser\'a utilizada en la \S~\ref{sec:generos}).

\begin{defDirichlet}\label{def:dirichlet:representacion:mod}
	Dados un discriminante $D\equiv 0,1\tmodulo[4]$, una clase de
	congruencia $c\in\Enterosmod[|D|]$ y una forma $f$, decimos que
	\emph{la clase $c$ es representada por $f$} (o que \emph{$f$ %
	representa la clase $c$}), si existe $m\in c$, perteneciente a la
	clase, tal que $m$ es representado por $f$.
	% Si existe $m\in c$ propiamente representado por $f$, decimos
	% que la clase $c$ es \emph{propiamente representada} por $f$.
\end{defDirichlet}

\begin{ejemDirichlet}\label{ejem:dirichlet:representacion:mod}
	La forma $\binaria{1,0,14}$ representa las clases
	\begin{displaymath}
		\clase 1\dispcomma\quad
		\clase{25}\dispcomma\quad
		\clase 9\dispcomma\quad
		\clase{15}\dispcomma\quad
		\clase{39}\dispand
		\clase{23}
	\end{displaymath}
	%
	m\'odulo $56$, porque, por ejemplo, representa \emph{los enteros}
	$1$, $25$, $9$, $15$, $39$ y $23$.
	Sin embargo, la forma $\binaria{2,0,7}$ tambi\'en representa
	estas clases: representa
	\begin{displaymath}
		9\,=\,2\cdot 1^2+7\dispcomma\quad
		15\,=\,2\cdot 2^2+7\dispcomma\quad
		25\,=\,2\cdot 3^2+7\dispand
		39\,=\,2\cdot 4^2+7
		\text{ ;}
	\end{displaymath}
	%
	pero tambi\'en representa
	\begin{displaymath}
		57\,=\,2\cdot 5^2+7\,\equiv\,1\modulo[56]
		\dispand
		79\,=\,2\cdot 6^2+7\,\equiv\,23\modulo[56]
		\dispstop
	\end{displaymath}
	%
\end{ejemDirichlet}

\begin{coroDirichlet}\label{coro:dirichlet}
	Sea $D\equiv 0,1\tmodulo[4]$ y sea $c\in\Unidadesmod[D]$.
	Entonces, la clase $c$ es representada (propiamente) por
	alguna forma (primitiva) de discriminante $D$, si $c\in\ker(\chi)$.
\end{coroDirichlet}

Rec\'{\i}procamente, el \lemaname~\ref{lema:generos} muestra, entre
otras cosas, que, si $c$ es representada por una forma de discriminante
$D$, \emph{entonces} $c\in\ker(\chi)$.

En definitiva, tenemos una condici\'on necesaria (y suficiente)
para determinar qu\'e \emph{clases de congruencia} son representadas
por \emph{alguna} forma de discriminante $D$.
En la \S~\ref{sec:generos}, veremos que no es posible, en general,
representar cualquier clase de congruencia (detro de las clases
representables) por cualquier forma cuadr\'atica.
El \emph{g\'enero} agrupa formas de acuerdo con las clases de
congruencia que cada una puede representar y clases de congruencia
de acuerdo con las formas que las representan.
% El \emph{g\'enero} servir\'a de herramienta para estudiar si una
% clase de congruencia espec\'{\i}fica es representada por una forma
% cuadr\'atica particular. En consecuencia, podremos precisar un poco
% m\'as qu\'e formas cuadr\'aticas --dentro de todas las formas de
% discriminante $D$-- podr\'{\i}an representar un primo dado.

Terminamos esta secci\'on con algunos ejemplos que ilustran
el uso del \teoname~\ref{teo:dirichlet} en este contexto.
Dado que formas equivalentes representan los mismos enteros,
cuando decimos que una clase es representada, o que un entero
es representado, por alguna forma de discriminante $D<0$ y
queremos encontrar dicha forma, podemos restringirnos a buscar
entre las formas reducidas.

\begin{ejemDirichlet}\label{ejem:dirichlet:representacion:tres}
	El grupo $\Unidadesmod[3]$ consta de dos clases de congruencia:
	\begin{displaymath}
		\clase 1\dispand\clase 2
		\dispstop
	\end{displaymath}
	%
	La clase $\clase 1$ contiene al primo $7$ y la clase
	$\clase 2$ contiene a $5$. Para decidir qu\'e clases
	se pueden representar por formas de discriminante $-3$,
	necesitamos conocer el valor de $\chi$ en ellas.
	Ahora,
	\begin{displaymath}
		\chi(1)\,=\,\chi(7)\,=\,\legendre{-3} 7\,=\,1
		\dispcomma
	\end{displaymath}
	%
	mientras que
	\begin{displaymath}
		\chi(2)\,=\,\chi(5)\,=\,\legendre{-3} 5\,=\,-1
		\dispstop
	\end{displaymath}
	%
	De hecho, podr\'{\i}amos deducir, usando las propiedades del
	s\'{\i}mbolo de Legendre, que $\tlegendre{-3} p=1$, si s\'olo si
	$p\equiv 1\tmodulo[3]$.
	Hay, por otro lado, s\'olo una forma reducida de discriminante
	$-3$:
	\begin{displaymath}
		x^2\,+\,xy\,+\,y^2
		\dispstop
	\end{displaymath}
	%
	Por el \coroname~\ref{coro:dirichlet}, la clase de congruencia
	m\'odulo $3$ $\clase 1$ es representable por esta forma
	cuadr\'atica. Efectivamente,
	\begin{displaymath}
		7\,=\,1^2+1\cdot (-3)+(-3)^2
		\dispstop
	\end{displaymath}
	%
	Ahora bien, nuevamente por el \coroname~\ref{coro:dirichlet},
	los primos pertenecientes a la clase $\clase 2$ no pueden
	ser representados por la forma $\binaria{1,1,1}$
	(y, por lo tanto, por ninguna forma de discriminante $-3$).
	Pero cabe, aun, la posibilidad de que la clase s\'{\i} sea
	representable, es decir, que exista alg\'un entero
	$m\equiv 2\tmodulo 3$ (no primo) que s\'{\i} sea representado
	en la forma $\binaria{1,1,1}$. La rec\'{\i}proca del
	\coroname~\ref{coro:dirichlet} garantiza que esto no puede
	pasar en general. De hecho, si no nos olvidamos de que la
	forma es $x^2+xy+y^2$, podemos ver que, si $\mcd{x,y}=1$, entonces
	\begin{displaymath}
		x^2+xy+y^2\,\equiv\,0,1\modulo[3]
		\dispcomma
	\end{displaymath}
	%
	con lo cual, la clase $\clase 2$ no es representable por
	formas de discriminante $-3$.
\end{ejemDirichlet}

\begin{ejemDirichlet}\label{ejem:dirichlet:representacion:quince}
	Las clases de congruencia m\'odulo $15$ en
	$\Unidadesmod[15]$ son:
	\begin{displaymath}
		\clase 1\dispcomma\quad
		\clase 2\dispcomma\quad
		\clase 4\dispcomma\quad
		\clase 7\dispcomma\quad
		\clase 8\dispcomma\quad
		\clase{11}\dispcomma\quad
		\clase{13}\dispand
		\clase{14}\dispstop
	\end{displaymath}
	%
	Queremos decidir cu\'ales de estas clases son representadas
	por alguna forma de discriminante $-15$ y, dentro de lo posible,
	determinar, dada una clase representable, cu\'al de las dos formas
	la representa. Empezamos calculando la funci\'on $\chi$.
	Para eso necesitamos evaluarla en cada una de las clases en
	$\Unidadesmod[15]$ (en las clases m\'odulo $15$ que no
	pertenecen a $\Unidadesmod[15]$, su valor es $0$).
	Pero, seg\'un el \lemaname~\ref{lema:ecuacion:kronecker},
	los \'unicos enteros en los que conocemos (m\'as o menos)
	expl\'{\i}citamente la funci\'on $\chi$ es en los primos (que,
	en este caso, no dividen a $15$).
	Dada una clase $c\in\Unidadesmod[15]$, si $p\in c$
	es primo perteneciente a la clase, entonces,
	\begin{displaymath}
		\chi(c)\,=\,\chi(p)\,=\,\legendre{-15} p
		\dispstop
	\end{displaymath}
	%
	Pero, por propiedades del s\'{\i}mbolo de Legendre,
	\begin{displaymath}
		\legendre{-15} p\,=\,\legendre{-3} p\,\legendre 5 p
		\dispstop
	\end{displaymath}
	%
	Entonces,
	\begin{displaymath}
		\legendre{-15} p\,=\,1
		\dispiff
		\left\{
			\begin{array}{l}
				p\,\equiv\,1\tmodulo[3]\dispand
					p\,\equiv\,\pm 1\tmodulo[5]
					\dispor \\
				p\,\equiv\,2\tmodulo[3]\dispand
					p\,\equiv\,\pm 3\tmodulo[5]
					\dispstop
			\end{array}
		\right.
	\end{displaymath}
	%
	O sea, $\chi(p)=1$, si y s\'olo si
	$p\equiv 1,4,2,8\tmodulo[15]$.
	La \tablename~\ref{tab:ejem:dirichlet:representacion:quince}
	muestra, para cada clase $c\in\Unidadesmod[15]$, un primo $p$
	perteneciente a $c$ y el valor de $\chi(p)$.
	\begin{table}
		\centering
		\begin{tabular}{r|cccccccc}
			$c$ & $1$ & $2$ & $4$ & $7$
				& $8$ & $11$ & $13$ & $14$ \\
			\hline
			$p\in c$ & $31$ & $17$ & $19$ & $37$
				& $23$ & $41$ & $43$ & $29$ \\
			\hline
			$\chi(p)$ & $1$ & $1$ & $1$ & $-1$
				& $1$ & $-1$ & $-1$ & $-1$ \\
			\hline
			$p=f(x,y)$ & $\binaria{1,1,4}$ & $\binaria{2,1,2}$
				& $\binaria{1,1,4}$ &
				& $\binaria{2,1,2}$ & & &
		\end{tabular}
		\caption{
			Valores de $\chi$ correspondientes a $D=-15$
			y la forma reducida $f$ que representa cada
			clase.
		}\label{tab:ejem:dirichlet:representacion:quince}
	\end{table}
	%
	Por otro lado, hay dos formas reducidas de discriminante $D=-15$;
	ellas son:
	\begin{displaymath}
		x^2\,+\,xy\,+4y^2
		\dispand
		2x^2\,+\,xy\,+\,2y^2
		\dispstop
	\end{displaymath}
	%
	Notamos que, si $\mcd{x,y}=1$, entonces
	\begin{displaymath}
		x^2\,+\,xy\,+\,4y^2\,\equiv\,1\tmodulo[3]
		\dispand
		2x^2\,+\,xy\,+\,2y^2\,\equiv\,2\tmodulo[3]
		\dispstop
	\end{displaymath}
	%
	En particular, no hay posibilidad de que ambas formas
	representen las mismas clases m\'odulo $15$.
	Complementando esta \'ultima observaci\'on,
	si $\mcd{x,y}=1$,
	\begin{displaymath}
		x^2\,+\,xy\,+\,4y^2\,\equiv\,1,4\tmodulo[5]
		\dispand
		2x^2\,+\,xy\,+\,2y^2\,\equiv\,2,3\tmodulo[5]
		\dispstop
	\end{displaymath}
	%
	De hecho, para los primos de la \tablename~%
	\ref{tab:ejem:dirichlet:representacion:quince},
	\begin{displaymath}
		\begin{aligned}
			31 & \,=\, 3^2+3\cdot 2+4\cdot 2^2 \\
			17 & \,=\, 2\cdot 1^2+1\cdot (-3)+2\cdot (-3)^2 \\
			19 & \,=\, 1^2+1\cdot 2+4\cdot 2^2 \\
			23 & \,=\,2\cdot 1^2+1\cdot 3+2\cdot 3^2
			\dispstop
		\end{aligned}
		%
	\end{displaymath}
	%
\end{ejemDirichlet}

