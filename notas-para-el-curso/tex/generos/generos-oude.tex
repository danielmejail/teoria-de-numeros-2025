\theoremstyle{plain}
\newtheorem{teoGeneros}{\teoname}[section]
\newtheorem{coroGeneros}[teoGeneros]{\coroname}
\newtheorem{lemaGeneros}[teoGeneros]{\lemaname}

\theoremstyle{definition}
\newtheorem{defGeneros}[teoGeneros]{\defname}
\newtheorem{ejemGeneros}[teoGeneros]{\ejemname}
\newtheorem{obsGeneros}[teoGeneros]{\obsname}

%-------------

La relaci\'on de equivalencia refleja, talvez, un aspecto m\'as bien
algebraico de las formas cuadr\'aticas.
En esta secci\'on, introduciremos la noci\'on de \emph{g\'enero}
de una forma cuadr\'atica (binaria, definida positiva y primitiva),
que se conecta m\'as con la idea de representar un entero por una forma
cuadr\'atica y que nos permitir\'a subdividir las formas cuadr\'aticas
teniendo en cuenta sus propiedades de representaci\'on.
Sabemos que formas equivalentes representan los mismos enteros
(el conjunto de enteros representados es un invariante de la
clase de equivalencia de una forma).
Con lo cual, la subdivisi\'on del conjunto de formas de acuerdo con su
g\'enero terminar\'a siendo una subdivisi\'on m\'as gruesa que la
subdivisi\'on por clases de equivalencia o clases de equivalencia
estricta.

\begin{ejemGeneros}\label{ejem:generos:menos-veinte}
	Hay dos clases de formas de discriminante $-20$.
	El argumento es similar al del \ejemname~\ref{ejem:reducidas}.
	Si $f=\binaria{a,b,c}$ es una forma cuadr\'atica primitiva,
	reducida y de discriminante $-20$, entonces
	$1\leq a\leq\sqrt{20/3}<7$, o sea $a\in\{1,2\}$. Adem\'as,
	como $D\equiv 0\tmodulo[4]$, debe ser $b\equiv 0\tmodulo[2]$. 
	Si $a=1$, entonces $b=0$ y $c=5$ es la \'unica posibilidad.
	Si $a=2$, entonces $b\in\{0,2\}$, pero $-20=b^2-4ac$ con
	$c\in\Enteros$ fuerza que $b=2$ y, as\'{\i}, $c=3$. O sea,
	$\nClases(-20)=2$ y las clases est\'an representadas por
	las formas primitivas reducidas
	\begin{displaymath}
		x^2\,+\,5y^2\dispand 2x^2\,+\,2xy\,+\,3y^2
		\dispstop
	\end{displaymath}
	%

	Por el \coroname~\ref{coro:reducidas:representaciones},
	un primo impar $p\neq 5$ est\'a representado por una
	forma reducida de discriminante $-20$, si y s\'olo si
	$\tlegendre{-5} p=1$. Pero
	\begin{displaymath}
		\tlegendre{-5} p\,=\,1\dispcomma\dispiff
		p\,\equiv\,1,9,7,3\tmodulo[20]
		\dispstop
	\end{displaymath}
	%
	Ahora, $f(x,y)=x^2+5y^2$ representa primos
	$p\equiv 1,9\tmodulo[20]$: por ejemplo,
	\begin{displaymath}
		\begin{aligned}
			f(6,1) & \,=\,41\dispcomma\quad
				f(4,3)\,=\,61 \dispcomma \\
			f(3,2) & \,=\,29\dispand f(3,4)\,=\,89
			\dispstop
		\end{aligned}
		%
	\end{displaymath}
	%
	Pero $f$ no representa primos $p\equiv 3,7\tmodulo[20]$
	(pues $f(x,y)\equiv 0,1,4\tmodulo[5]$).
	Sin embargo, la forma $g=2x^2+2xy+3y^2$ s\'{\i} los
	representa: por ejemplo,
	\begin{displaymath}
		\begin{aligned}
			g(0,1) & \,=\,3\dispcomma\quad
				g(1,-3)\,=\,23\dispcomma\quad
				g(4,1)\,=\,43\dispcomma\quad
				g(7,-3)\,=\,83\dispcomma \\
			g(1,1) & \,=\,7\dispcomma\quad
				g(5,-3)\,=\,47\dispcomma\dispand
				g(4,-5)\,=\,67
				\dispstop
		\end{aligned}
		%
	\end{displaymath}
	%
	Pero $g$ no representa primos $p\equiv 1,9\tmodulo[20]$
	(pues $g(x,y)\equiv 0,2,3\tmodulo[5]$).
	En definitiva, un primo impar $p\neq 5$ es representado
	por una forma reducida de discriminante $-20$, si y s\'olo si
	\begin{displaymath}
		p\,\equiv\,1,9,7,3\tmodulo[20]
		\dispiff
		\left\{
			\begin{array}{l}
				p\,=\,x^2+5y^2\dispor \\
				p\,=\,2x^2+2xy+3y^2
				\dispstop
			\end{array}
		\right.
	\end{displaymath}
	%
	M\'as precisamente, si $p\neq 5$ es un primo impar 
	\begin{displaymath}
		\begin{aligned}
			p & \,\equiv\,1,9\tmodulo[20]\dispiff
				p\,=\,x^2+5y^2\dispand \\
			p & \,\equiv\,3,7\tmodulo[20]\dispiff
				p\,=\,2x^2+2xy+3y^2
			\dispstop
		\end{aligned}
		%
	\end{displaymath}
	%
\end{ejemGeneros}

\begin{ejemGeneros}\label{ejem:generos:menos-cincuenta-y-seis}
	Hay cuatro clases de formas de discriminante $-56$. Las formas
	reducidas correspondientes son
	\begin{displaymath}
		\binaria{1,0,14} \dispcomma\quad
		\binaria{2,0,7} \dispcomma\quad
		\binaria{3,-2,5} \dispand
		\binaria{3,2,5}
		\dispstop
	\end{displaymath}
	%
	Por el \coroname~\ref{coro:reducidas:representaciones},
	un primo impar $p\neq 7$ est\'a representado por alguna de estas
	formas cuadr\'aticas, si y s\'olo si $\tlegendre{-14} p=1$. Pero
	\quedacomoejercicio,
	% % PARA EL EXAMEN
	% % Explicar o mostrar mediante ejemplos el significado de la
	% % afirmaci\'on: ``no es posible, conociendo \'unicamente la clase
	% % de congruencia de $p$ m\'odulo $56$, determinar si $p$ es o no
	% % de la forma $x^2+14y^2$''.
	\begin{itemize}
		\item si $p=x^2+14y^2$ o $p=2x^2+7y^2$, entonces
			$p\equiv 1,7\tmodulo[8]$ y
			$p\equiv 1,2,4\tmodulo[7]$, y,
		\item si $p=3x^2\pm 2xy+5y^2$, entonces debe ser
			$p\equiv 3,5\tmodulo[8]$ y
			$p\equiv 3,5,6\tmodulo[7]$.
	\end{itemize}
	%
	La condici\'on m\'odulo $7$ la podemos deducir, por ejemplo, de la
	condici\'on m\'odulo $8$ y de que, por Reciprocidad cuadr\'atica,
	\begin{displaymath}
		\tlegendre{-14} p\,=\,\tlegendre{-7} p\tlegendre 2 p\,=\,
			\tlegendre p 7\tlegendre 2 p
		\dispstop
	\end{displaymath}
	%
	Notamos, adem\'as, que $\binaria{1,0,14}$ y $\binaria{2,0,7}$ ambas
	representan primos en todas las clases
	$p\equiv 1,25,9,15,39,23\tmodulo[56]$,
	y que $\binaria{3,\pm 2,5}$ representan exactamente los mismos
	enteros.
\end{ejemGeneros}

\begin{obsGeneros}\label{obs:generos:ejemplos}
	En el \ejemname~\ref{ejem:generos:menos-veinte},
	podemos ver que la clase de congruencia de un primo $p$ m\'odulo $20$
	determina, en primer lugar, si $p$ es representado en \emph{alguna}
	forma cuadr\'atica de discriminante $-20$ y, en segundo lugar,
	determina tambi\'en la clase que lo representa.
	En el \ejemname~\ref{ejem:generos:menos-cincuenta-y-seis}, tambi\'en
	es cierto que la clase de congruencia de $p$ m\'odulo $56$
	determina si $p$ es representado por alguna forma de discriminante
	$-56$, pero no es posible distinguir exactamente cu\'al es la clase
	que lo representa. En particular, no podemos decir, dado
	$p\equiv 1,25,9,15,39,23\tmodulo[56]$, si $p$ es de la forma
	$x^2+14y^2$. M\'as aun, $p$ podr\'{\i}a estar % (y, de hecho, est\'a)
	representado por m\'as de una clase.
	En definitiva, hay un l\'{\i}mite a lo que podemos deducir a partir
	de (estas) congruencias.
\end{obsGeneros}

% En la \defname~\ref{def:definiciones:representacion}, definimos lo que
% significa que una forma cuadr\'atica represente un entero: $f$ representa
% $m\in\Enteros$, si existen $x,y\in\Enteros$ tales que $f(x,y)=m$.
% Teniendo en cuenta la \obsname~\ref{obs:generos:ejemplos}, introducimos
% la siguiente variante de este concepto de representaci\'on.
% 
% \begin{defGeneros}\label{def:generos:representacion:mod}
	% Dados $D\in\Enteros$, una clase de congruencia
	% $c\in\Enterosmod[|D|]$ y una forma $f$, decimos que
	% \emph{la clase $c$ es representada por $f$} (o que \emph{$f$ %
	% representa la clase $c$}), si existe $m\in c$, perteneciente a la
	% clase, tal que $m$ es representado por $f$.
% \end{defGeneros}

% \begin{ejemGeneros}\label{ejem:generos:representacion:mod}
	% La forma $\binaria{1,0,14}$ representa las clases
	% \begin{displaymath}
		% \clase 1\dispcomma\quad
		% \clase{25}\dispcomma\quad
		% \clase 9\dispcomma\quad
		% \clase{15}\dispcomma\quad
		% \clase{39}\dispand
		% \clase{23}
	% \end{displaymath}
	% %
	% m\'odulo $56$, porque, por ejemplo, representa a \emph{los enteros}
	% $1$, $25$, $9$, $15$, $39$ y $23$. Sin embargo, la forma
	% $\binaria{2,0,7}$ tambi\'en representa estas clases:
	% representa
	% \begin{displaymath}
		% 9\,=\,2\cdot 1^2+7\dispcomma\quad
		% 15\,=\,2\cdot 2^2+7\dispcomma\quad
		% 25\,=\,2\cdot 3^2+7\dispand
		% 39\,=\,2\cdot 4^2+7
		% \text{ ;}
	% \end{displaymath}
	% %
	% pero tambi\'en representa
	% \begin{displaymath}
		% 57\,=\,2\cdot 5^2+7\,\equiv\,1\modulo[56]
		% \dispand
		% 79\,=\,2\cdot 6^2+7\,\equiv\,23\modulo[56]
		% \dispstop
	% \end{displaymath}
	% %
% \end{ejemGeneros}

De la teor\'{\i}a desarrollada en la \S~\ref{sec:representaciones}, sabemos
que formas equivalentes representan los mismos enteros.

\begin{defGeneros}\label{def:generos:clases-coprimas}
	Dados $D\in\Enteros$ y una clase $c\in\Enterosmod[D]$, decimos
	que \emph{$c$ es coprima con $D$}, si los enteros de la clase
	$c$ son coprimos con $D$.
\end{defGeneros}

\begin{obsGeneros}\label{obs:generos:clases-coprimas}
	Si $m\equiv n\tmodulo[D]$, entonces $\mcd{m,D}=\mcd{n,D}$.
	En particular, cuando $m\equiv n$, $m$ es coprimo con $D$, si
	y s\'olo si $n$ lo es. En consecuencia,
	si $D\in\Enteros$ y $c\in\Enterosmod[D]$, entonces $c$ es coprima
	con $D$, si y s\'olo si \emph{al menos uno de los enteros %
	pertenecientes a $c$} es coprimo con $D$.
\end{obsGeneros}

\begin{defGeneros}\label{def:generos}
	Dos formas cuadr\'aticas % \emph{primitivas} definidas positivas
	de discriminante $D$ \emph{pertenecen al mismo g\'enero},
	si representan las mismas clases coprimas con $D$.
	% es decir, las mismas clases en $\Unidadesmod[D]$.
\end{defGeneros}

\begin{ejemGeneros}\label{ejem:generos:representacion:mod:bis}
	Las clases coprimas con $-56$ son:
	\begin{displaymath}
		1,\,3,\,5,\,9,\,
		11,\,13,\,15,\,17,\,
		19,\,23,\,25,\,27,\,
		29,\,31,\,33,\,37,\,
		39,\,41,\,43,\,45,\,
		47,\,51,\,53,\,55
		\dispstop
	\end{displaymath}
	%
	De ellas, las formas $\binaria{1,0,14}$ y $\binaria{2,0,7}$
	representan
	\begin{displaymath}
		\dots
		\dispcomma
	\end{displaymath}
	%
	mientras que las formas $\binaria{3,\pm 2,5}$ representan
	\begin{displaymath}
		\dots
		\dispstop
	\end{displaymath}
	%
	Las formas $\binaria{1,0,14}$ y $\binaria{2,0,7}$ pertenecen al
	mismo g\'enero. Las formas $\binaria{3,\pm 2,5}$ pertenecen a
	un mismo g\'enero, pero no pertenecen al mismo g\'enero que
	las dos formas anteriores. Decimos entonces que las formas
	(o, tambi\'en, las clases de formas) de discriminante $-56$
	se dividen en dos g\'eneros:
	\begin{displaymath}
		\big\{
			\binaria{1,0,14},\,\binaria{2,0,7}
		\big\}\dispand
		\big\{
			\binaria{3,-2,5},\,\binaria{3,2,5}
		\big\}
		\dispstop
	\end{displaymath}
	%
\end{ejemGeneros}

