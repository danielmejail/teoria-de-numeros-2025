En esta secci\'on hacemos una disgresi\'on por el
Teorema de Dirichlet sobre primos en progresiones aritm\'eticas y
lo relacionaremos con el estudio de la representabilidad por formas
cuadr\'aticas. Esto nos ser\'a \'util en la \S~\ref{sec:generos},
cuando hablemos de \emph{g\'enero} de una forma cuadr\'atica.

Por el \lemaname~\ref{lema:representaciones:primitiva}, sabemos que una
condici\'on necesaria y suficiente para que un entero $m$ sea representado
por una forma de discriminante $D$ es que $D$ sea un cuadrado m\'odulo $m$.
Fijado $m$ podemos decidir f\'acilmente qu\'e discriminantes lo representan.
Por otro lado, el \teoname~\ref{teo:representaciones} nos da,
\emph{rec\'{\i}procamente}, un criterio sencillo para determinar, fijado $D$,
qu\'e \emph{primos} podemos representar por formas de discriminante $D$.
Dicha condici\'on involucra la funci\'on $\chi$ del \lemaname~%
\ref{lema:ecuacion:kronecker} y, en definitiva, se reduce a una condici\'on
sobre la clase de congruencia del primo m\'odulo $D$:
los primos representables son aquellos cuyas clases de congruencia m\'odulo
$D$ cumplen $\chi(\clase p)=1$.

% La noci\'on de g\'enero de una forma cuadr\'atica tiene que ver con
% \emph{qu\'e clases de congruencia} son representadas por la forma en
% cuesti\'on; precisamente, qu\'e clases de congruencia m\'odulo el
% discriminante de la forma. Si $D\equiv 0,1\tmodulo[4]$ es un discriminante
% y $m\in\Entero$ 

