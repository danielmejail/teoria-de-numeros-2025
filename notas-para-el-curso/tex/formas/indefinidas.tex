\theoremstyle{plain}
\newtheorem{teoIndefinidas}{\teoname}[section]
\newtheorem{lemaIndefinidas}[teoIndefinidas]{\lemaname}

\theoremstyle{definition}
\newtheorem{defIndefinidas}[teoIndefinidas]{\defname}
\newtheorem{ejemIndefinidas}[teoIndefinidas]{\ejemname}

%-------------

El objetivo de esta secci\'on es describir un procedimiento que nos permita
determinar la clase de equivalencia propia de una forma cuadr\'atica
indefinida, que tenga el papel que juega la teor\'{\i}a de reducci\'on
de formas cuadr\'aticas definidas estudiada en la \S~\ref{sec:definidas}.

\subsection{Preliminares}
En la relaci\'on de equivalencia entre formas cuadr\'aticas est\'a
impl\'{\i}cita cierta ``din\'amica'', la acci\'on de un grupo,
$\GL(2,\Enteros)$, actuando por cambio de variables.
M\'as en general, podemos considerar la siguiente operaci\'on:
dada una forma cuadr\'atica binaria con coeficientes enteros,
$g=\binaria{a,b,c}$, y una matriz con coeficientes enteros,
$\gamma=\sbmatrix{ p & q \\ r & s }$, definimos una nueva
% \emph{funci\'on}
forma cuadr\'atica
\begin{equation}
	\label{eq:formas:accion}
	\big(g\cdot\gamma\big)(x,y)\,:=\,g(px+qy,rx+sy)
	\dispstop
\end{equation}
%
La expresi\'on \eqref{eq:formas:accion} define una forma cuadr\'atica:
si llamamos $f=g\cdot\gamma$, entonces $f$ es un polinomio en las variables
$x$ e $y$ y sus coeficientes y discriminante est\'an dados por las
f\'ormulas de la \obsname~\ref{obs:definiciones:equivalencia:invariantes}.
% :
% \begin{displaymath}
	% f(x,y) \,=\,a\,(px+qy)^2+b\,(px+qy)\,(rx+sy)+c\,(rx+sy)^2
% \end{displaymath}
% %
% y $f=\binaria{a',b',c'}$, donde
% \begin{displaymath}
	% \begin{aligned}
		% a' & \,=\,ap^2+bpr+cr^2\,=\,g(p,r)\dispcomma \\
		% b' & \,=\,2apq+b\,(ps+qr)+2crs\dispand \\
		% c' & \,=\,aq^2+bqs+cs^2\,=\,g(q,s)\dispstop
	% \end{aligned}
	% %
% \end{displaymath}
% %
% El discriminante de la forma $f$ es igual a
% \begin{displaymath}
	% \discriminante(f)\,=\,(ps-qr)^2\,\discriminante(g)
	% \dispstop
% \end{displaymath}
%
% Podr\'{\i}amos omitir la condici\'on de que $\gamma$ tenga coeficientes
% enteros; por ejemplo, podr\'{\i}amos considerar todas las matrices
% inversibles con coeficientes en $\Racionales$.
% Pero, en ese caso, la forma cuadr\'atica $g\cdot\gamma$ podr\'{\i}a
% tener coeficientes no enteros.

Si las formas $f$ y $g$ est\'an relacionadas por $f=g\cdot\gamma$
con $\gamma\in\Mat(2\times 2,\Enteros)$, Gauss hubiese dicho que
\emph{$f$ est\'a incluida en $g$} o que \emph{$g$ implica $f$};
adem\'as, hubiese dicho que la transformaci\'on $g\mapsto g\cdot\gamma$
es una \emph{transfomaci\'on propia}, si $\det(\gamma)>0$, y que
es una \emph{transformaci\'on impropia}, si $\det(\gamma)<0$.
\cite[\S~157]{Gauss}
Si $\det(\gamma)=\pm 1$, entonces $\gamma\in\GL(2,\Enteros)$ y
decimos que $f$ es equivalente a $g$; si $\det(\gamma)=1$, son
propiamente equivalentes y, si $\det(\gamma)=-1$, son impropiamente
equivalentes.

En la teor\'{\i}a de reducci\'on de formas cuadr\'aticas definidas,
dos transformaciones fueron de gran utilidad:
\begin{displaymath}
	S\,=\,\begin{bmatrix} & -1 \\ 1 & \end{bmatrix}\dispand
	T\,=\,\begin{bmatrix} 1 & 1 \\ & 1 \end{bmatrix}
		\dispcomma
\end{displaymath}
%
as\'{\i} tambi\'en como las potencias $T^h=\sbmatrix{ 1 & h \\ & 1 }$.
Todas ellas son transformaciones propias. Y, m\'as aun,
\begin{displaymath}
	\SL(2,\Enteros)\,=\,\generado{S,T}
	\dispcomma
\end{displaymath}
%
con lo cual, el proceso de reducci\'on de formas cuadr\'aticas definidas
positivas se puede describir enteramente con las matrices $S$ y $T$.%
\footnote{
	De hecho, las matrices $\pm\Id$ act\'uan trivialmente en el conjunto
	de formas cuadr\'aticas. El grupo que est\'a actuando es, en realidad,
	$\PSL(2,\Enteros)=\SL(2,\Enteros)/\{\pm\Id\}$.
	Las matrices $\pm\Id$ son las \'unicas que act\'uan trivialmente,
	pero hay otras que tienen ``puntos fijos'': por ejemplo,
	la matriz $\gamma=\sbmatrix{ & -1 \\ 1 & 1 }$ y la forma
	$f=\binaria{1,1,1}$ cumplen $f\cdot\gamma=f$.
}
El resultado principal de \S~\ref{sec:definidas} es que:
\begin{itemize}
	\item existe un conjunto finito de ``formas reducidas'',
	\item actuando por las matrices $S$ y $T$, podemos empujar
		una forma cuadr\'atica definida positiva hacia
		este conjunto, y
	\item no es posible pasar de una forma reducida a otra mediante
		las matrices $S$ y $T$.%
		\footnote{
			Aunque s\'{\i} talvez mendiante alguna
			transformaci\'on impropia; precisamente, mediante
			$\sbmatrix{ 1 & \\ & -1 }$.
		}
	\item En particular, dos formas definidas postivas son estrictamente
		equivalentes, si y s\'olo si son estrictamente equivalentes
		a una misma forma reducida.
\end{itemize}
%

