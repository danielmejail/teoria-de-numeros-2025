\theoremstyle{plain}
\newtheorem{teoIndefinidas}{\teoname}[section]
\newtheorem{lemaIndefinidas}[teoIndefinidas]{\lemaname}
\newtheorem{coroIndefinidas}[teoIndefinidas]{\coroname}

\theoremstyle{definition}
\newtheorem{defIndefinidas}[teoIndefinidas]{\defname}
\newtheorem{obsIndefinidas}[teoIndefinidas]{\obsname}
\newtheorem{ejemIndefinidas}[teoIndefinidas]{\ejemname}

%-------------

El objetivo de esta secci\'on es describir un procedimiento que nos permita
determinar la clase de equivalencia propia de una forma cuadr\'atica
indefinida, que tenga el papel que juega la teor\'{\i}a de reducci\'on
de formas cuadr\'aticas definidas estudiada en la \S~\ref{sec:definidas}.

% \subsection{Preliminares}\label{subsec:indefinidas:preliminares}
En la relaci\'on de equivalencia entre formas cuadr\'aticas est\'a
impl\'{\i}cita cierta ``din\'amica'', la acci\'on de un grupo,
$\GL(2,\Enteros)$, actuando por cambio de variables.
M\'as en general, podemos considerar la siguiente operaci\'on:
dada una forma cuadr\'atica binaria con coeficientes enteros,
$g=\binaria{a,b,c}$, y una matriz con coeficientes enteros,
$\gamma=\sbmatrix{ p & q \\ r & s }$, definimos una nueva
% \emph{funci\'on}
forma cuadr\'atica
\begin{equation}
	\label{eq:formas:accion}
	\big(g\cdot\gamma\big)(x,y)\,:=\,g(px+qy,rx+sy)
	\dispstop
\end{equation}
%
La expresi\'on \eqref{eq:formas:accion} define una forma cuadr\'atica:
si llamamos $f=g\cdot\gamma$, entonces $f$ es un polinomio en las variables
$x$ e $y$ y sus coeficientes y discriminante est\'an dados por las
f\'ormulas de la \obsname~\ref{obs:definiciones:equivalencia:invariantes}.
% :
% \begin{displaymath}
	% f(x,y) \,=\,a\,(px+qy)^2+b\,(px+qy)\,(rx+sy)+c\,(rx+sy)^2
% \end{displaymath}
% %
% y $f=\binaria{a',b',c'}$, donde
% \begin{displaymath}
	% \begin{aligned}
		% a' & \,=\,ap^2+bpr+cr^2\,=\,g(p,r)\dispcomma \\
		% b' & \,=\,2apq+b\,(ps+qr)+2crs\dispand \\
		% c' & \,=\,aq^2+bqs+cs^2\,=\,g(q,s)\dispstop
	% \end{aligned}
	% %
% \end{displaymath}
% %
% El discriminante de la forma $f$ es igual a
% \begin{displaymath}
	% \discriminante(f)\,=\,(ps-qr)^2\,\discriminante(g)
	% \dispstop
% \end{displaymath}
%
% Podr\'{\i}amos omitir la condici\'on de que $\gamma$ tenga coeficientes
% enteros; por ejemplo, podr\'{\i}amos considerar todas las matrices
% inversibles con coeficientes en $\Racionales$.
% Pero, en ese caso, la forma cuadr\'atica $g\cdot\gamma$ podr\'{\i}a
% tener coeficientes no enteros.

Si las formas $f$ y $g$ est\'an relacionadas por $f=g\cdot\gamma$
con $\gamma\in\Mat(2\times 2,\Enteros)$, Gauss hubiese dicho que
\emph{$f$ est\'a incluida en $g$} o que \emph{$g$ implica $f$};
adem\'as, hubiese dicho que la transformaci\'on $g\mapsto g\cdot\gamma$
es una \emph{transfomaci\'on propia}, si $\det(\gamma)>0$, y que
es una \emph{transformaci\'on impropia}, si $\det(\gamma)<0$.
\cite[\S~157]{Gauss}
Si $\det(\gamma)=\pm 1$, entonces $\gamma\in\GL(2,\Enteros)$ y
decimos que $f$ es equivalente a $g$; si $\det(\gamma)=1$, son
propiamente equivalentes y, si $\det(\gamma)=-1$, son impropiamente
equivalentes.

La teor\'{\i}a de reducci\'on de formas definidas positivas de
\S~\ref{sec:definidas} se puede resumir en que
\begin{itemize}
	\item[(\ejername~\ref{ejer:reducidas:numero})]
		existe un conjunto finito de ``formas reducidas'',
	\item[(\lemaname~\ref{lema:reducidas})]
		actuando por transformaciones propias,
		podemos empujar una forma cuadr\'atica definida positiva
		dentro de este conjunto, pero
	\item[(\teoname~\ref{teo:reducidas})]
		no es posible pasar de una forma reducida a otra mediante
		transformaciones propias.%
		\footnote{
			Aunque s\'{\i} talvez mendiante alguna
			transformaci\'on impropia; precisamente, mediante
			$\sbmatrix{ 1 & \\ & -1 }$.
		}
	% \item En particular, dos formas definidas postivas son propiamente
		% equivalentes, si y s\'olo si son propiamente equivalentes
		% a una misma forma reducida.
\end{itemize}
%
En particular, hay un procedimiento finito que nos permite determinar
si dos formas son equivalentes: hallar sus formas reducidas y compararlas.

Si $D>0$ es un n\'umero real positivo, $\sqrt D$ denota la ra\'{\i}z
cuadrada \emph{positiva}.

\begin{obsIndefinidas}\label{obs:coeficientes:dos-de-tres}
	Si $f=\binaria{a,b,*}$ es una forma cuadr\'atica de discriminante
	$D$, entonces el coeficiente restante, $c$, est\'a determinado
	por la relaci\'on $D=b^2-4ac$.
\end{obsIndefinidas}

\begin{defIndefinidas}\label{def:indefinidas:reducida}
	Una forma indefinida $f=\binaria{a,b,c}$ de discriminante $D>0$
	no cuadrado \emph{est\'a reducida} (o \emph{es una forma reducida}),
	si sus coeficientes cumplen con las condiciones siguientes:
	\begin{displaymath}
		0\,<\,b\,<\,\sqrt D\dispand
		\big|\sqrt D-|2a|\big|\,<\,b
		\dispstop
	\end{displaymath}
	%
\end{defIndefinidas}

\begin{obsIndefinidas}\label{obs:indefinidas:reducida}
	Como asumimos que $\sqrt D\not\in\Enteros$ (y, entonces,
	$\sqrt D\not\in\Racionales$), podemos escribir las condiciones
	de la \defname~\ref{def:indefinidas:reducida} simplemente como
	\begin{displaymath}
		0\,<\,\big|\sqrt D-|2a|\big|\,<\,b\,<\,\sqrt D
		\dispstop
	\end{displaymath}
	%
	Adem\'as, fijado $D>0$, hay finitas formas reducidas de
	discriminante $D$ \quedacomoejercicio.
\end{obsIndefinidas}

\begin{lemaIndefinidas}\label{lema:vecina}
	Sea $f=\binaria{a,b,c}$ una forma indefinida de discriminante
	$D>0$ no cuadrado. Entonces, existe una \'unica forma indefinida
	de discriminante $D$, $\binaria{a',b',c'}$, determinada por las
	siguientes condiciones:
	\begin{enumerate}[(i)]
		\item\label{item:lema:vecina:primer}
			$a'=c$ y
		\item\label{item:lema:vecina:segundo}
			$b'\equiv -b\tmodulo[2a']$ y
			$\sqrt D-|2a'|<b'<\sqrt D$.
	\end{enumerate}
	%
	Adem\'as, dicha forma es propiamente equivalente a $f$.
\end{lemaIndefinidas}

\begin{proof}
	En cuanto a la \'ultima afirmaci\'on,
	si $b+b'=2c\delta$, $\delta\in\Enteros$, entonces
	\begin{displaymath}
		\binaria{a',b',c'}\,=\, \binaria{c,-b,a}\cdot
			\begin{bmatrix}
				1 & \delta \\ & 1
			\end{bmatrix}
			\,=\,\binaria{a,b,c}\cdot
				\begin{bmatrix}
					& -1 \\ 1 &
				\end{bmatrix}\,
				\begin{bmatrix}
					1 & \delta \\ & 1
				\end{bmatrix}
		\dispstop
	\end{displaymath}
	%
	El valor de $c'$ es el que tenga que ser, de acuerdo con la
	\obsname~\ref{obs:coeficientes:dos-de-tres}; conociendo $\delta$,
	$c'=\delta^2c-\delta b+a$.
\end{proof}

\begin{defIndefinidas}\label{def:vecina}
	Si $f$ es una forma indefinida de discriminante $D>0$ no cuadrado, la
	forma indefinida de discriminante $D$ que queda determinada por las
	condiciones del \lemaname~\ref{lema:vecina} se denomina \emph{forma %
	vecina a la derecha de $f$} y la denotamos por $\rneigh f$.
\end{defIndefinidas}

A cada forma indefinida de discriminante positivo no cuadrado,
le asociamos la sucesi\'on de vecinas: empezando por una forma $f$,
definimos $\{f_n\}_{n\geq 0}$ por:
\begin{displaymath}
	f_0\,:=\,f\dispand f_n\,:=\,\rneigh f_{n-1}
		\quad\text{(\phantom)si } n\geq 1\text{\phantom().}
\end{displaymath}
%
($\rneigh^n f:=f_n$).
Todos los elementos de la sucesi\'on son propiamente equivalentes a $f$.

\begin{ejemIndefinidas}\label{ejem:vecina:cuarenta}
	Sea $f=\binaria{2,8,3}$. Su discriminante es $D=40$. La forma
	vecina a la derecha es igual $\rneigh f=\binaria{3,4,*}$.
	Como $a'=c=3$, el valor de $b'$, el segundo coeficiente de 
	$\rneigh f$, debe cumplir $b'\equiv -8\tmodulo[6]$ y estar en el
	rango $\sqrt{40}-6<b'<\sqrt{40}$. Como $\piso{\sqrt{40}}=6$,
	$b'=4$. Si queremos la sucesi\'on de formas vecinas, debemos
	hallar $c'$. La condici\'on $(b')^2-4a'c'=b^2-4ac=40$ implica que
	$c'=-2$. Los primeros t\'erminos de la sucesi\'on $\rneigh^n f$ son:
	\begin{displaymath}
		\begin{aligned}
			\binaria{2,8,3}\dispcomma\quad &
				\binaria{3,4,-2}\dispcomma\quad
				\binaria{-2,4,3}\dispcomma\quad
				\binaria{3,2,-3}\dispcomma \\
			& \binaria{-3,4,2}\dispcomma\quad
				\binaria{2,4,-3}\dispcomma\quad
				\binaria{-3,2,3}\dispcomma \\
			& \binaria{3,4,-2}\dispcomma\quad\dots
		\end{aligned}
		%
	\end{displaymath}
	%
\end{ejemIndefinidas}

\begin{ejemIndefinidas}\label{ejem:vecina:cuarenta:principal}
	Sea $f=\binaria{1,0,-10}$. De nuevo, $\discriminante(f)=40$.
	La sucesi\'on de vecinas de $f$ es:
	\begin{displaymath}
		\begin{aligned}
			\binaria{1,0,-10}\dispcomma\quad
				\binaria{-10,0,1}\dispcomma\quad &
				\binaria{1,6,-1}\dispcomma\quad
				\binaria{-1,6,1}\dispcomma \\
			& \binaria{1,6,-1}\dispcomma\quad\dots
		\end{aligned}
		%
	\end{displaymath}
	%
\end{ejemIndefinidas}

\begin{ejemIndefinidas}\label{ejem:reducida:cuarenta}
	Las formas $\binaria{1,0,-10}$, $\binaria{-10,0,1}$ y
	$\binaria{2,8,3}$, todas de discriminante $40$,
	no son reducidas. Las formas
	$\binaria{1,6,-1}$, $\binaria{-1,6,1}$,
	$\binaria{3,4,-2}$, $\binaria{-2,4,3}$, $\binaria{3,2,-3}$,
	$\binaria{-3,4,2}$, $\binaria{2,4,-3}$ y $\binaria{-3,2,3}$
	est\'an reducidas. Se puede probar que son todas las posibles
	formas reducidas de discriminante $40$ \quedacomoejercicio.
	Adem\'as, del \ejemname~\ref{ejem:vecina:cuarenta}, las \'ultimas
	seis son propiamente equivalentes y, del \ejemname~%
	\ref{ejem:vecina:cuarenta:principal}, las primeras dos son
	propiamente equivalentes. As\'{\i}, hay, como mucho dos clases
	de equivalencia propia ?`Hay una s\'ola? ?`Son, todas estas ocho
	formas, propiamente equivalentes?
	% ?`Y equivalentes a secas?
\end{ejemIndefinidas}

% \begin{obsIndefinidas}\label{obs:indefinidas:reducida:simetria}
	% Si $\binaria{a,b,c}$ es una forma indefinida de discriminante
	% no cuadrado reducida, entonces $\binaria{c,b,a}$ tambi\'en
	% est\'a reducida: por un lado, asumiendo $0<\sqrt D<b$,
	% la condici\'on $\big|\sqrt D-|2a|\big|<b$ se puede reescribir
	% como $\sqrt D-b<|2a|<\sqrt D+b$;
	% por otro, como $(2a)\,(2c)=(\sqrt D-b)\,(\sqrt D+b)$,
	% vale que $|2a|\,|2c|=(\sqrt D-b)\,(\sqrt D+b)$, si
	% $0<b<\sqrt D$.
% \end{obsIndefinidas}
% 
% \begin{obsIndefinidas}\label{obs:indefinidas:reducida:signo}
	% Si $\binaria{a,b,c}$ es una forma indefinida de discriminante
	% no cuadrado reducida, entonces $b<\sqrt D$ y, por lo tanto, $ac<0$.
% \end{obsIndefinidas}

\begin{lemaIndefinidas}\label{lema:indefinidas:reducida}
	Sea $\binaria{a,b,c}$ una forma indefinida de discriminante
	positivo no cuadrado. Entonces,
	\begin{enumerate}[(i)]
		\item\label{item:lema:indefinidas:reducida:simetria}
			la forma $\binaria{c,b,a}$ tambi\'en est\'a reducida;
		\item\label{item:lema:indefinidas:reducida:signo}
			se cumple $ac<0$.
	\end{enumerate}
	%
\end{lemaIndefinidas}

\begin{coroIndefinidas}\label{coro:indefinidas:reducidas}
	Si $D\equiv 0,1\tmodulo[4]$, $D>0$, no cuadrado, entonces
	la cantidad de formas reducidas de discriminante $D$ es par.
	Adem\'as, para enumerar las formas reducidas, es suficiente
	enumerar aquellas con $a>0$ y luego simetrizar.%
	\footnote{
		$\binaria{a,b,c}\mapsto\binaria{c,b,a}$.
	}
\end{coroIndefinidas}

\begin{ejemIndefinidas}\label{ejem:reducidas:cinco}
	Hay dos formas reducidas de discriminante $5$:
	$\binaria{1,1,-1}$ y $\binaria{-1,1,1}$. Adem\'as, son
	formas propiamente equivalentes: son vecinas.
\end{ejemIndefinidas}

\begin{ejemIndefinidas}\label{ejem:reducidas:veintinueve}
	Hay ocho formas reducidas de discriminante $33$.
	Adem\'as, empezando por $\binaria{1,5,-2}$ obtenemos la sucesi\'on
	de formas vecinas
	\begin{displaymath}
		\binaria{1,5,-2}\dispcomma\quad
			\binaria{-2,3,3}\dispcomma\quad
			\binaria{3,3,-2}\dispcomma\quad
			\binaria{-2,5,1}\dispcomma\quad
			\binaria{1,5,-2}\dispcomma\quad\dots
	\end{displaymath}
	%
	y, empezando por $\binaria{2,5,-1}$ obtenemos la sucesi\'on
	\begin{displaymath}
		\binaria{2,5,-1}\dispcomma\quad
			\binaria{-1,5,2}\dispcomma\quad
			\binaria{2,3,-3}\dispcomma\quad
			\binaria{-3,3,2}\dispcomma\quad
			\binaria{2,5,-1}\dispcomma\quad\dots
	\end{displaymath}
	%
	A diferencia de lo que ocurre en el \ejemname~%
	\ref{ejem:vecina:cuarenta} y en el \ejemname~%
	\ref{ejem:vecina:cuarenta:principal} de formas de discriminante $40$,
	las sucesiones de formas vecinas de discriminante $33$
	\emph{no contienen} formas $\binaria{a,b,c}$ y $\binaria{-a,b,-c}$
	simult\'aneamente.
	% ; si $\binaria{a,b,c}$ es reducida,
	% $\binaria{-a,b,-c}$ tambi\'en lo es, pero podr\'{\i}an
	% no ser vecinas.
\end{ejemIndefinidas}

\begin{teoIndefinidas}\label{teo:indefinidas}
	Sea $D\equiv 0,1\tmodulo[4]$, $D>0$, no cuadrado. Entonces,
	\begin{enumerate}[(i)]
		\item\label{item:teo:indefinidas:vecina}
			la operaci\'on $\rneigh$ induce una permutaci\'on
			del conjunto (finito) de formas reducidas
			de discriminante $D$;
		\item\label{item:teo:indefinidas:reducida}
			una forma de discriminante $D$ est\'a reducida,
			si y s\'olo si existe $N\geq 1$ tal que $f_N=f$;
		\item\label{item:teo:indefinidas:equivalencia}
			dos formas de discriminante $D$ reducidas, $f$ y $g$,
			son propiamente equivalentes, si y s\'olo si
			existe $M\geq 0$ tal que $f_M=g$.
	\end{enumerate}
	%
\end{teoIndefinidas}

\begin{coroIndefinidas}\label{coro:indefinidas}
	Sea $D\equiv 0,1\tmodulo[4]$, $D>0$, no cuadrado y sea $f$ una
	forma de discriminante $D$. Entonces,
	\begin{enumerate}[(i)]
		\item\label{item:coro:indefinidas:reduccion}
			existe $m\geq 0$ tal que $f_m=\binaria{a,b,c}$
			est\'a reducida y $|2a|<\sqrt D$;
		\item\label{item:coro:indefinidas:orbita}
			la sucesi\'on de formas vecinas $\{f_n\}_{n\geq 0}$
			contiene s\'olo una cantidad finita de formas
			distintas;
		\item\label{item:coro:indefinidas:comparacion}
			una forma de discriminante $D$, $g$, es propiamente
			equivalente a $f$, si y s\'olo si existen
			$M,N\geq 0$ tales que $f_M=g_N$.
	\end{enumerate}
	%
	En particular, de \eqref{item:coro:indefinidas:reduccion},
	se deduce que toda forma de discriminante $D$ es propiamente
	equivalente a una forma reducida.
\end{coroIndefinidas}

\subsection*{Ejercicios}
\theoremstyle{definition}
\newtheorem{ejerIndefinidas}{\ejername}[section]

%-------------

\begin{ejerIndefinidas}
	Hallar todas las formas reducidas de discriminante
	\begin{displaymath}
		\begin{aligned}
			D & \in\{5,13,17,21,29,33,37,41,45,53,57\}\dispand \\
			D & \in\{8,12,20,24,28,32,40,44,48,52,56\}
			\dispstop
		\end{aligned}
		%
	\end{displaymath}
	%
	Para cada forma reducida hallar su sucesi\'on de formas vecinas.
	% Para $D=32$ hay tres \'orbitas:
	% $\{\binaria{1,4,-4},\binaria{-4,4,1}\}$
	% $\{\binaria{-1,4,4},\binaria{4,4,-1}\}$
	% $\{\binaria{2,4,-2},\binaria{-2,4,2}\}$
	% Las pirmeras dos no contienen formas $\binaria{a,b,c}$ y
	% $\binaria{-a,b,-c}$ simult\'aneamente; la tercera s\'{\i}\dots
	% !`Pero esta \'ultima no es de formas primitivas!
	Determinar las clases de equivalencia propia y la cantidad de
	clases que contienen formas primitivas.
\end{ejerIndefinidas}



