\theoremstyle{plain}
\newtheorem{teoGenerosB}{\teoname}[section]
\newtheorem{lemaGenerosB}[teoGenerosB]{\lemaname}
\newtheorem{coroGenerosB}[teoGenerosB]{\coroname}

\theoremstyle{definition}
\newtheorem{obsGenerosB}[teoGenerosB]{\obsname}
\newtheorem{defGenerosB}[teoGenerosB]{\defname}

%-------------

\begin{obsGenerosB}\label{obs:generos:representacion}
	Sea $D\equiv 0,1\tmodulo[4]$, $D<0$, y sean $f,g$ formas
	de discriminante $D$ pertenecientes al mismo g\'enero.
	Si $c\in\Unidadesmod[D]$ es representada por $f$, entonces
	es representada por $g$. Dicho de otra manera, si
	$m\in c$ y $m=f(x,y)$ para ciertos $x,y\in\Enteros$, entonces
	existen $n\in c$ y $x',y'\in\Enteros$ tales que
	$n=g(x',y')$.
\end{obsGenerosB}

\begin{defGenerosB}\label{def:generos:representacion}
	Sea $D\equiv 0,1\tmodulo[4]$, $D<0$, y sea $f$ una forma
	de discriminante $D$. Si $c\in\Unidadesmod[D]$ es
	representada por $f$, decimos tambi\'en que
	\emph{la clase $c$ es representada por el g\'enero de $f$}.
\end{defGenerosB}

\begin{defGenerosB}\label{def:generos:principal}
	Sea $D\equiv 0,1\tmodulo[4]$, $D<0$. Llamamos
	\emph{forma principal (de discriminante $D$)} a la forma
	\begin{itemize}
		\item $\binaria{1,0,-D/4}$, si $D\equiv 0$, y a la forma
		\item $\binaria{1,1,(1-D)/4}$, si $D\equiv 1$.
	\end{itemize}
	%
	El \emph{g\'enero principal (de discriminante $D$)} es el
	g\'enero al que pertenece la forma principal.
\end{defGenerosB}

\begin{obsGenerosB}\label{obs:generos:principal}
	Dado un discriminante, la forma principal de dicho discriminante
	es una forma primitiva reducida.
\end{obsGenerosB}

\begin{lemaGenerosB}\label{lema:generos}
	Sea $D\equiv 0,1\tmodulo[4]$, $D<0$, y sea $f$ una forma
	de discriminante $D$. Entonces,
	\begin{enumerate}[(i)]
		\item\label{item:lema:generos:principal}
			los valores en $\Unidadesmod[D]$ representados por
			el g\'enero principal constituyen un subgrupo
			$H\subgrpeq\ker(\chi)$;
		\item\label{item:lema:generos:coclases}
			los valores en $\Unidadesmod[D]$ representados por
			el g\'enero de $f$ constituyen una coclase de $H$
			en $\ker(\chi)$.
	\end{enumerate}
	%
\end{lemaGenerosB}

\begin{teoGenerosB}\label{teo:generos}
	Sea $D\equiv 0,1\tmodulo[4]$, $D<0$, y sea $H\subgrpeq\ker(\chi)$
	como en el \lemaname~\ref{lema:generos}. Si $H'\subset\ker(\chi)$
	es una coclase de $H$ en $\ker(\chi)$ y $p$ es un primo impar
	que no divide a $D$, entonces su clase de congruencia
	$\clase p\in H'$, si y s\'olo si $p$ es representado por una forma
	de discriminante $D$ perteneciente al g\'enero correspondiente a $H'$.
\end{teoGenerosB}

\begin{obsGenerosB}\label{obs:generos:correspondencia}
	Dada una coclase $H'$ de $H$ en $\ker(\chi)$, las clases de
	congruencia contenidas en $H'$ son representadas por formas
	de discriminante $D$ (pues $H'\subset\ker(\chi)$; ver el
	\coroname~\ref{coro:dirichlet}).
	Dada una forma (primitiva) de discriminante $D$, $f$,
	las clases en $\Unidadesmod[D]$ representadas por $f$
	constituyen una coclase de $H$ en $\ker(\chi)$
	(\lemaname~\ref{lema:generos}~\eqref{item:lema:generos:coclases}).
	Existe, entonces, una biyecci\'on
	\begin{displaymath}
		\big(
			\text{coclases de }
			H\text{ en }\ker(\chi)
		\big)\,\simeq\,
		\big(
			\text{g\'eneros de formas primitivas %
				de discriminante } D
		\big)
	\end{displaymath}
	%
	dada por asignar, a una forma (primitiva de discriminante $D$) $f$,
	el conjunto de enteros que ella representa
	% y, luego, las clases de congruencia m\'odulo $D$
	y, a cada entero $m$ coprimo con $D$, el conjunto de formas
	(primitivas de discriminante $D$) que lo representan.
	% y, luego, el g\'enero al cual estas formas pertenecen
\end{obsGenerosB}

\subsection*{Ejercicios}
\theoremstyle{definition}
\newtheorem{ejerGenerosB}{\ejername}[section]

%-------------

\begin{ejerGenerosB}\label{ejer:generos:b}
	Para $D\in\{-40,-60,-84,-88,-92,-120\}$, describir los g\'eneros
	de formas cuadr\'aticas de discriminante $D$.%
	\hint{
		Recordar que dos coclases distintas son disjuntas, tiene
		todas el mismo cardinal y su uni\'on es todo.
	}
\end{ejerGenerosB}

\begin{ejerGenerosB}\label{ejer:generos:b:primos}
	Dar condiciones sobre un primo $p$ para que se cumpla
	$p=x^2+ny^2$, para $n\in\{6,10,13,15,21,22,30\}$.
\end{ejerGenerosB}

\begin{ejerGenerosB}\label{ejer:generos:b:congruencias}
	Sea $D\equiv 0,1\tmodulo[4]$ un discriminante y sea
	$\tilde H\subset\Unidadesmod[D]$ el subconjunto de clases
	$x$ que verifican las siguientes condiciones:
	\begin{enumerate}[(i)]
		\item\label{item:generos:b:congruencias:condiciones:i}
			para todo primo impar $q\mid D$,
			$\tlegendre x q=1$, y
		\item\label{item:generos:b:congruencias:condiciones:ii}
			la clase es
			\begin{displaymath}
				x\,\equiv\,
				\left\{
					\begin{array}{ll}
						x\equiv 1\tmodulo[4] \dispcomma
							& \text{si }D\equiv 12 \tmodulo[16] \dispcomma \\
						x\equiv 1\tmodulo[4] \dispcomma
							& \text{si }D\equiv 16 \tmodulo[32] \dispcomma \\
						x\equiv 1\tmodulo[8] \dispcomma
							& \text{si }D\equiv 0 \tmodulo[32] \dispcomma \\
						x\equiv 1,7\tmodulo[8]\dispcomma
							& \text{si }D\equiv 8 \tmodulo[32] \quad\text{y} \\
						x\equiv 1,3\tmodulo[8]\dispcomma
							& \text{si }D\equiv 24 \tmodulo[32]
				\dispstop

					\end{array}
				\right.
			\end{displaymath}
			%
	\end{enumerate}
	%
	Notar que la condici\'on
	\eqref{item:generos:b:congruencias:condiciones:ii}
	es vac\'{\i}a, si $D\equiv 1\tmodulo[4]$ o si $D=-4n$ con
	$n\equiv 3\tmodulo[4]$.
	\begin{enumerate}[(i)]
		\item\label{item:generos:b:congruencias:i}
			Probar que $\tilde H$ es subgrupo de
			$\Unidadesmod[D]$ y
			que $\tilde H\subset\ker(\chi)$.
		\item\label{item:generos:b:congruencias:ii}
			Probar que $\tilde H=H$,%
			\hint{
				En el caso $D=-4n$,
				traducir la condici\'on sobre $D$ a $n$ y
				mostrar que, si $n'$ es la parte impar
				de $n$, entonces la condici\'on
				$\tlegendre x q=1$ para todo primo impar
				$q\mid n$ equivale a que $x$ sea
				cuadrado m\'odulo $n'$.
			}
			donde $H\subset\Unidadesmod[D]$ es el subconjunto
			\begin{displaymath}
				H\,=\,\left\{
				\begin{array}{ll}
					\big\{
						x\equiv\beta^2\text{ o }
						\beta^2+n\tmodulo[4n]\,:\,
						\beta\in\Enteros\big\}
						\dispcomma &
						\text{si }D=-4n\dispcomma
						\\[5pt]
					\big\{
						x\equiv\beta^2\tmodulo[D]\,:\,
						\beta\in\Enteros\big\}
						\dispcomma &
						\text{si }D\equiv 1
							\tmodulo[4]
						\dispstop
				\end{array}
				%
				\right.
			\end{displaymath}
			%
			\dispstop
	\end{enumerate}
	%
	% (Notar que, por el \coroname~\ref{coro:generos:cero} y el
	% \coroname~\ref{coro:generos:uno}, $H\subgrpeq\ker(\chi)$,
	% con lo que \eqref{item:generos:b:congruencias:ii} implica
	% \eqref{item:generos:b:congruencias:i}
	% pero que tambi\'en
	% \eqref{item:generos:b:congruencias:i}
	% se puede demostrar de manera independiente).
\end{ejerGenerosB}



