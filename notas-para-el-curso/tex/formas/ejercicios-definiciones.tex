\theoremstyle{definition}
\newtheorem{ejerDefiniciones}{\ejername}[section]

%-------------

\begin{ejerDefiniciones}\label{ejer:definiciones:primitivamente}
	Si $f(x,y)=m$ es una representaci\'on de $m$ por $f$ y
	$g=\mcd{x,y}$, entonces $g^2\mid m$ y $f$ representa
	primitivamente a $m/g^2$.
	En particular, si $m$ es libre de cuadrados (por ejemplo,
	si $m$ es primo o $m=\pm 1$), s\'olo tiene sentido hablar
	de representaciones primitivas de $m$.
\end{ejerDefiniciones}

	Las partes \eqref{item:definiciones:invariantes:discriminante}
	y \eqref{item:definiciones:invariantes:contenido} del \teoname~%
	\ref{teo:definiciones:equivalencia:invariantes}
	se pueden expresar de la siguiente manera: el discriminante y
	el contenido de una forma cuadr\'atica son invariantes de la
	clase de equivalencia de la forma cuadr\'atica (todas las formas
	en la misma clase tienen igual discriminante e igual contenido).
	Sea $\varClases(D)$ el conjunto de clases de formas cuadr\'aticas
	de discriminante $D$, positivas, si $D<0$,
	y sea $\varClases[g](D)$ el subconjunto de
	clases de discriminante $D$ y contenido $g$.
	En particular, con esta notaci\'on $\varClases[1](D)$ denota el
	conjunto de clases de formas cuadr\'aticas primitivas de
	discriminante $D$ (positivas, si $D<0$).%
	\footnote{
		Comparar con la \defname~%
		\ref{def:representaciones:gaussianas}.
	}

\begin{ejerDefiniciones}\label{ejer:definiciones:clases}
	Mostrar que, si $f$ es una forma de contenido $g$ y discriminante
	$D$, entonces $g^2\mid D$. Concluir que
	\begin{displaymath}
		\varClases(D)\,=\,
		\bigsqcup_{\smatrix{g>0 \\ g^2\mid D}}\,\varClases[g](D)
			\,=\,\bigsqcup_{\smatrix{g>0 \\ g^2\mid D}}\,
				\varClases[1](D/g^2)
		\text{ .}
	\end{displaymath}
	%
\end{ejerDefiniciones}

	Un \emph{discriminante fundamental} es un entero $D\neq 0$ que cumple
	\begin{itemize}
		\item $D\equiv 1\tmodulo[4]$ y es libre de cuadrados,
			o bien
		\item $D=4m$, donde $m\equiv 2,3\tmodulo[4]$ y es libre
			de cuadrados.
	\end{itemize}
	%
	Equivalentemente, un discriminante fundamental es un discriminante
	\emph{minimal}, es decir,
	$D\equiv 0,1\tmodulo[4]$, $D\neq 0$, y no existen enteros
	$D_0\equiv 0,1\tmodulo[4]$ y $f>1$ tales que $D=D_0f^2$.

\begin{ejerDefiniciones}\label{ejer:definiciones:fundamental}
	Probar que un entero $D\equiv 0,1\tmodulo[4]$, $D\neq 0$,
	es un discriminante fundamental, si y s\'olo si toda forma
	de discriminante $D$ es primitiva.
\end{ejerDefiniciones}

El producto de formas lineales da lugar a formas cuadr\'aticas:
\begin{equation}
	\label{eq:definiciones:descomponible}
	(kx+ly)\,(mx+ny)\,=\,kmx^2\,+\,(kn+lm)\,xy\,+\,lny^2
\end{equation}
%
es una forma cuadr\'atica. Veremos que \'estas son muy especiales
dentro del conjunto de todas las formas cuadr\'aticas. Desde el punto
de vista del problema de representabilidad, determinar si una forma
como en \eqref{eq:definiciones:descomponible} representa un entero
$z\in\Enteros$ se reduce a descomponerlo como producto de enteros
de alguna manera, $z=mn$, $m,n\in\Enteros$, y determinar si es posible
representar los factores $m$ y $n$ por cada uno de los factores lineales
de la forma.

\begin{ejerDefiniciones}\label{ejer:definiciones:descomponible}
	Una forma cuadr\'atica es un producto de formas lineales,
	si y s\'olo si su discriminante es un cuadrado perfecto.
	\begin{enumerate}[(i)]
		\item\label{item:definiciones:descomponible:i}
			Probar que el discriminante de la forma
			\eqref{eq:definiciones:descomponible} es igual a
			$(kn-lm)^2$, o sea, un cuadrado perfecto.
		\item\label{item:definciones:descomponible:ii}
			Probar que, si $f=\binaria{a,b,c}$ y
			$\discriminante(f)=h^2$ es un cuadrado, entonces
			\begin{displaymath}
				4af\,=\,\big(2ax+(b+h)\,y\big)\,
					\big(2ax+(b-h)\,y\big)
				\text{ ;}
			\end{displaymath}
			%
			en particular,
			sobre $\Racionales$, $f$ se descompone como
			producto de factores lineales.
		\item\label{item:definciones:descomponible:iii}
			Si $f=\binaria{a,b,c}$ es una forma cuadr\'atica
			con coeficientes enteros y existe una
			factorizaci\'on
			\begin{displaymath}
				f\,=\,(\kappa x+\lambda y)\,(\mu x+\nu y)
			\end{displaymath}
			%
			con $\kappa,\lambda,\mu,\nu\in\Racionales$, entonces
			existe una factorizaci\'on
			\begin{displaymath}
				f\,=\,(kx+ly)\,(mx+ny)
			\end{displaymath}
			%
			con $k,l,m,n\in\Enteros$.%
			\hint{
				Escribir $\tilde nf=(kx+ly)(mx+ny)$ con
				$k,l,m,n,\tilde n\in\Enteros$; dividir
				por el m\'aximo com\'un divisor entre
				$k$, $l$, y $\tilde n$; dividir por el
				m\'aximo com\'un divisor entre
				$m$, $n$ y (el nuevo) $\tilde n$;
				comparar coeficientes:
				$\tilde na=km$, $\tilde nb=kn+lm$ y
				$\tilde nc=ln$; concluir que $\tilde n=1$.
			}
	\end{enumerate}
	%
\end{ejerDefiniciones}

A toda forma cuadr\'atica le podemos asociar una matriz, definida
a partir de los coeficientes de la forma: si $f=\binaria{a,b,c}$,
su \emph{matriz asociada} es%
\footnote{
	A veces se llama ``matriz asociada'' a la matriz
	\begin{math}
		\sbmatrix{a & b \\ b & c}
	\end{math}.
}
\begin{equation}
	\label{eq:definiciones:matriz-asociada}
	\begin{bmatrix}
		a & b/2 \\ b/2 & c
	\end{bmatrix}
	\text{ .}
\end{equation}
%
La matriz asociada a una forma cuadr\'atica es una matriz
sim\'etrica. Si $f$ es entera, los coeficientes de su matriz
asociada ser\'an enteros en la diagonal y medio enteros fuera de
la diagonal. Rec\'{\i}procamente, toda matriz sim\'etrica como
\eqref{eq:definiciones:matriz-asociada}, con $a,b,c\in\Enteros$
determina un forma cuadr\'atica entera:
\begin{equation}
	\label{eq:definiciones:forma-asociada}
	f(x,y)\,=\,\trnsp{\begin{bmatrix} x \\ y \end{bmatrix}}\,
		\begin{bmatrix} a & b/2 \\ b/2 & c \end{bmatrix}\,
		\begin{bmatrix} x \\ y \end{bmatrix}
	\text{ ,}
\end{equation}
%
donde $\trnsp v$ denota la matriz $v$ transpuesta. En definitiva,
hay una correspondencia entre formas cuadr\'aticas binarias con
coeficientes enteros y matrices ``medio enteras''.
El discriminante de una forma se expresa de manera
sencilla en t\'erminos del determinante de su matriz asociada:
dada una forma $f$ con matriz asociada $F$,
\begin{displaymath}
	\discriminante(f)\,=\,-4\det(F)
	\text{ .}
\end{displaymath}
%

\begin{ejerDefiniciones}\label{ejer:definiciones:equivalencia}
	Sea $f=\binaria{a,b,c}$ una forma cuadr\'atica.
	Probar que la matriz asociada a
	$f(px+qy,rx+sy)$ es igual a
	\begin{displaymath}
		\trnsp{\begin{bmatrix}
			p & q \\ r & s
		\end{bmatrix}}\,
		\begin{bmatrix}
			a & b/2 \\ b/2 & c
		\end{bmatrix}\,
		\begin{bmatrix}
			p & q \\ r & s
		\end{bmatrix}
		\text{ .}
	\end{displaymath}
	%
\end{ejerDefiniciones}

\begin{ejerDefiniciones}\label{ejer:definiciones:accion}
	Probar que el grupo $\GL(2,\Enteros)$ de matrices de tama\~no
	$2\times 2$ inversibles con coeficientes enteros act\'ua en el
	conjunto de matrices medio enteras v\'{\i}a
	$X\mapsto \trnsp AXA$ (la acci\'on es a derecha).
	Concluir que dos formas cuadr\'aticas, $f$ y $f_1$,
	son equivalentes, si y s\'olo si sus matrices asociadas, $F$ y
	$F_1$, cumplen que existe $A\in\GL(2,\Enteros)$ tal que
	$F_1=\trnsp AFA$. Probar que $f$ y $f_1$ son propiamente
	equivalentes si existe $A\in\SL(2,\Enteros)$ tal que
	$F_1=\trnsp AFA$. Demostrar que las relaciones de equivalencia y
	de equivalencia propia entre formas cuadr\'aticas son,
	efectivamente, relaciones de equivalencia.
\end{ejerDefiniciones}

Si $f$ es una forma cuadr\'atica y $\gamma=\sbmatrix{ p & q \\ r & s }$,
$f\cdot\gamma$ denota la forma
\begin{displaymath}
	(f\cdot\gamma)(x,y)\,=\,f(px+qy,rx+sy)
	\text{ .}
\end{displaymath}
%
% Esto define una acci\'on de $\GL(2,\Enteros)$ en el conjunto de formas
% cuadr\'aticas binarias con coeficientes enteros.
Dos formas, $f$ y $f_1$, son equivalentes, si y s\'olo si
existe $A\in\GL(2,\Enteros)$ tal que $f_1=f\cdot A$;
si $A\in\SL(2,\Enteros)$, entonces son estictamente equivalentes.

Dada una forma $f$, su \emph{grupo de isotrop\'{\i}a} es
\begin{displaymath}
	\Stab(f)^+\,=\,\big\{\gamma\in\SL(2,\Enteros)\,:\,
		f\cdot\gamma=f\big\}
	\text{ .}
\end{displaymath}
%

\begin{ejerDefiniciones}\label{ejer:definiciones:isotropia:conjugado}
	Probar que $\Stab(f)^+$ es un subgrupo de $\SL(2,\Enteros)$ y que,
	si $\gamma\in\SL(2,\Enteros)$, entonces
	\begin{displaymath}
		\Stab(f\cdot\gamma)^+\,=\,\gamma^{-1}\Stab(f)^+\gamma
		\text{ .}
	\end{displaymath}
	%
\end{ejerDefiniciones}

\begin{ejerDefiniciones}\label{ejer:definiciones:isotropia}
	Dada una forma cuadr\'atica $f=\binaria{a,b,c}$ de discriminante
	$D=b^2-4ac$, el grupo $\Stab(f)^+$ contiene todas las matrices
	de la forma
	\begin{equation}
		\label{eq:definiciones:isotropia}
		\begin{bmatrix}
			\frac{u-bv} 2 & -cv \\
			av & \frac{u+bv} 2
		\end{bmatrix}
		\text{ ,}
	\end{equation}
	%
	donde el par $u,v\in\Enteros$ satisface
	\begin{equation}
		\label{eq:definiciones:isotropia:pell}
		u^2\,-\,Dv^2\,=\,4
		\text{ .}
	\end{equation}
	%
	Si $f$ es primitiva, entonces \'estas son todas las matrices
	$\gamma\in\SL(2,\Enteros)$ tales que $f\cdot\gamma=f$.%
	\hint{
		Se puede corroborar que las matrices de la forma
		\eqref{eq:definiciones:isotropia} preservan $f$,
		a partir de las f\'ormulas
		\eqref{eq:definiciones:equivalencia:cambio} para los
		coeficientes de $f\cdot\gamma$.
		Sea $\gamma=\sbmatrix{ p & q \\ r & s }\in\Stab(f)^+$.
		Entonces,
		\begin{displaymath}
			\begin{aligned}
				a & \,=\,ap^2\,+\,bpr\,+\,cr^2
					\quad\text{y} \\
				b & \,=\,2apq\,+\,b\,(ps+qr)\,+\,2crs
					\,=\, 2apq\,+\,b\,(1+2qr)\,+\,2crs
				\text{ .}
			\end{aligned}
			%
		\end{displaymath}
		%
		De estas ecuaciones,
		\begin{displaymath}
			0\,=\,apq\,+\,bqr\,+\,crs
				\text{ ,}\quad
			aq\,=\,-cr \quad\text{y}\quad
			as\,=\,ap\,+\,br
			\text{ .}
		\end{displaymath}
		%
		O sea, $aq=-cr$ y $a\,(s-p)=br$. En particular,
		$a\mid cr$ y $a\mid br$. \emph{Ahora}, asumiendo
		$\mcd{a,b,c}=1$, se deduce que $a\mid r$.
		Si escribimos $r=av$, entonces $q=-cv$ y
		$s-p=bv$. De esto y de $ps-qr=1$, se puede ver que
		$(p+s)^2=Dv^2+4$. Elegir, entonces, $u=p+s$.
	}
\end{ejerDefiniciones}

% \begin{ejerDefiniciones}
	% Sea $f$ la forma cuadr\'atica $f=\binaria{2,1,3}$; su discriminante
	% es $-23$. Notar que
	% \begin{displaymath}
		% a\,-\,b\,+\,c\,=\,4
		% \dispcomma
		% f(\pm 1,0)\,=\,2\dispand
		% f(0,\pm 1)\,=\,3
	% \end{displaymath}
	% %
	% y probar que $f(x,y)\geq 4$ en cualquier otro caso.
	% Deducir que, si $f_1(x,y)=f(px+qy,rx+sy)$ es una forma
	% equivalente, donde $p,q,r,s\in\Enteros$ y $ps-qr=-1$,
% \end{ejerDefiniciones}

