\theoremstyle{definition}
\newtheorem{ejerRepresentaciones}{\ejername}[section]

%-------------

\begin{ejerRepresentaciones}\label{ejer:representaciones:kronecker}
	Sea $D\equiv 0,1\tmodulo[4]$, $D\neq 0$, y sea $p$
	un n\'umero primo. Las siguientes afirmaciones
	son equivalentes.%
	\footnote{
		El caso $p=2$ es especial.
	}
	\begin{enumerate}[(a)]
		\item\label{item:ejer:representaciones:kronecker:representable}
			Existe una forma cuadr\'atica de discriminante
			$D$ que representa $p$.
		\item\label{item:ejer:representaciones:kronecker:nucleo}
			O bien $p\mid D$, o bien $p\nmid D$ y
			$\varkronecker[D](p)=1$.
	\end{enumerate}
	%
\end{ejerRepresentaciones}

\begin{ejerRepresentaciones}\label{ejer:representaciones:equivalentes:primo}
	Sea $p$ un n\'umero primo y sean $f$ y $f_1$ formas
	cuadr\'aticas de igual discriminante que representan,
	ambas, $p$. Probar que $f$ y $f_1$ son equivalentes
	(pero no necesariamente propiamente equivalentes).%
	\footnote{
		Separar en casos $p=2$ y $p$ impar.
	}
\end{ejerRepresentaciones}

\begin{ejerRepresentaciones}\label{ejer:representaciones:equivalentes:uno}
	Probar que, si $f$ es una forma de discriminante $D$ que
	representa $1$, entonces $f$ es propiamente equivalente a
	$\binaria{1,0,-D/4}$, si $D\equiv 0$, o a
	$\binaria{1,1,(1-D)/4}$, si $D\equiv 1$.%
	\hint{
		Mostrar que toda forma $f$ es propiamente equivalente
		a una forma $\binaria{a,b,c}$ donde $a$ es el
		menor entero positivo primitivamente representado por $f$
		y $|b|\leq a$.
	}
\end{ejerRepresentaciones}

\begin{ejerRepresentaciones}
	Sea $p$ un primo representado por una forma $f$ de discriminante
	$D$. Probar que $f$ \emph{no es} primitiva, si y s\'olo si
	$p^2\mid D$ y $D/p^2$ no es un discriminante
	(es decir, $\not\equiv 0,1\tmodulo[4]$).
\end{ejerRepresentaciones}

% Talvez mejor dejar estos ejercicios para la \S~\ref{sec:definidas}.

\begin{ejerRepresentaciones}
	Probar que las formas
	$x^2+xy+6y^2$ y $2x^2+xy+3y^2$ tienen el mismo discriminante,
	que ambas representan $m=4$, pero que no son equivalentes.
\end{ejerRepresentaciones}

\begin{ejerRepresentaciones}
	Probar que la forma $x^2+xy+2y^2$ representa un primo $p$,
	si y s\'olo si $p=7$ o $p\equiv 1,2,4\tmodulo[7]$.
\end{ejerRepresentaciones}

\begin{ejerRepresentaciones}
	Describir el conjunto de primos representados por la forma
	$x^2+xy+3y^2$.
\end{ejerRepresentaciones}
