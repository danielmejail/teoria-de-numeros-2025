\theoremstyle{plain}
\newtheorem{teoDirichlet}{\teoname}[section]

\theoremstyle{definition}

%-------------

En esta secci\'on hacemos una disgresi\'on por el
Teorema de Dirichlet sobre primos en progresiones aritm\'eticas y
lo relacionaremos con el estudio de la representabilidad por formas
cuadr\'aticas. Esto nos ser\'a \'util en la \S~\ref{sec:generos},
cuando hablemos de \emph{g\'enero} de una forma cuadr\'atica.

% Por el \lemaname~\ref{lema:representaciones:primitiva}, sabemos que
% una condici\'on necesaria y suficiente para que un entero impar $m$
% sea representado por una forma de discriminante $D$ (coprimo con $m$)
% es que $D$ sea un cuadrado m\'odulo $m$.
% Fijado $m$ podemos decidir f\'acilmente qu\'e discriminantes lo representan,
% si entendemos qu\'e clases de congruencia m\'odulo $m$ son cuadrados
% m\'odulo $m$.
% Dicho de otra manera, fijado $m$, necesitamos entender el subgrupo
% \begin{displaymath}
	% \big\{x^2 \,:\,x\in\Unidadesmod[m]\big\}\,\leq\,\Unidadesmod[m]
	% \dispstop
% \end{displaymath}
% %
Por el \lemaname~\ref{lema:representaciones:primitiva}, fijado un entero
$m$ (impar), podemos decidir f\'acilmente qu\'e discriminantes (coprimos
con $m$) lo representan:
son aquellos que pertenecen al subgrupo de cuadrados,
\begin{displaymath}
	\big\{ x^2\,:\,x\in\Unidadesmod[m] \big\}\,\leq\,\Unidadesmod[m]
	\dispstop
\end{displaymath}
%
Si $m=p$ es primo (impar), entonces
\begin{displaymath}
	\big\{ x^2\,:\,x\in\Unidadesmod[p] \big\}\,=\,
		\big\{ y\in\Unidadesmod[p]\,:\,\tlegendre y p=1\big\}
	\dispstop
\end{displaymath}
%

\emph{Rec\'{\i}procamente}, fijado un discriminante $D$,
?`existen condiciones sobre las clases de congruencia m\'odulo $D$
para determinar qu\'e enteros $m$ se pueden representar por formas de
discriminante $D$?
% El \teoname~\ref{teo:representaciones} nos da un criterio sencillo
% para determinar qu\'e \emph{primos} podemos representar por formas de
% discriminante $D$.
% Dicha condici\'on involucra la funci\'on $\chi$ del \lemaname~%
% \ref{lema:ecuacion:kronecker} y, en definitiva, se reduce a una condici\'on
% sobre la clase de congruencia del primo m\'odulo $D$.
% La funci\'on $\chi$ permite pasar de la condici\'on $\tlegendre D{p}=1$,
% que $D$ sea un cuadrado m\'odulo $p$, a $\chi(p)=1$, que es una condici\'on
% sobre la clase de congruencia de $p$ m\'odulo $D$.
% En particular, si $p$ y $q$ son primos (impares) que no dividen a $D$,
% y si $p\equiv q\tmodulo[D]$, entonces la condici\'on para que $p$ est\'e
% representado por una forma de discriminante $D$ es la misma que para $q$.
% Dicho de otra manera, si nos interesa saber si podemos representar un
% primo particular $p$ por una forma de discriminante $D$, nos bastar\'a
% con decidir si podemos representar \emph{alg\'un} primo perteneciente a
% la clase de congruencia de $p$ m\'odulo $D$.
El \teoname~\ref{teo:representaciones} nos da un criterio sencillo
para determinar qu\'e \emph{primos} podemos representar por formas
de discriminante $D$:
son aquellos que pertenecen al n\'ucleo de la funci\'on $\chi$ del
\lemaname~\ref{lema:ecuacion:kronecker},
\begin{displaymath}
	\ker(\chi)\,=\,\big\{c\in\Unidadesmod[D]\,:\,\chi(c)=1\big\}
		\,\leq\,\Unidadesmod[D]
	\dispstop
\end{displaymath}
%
En particular, si $p$ y $q$ son primos (impares) que no dividen a $D$,
y si $p\equiv q\tmodulo[D]$, entonces la condici\'on para que $p$ est\'e
representado por una forma de discriminante $D$ es la misma que para $q$.
Dicho de otra manera, si nos interesa saber si podemos representar un
primo particular $p$ por una forma de discriminante $D$, nos bastar\'a
con decidir si podemos representar \emph{alg\'un} primo perteneciente a
la clase de congruencia de $p$ m\'odulo $D$.
Cuando restringimos el problema de representaci\'on de representar
enteros a representar primos, lo que importa es la clase de congruencia
del primo.

Ahora, si, en lugar de un entero, empezamos por una clase de congruencia

Para aplicar el \teoname~\ref{teo:representaciones}, lo que importa
entonces es saber qu\'e clases 

% La noci\'on de g\'enero de una forma cuadr\'atica tiene que ver con
% \emph{qu\'e clases de congruencia} son representadas por la forma en
% cuesti\'on; precisamente, qu\'e clases de congruencia m\'odulo el
% discriminante de la forma. Si $D\equiv 0,1\tmodulo[4]$ es un discriminante
% y $m\in\Entero$ 

\begin{teoDirichlet}[Dirichlet]\label{teo:dirichlet}
	Dados n\'umeros enteros coprimos $m$ y $D$, la sucesi\'on
	\begin{displaymath}
		m\dispcomma\quad
		m+D\dispcomma\quad
		m+2 D\dispcomma\quad\dots
	\end{displaymath}
	%
	contiene infinitos n\'umeros primos.
\end{teoDirichlet}

