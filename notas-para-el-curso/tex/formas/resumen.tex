\theoremstyle{plain}

\theoremstyle{definition}
\newtheorem{ejemResumen}{\ejemname}[section]

%-------------

Consideremos el problema de representar un primo $p$ en la forma
$\binaria{1,0,n}$ ($n>0$). Cuando $n=1,2,3$, la soluci\'on queda
sintetizada en los siguientes pasos:
\begin{itemize}
	\item[(descenso)]
		si existen $x,y\in\Enteros$, $\mcd{x,y}=1$,
		tales que $p$ divide $x^2+ny^2$, entonces $p$ se
		puede expresar en la forma $x^2+ny^2$ (posiblemente
		distintos $x$ e $y$);
	\item[(reciprocidad)]
		es posible hallar enteros $\alpha$, $\beta$, \dots,
		tales que, si
		\begin{displaymath}
			p\,\equiv\,\alpha,\,\beta,\,\dots
				\modulo[4n]
			\dispcomma
		\end{displaymath}
		%
		entonces $p$ divide $x^2+ny^2$ para ciertos
		$x,y\in\Enteros$, $\mcd{x,y}=1$.
\end{itemize}
%
Los resultados de la \S~\ref{sec:ecuacion} demuestran el paso
de reciprocidad:

\teoEcuacionKronecker*

Entonces, para un n\'umero primo, dividir un entero de la forma
$x^2+ny^2$, $\mcd{x,y}=1$, es equivalente a una serie de condiciones
de congruencia (\eqref{item:ecuacion:kronecker:equivalencias:nucleo}).
Esto \'ultimo no es especial de $n\in\{1,2,3\}$.

El paso de descenso se puede describir con el lenguage introducido en
la \S~\ref{sec:representaciones}.

\coroRepresentacionesDescenso*

Equivalentemente, por la \S~\ref{sec:reducidas}, podemos relacionarlo
con la idea de forma reducida y n\'umero de clases.

\coroReducidasDescenso*

Este resultado es especialmente \'util en los casos
$\nClases(-4n)=1$, pues, entonces, la \'unica forma reducida de
discriminante $-4n$ es la forma principal $\binaria{1,0,n}$.
Sin embargo, $\nClases(-4n)=1$, si y s\'olo si
$n\in\{1,2,3,4,7\}$. Lo \'unico que queda es describir $\ker(\chi)$
en estos cinco casos.

La teor\'{\i}a de g\'eneros entra al tratar de estudiar los casos
$\nClases(-4n)>1$. Resulta que las formas cuadr\'aticas se pueden
agrupar de acuerdo a los valores $c\in\Unidadesmod[4n]$ que ellas
representan. Adem\'as, hay una estrutura algebraica subyacente en
esta manera de agruparlas.

\lemaGenerosBCoclases*

En el caso $D\equiv 0\tmodulo[4]$, este resultado tiene la siguiente
consecuencia.

\coroGenerosBDescenso*

Nuevamente, esto es particularmente \'util si el g\'enero principal
consiste \'unicamente en la forma principal, es decir, si hay
exactamente una forma reducida por g\'enero.

\begin{ejemResumen}\label{ejem:resumen}
	El g\'enero principal de formas de discriminante $-4n$
	contiene solamente la forma principal, si
	\begin{displaymath}
		n\,\in\,\{6,10,13,15,21,22,30\}
		\dispstop
	\end{displaymath}
	%
\end{ejemResumen}

Lo que resta en estos (y posiblemente otros) casos es describir el
subgrupo $H\subgrpeq\ker(\chi)$ correspondiente al g\'enero pricipal.

% We summarise the main results so far vefore proceeding.
% \begin{itemize}
	% \item Lemma 1.14, Example 1.12;
	% \item Example 1.1 is useful, Example 1.13 shows Lemma 1.14
		% implies Quadratic Reciprocity, note Lemma 1.14 is
		% proved using QR (!);
	% \item Lemma 2.5, Proposition 2.15 using Theorem 2.9 and
		% Theorem 2.13;
	% \item Theorem 2.16;
	% \item Lemma 2.24, Theorem 2.26, Corollary 2.27 and Example 2.20
% \end{itemize}
% %

\subsection*{Ejercicios}
\theoremstyle{definition}
\newtheorem{ejerResumen}{\ejername}[section]

%-------------

\begin{ejerResumen}\label{ejer:resumen:congruencias}
	Sea $D\equiv 0,1\tmodulo[4]$ un discriminante y sea
	$\tilde H\subset\Unidadesmod[D]$ el subconjunto de clases
	$x$ que verifican las siguientes condiciones:
	\begin{itemize}
		\item%\label{item:generos:b:congruencias:condiciones:i}
			para todo primo impar $q\mid D$,
			$\tlegendre x q=1$, y
		\item%\label{item:generos:b:congruencias:condiciones:ii}
			la clase es
			\begin{displaymath}
				x\,\equiv\,
				\left\{
					\begin{array}{ll}
						x\equiv 1\tmodulo[4] \dispcomma
							& \text{si }D\equiv 12 \tmodulo[16] \dispcomma \\
						x\equiv 1\tmodulo[4] \dispcomma
							& \text{si }D\equiv 16 \tmodulo[32] \dispcomma \\
						x\equiv 1\tmodulo[8] \dispcomma
							& \text{si }D\equiv 0 \tmodulo[32] \dispcomma \\
						x\equiv 1,7\tmodulo[8]\dispcomma
							& \text{si }D\equiv 8 \tmodulo[32] \quad\text{y} \\
						x\equiv 1,3\tmodulo[8]\dispcomma
							& \text{si }D\equiv 24 \tmodulo[32]
				\dispstop

					\end{array}
				\right.
			\end{displaymath}
			%
	\end{itemize}
	%
	Notar que la \'ultima condici\'on
	% \eqref{item:generos:b:congruencias:condiciones:ii}
	es vac\'{\i}a, si $D\equiv 1\tmodulo[4]$ o si $D=-4n$ con
	$n\equiv 3\tmodulo[4]$.
	\begin{enumerate}[(i)]
		\item\label{item:generos:b:congruencias:i}
			Probar que $\tilde H$ es subgrupo de
			$\Unidadesmod[D]$ y
			que $\tilde H\subset\ker(\chi)$.
		\item\label{item:generos:b:congruencias:ii}
			Probar que $\tilde H=H$,%
			\hint{
				En el caso $D=-4n$,
				traducir la condici\'on sobre $D$ a $n$ y
				mostrar que, si $n'$ es la parte impar
				de $n$, entonces la condici\'on
				$\tlegendre x q=1$ para todo primo impar
				$q\mid n$ equivale a que $x$ sea
				cuadrado m\'odulo $n'$.
			}
			donde $H\subset\Unidadesmod[D]$ es el subconjunto
			\begin{displaymath}
				H\,=\,\left\{
				\begin{array}{ll}
					\big\{
						x\equiv\beta^2\text{ o }
						\beta^2+n\tmodulo[4n]\,:\,
						\beta\in\Enteros\big\}
						\dispcomma &
						\text{si }D=-4n\dispcomma
						\\[5pt]
					\big\{
						x\equiv\beta^2\tmodulo[D]\,:\,
						\beta\in\Enteros\big\}
						\dispcomma &
						\text{si }D\equiv 1
							\tmodulo[4]
						\dispstop
				\end{array}
				%
				\right.
			\end{displaymath}
			%
			\dispstop
	\end{enumerate}
	%
	% (Notar que, por el \coroname~\ref{coro:generos:cero} y el
	% \coroname~\ref{coro:generos:uno}, $H\subgrpeq\ker(\chi)$,
	% con lo que \eqref{item:generos:b:congruencias:ii} implica
	% \eqref{item:generos:b:congruencias:i}
	% pero que tambi\'en
	% \eqref{item:generos:b:congruencias:i}
	% se puede demostrar de manera independiente).
\end{ejerResumen}



