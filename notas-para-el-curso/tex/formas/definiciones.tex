\theoremstyle{plain}
\newtheorem{teoDefiniciones}{\teoname}[section]
\newtheorem{lemaDefiniciones}[teoDefiniciones]{\lemaname}
\newtheorem{coroDefiniciones}[teoDefiniciones]{\coroname}

\theoremstyle{definition}
\newtheorem{defDefiniciones}[teoDefiniciones]{\defname}
\newtheorem{obsDefiniciones}[teoDefiniciones]{\obsname}
\newtheorem{ejemDefiniciones}[teoDefiniciones]{\ejemname}

%-------------

Empecemos con la definici\'on de forma cuadr\'atica.

% \subsection{Definici\'on de forma cuadr\'atica}
	% \label{subsec:definicion-de-forma}
\begin{defDefiniciones}\label{def:definiciones:forma}
	Por \emph{forma cuadr\'atica binaria} entenderemos un
	polinomio en dos variables, homog\'eneo de grado $2$:
	\begin{displaymath}
		ax^2\,+\,bxy\,+\,cy^2
		\text{ .}
	\end{displaymath}
	%
	Los coeficientes del polinomio, $a$, $b$ y $c$, son los
	\emph{coeficientes} de la forma cuadr\'atica binaria.
	Usaremos la notaci\'on
	\begin{displaymath}
		\binaria{a,b,c}\,:=\,ax^2\,+\,bxy\,+\,cy^2
		\text{ .}
	\end{displaymath}
	%
\end{defDefiniciones}

% El concepto se puede generalizar a una cantidad arbitraria de
% variables. Dado que s\'olo estudiaremos formas cuadr\'aicas en
% dos variables, diremos ``forma cuadr\'atica'' en lugar de
% ``forma cuadr\'atica binaria''.

% \subsection{La funci\'on asociada a una forma cuadr\'atica}%
	% \label{subsec:la-funcion-asociada-a-una-forma}
% \begin{obsDefiniciones}\label{obs:definiciones:forma}
	% Toda forma cuadr\'atica se puede interpretar como una
	% funci\'on. Para eso, deber\'{\i}amos, por lo menos, especificar
	% su dominio. Si $a,b,c\in\Reales$, podemos pensar en la funci\'on
	% $f(x,y)=ax^2+bxy+cy^2$ cuyos argumentos $x$ e $y$ son n\'umeros
	% reales y obtenemos una funci\'on
	% $\Reales\times\Reales\rightarrow\Reales$. Sin embargo, la misma
	% expresi\'on puede interpretarse como una funci\'on
	% $\Complejos\times\Complejos\rightarrow\Complejos$, si permitimos
	% que los argumentos sean n\'umeros complejos.
	% Si $a,b,c\in\Complejos$, podemos pensar en la funci\'on
	% $\Reales\times\Reales\rightarrow\Complejos$ cuyos argumentos son
	% reales, o bien en la funci\'on de argumentos complejos.
	% A pesar de esta ambig\"uedad, nos resultar\'a \'util pensar en
	% una forma cuadr\'atica como una funci\'on, pero la situaci\'on
	% m\'as frecuente ser\'a la de una forma con coeficientes en
	% $\Enteros$ a la que interpretaremos como funci\'on de variables
	% tambi\'en enteras.
% \end{obsDefiniciones}

% \subsection{Formas enteras}%
	% \label{subsec:formas-enteras}
\begin{defDefiniciones}\label{def:definiciones:entera}
	Decimos que una forma cuadr\'atica es \emph{entera},
	si sus coeficientes son n\'umeros enteros.
	El \emph{contenido} de una forma es el m\'aximo com\'un
	divisor de sus coeficientes.
	Una forma \emph{primitiva} es una forma cuadr\'atica
	entera cuyos coeficientes son coprimos, es decir, si
	su contenido es igual a $1$.
\end{defDefiniciones}

% An\'alogamente, una forma es \emph{racional}, si tiene
% coeficientes racionales, \emph{real}, si sus coeficientes son
% n\'umeros reales y \emph{compleja}, si son n\'umeros complejos.
% De ahora en adelante, a menos que se mencione, las formas
% cuadr\'aticas ser\'an enteras.

\begin{ejemDefiniciones}\label{ejem:definiciones:primitiva}
	La forma $\binaria{1,0,1}=x^2+y^2$ es primitiva.
	La forma $\binaria{2,2,3}=2x^2+2xy+3y^2$, tambi\'en.
\end{ejemDefiniciones}

% \subsection{Representaci\'on por formas cuadr\'aticas}%
	% \label{subsec:representacion-por-formas-cuadraticas}
\begin{defDefiniciones}\label{def:definiciones:representacion}
	Dado $m\in\Enteros$ y una forma $f$, decimos que
	\emph{$m$ es representado por $f$} (o que \emph{$f$ %
	representa $m$}), si existen $x,y\in\Enteros$ tales que
	\begin{displaymath}
		f(x,y)\,=\,m
		\text{ ;}
	\end{displaymath}
	%
	el par $(x,y)$ soluci\'on de esta ecuaci\'on, o bien la
	expresi\'on $f(x,y)=m$ es una \emph{representaci\'on de $m$ %
	por la forma $f$}. Dicha representaci\'on se dice
	\emph{propia}, si $x$ e $y$ son coprimos. Decimos que
	\emph{$f$ representa propiamente a $m$} (o que
	\emph{$m$ es propiamente representado por $f$}), si
	existe una representaci\'on propia de $m$ por $f$.
\end{defDefiniciones}

\begin{ejemDefiniciones}\label{ejem:definiciones:representacion}
	La forma $f(x,y)=\binaria{1,0,1}$ representa $m=1$, pues
	$f(1,0)=1$; esta representaci\'on de $1$ es propia. La forma
	$f$ tambi\'en representa $m=5$: $f(2,1)=5$; la representaci\'on
	es propia. Tambi\'en representa propiamente a $m=10$, pues
	$f(1,3)=10$. Pero $f$ no representa a $m=7$. De hecho, si $f$
	representa $m$, entonces $m\equiv 0,1,2\tmodulo[4]$, y, si $f$
	representa propiamente a $m$, entonces $m\equiv 1,2\tmodulo[4]$.
\end{ejemDefiniciones}

% En este contexto, la posibilidad de expresar un primo en la forma
% $p=x^2+ny^2$ pasa a ser una propiedad de representabilidad de la
% forma cuadr\'atica entera $\binaria{1,0,n}$. Estas formas son
% primitivas y, como funci\'on de variable real, toman valores
% estrictamente positivos, salvo en $(0,0)$. Geom\'etricamente, los
% puntos $(x,y)$ tales que $f(x,y)=m$ describen una elipse en el plano.
% Las formas cuadr\'aticas con estas propiedades se denominan
% ``definidas positivas''. Nuestro objetivo en esta parte ser\'a
% deducir propiedades de las formas $\binaria{1,0,n}$ a partir de
% propiedades de las formas cuadr\'aticas (binarias enteras) primitivas
% y definidas positivas.

% \subsection{Discriminante de una forma cuadr\'atica}%
	% \label{subsec:discriminante-de-una-forma}
\begin{defDefiniciones}\label{def:definiciones:discriminante}
	Dada una forma cuadr\'atica $f(x,y)=\binaria{a,b,c}=ax^2+bxy+cy^2$,
	llamaremos \emph{discriminante} a
	\begin{displaymath}
		\discriminante(f)\,=\,b^2\,-\,4ac
		\text{ .}
	\end{displaymath}
	%
\end{defDefiniciones}

\begin{obsDefiniciones}\label{obs:definiciones:discriminante}
	El discriminante de una forma cuadr\'atica entera es un
	% n\'umero entero, pues es de la forma $D=b^2-4ac$ con
	% $a,b,c\in\Enteros$.%
	% \footnote{
		% M\'as en general, si los coeficientes de
		% $f=\binaria{a,b,c}$ son racionales, reales,
		% complejos, su discriminante, $D=b^2-4ac$ tambi\'en
		% es racional, real, complejo. Bien en general, si
		% $a$, $b$, $c$ pertenecen a cierto anillo $A$,
		% $D\in A$, tambi\'en.
	% }
	% Adem\'as, como $D\equiv b^2\tmodulo[4]$, debe ser
	% $D\equiv 0,1\tmodulo[4]$.
	n\'umero entero y congruente con $0$ 0 $1$ m\'odulo $4$;
	adem\'as, $D\equiv 0\tmodulo[4]$, si y s\'olo si
	$b\equiv 0\tmodulo[2]$.
	Rec\'{\i}procamente, si $D\equiv 0,1\tmodulo[4]$ existe una
	forma cuadr\'atica entera de discriminante $D$:
	si $D\equiv 0$, la forma $\binaria{1,0,\frac{-D} 4}$ tiene
	discriminante $D$, si $D\equiv 1$, la forma
	$\binaria{1,1,\frac{1-D} 4}$ tiene discriminante $D$.
	% Notemos que estas formas son, adem\'as, primitivas y que
	% el ``coeficiente $a$'' es positivo.
\end{obsDefiniciones}

\begin{lemaDefiniciones}\label{lema:definiciones:discriminante-no-cuadrado}
	Sea $f=\binaria{a,b,c}$ una forma cuadr\'atica entera
	y sea $D:=\discriminante(f)$ su discriminante.
	Si $D\neq 0$ no es un cuadrado perfecto, entonces $a\neq 0$,
	$c\neq 0$ y la \'unica soluci\'on de $f(x,y)=0$ con
	$x,y\in\Enteros$ es $x=y=0$.
\end{lemaDefiniciones}

% \begin{obsDefiniciones}\label{obs:definiciones:valores}
	% La conclusi\'on del \lemaname~%
	% \ref{lema:definiciones:discriminante-no-cuadrado} es que, si el
	% discriminante de una forma cuadr\'atica no es un
	% cuadrado perfecto, entonces la \'unica representaci\'on
	% de $0$ es la trivial.
% \end{obsDefiniciones}

\begin{proof}
	Si $a=0$ o $c=0$, entonces $D=b^2$ es cuadrado, con lo
	que asumimos que ni $a$ ni $c$ son nulos. En ese caso,
	sean $x,y\in\Enteros$ tales que $f(x,y)=0$.
	Si $x=0$, entonces $cy^2=0$ e $y=0$.
	An\'alogamente, si $y=0$, entonces $ax^2=0$ y $x=0$.
	Podemos suponer, entonces, que $x\neq 0$ e $y\neq 0$.
	Ahora, en general,
	\begin{equation}
		\label{eq:definiciones:completar}
		4af(x,y)\,=\,(2ax+by)^2\,-\,Dy^2
		\text{ ,}
	\end{equation}
	%
	con lo que, $f(x,y)=0$ implica $(2ax+by)^2=Dy^2$. Pero,
	como $y\neq 0$, por factorizaci\'on \'unica, $D$ debe
	ser un cuadrado.
\end{proof}

% \subsection{Formas de acuerdo a su discriminante}%
	% \label{subsec:formas-de-acuerdo-a-su-discriminante}
\begin{defDefiniciones}\label{def:definiciones:valores}
	Decimos que una forma es \emph{indefinida}, si toma valores
	tanto positivos como negativos. Si una forma toma valores
	no negativos decimos que es \emph{semidefinida positiva};
	si no toma valores positivos, \emph{semidefinida negativa}.
	Una forma semidefinida que s\'olo toma el valor $0$ en el
	origen se dice \emph{definida}.
\end{defDefiniciones}

% Abreviamos ``definida semidefinida positiva'' por ``definida
% positiva''; similarmente, ``definida negativa'' significa
% ``definida semidefinida negativa''.

\begin{teoDefiniciones}\label{teo:definiciones:valores}
	Sea $f$ una forma cuadr\'atica y sea $D:=\discriminante(f)$,
	su discriminante. Entonces,
	\begin{enumerate}[(A)]
		\item\label{item:definiciones:valores:indefinida}
			si $D>0$, $f$ es indefinida;
		\item\label{item:definiciones:valores:semidefinida}
			si $D=0$, $f$ es semidefinida, pero no definida;
		\item\label{item:definiciones:valores:definida}
			si $D<0$, $f$ es definida.
	\end{enumerate}
	%
	Adem\'as, si $f=\binaria{a,b,c}$ y $D<0$, entonces $a$ y $c$
	tienen igual signo y son distintos de cero y $f$ es positiva,
	si $a>0$, y negativa, si $a<0$.
\end{teoDefiniciones}

\begin{proof}
	Supongamos que $D<0$
	(entonces, $a$ y $c$ deben tener el mismo signo y
	ser no nulos).
	Por \eqref{eq:definiciones:completar} y el \lemaname~%
	\ref{lema:definiciones:discriminante-no-cuadrado},
	$4af(x,y)>0$ para todo par $x,y\in\Enteros$, excepto
	$x=y=0$. En particular, $f$ es definida (positiva o
	negativa). Dado que
	\begin{displaymath}
		a\,=\,f(1,0)\quad\text{y}\quad
			c\,=\,f(0,1)
		\text{ ,}
	\end{displaymath}
	%
	vemos que $a$ y $c$ tienen igual signo y que este
	signo concuerda con el ``signo'' de $f$.
\end{proof}

% \subsection{Equivalencia de formas cuadr\'aticas}%
	% \label{subsec:equivalencia-de-formas}
\begin{defDefiniciones}\label{def:definiciones:equivalencia}
	Dos formas $f$ y $g$ son \emph{equivalentes}, si existen
	$p,q,r,s\in\Enteros$ tales que
	\begin{displaymath}
		f(x,y)\,=\,g(px+qy,rx+sy)
		\quad\text{y}\quad
		ps-qr\,=\,\pm 1
		\text{ ;}
	\end{displaymath}
	%
	si $ps-qr=1$, entonces decimos que son
	\emph{estrictamente equivalentes}.
\end{defDefiniciones}

\begin{ejemDefiniciones}\label{ejem:definiciones:equivalencia}
	Las formas $\binaria{2,-1,3}$ y $\binaria{2,1,3}$ son equivalentes,
	pero, con los resultados de la \S~\ref{sec:reducidas}, podremos
	probar que no pueden ser estrictamente equivalentes.
	En particular, las clases de equivalencia estricta no coinciden,
	en general, con clases a secas.
\end{ejemDefiniciones}

\begin{ejemDefiniciones}\label{ejem:definiciones:equivalencia:traslacion}
	Sea $f=\binaria{1,2,4}$. Su discriminante es
	$2^2-4\dot 1\dot 4=-12$
	?`Qu\'e formas son estrictamente equivalentes a $f$?
	Otra forma de discriminante $-12$ con la que ya nos hemos
	encontrado es $\binaria{1,0,3}=x^2+3y^2$.
	Ambas son equivalentes:
	\begin{displaymath}
		f(x-y,y)\,=\,(x-y)^2+2(x-y)y+4y^2\,=\,
			x^2+(-2+2)xy+(1-2+4)y^2\,=\,
			x^2+3y^2
		\text{ .}
	\end{displaymath}
	%
	M\'as aun, en este caso, $p=1$, $q=-1$, $r=0$, $s=1$,
	con lo que $ps-qr=1$ y las formas son estrictamente
	equivalentes.
\end{ejemDefiniciones}

\begin{obsDefiniciones}\label{obs:definiciones:equivalencia:invariantes}
	\quedacomoejercicio.
	% Formas equivalentes comparten ciertas propiedades.
	Si $p,q,r,s\in\Enteros$ y $f$ y $f_1$ son formas relacionadas por
	\begin{equation}
		\label{eq:definiciones:equivalencia:cambio}
		f_1(x,y)\,=\,f(px+qy,rx+sy)
		\text{ ,}
	\end{equation}
	%
	entonces los enteros representados por $f_1$ tambi\'en son
	representados por $f$.
	Por otro lado, los coeficientes de $f$ y de $f_1$ est\'an
	relacionados de la siguiente manera:
	si $f=\binaria{a,b,c}$ y $f_1=\binaria{a_1,b_1,c_1}$,
	\begin{equation}
		\label{eq:definiciones:equivalencia:coeficientes}
		\begin{aligned}
			a_1 & \,=\, ap^2\,+\,bpr\,+\,cr^2\,=\,f(p,r)
				\text{ ,} \\
			b_1 & \,=\, 2apq\,+\,b\,(ps+qr)\,+\,2crs
				\quad\text{y} \\
			c_1 & \,=\, aq^2\,+\,bqs\,+\,cs^2\,=\,f(q,s)
				\text{ .}
		\end{aligned}
		%
	\end{equation}
	%
	En particular, sus discriminantes, $D=\discriminante(f)$ y
	$D_1=\discriminante(f_1)$, est\'an relacionados por:
	\begin{equation}
		\label{eq:definiciones:equivalencia:discriminantes}
		D_1\,=\,(ps-qr)^2\,D
		\text{ .}
	\end{equation}
	%
	% Supongamos que $f$ y $f_1$ son equivalentes, que
	% $p,q,r,s\in\Enteros$ en \eqref{eq:definiciones:equivalencia:cambio}
	% cumplen $ps-qr=\pm 1$. Entonces, 
\end{obsDefiniciones}

\begin{teoDefiniciones}\label{teo:definiciones:equivalencia:invariantes}
	La relaci\'on de equivalencia de formas cuadr\'aticas y la
	relaci\'on de equivalencia estricta son relaciones de equivalencia
	en el conjunto de formas cuadr\'aticas.
	Sean $f$ y $g$ formas cuadr\'aticas equivalentes. Entonces,
	\begin{enumerate}[(i)]
		\item\label{item:definiciones:invariantes:representados}
			un entero es representado por $g$, si y s\'olo si
			es representado por $f$;
		\item\label{item:definiciones:invariantes:propiamente}
			un entero es propiamente representado por $g$,
			si y s\'olo si es propiamente representado por $f$;
		\item\label{item:definiciones:invariantes:discriminante}
			$\discriminante(g)=\discriminante(f)$;
		\item\label{item:definiciones:invariantes:contenido}
			el contenido de $g$ es igual al contenido de $f$;
		% \item\label{item:definiciones:invariantes:primitiva}
			en particular,
			% la forma
			$g$ es primitiva, si y s\'olo si
			% la forma
			$f$ es primitiva.
	\end{enumerate}
	%
\end{teoDefiniciones}

\begin{proof}
	Son consecuencias de la \obsname~%
	\ref{obs:definiciones:equivalencia:invariantes}
	\quedacomoejercicio.
\end{proof}

\subsection*{Ejercicios}
\theoremstyle{definition}
\newtheorem{ejerDefiniciones}{\ejername}[section]

%-------------

\begin{ejerDefiniciones}\label{ejer:definiciones:primitivamente}
	Si $f(x,y)=m$ es una representaci\'on de $m$ por $f$ y
	$g=\mcd{x,y}$, entonces $g^2\mid m$ y $f$ representa
	primitivamente a $m/g^2$.
	En particular, si $m$ es libre de cuadrados (por ejemplo,
	si $m$ es primo o $m=\pm 1$), s\'olo tiene sentido hablar
	de representaciones primitivas de $m$.
\end{ejerDefiniciones}

	Las partes \eqref{item:definiciones:invariantes:discriminante}
	y \eqref{item:definiciones:invariantes:contenido} del \teoname~%
	\ref{teo:definiciones:equivalencia:invariantes}
	se pueden expresar de la siguiente manera: el discriminante y
	el contenido de una forma cuadr\'atica son invariantes de la
	clase de equivalencia de la forma cuadr\'atica (todas las formas
	en la misma clase tienen igual discriminante e igual contenido).
	Sea $\varClases(D)$ el conjunto de clases de formas cuadr\'aticas
	de discriminante $D$, positivas, si $D<0$,
	y sea $\varClases[g](D)$ el subconjunto de
	clases de discriminante $D$ y contenido $g$.
	En particular, con esta notaci\'on $\varClases[1](D)$ denota el
	conjunto de clases de formas cuadr\'aticas primitivas de
	discriminante $D$ (positivas, si $D<0$).%
	\footnote{
		Comparar con la \defname~%
		\ref{def:representaciones:gaussianas}.
	}

\begin{ejerDefiniciones}\label{ejer:definiciones:clases}
	Mostrar que, si $f$ es una forma de contenido $g$ y discriminante
	$D$, entonces $g^2\mid D$. Concluir que
	\begin{displaymath}
		\varClases(D)\,=\,
		\bigsqcup_{\smatrix{g>0 \\ g^2\mid D}}\,\varClases[g](D)
			\,=\,\bigsqcup_{\smatrix{g>0 \\ g^2\mid D}}\,
				\varClases[1](D/g^2)
		\text{ .}
	\end{displaymath}
	%
\end{ejerDefiniciones}

	Un \emph{discriminante fundamental} es un entero $D\neq 0$ que cumple
	\begin{itemize}
		\item $D\equiv 1\tmodulo[4]$ y es libre de cuadrados,
			o bien
		\item $D=4m$, donde $m\equiv 2,3\tmodulo[4]$ y es libre
			de cuadrados.
	\end{itemize}
	%
	Equivalentemente, un discriminante fundamental es un discriminante
	\emph{minimal}, es decir,
	$D\equiv 0,1\tmodulo[4]$, $D\neq 0$, y no existen enteros
	$D_0\equiv 0,1\tmodulo[4]$ y $f>1$ tales que $D=D_0f^2$.

\begin{ejerDefiniciones}\label{ejer:definiciones:fundamental}
	Probar que un entero $D\equiv 0,1\tmodulo[4]$, $D\neq 0$,
	es un discriminante fundamental, si y s\'olo si toda forma
	de discriminante $D$ es primitiva.
\end{ejerDefiniciones}

El producto de formas lineales da lugar a formas cuadr\'aticas:
\begin{equation}
	\label{eq:definiciones:descomponible}
	(kx+ly)\,(mx+ny)\,=\,kmx^2\,+\,(kn+lm)\,xy\,+\,lny^2
\end{equation}
%
es una forma cuadr\'atica. Veremos que \'estas son muy especiales
dentro del conjunto de todas las formas cuadr\'aticas. Desde el punto
de vista del problema de representabilidad, determinar si una forma
como en \eqref{eq:definiciones:descomponible} representa un entero
$z\in\Enteros$ se reduce a descomponerlo como producto de enteros
de alguna manera, $z=mn$, $m,n\in\Enteros$, y determinar si es posible
representar los factores $m$ y $n$ por cada uno de los factores lineales
de la forma.

\begin{ejerDefiniciones}\label{ejer:definiciones:descomponible}
	Una forma cuadr\'atica es un producto de formas lineales,
	si y s\'olo si su discriminante es un cuadrado perfecto.
	\begin{enumerate}[(i)]
		\item\label{item:definiciones:descomponible:i}
			Probar que el discriminante de la forma
			\eqref{eq:definiciones:descomponible} es igual a
			$(kn-lm)^2$, o sea, un cuadrado perfecto.
		\item\label{item:definciones:descomponible:ii}
			Probar que, si $f=\binaria{a,b,c}$ y
			$\discriminante(f)=h^2$ es un cuadrado, entonces
			\begin{displaymath}
				4af\,=\,\big(2ax+(b+h)\,y\big)\,
					\big(2ax+(b-h)\,y\big)
				\text{ ;}
			\end{displaymath}
			%
			en particular,
			sobre $\Racionales$, $f$ se descompone como
			producto de factores lineales.
		\item\label{item:definciones:descomponible:iii}
			Si $f=\binaria{a,b,c}$ es una forma cuadr\'atica
			con coeficientes enteros y existe una
			factorizaci\'on
			\begin{displaymath}
				f\,=\,(\kappa x+\lambda y)\,(\mu x+\nu y)
			\end{displaymath}
			%
			con $\kappa,\lambda,\mu,\nu\in\Racionales$, entonces
			existe una factorizaci\'on
			\begin{displaymath}
				f\,=\,(kx+ly)\,(mx+ny)
			\end{displaymath}
			%
			con $k,l,m,n\in\Enteros$.%
			\hint{
				Escribir $\tilde nf=(kx+ly)(mx+ny)$ con
				$k,l,m,n,\tilde n\in\Enteros$; dividir
				por el m\'aximo com\'un divisor entre
				$k$, $l$, y $\tilde n$; dividir por el
				m\'aximo com\'un divisor entre
				$m$, $n$ y (el nuevo) $\tilde n$;
				comparar coeficientes:
				$\tilde na=km$, $\tilde nb=kn+lm$ y
				$\tilde nc=ln$; concluir que $\tilde n=1$.
			}
	\end{enumerate}
	%
\end{ejerDefiniciones}

A toda forma cuadr\'atica le podemos asociar una matriz, definida
a partir de los coeficientes de la forma: si $f=\binaria{a,b,c}$,
su \emph{matriz asociada} es%
\footnote{
	A veces se llama ``matriz asociada'' a la matriz
	\begin{math}
		\sbmatrix{a & b \\ b & c}
	\end{math}.
}
\begin{equation}
	\label{eq:definiciones:matriz-asociada}
	\begin{bmatrix}
		a & b/2 \\ b/2 & c
	\end{bmatrix}
	\text{ .}
\end{equation}
%
La matriz asociada a una forma cuadr\'atica es una matriz
sim\'etrica. Si $f$ es entera, los coeficientes de su matriz
asociada ser\'an enteros en la diagonal y medio enteros fuera de
la diagonal. Rec\'{\i}procamente, toda matriz sim\'etrica como
\eqref{eq:definiciones:matriz-asociada}, con $a,b,c\in\Enteros$
determina un forma cuadr\'atica entera:
\begin{equation}
	\label{eq:definiciones:forma-asociada}
	f(x,y)\,=\,\trnsp{\begin{bmatrix} x \\ y \end{bmatrix}}\,
		\begin{bmatrix} a & b/2 \\ b/2 & c \end{bmatrix}\,
		\begin{bmatrix} x \\ y \end{bmatrix}
	\text{ ,}
\end{equation}
%
donde $\trnsp v$ denota la matriz $v$ transpuesta. En definitiva,
hay una correspondencia entre formas cuadr\'aticas binarias con
coeficientes enteros y matrices ``medio enteras''.
El discriminante de una forma se expresa de manera
sencilla en t\'erminos del determinante de su matriz asociada:
dada una forma $f$ con matriz asociada $F$,
\begin{displaymath}
	\discriminante(f)\,=\,-4\det(F)
	\text{ .}
\end{displaymath}
%

\begin{ejerDefiniciones}\label{ejer:definiciones:equivalencia}
	Sea $f=\binaria{a,b,c}$ una forma cuadr\'atica.
	Probar que la matriz asociada a
	$f(px+qy,rx+sy)$ es igual a
	\begin{displaymath}
		\trnsp{\begin{bmatrix}
			p & q \\ r & s
		\end{bmatrix}}\,
		\begin{bmatrix}
			a & b/2 \\ b/2 & c
		\end{bmatrix}\,
		\begin{bmatrix}
			p & q \\ r & s
		\end{bmatrix}
		\text{ .}
	\end{displaymath}
	%
\end{ejerDefiniciones}

\begin{ejerDefiniciones}\label{ejer:definiciones:accion}
	Probar que el grupo $\GL(2,\Enteros)$ de matrices de tama\~no
	$2\times 2$ inversibles con coeficientes enteros act\'ua en el
	conjunto de matrices medio enteras v\'{\i}a
	$X\mapsto \trnsp AXA$ (la acci\'on es a derecha).
	Concluir que dos formas cuadr\'aticas, $f$ y $f_1$,
	son equivalentes, si y s\'olo si sus matrices asociadas, $F$ y
	$F_1$, cumplen que existe $A\in\GL(2,\Enteros)$ tal que
	$F_1=\trnsp AFA$. Probar que $f$ y $f_1$ son propiamente
	equivalentes si existe $A\in\SL(2,\Enteros)$ tal que
	$F_1=\trnsp AFA$. Demostrar que las relaciones de equivalencia y
	de equivalencia propia entre formas cuadr\'aticas son,
	efectivamente, relaciones de equivalencia.
\end{ejerDefiniciones}

Si $f$ es una forma cuadr\'atica y $\gamma=\sbmatrix{ p & q \\ r & s }$,
$f\cdot\gamma$ denota la forma
\begin{displaymath}
	(f\cdot\gamma)(x,y)\,=\,f(px+qy,rx+sy)
	\text{ .}
\end{displaymath}
%
% Esto define una acci\'on de $\GL(2,\Enteros)$ en el conjunto de formas
% cuadr\'aticas binarias con coeficientes enteros.
Dos formas, $f$ y $f_1$, son equivalentes, si y s\'olo si
existe $A\in\GL(2,\Enteros)$ tal que $f_1=f\cdot A$;
si $A\in\SL(2,\Enteros)$, entonces son estictamente equivalentes.

Dada una forma $f$, su \emph{grupo de isotrop\'{\i}a} es
\begin{displaymath}
	\Stab(f)^+\,=\,\big\{\gamma\in\SL(2,\Enteros)\,:\,
		f\cdot\gamma=f\big\}
	\text{ .}
\end{displaymath}
%

\begin{ejerDefiniciones}\label{ejer:definiciones:isotropia:conjugado}
	Probar que $\Stab(f)^+$ es un subgrupo de $\SL(2,\Enteros)$ y que,
	si $\gamma\in\SL(2,\Enteros)$, entonces
	\begin{displaymath}
		\Stab(f\cdot\gamma)^+\,=\,\gamma^{-1}\Stab(f)^+\gamma
		\text{ .}
	\end{displaymath}
	%
\end{ejerDefiniciones}

\begin{ejerDefiniciones}\label{ejer:definiciones:isotropia}
	Dada una forma cuadr\'atica $f=\binaria{a,b,c}$ de discriminante
	$D=b^2-4ac$, el grupo $\Stab(f)^+$ contiene todas las matrices
	de la forma
	\begin{equation}
		\label{eq:definiciones:isotropia}
		\begin{bmatrix}
			\frac{u-bv} 2 & -cv \\
			av & \frac{u+bv} 2
		\end{bmatrix}
		\text{ ,}
	\end{equation}
	%
	donde el par $u,v\in\Enteros$ satisface
	\begin{equation}
		\label{eq:definiciones:isotropia:pell}
		u^2\,-\,Dv^2\,=\,4
		\text{ .}
	\end{equation}
	%
	Si $f$ es primitiva, entonces \'estas son todas las matrices
	$\gamma\in\SL(2,\Enteros)$ tales que $f\cdot\gamma=f$.%
	\hint{
		Se puede corroborar que las matrices de la forma
		\eqref{eq:definiciones:isotropia} preservan $f$,
		a partir de las f\'ormulas
		\eqref{eq:definiciones:equivalencia:cambio} para los
		coeficientes de $f\cdot\gamma$.
		Sea $\gamma=\sbmatrix{ p & q \\ r & s }\in\Stab(f)^+$.
		Entonces,
		\begin{displaymath}
			\begin{aligned}
				a & \,=\,ap^2\,+\,bpr\,+\,cr^2
					\quad\text{y} \\
				b & \,=\,2apq\,+\,b\,(ps+qr)\,+\,2crs
					\,=\, 2apq\,+\,b\,(1+2qr)\,+\,2crs
				\text{ .}
			\end{aligned}
			%
		\end{displaymath}
		%
		De estas ecuaciones,
		\begin{displaymath}
			0\,=\,apq\,+\,bqr\,+\,crs
				\text{ ,}\quad
			aq\,=\,-cr \quad\text{y}\quad
			as\,=\,ap\,+\,br
			\text{ .}
		\end{displaymath}
		%
		O sea, $aq=-cr$ y $a\,(s-p)=br$. En particular,
		$a\mid cr$ y $a\mid br$. \emph{Ahora}, asumiendo
		$\mcd{a,b,c}=1$, se deduce que $a\mid r$.
		Si escribimos $r=av$, entonces $q=-cv$ y
		$s-p=bv$. De esto y de $ps-qr=1$, se puede ver que
		$(p+s)^2=Dv^2+4$. Elegir, entonces, $u=p+s$.
	}
\end{ejerDefiniciones}

% \begin{ejerDefiniciones}
	% Sea $f$ la forma cuadr\'atica $f=\binaria{2,1,3}$; su discriminante
	% es $-23$. Notar que
	% \begin{displaymath}
		% a\,-\,b\,+\,c\,=\,4
		% \dispcomma
		% f(\pm 1,0)\,=\,2\dispand
		% f(0,\pm 1)\,=\,3
	% \end{displaymath}
	% %
	% y probar que $f(x,y)\geq 4$ en cualquier otro caso.
	% Deducir que, si $f_1(x,y)=f(px+qy,rx+sy)$ es una forma
	% equivalente, donde $p,q,r,s\in\Enteros$ y $ps-qr=-1$,
% \end{ejerDefiniciones}



