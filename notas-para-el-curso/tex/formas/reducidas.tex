\theoremstyle{plain}
\newtheorem{teoReducidas}{\teoname}[section]
\newtheorem{lemaReducidas}[teoReducidas]{\lemaname}
\newtheorem{coroReducidas}[teoReducidas]{\coroname}

\theoremstyle{definition}
\newtheorem{defReducidas}[teoReducidas]{\defname}
\newtheorem{obsReducidas}[teoReducidas]{\obsname}
\newtheorem{ejemReducidas}[teoReducidas]{\ejemname}

%-------------

En esta secci\'on estudiaremos la relaci\'on de equivalencia
entre formas definidas. Las formas definidas se pueden separar,
en una primera instancia, en dos grandes partes: las definidas
positivas y las definidas negativas; las que representan enteros
positivos y las que representan enteros negativos.
Estos dos subconjuntos son disjuntos y la subdivisi\'on
respeta la relaci\'on de equivalencia de formas cuadr\'aticas binarias
(los conjuntos de enteros representados por formas equivalentes son
iguales). Por otro lado, la funci\'on
\begin{displaymath}
	f\,=\,\binaria{a,b,c}\,\mapsto\,-f\,:=\,\binaria{-a,-b-c}
\end{displaymath}
%
determina una biyecci\'on entre el conjunto de formas definidas positivas
y el conjunto de formas definidas negativas. Adem\'as, esta
correspondencia tambi\'en respeta la relaci\'on de equivalencia
(estricta): si dos formas $f$ y $g$ son (estrictamente) equivalentes,
entonces las formas $-f$ y $-g$ tambi\'en lo son.
De esta manera, podemos concentrarnos en una de las dos partes;
elegimos concentrarnos en las formas definidas positivas.

El objetivo de esta secci\'on, entonces, es describir una manera de
determinar si dos formas cuadr\'aticas \emph{definidas positivas}
son estrictamente equivalentes.
La idea es, en primer lugar, definir una noci\'on de representante
``can\'onico'' para cada clase de equivalencia estricta; \'estos
ser\'an las ``formas reducidas''. Cada clase de equivalencia estricta
contiene exactamente una forma reducida. Luego, necesitamos una manera
de ``conectar'', de pasar de una forma cuadr\'atica definida positiva
al representante can\'onico de su clase; diremos que ``reducimos'' la
forma cuadr\'atica. Finalmente, para poder comparar dos formas, determinar
si son equivalentes estrictamente, o no, bastar\'a con reducirlas y
compararlas: la clase ser\'a la misma, si y s\'olo si tienen la misma
forma reducida.

La noci\'on de forma reducida que daremos en esta secci\'on
no se aplica a formas indefinidas.
Con lo cual, nos estar\'a faltando un procedimiento para
determinar si dos formas indefinidas son equivalentes.
Hay, sin embargo, una teor\'{\i}a de reducci\'on de formas indefinidas.
Pero tiene otras caracter\'{\i}sticas.

\begin{lemaReducidas}\label{lema:reducidas}
	Cada clase de equivalencia estricta de formas cuadr\'aticas
	de discriminante no cuadrado
	contiene, al menos, un representante $\binaria{a,b,c}$ que cumple
	$|b|\leq |a|\leq |c|$.
\end{lemaReducidas}

\begin{proof}
	Sea $f_0=\binaria{a_0,b_0,c_0}$ una forma cuadr\'atica arbitraria
	y sea $a\in\Enteros$ tal que:
	\begin{itemize}
		\item $a\neq 0$,
		\item $a$ es representado por $f_0$ y
		\item $|a|$ es m\'{\i}nimo entre los enteros representados
			por $f_0$
	\end{itemize}
	%
	Por minimalidad, si $a=f_0(p,r)$, entonces $\mcd{p,r}=1$
	\quedacomoejercicio~
	% si no fuese as\'{\i},
	% \begin{displaymath}
		% \left|f_0\left(\frac p{\mcd{p,r}},\frac r{\mcd{p,r}}
			% \right)\right|\,=\,\left|\frac a{\mcd{p,r}^2}\right|
			% \,<\,|a|
		% \text{ .}
	% \end{displaymath}
	% %
	y existen $q,s\in\Enteros$ tales que $ps-qr=1$. Sea $f'$ la forma
	\begin{displaymath}
		f'(x.y)\,=\,f_0(px+qy,rx+sy)
		\text{ .}
	\end{displaymath}
	%
	Por las f\'ormulas~\eqref{eq:definiciones:equivalencia:cambio},
	el primer coeficiente de $f'$ es $a=f_0(p,r)$, es decir,
	$f'=\binaria{a,b',c'}$ para ciertos $b',c'\in\Enteros$.
	Si, ahora, dado $h\in\Enteros$, definimos
	\begin{displaymath}
		f(x,y)\,=\,f'(x+hy,y)
		\text{ ,}
	\end{displaymath}
	%
	entonces la forma $f$ cumple:
	\begin{itemize}
		\item es estrictamente equivalente a $f'$ y,
			por lo tanto, a $f_0$,
		\item su primer coeficiente sigue siendo $a$,
			es decir, $f=\binaria{a,b,c}$ para ciertos
			$b,c\in\Enteros$, y
		\item $b=2ah+b'$.
	\end{itemize}
	%
	Eligiendo $h$ convenientemente, $|b|\leq |a|$. Finalmente,
	si $c\neq 0$, entonces debe cumplirse $|a|\leq |c|$, por
	minimalidad de $a$. Pero $c\neq 0$ pues, por~%
	\eqref{eq:definiciones:equivalencia:cambio}, $c=f'(h,1)$ y,
	como $\discriminante(f')=\discriminante(f_0)$ no es un cuadrado
	perfecto, por el \lemaname~%
	\ref{lema:definiciones:discriminante-no-cuadrado},
	$f'(x,y)=0$ s\'olo si $x=y=0$.
\end{proof}

% \begin{teoReducidas}\label{teo:reducidas:numero}
	% El n\'umero de clases estrictas de formas cuadr\'aticas
	% de discriminante dado es finito.
% \end{teoReducidas}
% 
% \begin{proof}
	% \dots
% \end{proof}

\begin{lemaReducidas}\label{lema:reducidas:definidas}
	Cada clase de equivalencia estricta de formas cuadr\'aticas
	definidas positivas contiene, al menos, un representante que
	cumple
	\begin{displaymath}
		-a\,<\,b\,\leq\,a\,<\,c\disporelse
		0\,\leq\,b\,\leq\,a\,=\,c
		\dispstop
	\end{displaymath}
	%
\end{lemaReducidas}

\begin{proof}
	En primer lugar, como el discriminante de una forma definida
	positiva es negativo, estamos en la situaci\'on del \lemaname~%
	\ref{lema:reducidas}. Sea $\binaria{a_0,b_0,c_0}$ una forma
	cuadr\'atica definida postiva. Entonces, $a_0,c_0>0$.
	Podemos suponer, por el \lemaname~\ref{lema:reducidas}, que
	$|b_0|\leq a_0\leq c_0$. Ahora, si $a_0=c_0$ pero $b_0<0$,
	obtenemos una forma estrictamente equivalente con $a=a_0=c=c_0$ y
	$b=|b_0|\geq 0$ definiendo:
	\begin{displaymath}
		f_0(y,-x)\,=\,a_0y^2\,+\,b_0y(-x)\,+\,c_0(-x)^2\,=\,
			c_0x^2\,+\,(-b_0)\,xy\,+\,a_0\,y^2
		\dispstop
	\end{displaymath}
	%
	Si $b_0=-a_0$, obtenemos una forma estrictamente equivalente con
	$a=b=a_0$ y $c=c_0$ definiendo:
	\begin{displaymath}
		\begin{aligned}
			f_0(x+y,y) & \,=\,
				a_0\,(x+y)^2\,+\,b_0(x+y)y\,+\,c_0y^2 \\
			& \,=\,a_0x^2\,+\,(2a_0+b_0)xy\,+\,(a_0+b_0+c_0)y^2 \\
			& \,=\,a_0x^2\,+\,a_0xy\,+\,c_0y^2
			\dispstop
		\end{aligned}
		%
	\end{displaymath}
	%
\end{proof}

\begin{defReducidas}\label{def:reducidas}
	Una forma cuadr\'atica definida positiva $f=\binaria{a,b,c}$
	\emph{est\'a reducida} (o \emph{es una forma reducida}), si
	sus coeficientes cumplen las condiciones del \lemaname~%
	\ref{lema:reducidas:definidas}
	% \begin{displaymath}
		% |b|\,\leq\,a\,\leq\,c\dispandif
		% |b|\,=\,a\dispor a\,=\,c\dispcomma
		% \dispthen
		% b\,\geq\,0
		% \dispstop
	% \end{displaymath}
	% %
\end{defReducidas}

\begin{obsReducidas}\label{obs:reducidas:definicion}
	Equivalentemente, $f=\binaria{a,b,c}$ (definida positiva)
	est\'a reducida, si sus coeficientes cumplen:
	\begin{displaymath}
		|b|\,\leq\,a\,\leq\,c\dispandif
		|b|\,=\,a\dispor a\,=\,c\dispcomma
		\dispthen
		b\,\geq\,0
		\dispstop
	\end{displaymath}
	%
\end{obsReducidas}

\begin{teoReducidas}\label{teo:reducidas}
	Toda forma cuadr\'atica primitiva definida positiva es
	estrictamente equivalente a una \'unica forma reducida.
\end{teoReducidas}

\begin{proof}
	Lo \'unico que falta probar es que cada clase de equivalencia
	estricta contiene \emph{a lo sumo} una forma reducida.
	Sean, entonces, $f=\binaria{a,b,c}$ y $f_1=\binaria{a_1,b_1,c_1}$
	dos formas cuadr\'aticas de discriminante $D<0$, ambas reducidas
	en el sentido de la \defname~\ref{def:reducidas}.
	Probaremos que, si $f$ y $f_1$ son estrictamente equivalentes,
	entonces deben ser iguales.
	Sin p\'erdida de generalidad, podemos asumir que $a\geq a_1$.	

	Veamos, primero, que $a=a_1$. Supongamos que $p,q,r,s\in\Enteros$
	cumplen que
	\begin{math}
		f_1(x,y)=f(px+qy,rx+sy)
	\end{math}.
	De las f\'ormulas~\eqref{eq:definiciones:equivalencia:cambio},
	como $a\leq c$ y $|b|\leq a$, vale que
	\begin{displaymath}
		a_1\,=\,ap^2\,+\,bpr\,+\,cr^2
		\,\geq\,ap^2\,-\,a|pr|\,+\,ar^2\,\geq\,a|pr|
		\dispstop
	\end{displaymath}
	%
	La \'ultima desigualdad es consecuencia de que
	$x^2+y^2\geq 2|xy|$ para todo par $x$ e $y$. En consecuencia,
	$|pr|\leq 1$. Asumiendo que $ps-qr=1$, vemos que
	$p,r\in\{-1,0,1\}$ y no son ambos nulos.
	% De hecho, para esta conclusi\'on parcial, basta con $\pm 1$
	Entonces, $p^2-|pr|+r^2=1$ y $a\geq a_1\geq a$, es decir, $a=a_1$.

	Para terminar, distinguimos dos casos: $c=a$ y $c_1=a_1$, o bien
	se cumple alguna de $c>a$ o $c_1>a_1$. En el primer caso, $c=c_1$
	y $D=b^2-4ac=b_1^2-4a_1c_1$ muestra que $b=\pm b_1$. Pero, como
	$f$ y $f_1$ est\'an reducidas, en esta situaci\'on $b,b_1\geq 0$.
	En particular, $b=b_1$, con lo que $f=f_1$. Supongamos, finalmente,
	que estamos en el segundo caso y que es $c>a$.
	Sean $p,q,r,s\in\Enteros$ tales que
	\begin{math}
		f_1(x,y)=f(px+qy,rx+sy)
	\end{math}
	y que $ps-qr=1$. Como antes, deducimos que $p^2-|pr|+r^2=1$.
	Si fuese $r\neq 0$, entonces ser\'{\i}a $cr^2>ar^2$ y
	\begin{displaymath}
		a\,=\,a_1\,ap^2\,+\,bpr\,+\,cr^2\,>\,
			a\,(p^2-|pr|+r^2)\,=\,a
		\dispcomma
	\end{displaymath}
	%
	lo que es absurdo. As\'{\i}, debe ser $r=0$. Pero, ahora,
	$ps-qr=1$ implica $ps=1$ y, por lo tanto, de
	\eqref{eq:definiciones:equivalencia:cambio},
	\begin{displaymath}
		b_1\,=\,2apq\,+\,b\,(ps+qr)\,+\,2crs\,\equiv\,b\modulo[2a]
		\dispstop
	\end{displaymath}
	%
	Pero $b$ y $b_1$ est\'an ambos en el rango
	$-a=-a_1<b,b_1\leq a=a_1$, lo que implica, junto con
	$b_1\equiv b\tmodulo[2a]$, que $b=b_1$. Por \'ultimo,
	$D=b^2-4ac=b_1^2-4a_1c_1$ implica que $c=c_1$ tambi\'en.
\end{proof}

Como consecuencia del \teoname~\ref{teo:reducidas},
podemos deducir versiones un poco m\'as refinadas de
los resultados de la \S~\ref{sec:representaciones}.

\begin{teoReducidas}\label{teo:reducidas:representaciones}
	Sea $D\equiv 0,1\tmodulo[4]$, $D<0$ y sea $p$ un primo positivo
	impar que no divide a $D$. Entonces, $p$ es representado por
	una forma primitiva \emph{reducida} de discriminante $D$,
	si y s\'olo si $\chi(p)=1$, donde $\chi$ es la funci\'on
	del \lemaname~\ref{lema:ecuacion:kronecker}.
\end{teoReducidas}

\begin{proof}
	La diferencia con el \teoname~\ref{teo:representaciones}
	est\'a en que aqu\'{\i} agregamos que la forma que representa
	$p$ debe ser reducida. Pero si $p$ es representado por
	una forma definida positiva de discriminante $D$, entonces,
	cambiando dicha forma por una (la) forma reducida y
	(estrictamente) equivalente a ella, vemos que $p$ es
	representado por una forma reducida.
\end{proof}

El siguiente resultado refina el \coroname~\ref{coro:representaciones}.

\begin{restatable}{coroReducidas}{coroReducidasDescenso}
	\label{coro:reducidas:representaciones}
	Sea $n\in\Enteros$, $n>0$, y sea $p$ un primo positivo impar
	que no divide a $n$. Entonces, $p$ es representado por una
	forma primitiva \emph{reducida} de discriminante $-4n$,
	si y s\'olo si $p$ es un divisor de $x^2+ny^2$, $\mcd{x,y}=1$.
\end{restatable}

% \begin{defReducidas}\label{def:reducidas:numero}
	% Denominaremos \emph{clases} de formas cuadr\'aticas a las
	% clases de equivalencia estricta en el conjunto de formas
	% cuadr\'aticas binarias enteras. Dado $D\equiv 0,1\tmodulo[4]$,
	% el \emph{n\'umero de clases} de formas de discriminante $D$
	% es la cantidad de clases de formas cuadr\'aticas
	% \emph{primitivas}%
	% \footnote{
		% Ver el \ejername~\ref{ejer:definiciones:clases}
		% para una justificaci\'on de por qu\'e s\'olo
		% consideramos clases de formas primitivas.
		% % Ver tambi\'en el \ejername~%
		% % \ref{ejer:definiciones:propiamente}.
	% }
	% de discriminante $D$; lo denotamos $\nClases(D)$.
% \end{defReducidas}

\begin{obsReducidas}\label{obs:reducidas}
	Si $D<0$, el n\'umero de clases $\nClases(D)$ coincide
	con la cantidad de formas reducidas de discriminante $D$
	(\teoname~\ref{teo:reducidas}).
	Las formas primitivas $x^2+ny^2$ con $n>0$ son formas
	reducidas. Pero eso no quiere decir que cada una sea la
	\'unica forma primitiva reducida de discriminante $-4n$.
	Si $\nClases(-4n)=1$, entonces la condici\'on
	$\tlegendre{-n} p=1$ caracteriza aquellos primos que
	pueden ser expresados en la forma $p=x^2+ny^2$
	(\coroname~\ref{coro:reducidas:representaciones})
	y esta condici\'on puede ser formulada como una serie de
	condiciones de congruencia sobre $p$, m\'odulo $4n$.
\end{obsReducidas}

\begin{ejemReducidas}\label{ejem:reducidas}
	Veamos que $x^2+7y^2$ es la \'unica forma cuadr\'atica
	reducida de discriminante $-28$. Si $f=\binaria{a,b,c}$
	tiene discriminante $-28$, entonces $b$ es par. Si est\'a
	reducida, entonces $|b|\leq a\leq c$. Como $-28=b^2-4ac$,
	se deduce que $-28\leq-3a^2$ y, por lo tanto, que
	$a\in\{1,2,3\}$. Por otro lado, debe ser $-a<b\leq a$.
	Si $a=3$, entonces $b\in\{-2,0,2\}$. Pero
	$-28\equiv 2\tmodulo[3]$, mientras que
	$b^2\equiv 0,1\tmodulo[3]$. As\'{\i} que debe ser
	$a\in\{1,2\}$. Si $a=2$, la \'unica posibilidad ser\'{\i}a
	$b=0$, pero $-28\equiv 4\tmodulo[8]$, mientras que
	$b^2\equiv 0\tmodulo[8]$. Finalmente, la \'unica posibilidad
	restante es $a=1$. Entonces, $b=0$ y $c=7$.
	As\'{\i}, $\nClases(-28)=1$.

	En particular, como $x^2+7y^2$ es la \'unica forma
	cuadr\'atica reducida de discriminante $-28$, por el
	\coroname~\ref{coro:reducidas:representaciones},
	$\tlegendre{-7} p=1$, si y s\'olo si $p=x^2+7y^2$.
	Si ahora logramos describir qu\'e primos $p$ satisfacen
	$\tlegendre{-7} p=1$, habremos encontrado condiciones
	que garantizan que $p$ se pueda expresar en la forma
	$p=x^2+7y^2$. Ahora,
	\begin{math}
		\tlegendre{-7} p=1
	\end{math},
	si y s\'olo si
	\begin{math}
		\tlegendre{-1} p\tlegendre 7 p=1
	\end{math}.
	Equivalentemente,
	\begin{math}
		\tlegendre{-1} p=\tlegendre 7 p=1
	\end{math},
	o bien
	\begin{math}
		\tlegendre{-1} p=\tlegendre 7 p=-1
	\end{math}.
	Es decir,
	\begin{displaymath}
		\begin{aligned}
			p & \,\equiv\,1\tmodulo[4]\dispand
			p\,\equiv\,1,2,4\tmodulo[7]\dispcomma\disporelse \\
			p & \,\equiv\,3\tmodulo[4]\dispand
			p\,\equiv\,1,2,4\tmodulo[7]
			\dispstop
		\end{aligned}
		%
	\end{displaymath}
	%
	O sea, un primo impar $p$ se expresa en la forma $p=x^2+7y^2$,
	si y s\'olo si $p\equiv 1,2,4\tmodulo[7]$.
\end{ejemReducidas}

\begin{teoReducidas}[Landau]\label{teo:landau}
	Para $n\in\Enteros$, $n>0$, se cumple
	$\nClases(-4n)=1$, si y s\'olo si $n\in\{1,2,3,4,7\}$.
\end{teoReducidas}

\begin{proof}
	Si $n\in\{1,2,3,4,7\}$, entonces $x^2+ny^2$ es la \'unica
	forma primitiva reducida de discriminante $-4n$
	\quedacomoejercicio.%
	\hint{
		Argumentar como en el \ejemname~\ref{ejem:reducidas}.
		% Usar que, si $\binaria{a,b,c}$ es reducida
		% de discriminante $-4n$, entonces
		% $0<a\leq\sqrt{(4n)/3}$.
	}
	Rec\'{\i}procamente, si $n\not\in\{1,2,3,4,7\}$, encontraremos
	una segunda forma reducida, primitiva y de discriminante $-4n$.

	Supongamos, primero, que $n$ no es una potencia de un n\'umero
	primo y que $n=ac$ con $1<a<c$, $\mcd{a,c}=1$. Entonces, la
	forma $ax^2+cy^2$ es una forma primitiva, reducida y de
	discriminante $-4ac=-4n$.

	Supongamos, ahora, que $n=2^r$. Si $r\in\{0,1,2\}$, entonces
	$n\in\{1,2,4\}$. Si $r=3$, entonces $n=8$ y se verifica que
	$\nClases(-4\cdot 8)=2$ \quedacomoejercicio. Si $r\geq 4$,
	la forma
	\begin{displaymath}
		4x^2\,+\,4xy\,+\,(2^{r-2}+1)\,y^2
	\end{displaymath}
	%
	es primitiva, reducida y de discriminante
	$4^2-4\cdot 4\cdot(2^{r-2}+1)=-4\cdot 2^r=-4n$.

	Supongamos, por \'ultimo, que $n=p^r$, $p$ primo impar.
	Si $n+1$ no es potencia de primo, entonces $n+1=ac$, donde
	$1<a<c$, $\mcd{a,c}=1$ y la forma
	\begin{math}
		ax^2+2xy+cy^2
	\end{math}
	es primitiva, reducida y de discriminante $4-4ac=-4n$.
	Si $n+1$ es potencia de primo, entonces, como es par,
	$n+1=2^s$. Si $s\in\{1,2,3\}$, entonces $n\in\{1,3,7\}$.
	Si $s=4$, $n=15$ no es potencia de primo. Si $s=5$, entonces
	$n=31$ y se verifica que $\nClases(-4\cdot 31)=3$
	\quedacomoejercicio. Si $s\geq 6$, entonces a forma
	\begin{displaymath}
		8x^2\,+\,6xy\,+\,(2^{s-3}+1)\,y^2
	\end{displaymath}
	%
	es primitiva, reducida y de discriminante
	$6^2-4\cdot 8\cdot(2^{s-3}+1)=4-2^{s+2}=-4\cdot(2^s-1)=-4n$.
\end{proof}

\subsection*{Ejercicios}
\theoremstyle{definition}
\newtheorem{ejerDefinidas}{\ejername}[section]

%-------------

% \begin{ejerDefinidas}
	% Probar que, en la demostraci\'on del \teoname~%
	% \ref{teo:reducidas:numero}, la cota para el coeficiente $a$ es:
	% \begin{itemize}
		% \item $|a|\leq \sqrt D/2$, si $D>0$, y
		% \item $a\leq \sqrt{|D|/3}$, si $D<0$.
	% \end{itemize}
	% %
% \end{ejerDefinidas}

\begin{ejerDefinidas}\label{ejer:definidas:numero}
	Sea $\binaria{a,b,c}$ una forma cuadr\'atica que verifica
	$|b|\leq |a|\leq |c|$ y sea $D=b^2-4ac$ su discriminante.
	Probar que,%
	\hint{
		Ver el \ejemname~\ref{ejem:reducidas}.
	}
	\begin{enumerate}[(A)]
		\item\label{ejer:definidas:numero:indefinida}
			si $D>0$, entonces $0\leq |a|\leq\sqrt{D/4}$ y que,
		\item\label{ejer:definidas:numer:definida}
			si $D<0$, entonces $0<a\leq\sqrt{|D|/3}$.
	\end{enumerate}
	%
	Concluir, como consecuencia del \lemaname~\ref{lema:reducidas},
	que el n\'umero de clases de formas cuadr\'aticas de
	discriminante no cuadrado es finito.
\end{ejerDefinidas}

\begin{ejerDefinidas}\label{ejer:definidas:equivalencia}
	Probar que $\binaria{a,b,c}$ es propiamente equivalente a
	las formas:
	\begin{displaymath}
		\binaria{c,-b,a}\dispand\binaria{a,2ah+b,ah^2+bh+c}
		\dispstop
	\end{displaymath}
	%
	Mostrar que $\binaria{a,b,c}$ y $\binaria{a,-b,c}$ son
	equivalentes, pero no necesariamente propiamente equivalentes
	(asumir que $D<0$).
\end{ejerDefinidas}

\begin{ejerDefinidas}\label{ejer:definidas:equivalencia:caso}
	Probar que $\binaria{a,-a,c}$ y $\binaria{a,a,c}$ son
	propiamente equivalentes.
	% $(x+y,y)$
\end{ejerDefinidas}

\begin{ejerDefinidas}\label{ejer:definidas:cota}
	Sea $f=\binaria{a,b,c}$ una forma cuadr\'atica. Probar que
	\begin{enumerate}[(i)]
		\item\label{item:ejer:definidas:cota:i}
			si $|b|\leq a\leq c$, entonces
			\begin{displaymath}
				|f(x,y)|\,\geq\,
					% a|x|^2\,-\,|b||xy|\,+\,c|y|^2\,\geq\,
					(a-|b|+c)\, \min\{|x|^2,|y|^2\}
				\disptext{y que}
			\end{displaymath}
			%
		\item\label{item:ejer:definidas:cota:ii}
			si $|b|<a<c$,
			las \'unicas representaciones primitivas
			$f(x,y)=a$ son $(x,y)=(\pm 1,0)$ y
			las \'unicas representaciones primitivas
			$f(x,y)=c$ son $(x,y)=(0,\pm 1)$
		\item\label{item:ejer:definidas:cota:iii}
			?`Qu\'e se puede decir en los casos $|b|=a<c$ y
			$|b|<a=c$?
	\end{enumerate}
	%
\end{ejerDefinidas}

\begin{ejerDefinidas}
	Hallar condiciones sobre un primo impar $p$ que garantizan
	que $p=x^2+ny^2$, para $n\in\{1,2,3,4,7\}$ (el caso
	$n=7$ lo hicimos en el \ejemname~\ref{ejem:reducidas}).
\end{ejerDefinidas}

\begin{ejerDefinidas}
	Hallar la forma reducida propiamente equivalente a
	$\binaria{126,74,13}$.
\end{ejerDefinidas}

\begin{ejerDefinidas}\label{ejer:definidas:varios}
	Determinar las formas primitivas reducidas de discriminantes
	$-20$, $-56$, $-3$, $-15$, $-24$, $-31$, $-52$.
\end{ejerDefinidas}

\begin{ejerDefinidas}\label{ejer:definidas:representaciones}
	Probar que, si $f=ax^2+cy^2$ y $f_1$ es una forma
	equivalente a $f$, entonces $f_1$ es propiamente
	equivalente a $f$. Concluir que, si $f=x^2+ny^2$ y
	$f_1$ es una forma reducida equivalente a $f$, entonces
	$f_1=f$.
\end{ejerDefinidas}

\begin{ejerDefinidas}\label{ejer:definidas:menos-veinte}
	En este \ejername estudiaremos represantabilidad por formas
	de discriminante $-20$.
	\begin{enumerate}[(i)]
		\item\label{item:ejer:definidas:menos-veinte:i}
			Sea $p\neq 2,5$ un n\'umero primo.
			\begin{enumerate}[(a)]
				\item Probar que $p$ es representado
					por una forma de disriminante
					$-20$, si y s\'olo si
					$p\equiv 1,3,7,9\tmodulo[20]$.
				\item Probar que $p$ es representado
					por $x^2+5y^2$, si y s\'olo si
					$p\equiv 1,9\tmodulo[20]$ y que
					es representado por
					$2x^2+2xy+3y^2$, si y s\'olo si
					$p\equiv 3,7\tmodulo[20]$.
			\end{enumerate}
			%
		\item\label{item:ejer:definidas:menos-veinte:ii}
			Verificar la identidad
			\begin{displaymath}
				\big(2x^2+2xy+3y^2\big)\,
				\big(2z^2+2zw+3w^2\big)\,=\,
				\big(\big)^2\,+\,
				5\,\big(\big)^2
				\dispstop
			\end{displaymath}
			%
		\item\label{item:ejer:definidas:menos-veinte:iii}
			Si
			\begin{math}
				m=5^b\,
				\big(p_1^{c_1}\,\cdots\,p_r^{c_r}\big)\,
				\big(q_1^{d_1}\,\cdots\,q_s^{d_s}\big)\,
				\big(u_1^{e_1}\,\cdots\,u_t^{e_t}\big)
			\end{math}
			es la factorizaci\'on de $m$ como producto de
			primos a potencias, donde
			$p_i\equiv 1,9\tmodulo[20]$,
			$q_j\equiv 3,7\tmodulo[20]$,
			la suma $d_1+\,\cdots\,+d_s$ es par y
			los $e_k$ son pares, \emph{entonces}
			$m$ es representado por $x^2+5y^2$.
	\end{enumerate}
	%
\end{ejerDefinidas}

\begin{ejerDefinidas}\label{ejer:definidas:menos-doce}
	Sea $p\equiv 1\tmodulo[12]$ un primo.
	\begin{enumerate}[(i)]
		\item\label{item:ejer:definidas:menos-doce:i}
			Probar que existen enteros $a,b,t,u>0$,
			$a$ y $t$ impares, $b$ y $u$ pares, tales que
			$p=a^2+b^2=t^2+3u^2$.
		\item\label{item:ejer:definidas:menos-doce:ii}
			Probar que, con la notaci\'on de
			\eqref{item:ejer:definidas:menos-doce:i},
			\begin{displaymath}
				(-3)^{(p-1)/4}\,\equiv\,
					(-1)^{(t-1)/2}\,\legendre 2 p\,
						\legendre t 3
						\modulo[p]
				\dispstop
			\end{displaymath}
			%
			% Probar que
			% $\tlegendre t p =(-1)^{(t-1)/2}\,\tlegendre t 3$
			% y que
			% $\tlegendre u p = \tlegendre 2 p$.
		\item\label{item:ejer:definidas:menos-doce:iii}
			Probar que $3\mid a$ o $3\mid b$, pero no ambas.
		\item\label{item:ejer:definidas:menos-doce:iv}
			Probar que $(t+b)\,(t-b)=a^2-3u^2$.
		\item\label{item:ejer:definidas:menos-doce:v}
			Suponiendo que $3\mid b$, probar que
			\begin{enumerate}[(a)]
				\item $3\nmid t+b$,
				\item
					\begin{math}
						\tlegendre t 3 =
						\tlegendre{(t+b)} 3=
						(-1)^{(t-1)/2}\,\tlegendre 2 p
					\end{math} y
				\item $(-3)^{(p-1)/4}\equiv 1\tmodulo[p]$.
			\end{enumerate}
			%
		\item\label{item:ejer:definidas:menos-doce:vi}
			Suponiendo que $3\mid a$, probar que
			\begin{enumerate}[(a)]
				\item
					\begin{math}
						-\tlegendre t 3=
						\tlegendre{(t\pm b)} 3=
						(-1)^{(t-1)/2}\,\tlegendre 2 p
					\end{math},
					eligiendo $\pm$ de manera que
					$3\nmid t\pm b$, y
				\item $(-3)^{(p-1)/4}\equiv -1\tmodulo[p]$.
			\end{enumerate}
			%
		\item\label{item:ejer:definidas:menos-doce:vii}
			Probar que $p\equiv 1\tmodulo[12]$, primo,
			\begin{enumerate}[(A)]
				\item es representado por $x^2+36y^2$,
					si (y s\'olo si) la congruencia
					$x^4\equiv -3\tmodulo[p]$ tiene
					soluci\'on, o bien
				\item es representado por $9x^2+4y^2$,
					si (y s\'olo si) la congruencia
					$x^4\equiv -3\tmodulo[p]$
					\emph{no tiene} soluci\'on.
			\end{enumerate}
			%
	\end{enumerate}
	%
	(Para el \'{\i}tem~\eqref{item:ejer:definidas:menos-doce:vii}
	sirve conocer la estructura del grupo $\Unidadesmod[p]$,
	es decir, que es \emph{c\'{\i}clico}; como $p\equiv 1\tmodulo[12]$,
	la condici\'on $(-3)^{(p-1)/4}\equiv 1\tmodulo[p]$ equivale a
	que $x^4\equiv -3\tmodulo[p]$ tenga soluci\'on (y, en el caso en que
	tiene soluci\'on, tiene $4$ soluciones)).
	% ?`Se puede concluir de esto que $-3$ no es una ra\'{\i}z primitiva
	% m\'odulo $p$, si $p\equiv 1\tmodulo[12]$?
\end{ejerDefinidas}



