\theoremstyle{definition}
\newtheorem{ejerReducidas}{\ejername}[section]

%-------------

% \begin{ejerReducidas}
	% Probar que, en la demostraci\'on del \teoname~%
	% \ref{teo:reducidas:numero}, la cota para el coeficiente $a$ es:
	% \begin{itemize}
		% \item $|a|\leq \sqrt D/2$, si $D>0$, y
		% \item $a\leq \sqrt{|D|/3}$, si $D<0$.
	% \end{itemize}
	% %
% \end{ejerReducidas}

\begin{ejerReducidas}\label{ejer:reducidas:numero}
	Sea $\binaria{a,b,c}$ una forma cuadr\'atica que verifica
	$|b|\leq |a|\leq |c|$ y sea $D=b^2-4ac$ su discriminante.
	Probar que,%
	\hint{
		Ver el \ejemname~\ref{ejem:reducidas}.
	}
	\begin{enumerate}[(A)]
		\item\label{ejer:reducidas:numero:indefinida}
			si $D>0$, entonces $0\leq |a|\leq\sqrt{D/4}$ y que,
		\item\label{ejer:reducidas:numer:definida}
			si $D<0$, entonces $0<a\leq\sqrt{|D|/3}$.
	\end{enumerate}
	%
	Concluir, como consecuencia del \lemaname~\ref{lema:reducidas},
	que el n\'umero de clases de formas cuadr\'aticas de
	discriminante no cuadrado es finito.
\end{ejerReducidas}

\begin{ejerReducidas}\label{ejer:reducidas:equivalencia}
	Probar que $\binaria{a,b,c}$ es propiamente equivalente a
	las formas:
	\begin{displaymath}
		\binaria{c,-b,a}\dispand\binaria{a,2ah+b,ah^2+bh+c}
		\dispstop
	\end{displaymath}
	%
	Mostrar que $\binaria{a,b,c}$ y $\binaria{a,-b,c}$ son
	equivalentes, pero no necesariamente propiamente equivalentes
	(asumir que $D<0$).
\end{ejerReducidas}

\begin{ejerReducidas}\label{ejer:reducidas:equivalencia:caso}
	Probar que $\binaria{a,-a,c}$ y $\binaria{a,a,c}$ son
	propiamente equivalentes.
	% $(x+y,y)$
\end{ejerReducidas}

\begin{ejerReducidas}\label{ejer:reducidas:cota}
	Sea $f=\binaria{a,b,c}$ una forma cuadr\'atica. Probar que
	\begin{enumerate}[(i)]
		\item\label{item:ejer:reducidas:cota:i}
			si $|b|\leq a\leq c$, entonces
			\begin{displaymath}
				|f(x,y)|\,\geq\,
					% a|x|^2\,-\,|b||xy|\,+\,c|y|^2\,\geq\,
					(a-|b|+c)\, \min\{|x|^2,|y|^2\}
				\disptext{y que}
			\end{displaymath}
			%
		\item\label{item:ejer:reducidas:cota:ii}
			si $|b|<a<c$,
			las \'unicas representaciones primitivas
			$f(x,y)=a$ son $(x,y)=(\pm 1,0)$ y
			las \'unicas representaciones primitivas
			$f(x,y)=c$ son $(x,y)=(0,\pm 1)$
		\item\label{item:ejer:reducidas:cota:iii}
			?`Qu\'e se puede decir en los casos $|b|=a<c$ y
			$|b|<a=c$?
	\end{enumerate}
	%
\end{ejerReducidas}

\begin{ejerReducidas}
	Hallar condiciones sobre un primo impar $p$ que garantizan
	que $p=x^2+ny^2$, para $n\in\{1,2,3,4,7\}$ (el caso
	$n=7$ lo hicimos en el \ejemname~\ref{ejem:reducidas}).
\end{ejerReducidas}

\begin{ejerReducidas}
	Hallar la forma reducida propiamente equivalente a
	$\binaria{126,74,13}$.
\end{ejerReducidas}

\begin{ejerReducidas}\label{ejer:reducidas:varios}
	Determinar las formas primitivas reducidas de discriminantes
	$-20$, $-56$, $-3$, $-15$, $-24$, $-31$, $-52$.
\end{ejerReducidas}

\begin{ejerReducidas}\label{ejer:reducidas:representaciones}
	Sea $p$ un n\'umero primo y sean $f$ y $f_1$ formas
	cuadr\'aticas de igual discriminante que representan,
	ambas, $p$. Probar que $f$ y $f_1$ son equivalentes
	(pero no necesariamente propiamente equivalentes).
\end{ejerReducidas}

\begin{ejerReducidas}\label{ejer:reducidas:representaciones:bis}
	Probar que, si $f=ax^2+cy^2$ y $f_1$ es una forma
	equivalente a $f$, entonces $f_1$ es propiamente
	equivalente a $f$. Concluir que, si $f=x^2+ny^2$ y
	$f_1$ es una forma reducida equivalente a $f$, entonces
	$f_1=f$.
\end{ejerReducidas}

\begin{ejerReducidas}\label{ejer:reducidas:representaciones:bisbis}
	Probar que, si $f$ es una forma de discriminante $D$ que
	representa $1$, entonces $f$ es propiamente equivalente a
	$\binaria{1,0,-D/4}$, si $D\equiv 0$, o a
	$\binaria{1,1,(1-D)/4}$, si $D\equiv 1$.%
	\hint{
		Mostrar que toda forma $f$ es propiamente equivalente
		a una forma $\binaria{a,b,c}$ donde $a$ es el
		menor entero positivo primitivamente representado por $f$
		y $|b|\leq a$.
	}
\end{ejerReducidas}

