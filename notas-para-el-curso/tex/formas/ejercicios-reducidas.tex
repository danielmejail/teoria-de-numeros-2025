\theoremstyle{definition}
\newtheorem{ejerDefinidas}{\ejername}[section]

%-------------

% \begin{ejerDefinidas}
	% Probar que, en la demostraci\'on del \teoname~%
	% \ref{teo:reducidas:numero}, la cota para el coeficiente $a$ es:
	% \begin{itemize}
		% \item $|a|\leq \sqrt D/2$, si $D>0$, y
		% \item $a\leq \sqrt{|D|/3}$, si $D<0$.
	% \end{itemize}
	% %
% \end{ejerDefinidas}

\begin{ejerDefinidas}\label{ejer:definidas:numero}
	Sea $\binaria{a,b,c}$ una forma cuadr\'atica que verifica
	$|b|\leq |a|\leq |c|$ y sea $D=b^2-4ac$ su discriminante.
	Probar que,%
	\hint{
		Ver el \ejemname~\ref{ejem:reducidas}.
	}
	\begin{enumerate}[(A)]
		\item\label{ejer:definidas:numero:indefinida}
			si $D>0$, entonces $0\leq |a|\leq\sqrt{D/4}$ y que,
		\item\label{ejer:definidas:numer:definida}
			si $D<0$, entonces $0<a\leq\sqrt{|D|/3}$.
	\end{enumerate}
	%
	Concluir, como consecuencia del \lemaname~\ref{lema:reducidas},
	que el n\'umero de clases de formas cuadr\'aticas de
	discriminante no cuadrado es finito.
\end{ejerDefinidas}

\begin{ejerDefinidas}\label{ejer:definidas:equivalencia}
	Probar que $\binaria{a,b,c}$ es propiamente equivalente a
	las formas:
	\begin{displaymath}
		\binaria{c,-b,a}\dispand\binaria{a,2ah+b,ah^2+bh+c}
		\dispstop
	\end{displaymath}
	%
	Mostrar que $\binaria{a,b,c}$ y $\binaria{a,-b,c}$ son
	equivalentes, pero no necesariamente propiamente equivalentes
	(asumir que $D<0$).
\end{ejerDefinidas}

\begin{ejerDefinidas}\label{ejer:definidas:equivalencia:caso}
	Probar que $\binaria{a,-a,c}$ y $\binaria{a,a,c}$ son
	propiamente equivalentes.
	% $(x+y,y)$
\end{ejerDefinidas}

\begin{ejerDefinidas}\label{ejer:definidas:cota}
	Sea $f=\binaria{a,b,c}$ una forma cuadr\'atica. Probar que
	\begin{enumerate}[(i)]
		\item\label{item:ejer:definidas:cota:i}
			si $|b|\leq a\leq c$, entonces
			\begin{displaymath}
				|f(x,y)|\,\geq\,
					% a|x|^2\,-\,|b||xy|\,+\,c|y|^2\,\geq\,
					(a-|b|+c)\, \min\{|x|^2,|y|^2\}
				\disptext{y que}
			\end{displaymath}
			%
		\item\label{item:ejer:definidas:cota:ii}
			si $|b|<a<c$,
			las \'unicas representaciones primitivas
			$f(x,y)=a$ son $(x,y)=(\pm 1,0)$ y
			las \'unicas representaciones primitivas
			$f(x,y)=c$ son $(x,y)=(0,\pm 1)$
		\item\label{item:ejer:definidas:cota:iii}
			?`Qu\'e se puede decir en los casos $|b|=a<c$ y
			$|b|<a=c$?
	\end{enumerate}
	%
\end{ejerDefinidas}

\begin{ejerDefinidas}
	Hallar condiciones sobre un primo impar $p$ que garantizan
	que $p=x^2+ny^2$, para $n\in\{1,2,3,4,7\}$ (el caso
	$n=7$ lo hicimos en el \ejemname~\ref{ejem:reducidas}).
\end{ejerDefinidas}

\begin{ejerDefinidas}
	Hallar la forma reducida propiamente equivalente a
	$\binaria{126,74,13}$.
\end{ejerDefinidas}

\begin{ejerDefinidas}\label{ejer:definidas:varios}
	Determinar las formas primitivas reducidas de discriminantes
	$-20$, $-56$, $-3$, $-15$, $-24$, $-31$, $-52$.
\end{ejerDefinidas}

\begin{ejerDefinidas}\label{ejer:definidas:representaciones}
	Probar que, si $f=ax^2+cy^2$ y $f_1$ es una forma
	equivalente a $f$, entonces $f_1$ es propiamente
	equivalente a $f$. Concluir que, si $f=x^2+ny^2$ y
	$f_1$ es una forma reducida equivalente a $f$, entonces
	$f_1=f$.
\end{ejerDefinidas}

\begin{ejerDefinidas}\label{ejer:definidas:menos-veinte}
	En este \ejername estudiaremos represantabilidad por formas
	de discriminante $-20$.
	\begin{enumerate}[(i)]
		\item\label{item:ejer:definidas:menos-veinte:i}
			Sea $p\neq 2,5$ un n\'umero primo.
			\begin{enumerate}[(a)]
				\item Probar que $p$ es representado
					por una forma de disriminante
					$-20$, si y s\'olo si
					$p\equiv 1,3,7,9\tmodulo[20]$.
				\item Probar que $p$ es representado
					por $x^2+5y^2$, si y s\'olo si
					$p\equiv 1,9\tmodulo[20]$ y que
					es representado por
					$2x^2+2xy+3y^2$, si y s\'olo si
					$p\equiv 3,7\tmodulo[20]$.
			\end{enumerate}
			%
		\item\label{item:ejer:definidas:menos-veinte:ii}
			Verificar la identidad
			\begin{displaymath}
				\big(2x^2+2xy+3y^2\big)\,
				\big(2z^2+2zw+3w^2\big)\,=\,
				\big(\big)^2\,+\,
				5\,\big(\big)^2
				\dispstop
			\end{displaymath}
			%
		\item\label{item:ejer:definidas:menos-veinte:iii}
			Si
			\begin{math}
				m=5^b\,
				\big(p_1^{c_1}\,\cdots\,p_r^{c_r}\big)\,
				\big(q_1^{d_1}\,\cdots\,q_s^{d_s}\big)\,
				\big(u_1^{e_1}\,\cdots\,u_t^{e_t}\big)
			\end{math}
			es la factorizaci\'on de $m$ como producto de
			primos a potencias, donde
			$p_i\equiv 1,9\tmodulo[20]$,
			$q_j\equiv 3,7\tmodulo[20]$,
			la suma $d_1+\,\cdots\,+d_s$ es par y
			los $e_k$ son pares, \emph{entonces}
			$m$ es representado por $x^2+5y^2$.
	\end{enumerate}
	%
\end{ejerDefinidas}

\begin{ejerDefinidas}\label{ejer:definidas:menos-doce}
	Sea $p\equiv 1\tmodulo[12]$ un primo.
	\begin{enumerate}[(i)]
		\item\label{item:ejer:definidas:menos-doce:i}
			Probar que existen enteros $a,b,t,u>0$,
			$a$ y $t$ impares, $b$ y $u$ pares, tales que
			$p=a^2+b^2=t^2+3u^2$.
		\item\label{item:ejer:definidas:menos-doce:ii}
			Probar que, con la notaci\'on de
			\eqref{item:ejer:definidas:menos-doce:i},
			\begin{displaymath}
				(-3)^{(p-1)/4}\,\equiv\,
					(-1)^{(t-1)/2}\,\legendre 2 p\,
						\legendre t 3
						\modulo[p]
				\dispstop
			\end{displaymath}
			%
			% Probar que
			% $\tlegendre t p =(-1)^{(t-1)/2}\,\tlegendre t 3$
			% y que
			% $\tlegendre u p = \tlegendre 2 p$.
		\item\label{item:ejer:definidas:menos-doce:iii}
			Probar que $3\mid a$ o $3\mid b$, pero no ambas.
		\item\label{item:ejer:definidas:menos-doce:iv}
			Probar que $(t+b)\,(t-b)=a^2-3u^2$.
		\item\label{item:ejer:definidas:menos-doce:v}
			Suponiendo que $3\mid b$, probar que
			\begin{enumerate}[(a)]
				\item $3\nmid t+b$,
				\item
					\begin{math}
						\tlegendre t 3 =
						\tlegendre{(t+b)} 3=
						(-1)^{(t-1)/2}\,\tlegendre 2 p
					\end{math} y
				\item $(-3)^{(p-1)/4}\equiv 1\tmodulo[p]$.
			\end{enumerate}
			%
		\item\label{item:ejer:definidas:menos-doce:vi}
			Suponiendo que $3\mid a$, probar que
			\begin{enumerate}[(a)]
				\item
					\begin{math}
						-\tlegendre t 3=
						\tlegendre{(t\pm b)} 3=
						(-1)^{(t-1)/2}\,\tlegendre 2 p
					\end{math},
					eligiendo $\pm$ de manera que
					$3\nmid t\pm b$, y
				\item $(-3)^{(p-1)/4}\equiv -1\tmodulo[p]$.
			\end{enumerate}
			%
		\item\label{item:ejer:definidas:menos-doce:vii}
			Probar que $p\equiv 1\tmodulo[12]$, primo,
			\begin{enumerate}[(A)]
				\item es representado por $x^2+36y^2$,
					si (y s\'olo si) la congruencia
					$x^4\equiv -3\tmodulo[p]$ tiene
					soluci\'on, o bien
				\item es representado por $9x^2+4y^2$,
					si (y s\'olo si) la congruencia
					$x^4\equiv -3\tmodulo[p]$
					\emph{no tiene} soluci\'on.
			\end{enumerate}
			%
	\end{enumerate}
	%
	(Para el \'{\i}tem~\eqref{item:ejer:definidas:menos-doce:vii}
	sirve conocer la estructura del grupo $\Unidadesmod[p]$,
	es decir, que es \emph{c\'{\i}clico}; como $p\equiv 1\tmodulo[12]$,
	la condici\'on $(-3)^{(p-1)/4}\equiv 1\tmodulo[p]$ equivale a
	que $x^4\equiv -3\tmodulo[p]$ tenga soluci\'on (y, en el caso en que
	tiene soluci\'on, tiene $4$ soluciones)).
	% ?`Se puede concluir de esto que $-3$ no es una ra\'{\i}z primitiva
	% m\'odulo $p$, si $p\equiv 1\tmodulo[12]$?
\end{ejerDefinidas}

