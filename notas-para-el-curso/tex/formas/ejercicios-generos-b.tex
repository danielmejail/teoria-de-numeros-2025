\theoremstyle{definition}
\newtheorem{ejerGenerosB}{\ejername}[section]

%-------------

\begin{ejerGenerosB}\label{ejer:generos:b}
	Para $D\in\{-40,-60,-84,-88,-92,-120\}$, describir los g\'eneros
	de formas cuadr\'aticas de discriminante $D$.%
	\hint{
		Recordar que dos coclases distintas son disjuntas, tiene
		todas el mismo cardinal y su uni\'on es todo.
	}
\end{ejerGenerosB}

\begin{ejerGenerosB}\label{ejer:generos:b:primos}
	Dar condiciones sobre un primo $p$ para que se cumpla
	$p=x^2+ny^2$, para $n\in\{6,10,13,15,21,22,30\}$.
\end{ejerGenerosB}

\begin{ejerGenerosB}\label{ejer:generos:b:congruencias}
	Sea $D\equiv 0,1\tmodulo[4]$ un discriminante y sea
	$\tilde H\subset\Unidadesmod[D]$ el subconjunto de clases
	$x$ que verifican las siguientes condiciones:
	\begin{enumerate}[(i)]
		\item\label{item:generos:b:congruencias:condiciones:i}
			para todo primo impar $q\mid D$,
			$\tlegendre x q=1$, y
		\item\label{item:generos:b:congruencias:condiciones:ii}
			la clase es
			\begin{displaymath}
				x\,\equiv\,
				\left\{
					\begin{array}{ll}
						x\equiv 1\tmodulo[4] \dispcomma
							& \text{si }D\equiv 12 \tmodulo[16] \dispcomma \\
						x\equiv 1\tmodulo[4] \dispcomma
							& \text{si }D\equiv 16 \tmodulo[32] \dispcomma \\
						x\equiv 1\tmodulo[8] \dispcomma
							& \text{si }D\equiv 0 \tmodulo[32] \dispcomma \\
						x\equiv 1,7\tmodulo[8]\dispcomma
							& \text{si }D\equiv 8 \tmodulo[32] \quad\text{y} \\
						x\equiv 1,3\tmodulo[8]\dispcomma
							& \text{si }D\equiv 24 \tmodulo[32]
				\dispstop

					\end{array}
				\right.
			\end{displaymath}
			%
	\end{enumerate}
	%
	Notar que la condici\'on
	\eqref{item:generos:b:congruencias:condiciones:ii}
	es vac\'{\i}a, si $D\equiv 1\tmodulo[4]$ o si $D=-4n$ con
	$n\equiv 3\tmodulo[4]$.
	\begin{enumerate}[(i)]
		\item\label{item:generos:b:congruencias:i}
			Probar que $\tilde H$ es subgrupo de
			$\Unidadesmod[D]$ y
			que $\tilde H\subset\ker(\chi)$.
		\item\label{item:generos:b:congruencias:ii}
			Probar que $\tilde H=H$,%
			\hint{
				En el caso $D=-4n$,
				traducir la condici\'on sobre $D$ a $n$ y
				mostrar que, si $n'$ es la parte impar
				de $n$, entonces la condici\'on
				$\tlegendre x q=1$ para todo primo impar
				$q\mid n$ equivale a que $x$ sea
				cuadrado m\'odulo $n'$.
			}
			donde $H\subset\Unidadesmod[D]$ es el subconjunto
			\begin{displaymath}
				H\,=\,\left\{
				\begin{array}{ll}
					\big\{
						x\equiv\beta^2\text{ o }
						\beta^2+n\tmodulo[4n]\,:\,
						\beta\in\Enteros\big\}
						\dispcomma &
						\text{si }D=-4n\dispcomma
						\\[5pt]
					\big\{
						x\equiv\beta^2\tmodulo[D]\,:\,
						\beta\in\Enteros\big\}
						\dispcomma &
						\text{si }D\equiv 1
							\tmodulo[4]
						\dispstop
				\end{array}
				%
				\right.
			\end{displaymath}
			%
			\dispstop
	\end{enumerate}
	%
	% (Notar que, por el \coroname~\ref{coro:generos:cero} y el
	% \coroname~\ref{coro:generos:uno}, $H\subgrpeq\ker(\chi)$,
	% con lo que \eqref{item:generos:b:congruencias:ii} implica
	% \eqref{item:generos:b:congruencias:i}
	% pero que tambi\'en
	% \eqref{item:generos:b:congruencias:i}
	% se puede demostrar de manera independiente).
\end{ejerGenerosB}

