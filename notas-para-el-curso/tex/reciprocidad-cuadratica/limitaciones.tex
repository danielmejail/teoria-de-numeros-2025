\theoremstyle{plain}

\theoremstyle{definition}

%-------------

Volvamos al problema de determinar si un primo impar $p$ se puede,
o no, expresar en la forma $p=x^2+ny^2$.
La condici\'on $\tlegendre{-n} p=1$ garantiza que $p$ \emph{divide}
un entero $N=a^2+nb^2$, $\mcd{a,b}=1$.
% Usando las propiedades del s\'{\i}mbolo de Legendre, podemos
% determinar el valor de $\tlegendre{-n} p$ de manera m\'as o
% menos sencilla.
Pero no es cierto en general que $\tlegendre{-n} p=1$ implique que
$p$ sea de esa forma. Por ejemplo, con $n=5$, si $p\neq 5$ es un
primo impar y existen $x,y\in\Enteros$ tales que $p=x^2+5y^2$,
entonces $p\equiv 1\tmodulo[4]$ y $p\equiv 1,4\tmodulo[5]$, o sea
\begin{displaymath}
	p\,=\,x^2+5y^2\quad\text{implica}\quad
		p\,\equiv\, 1,9\modulo[20]
	\text{ .}
\end{displaymath}
%
% M\'as aun, calculando unos cuantos valores primos de la forma
% $x^2+5y^2$, podemos conjeturar que son equivalentes, es decir,
% si $p\equiv 1,9\tmodulo[20]$, entonces existen $x,y\in\Enteros$
% tales que $p=x^2+5y^2$... NO, NO.
Pero,
\begin{displaymath}
	\tlegendre{-5} p\,=\,1\quad\text{si y s\'olo si}\quad
		p\,\equiv\,1,3,7,9\tmodulo[20]
	\text{ .}
\end{displaymath}
%
Al respecto, Euler conjetur\'o lo siguiente:
% las clases de congruencia $1$, $3$, $7$ y $9$ m\'odulo $20$
% se separan en dos \emph{grupos}
\begin{displaymath}
	\begin{aligned}
		p\,=\,x^2+5y^2\text{ ,} &
			\quad\text{si y s\'olo si}\quad
		p\,\equiv\,1,9\tmodulo[20]\quad\text{y} \\
		2p\,=\,x^2+5y^2\text{ ,} &
			\quad\text{si y s\'olo si}\quad
		p\,\equiv\,3,7\tmodulo[20]
		\text{ .}
	\end{aligned}
	%
\end{displaymath}
%
Algo similar, pero, como veremos m\'as adelante, con una
dificultad adicional, ocurre en el caso $n=14$. Si $p\neq 7$
es un primo impar, Euler conjetur\'o que:
\begin{displaymath}
	\begin{aligned}
		p\,=\,	\begin{cases}
				x^2+14y^2 & \text{o} \\
				2x^2+7y^2 & \text{ ,}
			\end{cases}
			& \quad\text{si y s\'olo si}\quad
		p\,\equiv\,1,9,15,23,25,39\tmodulo[56]
			\quad\text{y} \\
		3p\,=\,x^2+14y^2\text{ ,}
			& \quad\text{si y s\'olo si}\quad
		p\,\equiv\,3,5,13,19,27,45\tmodulo[56]
		\text{ .}
	\end{aligned}
	%
\end{displaymath}
%
Nuevamente, sabiendo \'unicamente que $\tlegendre{-14} p=1$,
no podemos determinar si $p$ cae en el primer grupo o en el
segundo. E incluso si supi\'esemos que $p$ pertenece al primero,
?`c\'omo podr\'{\i}amos distinguir entre los primos de la forma
$x^2+14y^2$ y aquellos de la forma $2x^2+7y^2$?
Saber su clase de congruencia m\'odulo $56$ no es suficiente.
En definitiva, lo que se puede observar en el caso general es
que las clases de congruencia para las cuales $\tlegendre{-n} p=1$
se agrupan en lo que llamaremos ``g\'eneros'', cada uno de los
cuales tendr\'a distintas propiedades de representabilidad.
Para precisar y poder explicar este fen\'omeno, introduciremos
las formas cuadr\'aticas.


Por \'ultimo, mencionamos otras dos conjeturas de Euler que
motivaron a Gauss a estudiar leyes de reciprocidad similares
a las del s\'{\i}mbolo de Legendre, de las cuales hablaremos
en la \partname~\ref{pt:reciprocidad-superior}.
Sea $p$ un primo impar. Entonces,
\begin{displaymath}
	\begin{aligned}
		p\,=\,x^2+27y^2\text{ ,} &
			\quad\text{si y s\'olo si}\quad
			\left\{
			\begin{array}{l}
				p\equiv 1\tmodulo[3]
					\text{ y} \\[5pt]
				2\equiv x^3\tmodulo[p]
				\text{ tiene soluci\'on, y}
			\end{array}
			\right. \\[10pt]
		p\,=\,x^2+64y^2\text{ ,} &
			\quad\text{si y s\'olo si}\quad
			\left\{
			\begin{array}{l}
				p\equiv 1\tmodulo[4]
					\text{ y} \\[5pt]
				2\equiv x^4\tmodulo[p]
				\text{ tiene soluci\'on.}
			\end{array}
			\right.
	\end{aligned}
	%
\end{displaymath}
%
Estas afirmaciones caracterizan los primos $p$ que se pueden
expresar en la forma $p=x^2+27y^2$ y aquellos que se pueden
expresar en la forma $p=x^2+64y^2$.
% El s\'{\i}mbolo de Legendre captura la relaci\'on entre primos
% v\'{\i}a residuos cuadr\'aticos.
Sin embargo, adem\'as de una condici\'on de congruencia
($p\equiv 1\tmodulo[3]$ y $p\equiv 1\tmodulo[4]$, respectivamente),
estas caracterizaciones involucran los conceptos de residuos
c\'ubicos y bicuadr\'aticos.

\subsection*{Ejercicios}
\theoremstyle{definition}
\newtheorem{ejerLimitaciones}{\ejername}[section]

%-------------

\begin{ejerLimitaciones}
	Probar que, si $p\neq 5$ es un primo impar y existen
	$x,y\in\Enteros$ tales que $2p=x^2+5y^2$, entonces
	$p\equiv 3,7\tmodulo[20]$.%
	\hint{
		Si $2p=x^2+5y^2$, entonces $x$ e $y$ son impares.
		Luego, mirar m\'odulo $8$.
	}
\end{ejerLimitaciones}



