\theoremstyle{definition}
\newtheorem{ejerResiduos}{\ejername}[section]

%-------------

\begin{ejerResiduos}\label{ejer:residuos:potencia}
	Sean $N\in\Enteros$, $p$ un primo que no divide a $N$ y $l\geq 1$.
	La cantidad de soluciones a la ecuaci\'on de congruencia
	\begin{equation}
		\label{eq:ejer:residuos:potencia}
		x^2\,\equiv\,N\modulo[p^l]
	\end{equation}
	%
	es igual a:
	\begin{itemize}
		\item $1$, si $p=2$ y $l=1$,
		\item $0$, si $p=2$, $l=2$ y $N\equiv 3\tmodulo[4]$,
		\item $2$, si $p=2$, $l=2$ y $N\equiv 1\tmodulo[4]$,
		\item $0$, si $p=2$, $l\geq 3$ y $N\not\equiv 1\tmodulo[8]$,
		\item $4$, si $p=2$, $l\geq 3$ y $N\equiv 1\tmodulo[8]$ y
		\item $1+\tlegendre N{p}$, si $p$ es impar ($l$ arbitrario).
	\end{itemize}
	%
\end{ejerResiduos}

\begin{ejerResiduos}\label{ejer:residuos:compuesto}
	Sean $k,d\in\Enteros$, $k\geq 1$ y $\mcd{d,k}=1$. Entonces, la
	cantidad de soluciones a la ecuaci\'on de congruencia
	\begin{equation}
		\label{eq:residuos:compuesto}
		x^2\,\equiv\,d\modulo[4k]
	\end{equation}
	%
	es igual a
	\begin{displaymath}
		2\,\sum_{f\mid k}\,\jacobi d{f}
		\text{ ,}
	\end{displaymath}
	%
	donde $f$ recorre los divisores positivos libres de cuadrados
	($f=1$ inclusive) y $\tjacobi d{f}$ denota el s\'{\i}mbolo de
	Jacobi. Concluir de $(x+2k)^2\equiv x^2\tmodulo[4k]$ que la sumatoria
	$\sum_{f\mid k}\,\tjacobi d{f}$ es igual a la cantidad de enteros $x$
	en el rango $0\leq x<2k$ que satisfacen \eqref{eq:residuos:compuesto}.
\end{ejerResiduos}

\begin{ejerResiduos}
	Un entero $a$ se dice \emph{residuo bicuadr\'atico m\'odulo $p$},
	si existe $x\in\Enteros$ tal que $a\equiv x^4\tmodulo[p]$. Probar
	que $-4$ es residuo bicuadr\'atico m\'odulo $p$, si y s\'olo si
	$p\equiv 1\tmodulo[4]$.%
	\hint{
		$x^4+4=((x+1)^2+1)((x-1)^2+1)$.
	}
\end{ejerResiduos}

\begin{ejerResiduos}
	Probar que, si $m$ y $n$ son enteros coprimos, entonces
	\begin{displaymath}
		\frac{m-1} 2\frac{n-1} 2\,=\,
			\sum_{x=1}^{(m-1)/2}\,\piso{\frac{xn} m}\,+\,
			\sum_{y=1}^{(n-1)/2}\,\piso{\frac{ym} n}
		\text{ .}
	\end{displaymath}
	%
\end{ejerResiduos}

