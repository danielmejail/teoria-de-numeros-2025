\theoremstyle{definition}
\newtheorem{ejerResiduos}{\ejername}[section]

%-------------

\begin{ejerResiduos}\label{ejer:residuos:potencia}
	Sean $N\in\Enteros$, $p$ un primo que no divide a $N$ y $l\geq 1$.
	La cantidad de soluciones a la ecuaci\'on de congruencia
	\begin{equation}
		\label{eq:ejer:residuos:potencia}
		x^2\,\equiv\,N\modulo[p^l]
	\end{equation}
	%
	es igual a:
	\begin{itemize}
		\item $1$, si $p=2$ y $l=1$,
		\item $0$, si $p=2$, $l=2$ y $N\equiv 3\tmodulo[4]$,
		\item $2$, si $p=2$, $l=2$ y $N\equiv 1\tmodulo[4]$,
		\item $0$, si $p=2$, $l\geq 3$ y $N\not\equiv 1\tmodulo[8]$,
		\item $4$, si $p=2$, $l\geq 3$ y $N\equiv 1\tmodulo[8]$ y
		\item $1+\tlegendre N{p}$, si $p$ es impar ($l$ arbitrario).
	\end{itemize}
	%
\end{ejerResiduos}

\begin{ejerResiduos}\label{ejer:residuos:compuesto}
	Sean $k,d\in\Enteros$, $k\geq 1$ y $\mcd{d,k}=1$. Entonces, la
	cantidad de soluciones a la ecuaci\'on de congruencia
	\begin{equation}
		\label{eq:residuos:compuesto}
		x^2\,\equiv\,d\modulo[4k]
	\end{equation}
	%
	es igual a
	\begin{displaymath}
		2\,\sum_{f\mid k}\,\jacobi d{f}
		\text{ ,}
	\end{displaymath}
	%
	donde $f$ recorre los divisores positivos libres de cuadrados
	($f=1$ inclusive) y $\tjacobi d{f}$ denota el s\'{\i}mbolo de
	Jacobi. Concluir de $(x+2k)^2\equiv x^2\tmodulo[4k]$ que la sumatoria
	$\sum_{f\mid k}\,\tjacobi d{f}$ es igual a la cantidad de enteros $x$
	en el rango $0\leq x<2k$ que satisfacen \eqref{eq:residuos:compuesto}.
\end{ejerResiduos}

\begin{ejerResiduos}
	Un entero $a$ se dice \emph{residuo bicuadr\'atico m\'odulo $p$},
	si existe $x\in\Enteros$ tal que $a\equiv x^4\tmodulo[p]$. Probar
	que $-4$ es residuo bicuadr\'atico m\'odulo $p$, si y s\'olo si
	$p\equiv 1\tmodulo[4]$.%
	\hint{
		$x^4+4=((x+1)^2+1)((x-1)^2+1)$.
	}
\end{ejerResiduos}

\begin{ejerResiduos}
	Probar que, si $m$ y $n$ son enteros coprimos, entonces
	\begin{displaymath}
		\frac{m-1} 2\frac{n-1} 2\,=\,
			\sum_{x=1}^{(m-1)/2}\,\piso{\frac{xn} m}\,+\,
			\sum_{y=1}^{(n-1)/2}\,\piso{\frac{ym} n}
		\text{ .}
	\end{displaymath}
	%
\end{ejerResiduos}

\begin{ejerResiduos}\label{ejer:residuos:generador}
	Sea $p$ un primo impar y sea $g\in\Enteros$ tal que
	$g^{p-1}\equiv 1\tmodulo[p]$, pero $g^k\not\equiv 1\tmodulo[p]$
	para $1\leq k\leq p-2$.
	Probar que $\tlegendre g p =-1$.
\end{ejerResiduos}

\begin{ejerResiduos}\label{ejer:residuos:minimo-noresiduo}
	Sea $p$ un primo impar. Probar que el m\'{\i}nimo
	entero positivo no residuo m\'odulo $p$ es un n\'umero primo.
\end{ejerResiduos}

\begin{ejerResiduos}\label{ejer:residuos:residuos-consecutivos}%
	\hint{
		Separar en los tres casos:
		\begin{itemize}
			\item $\tlegendre 2 p=1$,
			\item $\tlegendre 5 p=1$ y
			\item $\tlegendre 2 p=\tlegendre 5 p=-1$.
		\end{itemize}
		%
	}
	Probar que para todo primo impar $p\geq 7$, existe
	$n\in\Enteros$, $1\leq n\leq 9$ tal que
	\begin{displaymath}
		\legendre n p \,=\,\legendre{n+1} p\,=\,1
	\end{displaymath}
	%
\end{ejerResiduos}

\begin{ejerResiduos}%
	\hint{
		Notar que existen $a,u\in\Enteros$ tales que
		$2\mid a$, $1\leq a\leq p-1$ y $a^2=5+up$.
		Separar en casos $5\nmid u$ y $5\mid u$.
	}
	Sea $p\neq 5$ un primo (positivo) impar, tal que
	$\tlegendre 5 p=1$. Probar que, entonces,
	$p\equiv\pm 1\tmodulo[5]$.
\end{ejerResiduos}

\begin{ejerResiduos}\label{ejer:residuos:gauss}
	Dado un primo (positivo) impar $p$, definimos los conjuntos:
	\begin{displaymath}
		\begin{aligned}
			RN & \,=\,
				\cardinal{\Big\{ n\in\Enteros\,:\,1\leq n\leq p-2,\,\tlegendre n p=1,\,\tlegendre{n+1} p=-1\Big\}}
				\dispcomma \\
			RR & \,=\,
				\cardinal{\Big\{ n\in\Enteros\,:\,1\leq n\leq p-2,\,\tlegendre n p=1,\,\tlegendre{n+1} p=1\Big\}}
				\dispcomma \\
			NR & \,=\,
				\cardinal{\Big\{ n\in\Enteros\,:\,1\leq n\leq p-2,\,\tlegendre n p=-1,\,\tlegendre{n+1} p=1\Big\}}
				\dispand \\
			NN & \,=\,
				\cardinal{\Big\{ n\in\Enteros\,:\,1\leq n\leq p-2,\,\tlegendre n p=-1,\,\tlegendre{n+1} p=-1\Big\}}
				\dispstop
		\end{aligned}
		%
	\end{displaymath}
	%
	Probar las siguientes afirmaciones.
	\begin{enumerate}[(i)]
		\item\label{item:residuos:gauss:i}
			Se cumple que
			\begin{displaymath}
				RR+RN\,=\,\left\{
					\begin{cases}
						-1+(p-1)/2\dispcomma &
							\text{si }
							p\equiv 1\tmodulo[4]
							\dispand \\
						(p-1)/2\dispcomma &
							\text{si }
							p\equiv 3\tmodulo[4]
							\dispstop
					\end{cases}
					%
			\end{displaymath}
			%
			Demostrar f\'ormulas para
			$NR+NN$, $RR+NR$ y para $RN+NN$.
		\item\label{item:residuos:gauss:ii}
			Se cumple que
			$RR+NN-RN-NR=-1$.
		\item\label{item:residuos:gauss:iii}
			Se cumple que
			\begin{displaymath}
				(RR,RN,NR,NN)\,=\,
					\begin{cases}
						(x-1,x,x,x)\dispcomma &
							\text{si }
							p=4x+1\dispand \\
						(x,x+1,x,x)\dispcomma &
							\text{si }
							p=4x+3\dispstop
					\end{cases}
					%
			\end{displaymath}
			%
	\end{enumerate}
	%
\end{ejerResiduos}

