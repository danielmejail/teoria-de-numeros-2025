\theoremstyle{definition}
\newtheorem{ejerFermat}{\ejername}[section]

%-------------

\begin{ejerFermat}\label{ejer:fermat:identidades}
	Verificar las siguientes identidades:
	\begin{enumerate}[(i)]
		\item\label{item:ejer:fermat:identidades:cuadrados}
			\begin{math}
				(x^2+y^2)\,(z^2+w^2)=
					(xz\pm yw)^2+(xw\mp yz)^2
			\end{math};
		\item\label{item:ejer:fermat:identidades}
			\begin{math}
				(x^2+ny^2)\,(z^2+nw^2)=
					(xz\pm nyw)^2+n\,(xw\mp yz)^2
			\end{math}.
	\end{enumerate}
	%
	Generalizar y hallar una identidad del estilo
	\begin{displaymath}
		(ax^2+cy^2)\,(az^2+cw^2)\,=\,(\text{?`?})^2+ac\,(\text{?`?})^2
		\text{ .}
	\end{displaymath}
	%
\end{ejerFermat}

\begin{ejerFermat}\label{ejer:fermat:soluciones:euler}
	Sean $p$ un n\'umero primo y
	\begin{displaymath}
		f\,=\,a_0\,+\,a_1\,X\,+\,\cdots\,+\,a_{d-1}\,X^{d-1}\,+\,X^d
	\end{displaymath}
	%
	un polinomio m\'onico (coeficiente principal $=1$) de grado $d<p$.
	Una de las conclusiones de este ejercicio ser\'a que
	$f(x)\not\equiv 0\tmodulo[p]$ tiene soluci\'on, es decir, existe
	$x\in\Enteros$ tal que $f(x)\not\equiv 0\tmodulo[p]$. Dicho de otra
	manera, $f(x)\equiv 0\tmodulo[p]$, con $f$ de grado $d<p$, no puede
	tener $p$ soluciones disitintas m\'odulo $p$.
	Con este objetivo, definimos $\Delta f$ como el polinomio
	\begin{displaymath}
		\Delta f\,=\,f(X+1)\,-\,f(X)
		\text{ ;}
	\end{displaymath}
	%
	podemos iterar y definir $\Delta^{k+1}f=\Delta(\Delta^k f)$.
	Probar las siguientes afirmaciones:
	\begin{enumerate}[(i)]
		\item\label{item:ejer:fermat:soluciones:euler:i}
			si $k\geq 1$, entonces $\Delta^k f$ es una
			combinaci\'on lineal de $f(X)$, $f(X+1)$, \dots,
			$f(X+k)$ con coeficientes enteros;
		\item\label{item:ejer:fermat:soluciones:euler:ii}
			si $k=d$, el polinomio $\Delta^d f$ es constante:
			$\Delta^d f(X)=d!$;
		\item\label{item:ejer:fermat:soluciones:euler:iii}
			existe $x\in\Enteros$ tal que
			$f(x)\not\equiv 0\tmodulo[p]$.%
			\hint{
				si $f(x)\not\equiv 0\tmodulo[p]$
				\emph{no admitiera soluciones}, entonces
				por
				\eqref{item:ejer:fermat:soluciones:euler:i},
				$p\mid\Delta^d f$ y, por
				\eqref{item:ejer:fermat:soluciones:euler:ii}
				(y $d<p$) esto es imposible.
			}
	\end{enumerate}
	%
\end{ejerFermat}

\begin{ejerFermat}\label{ejer:fermat:descenso:pre}
	Probar la siguiente versi\'on an\'aloga del \lemaname~%
	\ref{lema:fermat:descenso:pre}:
	\begin{quote}
		\itshape
		Sea $n>0$ un n\'umero entero positivo y sea $N$ un natural
		que se puede expresar como $N=a^2+nb^2$ con $a,b\in\Enteros$
		y $\mcd{a,b}=1$. Si $q$ es un divisor primo de $N$ que no
		divide a $n$ y se puede expresar como $q=x^2+ny^2$,
		entonces el cociente $N/q$ tambi\'en se puede expresar como
		$c^2+nd^2$ con $c,d\in\Enteros$ y $\mcd{c,d}=1$.
	\end{quote}
	Probar, adem\'as, que esto es cierto tambi\'en si $n=3$ y $q=4$.%
	% \hint{
		% Emular el argumento del \lemaname~%
		% \ref{lema:fermat:descenso:pre}
	% }
\end{ejerFermat}

% \begin{solucion}
	% Si $N=a^2+nb^2$, $\mcd{a,b}=1$, y $q=x^2+ny^2$, entonces
	% \begin{displaymath}
		% Nx^2-a^2q\,=\,(a^2+nb^2)\,x^2-a^2\,(x^2+ny^2)\,=\,
			% n\,(bx-ay)\,(bx+ay)
		% \text{ .}
	% \end{displaymath}
	% %
	% Como $q$ es primo, $q\mid N$ y $q\nmid n$, se deduce que
	% $q\mid (bx-ay)$, o bien $q\mid (bx+ay)$. Cambiando el signo de $a$,
	% podemos asumir que estamos en el primer caso. Esto quiere decir que
	% existe $d\in\Enteros$ tal que $bx-ay=dq$. Ahora bien,
	% \begin{displaymath}
		% bx-dx^2\,=\,bx-dq+ndy^2\,=\,ay+ndy^2\,=\,(a+ndy)\,y
		% \text{ .}
	% \end{displaymath}
	% %
	% Como $x\mid bx-dx^2$, se cumple que $x\mid (a+ndy)\,y$. Pero
	% $\mcd{x,y}=1$, con lo que $x\mid a+ndy$. Existe, entonces,
	% $c\in\Enteros$ tal que $a+ndy=cx$. O sea,
	% \begin{displaymath}
		% a\,=\,cx-ndy\quad\text{y}\quad b\,=\,dx+cy
		% \text{ .}
		% % bx\,=\,dq + ay\,=\,d\,(x^2+ny^2)+(cx-ndy)\,y\,=\,
		% % dx^2+cxy\,=\,(dx+cy)\,x
	% \end{displaymath}
	% %
	% Pero, entonces,
	% \begin{displaymath}
		% N\,=\,a^2+nb^2\,=\,(cx-ndy)^2+(dx+cy)^2
		% \,=\,(c^2+nd^2)\,(x^2+ny^2)\,=\,(c^2+nd^2)\,q
		% \text{ .}
	% \end{displaymath}
	% %
	% As\'{\i}, $N/q=c^2+nd^2$, pero, adem\'as, $\mcd{a,b}=1$ implica
	% que $\mcd{c,d}=1$.
% 
	% Si $n=3$ y $q=4$, entonces la situaci\'on es la siguiente:
	% $x=y=1$, $N=a^2+3b^2$, $a,b\in\Enteros$, $\mcd{a,b}=1$ y $4\mid N$
	% (y $\mcd{q,n}=1$).
	% Si $4\mid b-a$, entonces $b=a+4d$. En ese caso, definimos
	% $c=a+3d$ y, as\'{\i},
	% \begin{displaymath}
		% a\,=\,c-3d\quad\text{y}\quad b\,=\,c+d
		% \text{ .}
	% \end{displaymath}
	% %
	% Pero, entonces,
	% \begin{displaymath}
		% N\,=\,a^2+3b^2\,=\,(c-3d)^2+3\,(d+c)^2\,=\,
			% 4\,(c^2+3d^2)
		% \text{ .}
	% \end{displaymath}
	% %
	% Si $4\nmid b+a$, entonces $2\mid b-a$; si, adem\'as, $4\nmid b-a$,
	% entonces
	% \begin{displaymath}
		% b\,\equiv\,a+2\tmodulo[4]\quad\text{y}\quad
			% b+a\,\equiv\,2\,(a+1)\tmodulo[4]
		% \text{ .}
	% \end{displaymath}
	% %
	% Como $2\mid a$ implica $2\mid b$, lo que contradir\'{\i}a
	% $\mcd{a,b}=1$, debe ser $2\nmid a$. Pero, entonces ser\'{\i}a
	% $a+b\equiv 0\tmodulo[4]$. Otra contradicci\'on. Debe ser entonces
	% $4\mid b+a$. Cambiando el signo de $a$, podemos asumir $4\mid b-a$.
	% En definitiva, si $N=a^2+3b^2$, $\mcd{a,b}=1$ y $4\mid N$, entonces
	% $N=4\,(c^2+3d^2)$ para ciertos $c,d\in\Enteros$, $\mcd{c,d}=1$.
% \end{solucion}
% 

\begin{ejerFermat}\label{ejer:fermat:descenso:caso}
	Sea $q$ un primo y sea $N=a^2+mqb^2$ un natural ($n=mq$) donde
	$a,b,m\in\Enteros$.
	Probar que, si $q$ divide a $N$, entonces $N/q=mc^2+qd^2$ donde
	$c,d\in\Enteros$.
	Probar, adem\'as, que, si $\mcd{a,b}=1$, entonces $\mcd{c,d}=1$.
\end{ejerFermat}

\begin{ejerFermat}\label{ejer:fermat:descenso}
	Probar las siguientes versiones an\'alogas del \lemaname~%
	\ref{lema:fermat:descenso} para $n=2,3$:%
	% \hint{
		% Emular el argumento del \lemaname~\ref{lema:fermat:descenso}
	% }
	\begin{quote}
		\itshape
		Sea $p$ un primo que divide a una expresi\'on
		del tipo $a^2+2b^2$ donde $a,b\in\Enteros$ y
		$\mcd{a,b}=1$. Entonces, existen
		$x,y\in\Enteros$ tales que $p=x^2+2y^2$.
	\end{quote}
	\begin{quote}
		\itshape
		Sea $p$ un primo impar que divide a una
		expresi\'on del tipo $a^2+3b^2$ donde
		$a,b\in\Enteros$ y $\mcd{a,b}=1$. Entonces,
		existen $x,y\in\Enteros$ tales que
		$p=x^2+3y^2$.
	\end{quote}
\end{ejerFermat}

\begin{ejerFermat}
	Probar la siguiente versi\'on an\'aloga del \lemaname~%
	\ref{lema:fermat:reciprocidad}:
	\begin{quote}
		\itshape
		Si $p$ es primo y $p\equiv 1\tmodulo[3]$, entonces $p$ divide
		a una expresi\'on del tipo $a^2+3b^2$ donde $a,b\in\Enteros$
		y $\mcd{a,b}=1$,
	\end{quote}
	usando, por ejemplo, la siguiente identidad:
	\begin{displaymath}
		4\,(x^{3k}-1)\,=\,4\,(x^{2k}+x^k+1)\,(x^k-1)\,=\,
			((2x^k+1)^2+3)\,(x^k-1)
		\text{ .}
	\end{displaymath}
	%
\end{ejerFermat}

\begin{ejerFermat}
	Probar que si $p\equiv 1\tmodulo[8]$ es primo, entonces $p$ divide a
	una expresi\'on del tipo $a^2+2b^2$ donde $a,b\in\Enteros$ y
	$\mcd{a,b}=1$, usando la siguiente identidad:
	\begin{displaymath}
		x^{8k}-1\,=\, ((x^{2k}-1)^2+2x^{2k})\,(x^{4k}-1)
		\text{ .}
	\end{displaymath}
	%
	Mostrar que existen primos $p\equiv 3\tmodulo[8]$ que dividen a
	n\'umeros de la forma $a^2+2b^2$.
	% con $a,b\in\Enteros$ y $\mcd{a,b}=1$.
\end{ejerFermat}

\begin{ejerFermat}
	Para cada primo $p\equiv 1\tmodulo[3]$, $p\leq 50$, buscar todos los
	valores $x,y\in\Enteros$ tales que $p=x^2+3y^2$.
\end{ejerFermat}

\begin{ejerFermat}\label{ejer:fermat:cinco}
	En este ejercicio investigaremos condiciones que sirvan para
	determinar si un primo divide a una expresi\'on de la forma
	$a^2+5b^2$ donde $a,b\in\Enteros$ y $\mcd{a,b}=1$.
	\begin{enumerate}[(i)]
		\item\label{item:ejer:fermat:cinco:i}
			Calcular que, si $p$ es un primo que divide a un
			natural de la forma $a^2+5b^2$ con $a,b\leq 40$
			y $\mcd{a,b}=1$, entonces $p$ puede ser
			$\equiv 1,3,2,4\tmodulo[5]$.
			O sea, en este caso, mirar congruencia m\'odulo $5$
			no parece dar informaci\'on acerca de si un primo
			divide o no a un natural de la forma $a^2+5b^2$.
		\item\label{item:ejer:fermat:cinco:ii}
			Hallar primos $p\equiv 2,4\tmodulo[5]$ que no
			dividen a ning\'un natural de la forma $a^2+5b^2$
			con $a,b\leq 40$ y $\mcd{a,b}=1$
			(dicho de otra manera, que no aparecen entre los
			divisores de ninguno de estos n\'umeros).
		\item\label{item:ejer:fermat:cinco:iii}
			Calcular que, si $p$ es un primo que divide a un
			natural de la forma $a^2+5b^2$ con $a,b\leq 40$
			y $\mcd{a,b}=1$, entonces
			$p\equiv 1,3,7,9\tmodulo[20]$.
	\end{enumerate}
	%
\end{ejerFermat}

\begin{ejerFermat}\label{ejer:fermat:otras}
	En este ejercicio investigamos otras formas de representar.
	En las siguientes afirmaciones, $a,b\in\Enteros$, $\mcd{a,b}=1$
	y $|a|,|b|\leq 40$.
	\begin{enumerate}[(i)]
		\item\label{item:ejer:fermat:siete}
			Determinar las posibles clases de congruencia
			m\'odulo $28$ de los primos $p$ que dividen a
			los enteros $a^2+7b^2$.
		\item\label{item:ejer:fermat:menos-tres}
			Determinar las posibles clases de congruencia
			m\'odulo $12$ de los primos $p$ que dividen a
			los enteros $a^2-3b^2$.
		\item\label{item:ejer:fermat:menos-cinco}
			Determinar las posibles clases de congruencia
			m\'odulo $20$ de los primos $p$ que dividen a
			los enteros $a^2-5b^2$.
		\item\label{item:ejer:fermat:menos-siete}
			Determinar las posibles clases de congruencia
			m\'odulo $28$ de los primos $p$ que dividen a
			los enteros $a^2-7b^2$.
	\end{enumerate}
	%
	Notar que en los casos \eqref{item:ejer:fermat:menos-tres},
	\eqref{item:ejer:fermat:menos-cinco} y
	\eqref{item:ejer:fermat:menos-siete} las clases obtenidas se pueden
	representar como $\pm\beta^2$ donde $\beta$ es un n\'umero impar.
\end{ejerFermat}
