\theoremstyle{plain}
\newtheorem{lemaResiduos}{\lemaname}[section]

\theoremstyle{definition}
\newtheorem{ejemResiduos}[lemaResiduos]{\ejemname}

%-------------

En esta secci\'on estudiaremos la resolubilidad de
$x^2\equiv n\tmodulo[m]$ especialmente en el caso $m=p$ un primo impar.
Recordemos que $n\in\Enteros$ es un \emph{residuo (cuadr\'atico) m\'odulo $m$},
si la ecuaci\'on tiene soluci\'on y un \emph{no residuo}, en caso contrario.
Nuestro objetivo ahora ser\'a demostrar las propiedades del \teoname~%
\ref{teo:ecuacion:legendre}:

\teoEcuacionLegendre*

La propiedad~\eqref{item:ecuacion:legendre:clases} es casi inmediata:
si $n\equiv n'\tmodulo[p]$, entonces $\mcd{n,p}=\mcd{n',p}$, con lo que
$\tlegendre n p=0$, si y s\'olo si $\tlegendre{n'} p=0$, si y s\'olo si $p$
divide a $n$ y a $n'$; en otro caso, si $x$ es soluci\'on de $x^2\equiv n$,
entonces $x^2\equiv n'$, y viceversa, con lo que $\tlegendre n p=1$, si y
s\'olo si $\tlegendre{n'} p=1$. Como el s\'{\i}mbolo de Legendre toma valores
$0$, $1$ o $-1$ (como $x^2\equiv n$ tiene soluci\'on o no la tiene),
concluimos que $\tlegendre n p=\tlegendre{n'} p$, si $n\equiv n'\tmodulo[p]$.

Empezamos por el siguiente resultado general acerca de la cantidad de
soluciones a una ecuaci\'on polinomial m\'odulo un primo.

\begin{lemaResiduos}\label{lema:residuos:polinomios}
	Sea $p$ un primo y sea $f\in\polinomios\Enteros X$,
	$f=c_0+c_1X+\,\cdots\,+c_nX^n$, $n\geq 0$ y $p\nmid c_n$.%
	\footnote{
		El \lemaname~\ref{lema:residuos:polinomios} se puede omitir,
		teniendo en cuenta los resultados de la \S~%
		\ref{sec:polinomios}, espec\'{\i}ficamente, el \coroname~%
		\ref{coro:polinomios:algoritmo}, \'{\i}tem~%
		\eqref{item:coro:polinomios:algoritmo:raices:bis}. Es, sin
		embargo, una versi\'on elemental, en el sentido de que no
		hace uso de los conceptos desarrollados en la \S~%
		\ref{sec:polinomios}. Esencialmente, la hip\'otesis sobre $f$
		es que el polinomio en $\polinomios{\Enterosmod[p]} X$ que se
		obtiene reduciendo sus coeficientes m\'odulo $p$ tiene grado
		$n\geq 0$.
	}
	Entonces, la congruencia $f(x)\equiv 0\tmodulo[p]$ tiene, a lo sumo,
	$n$ soluciones distintas.
\end{lemaResiduos}

\begin{proof}
	Si $n=0$, $c_0\not\equiv 0$ y $f(x)\equiv 0$ no tiene soluciones.
	Si $n>0$ y $x_0\in\Enteros$ cumple $f(x_0)\equiv 0$, entonces
	\begin{displaymath}
		f(x)\,-\,f(x_0)\,\equiv\,(x-x_0)\,g(x)
		\text{ ,}
	\end{displaymath}
	%
	para cierto polinomio $g\in\polinomios\Enteros X$,
	$g=b_0+b_1X+\,\cdots\,+b_{n-1}X^{n-1}$, donde
	$b_{n-1}=c_n\not\equiv 0$, es decir, no es divisible por $p$
	\quedacomoejercicio.%
	\hint{
		Usar que
		\begin{math}
			x^t-x_0^t=(x-x_0)\,
				(x^{t-1}+x^{t-2}x_0+\,\cdots\,+x_0^{t-1})
		\end{math}.
	}
	Pero, inductivamente, $g$ no posee m\'as de $n-1$ ra\'{\i}ces
	distintas m\'odulo $p$, con lo que $f$ no puede poseer m\'as de
	$n$ ra\'{\i}ces distintas m\'odulo $p$.
\end{proof}

\begin{lemaResiduos}\label{lema:residuos:residuos}
	Sea $p$ un primo positivo impar. Entonces,
	\begin{enumerate}[(i)]
		\item\label{item:residuos:residuos:cantidad}
			excluyendo $0$, existen exactamente $\frac{p-1} 2$
			residuos cuadr\'aticos m\'odulo $p$ y, por lo tanto,
			$\frac{p-1} 2$ no residuos en cualquier sistema
			reducido de representantes de las clases m\'odulo $p$;
		\item\label{item:residuos:residuos:cuadrados}
			las clases m\'odulo $p$ de los enteros $1^2$, $2^2$,
			\dots, $(\tfrac{p-1} 2)^2$ son distintas;
		\item\label{item:residuos:residuos:representantes}
			en particular, exceptuando $0$, los residuos
			m\'odulo $p$ est\'an representados por las
			$\frac{p-1} 2$ clases del \'{\i}tem~%
			\eqref{item:residuos:residuos:cuadrados}.
	\end{enumerate}
	%
\end{lemaResiduos}

\begin{proof}
	Busquemos soluciones a la ecuaci\'on de congruencia
	\begin{equation}
		\label{eq:residuos:residuos:congruencia}
		x^2\,\equiv\,n\modulo[p]
		\text{ .}
	\end{equation}
	%
	Si $n\equiv 0$, entonces la \'unica soluci\'on es $x\equiv 0$.
	Supongamos que $n\not\equiv 0$ y que
	\eqref{eq:residuos:residuos:congruencia} admite una soluci\'on.
	Tomando resto de divisi\'on por $p$,
	% podemos ver que
	existe al menos una soluci\'on en el rango $0\leq x\leq p-1$;
	como $n$ no es divisible por $p$, $x=0$ no es una de ellas.
	Si $x>p/2$, cambiando $x$ por $p-x$,
	% vemos que
	existe al menos una soluci\'on en el rango $1\leq x\leq\frac{p-1} 2$.
	Por lo tanto, \eqref{eq:residuos:residuos:congruencia} admite
	soluci\'on, si y s\'olo si $n$ es congruente a alguna de las clases
	\begin{equation}
		\label{eq:residuos:residuos:cuadrados}
		1^2\text{ , }2^2\text{ , \dots, }(\tfrac{p-1} 2)^2
		\text{ .}
	\end{equation}
	%
	En particular, exceptuando $0$, hay a lo sumo $\frac{p-1} 2$ residuos
	cuadr\'aticos m\'odulo $p$.
	Pero las clases \eqref{eq:residuos:residuos:cuadrados} son todas
	distintas \quedacomoejercicio%
	\hint{
		Si $x$ e $y$ son soluciones a
		\eqref{eq:residuos:residuos:congruencia}, entonces
		$x\equiv y$, o bien $x\equiv -y$, y estas posibilidades
		se excluyen mutuamente, si $n\not\equiv 0$.
		Si $1\leq x,y\leq\frac{p-1} 2$ e $y\neq x$, entonces
		tendr\'{\i}amos cuatro soluciones distintas m\'odulo $p$,
		contradiciendo el \lemaname~\ref{lema:residuos:polinomios}.
	}
	y, en consecuencia, hay $\frac{p-1} 2$ residuos cuadr\'aticos
	m\'odulo $p$ (exceptuando $0$).
	% % Tomando restos de divisi\'on por $p$, podemos elegir, como sistema
	% % de representantes de las clases m\'odulo $p$, los enteros $x$ en el
	% % rango $0\leq x\leq p-1$ y, como sistema reducido, aquellos $x\neq 0$.
	% Por el \lemaname~\ref{lema:residuos:polinomios}, la ecuaci\'on de
	% congruencia
	% \begin{equation}
		% \label{eq:residuos:residuos}
		% x^2\,\equiv\,n\modulo[p]
	% \end{equation}
	% %
	% admite, a lo sumo, dos soluciones en el intervalo $0\leq x\leq p-1$ y,
	% si $n\not\equiv 0$, $x=0$ no es una de ellas; adem\'as, si
	% \eqref{eq:residuos:residuos} tiene soluci\'on, podemos garantizar que
	% existe, al menos, una soluci\'on en el intervalo $0\leq x\leq p-1$,
	% tomando resto de divisi\'on por $p$.
% 
	% Supongamos que $p$ no divide a $n$ y que $1\leq x\leq p-1$.
	% Si $x$ es soluci\'on de \eqref{eq:residuos:residuos}, $p-x$
	% tambi\'en lo es, pues $(p-x)^2\equiv (-x)^2=x^2$, y, adem\'as, se
	% cumple que $1\leq p-x\leq p-1$. Pero, si $x>\frac{p-1} 2$, entonces
	% $1\leq p-x\leq\frac{p-1} 2$, con lo cual, si
	% \eqref{eq:residuos:residuos} tiene soluci\'on, podemos garantizar que
	% existe, al menos, una soluci\'on en el intervalo
	% $1\leq x\leq\frac{p-1} 2$. En particular, las clases m\'odulo $p$
	% de los $\frac{p-1} 2$ enteros $1^2$, $2^2$, \dots, $(\tfrac{p-1} 2)^2$
	% representan todos los posibles residuos cuadr\'aticos m\'odulo $p$,
	% con lo cual, exceptuando $0$, hay a lo sumo $\frac{p-1} 2$ residuos
	% cuadr\'aticos m\'odulo $p$.
	% Pero podr\'{\i}an haber repeticiones. Veamos que no es as\'{\i}, que
	% las clases son distintas y que, por lo tanto, por ser $\frac{p-1} 2$
	% clases distintas, hay exactamente $\frac{p-1} 2$ residuos
	% cuadr\'aticos m\'odulo $p$ (exceptuando $0$).
% 
	% Sigamos asumiendo que $p\nmid n$ y que $1\leq x\leq\frac{p-1} 2$
	% es solu
	% $\not\equiv 0$, $p-x\not\equiv x$. Es decir,
	% por cada $x$ en el rango $1\leq x\leq p-1$, tenemos dos soluciones
	% distintas m\'odulo $p$: $x$ y $p-x$ en el mismo rango.
	% En particular, si \eqref{eq:residuos:residuos} tiene soluci\'on, debe
	% tener exactamente una soluci\'on en el rango
	% $1\leq x\leq \frac{p-1} 2$ \quedacomoejercicio.
	% En consecuencia, las clases m\'odulo $p$ de los $\frac{p-1} 2$
	% enteros $1^2$, $2^2$, \dots, $(\tfrac{p-1} 2)^2$ son todas distintas.
\end{proof}

\begin{lemaResiduos}[Criterio de Euler]\label{lema:residuos:euler}
	Si $p$ es un primo positivo impar, entonces
	\begin{displaymath}
		\legendre n p\,\equiv\,n^{\frac{p-1} 2}\modulo[p]
		\text{ .}
	\end{displaymath}
	%
\end{lemaResiduos}

\begin{proof}
	Si $n\equiv 0$, no hay nada que probar. Supongamos que $n\not\equiv 0$.
	El \teoname~\ref{teo:congruencias:fermatito} implica que
	$n^{p-1}\equiv 1$. En particular,
	\begin{displaymath}
		\big(n^{\frac{p-1} 2}-1\big)\,
		\big(n^{\frac{p-1} 2}+1\big)\,\equiv\,0
		\text{ ,}
	\end{displaymath}
	%
	o sea que $n^{\frac{p-1} 2}\equiv\pm 1$. La afirmaci\'on es que este
	signo coincide con $\tlegendre n p$. Notamos que no pueden darse
	simult\'aneamente ambas posibilidades.

	Por un lado, si $\tlegendre n p=1$, existe $x\in\Enteros$ tal que
	$n\equiv x^2$ y, entonces
	\begin{displaymath}
		n^{\frac{p-1} 2}\,\equiv\,x^{p-1}\,\equiv\,1
		\text{ .}
	\end{displaymath}
	%
	Por otro lado, por el \lemaname~\ref{lema:residuos:polinomios}, la
	congruencia $n^{\frac{p-1} 2}\equiv 1$ tiene, a lo sumo,
	$\frac{p-1} 2$ soluciones distintas m\'odulo $p$. Pero, por el
	\lemaname~\ref{lema:residuos:residuos}, los enteros
	$1^2$, $2^2$, \dots, $(\tfrac{p-1} 2)^2$ representan $\frac{p-1} 2$
	clases distintas y son, por lo tanto, $\frac{p-1} 2$ soluciones
	distintas (m\'odulo $p$) a dicha ecuaci\'on.
	As\'{\i}, si $\tlegendre n p=-1$, $n\not\equiv x^2$ para ning\'un
	$1\leq x\leq\frac{p-1} 2$ y, en consecuencia,
	$n^{\frac{p-1} 2}-1\not\equiv 0$. Como $p$ es primo, debe cumplirse,
	en este caso,
	\begin{displaymath}
		n^{\frac{p-1} 2}\,\equiv\,-1
		\text{ .}
	\end{displaymath}
	%
\end{proof}

Del \lemaname~\ref{lema:residuos:euler}, deducimos, eligiendo $a=-1$ que
$\tlegendre{-1} p\equiv(-1)^{\frac{p-1} 2}\tmodulo[p]$. Pero, dado que
ambos lados de esta congruencia son iguales a $\pm 1$ y que $p$ es impar,
concluimos que, en realidad, $\tlegendre{-1} p=(-1)^{\frac{p-1} 2}$. Esto
prueba el \'{\i}tem~\eqref{item:ecuacion:legendre:menos-uno}.

De manera similar, usando el \lemaname~\ref{lema:residuos:euler}, tambi\'en
podemos probar el \'{\i}tem~\eqref{item:ecuacion:legendre:multiplicativo}:
\begin{displaymath}
	\legendre{ab} p\,\equiv\,(ab)^{\frac{p-1} 2}\,=\,
		a^{\frac{p-1} 2}\,b^{\frac{p-1} 2}\,\equiv\,
		\legendre a p\legendre b p\modulo[p]
	\text{ ,}
\end{displaymath}
%
o sea que $\tlegendre{ab} p\equiv\tlegendre a p\tlegendre b p\tmodulo[p]$.
Pero, al igual que antes, ambos lados de la congruencia son iguales a $\pm 1$
y $p$ es impar, con lo cual, $\tlegendre{ab} p=\tlegendre a p\tlegendre b p$.

Antes de pasar a la demostraci\'on del \teoname~\ref{teo:ecuacion:legendre},
demostramos un \'ultimo resultado preliminar.

\begin{lemaResiduos}[Lema de Gauss]\label{lema:residuos:gauss}
	% Sea $p$ un primo positivo impar y sea $n\in\Enteros$,
	% $\mcd{n,p}=1$. De los restos $0<r<p$ m\'odululo $p$ obtenidos de los
	% $\frac{p-1} 2$ enteros $n$, $2n$, \dots, $\tfrac{p-1} 2 n$, sea
	% $m\geq 0$ la cantidad de los mismos que son mayores que $p/2$.
	Dado un primo positivo impar $p$ y un n\'umero entero $n$ coprimo
	con $p$, de los $\frac{p-1} 2$ restos de divisi\'on por $p$
	provenientes de los enteros $n$, $2n$, \dots, $\tfrac{p-1} 2 n$,
	sea $m\geq 0$ la cantidad de los mismos que pertenecen al rango
	$p/2<r<p$. Entonces,
	\begin{displaymath}
		\legendre n p\,=\,(-1)^m
		\text{ .}
	\end{displaymath}
	%
\end{lemaResiduos}

\begin{ejemResiduos}\label{ejem:residuos:gauss}
	Si $p=7$, $n=10\equiv 3\tmodulo[7]$, entonces $3\equiv x^2\tmodulo[7]$
	no tiene soluci\'on --los cuadrados son $1$, $4$ y $2$. Por otro lado,
	$10$, $20$ y $30$ tienen restos $3$, $6$ y $2$, respectivamente, al
	dividir por $7$, con lo que $m=1$ y se verifica que
	$\tlegendre{10} 7=\tlegendre 3 7=-1=(-1)^m$.
\end{ejemResiduos}

\begin{ejemResiduos}\label{ejem:residuos:gauss:trivial}
	Si $n=1$, $m=0$ y, entonces $\tlegendre 1 p=1=(-1)^m$ tambi\'en
	en este caso.
\end{ejemResiduos}

\begin{proof}
	En primer lugar, como $\mcd{n,p}=1$, los enteros
	$n$, $2n$, \dots, $\tfrac{p-1} 2n$ pertenecen a clases distintas
	m\'odulo $p$ y, en particular, sus restos de dividir por $p$ son
	distintos (y son $\frac{p-1} 2$ en cantidad).
	Sea, ahora, $m\geq 0$ como en el enunciado y sea $l=\frac{p-1} 2-m$.
	Sea $\resto(k)=\resto[p](k)$ el resto de dividir $k\in\Enteros$ por
	$p$; si $k\not\equiv 0$, entonces $0<\resto(k)<p$.
	Sea $\{\lista a{l}\}$ el subconjunto de restos $\resto(nx)<p/2$ y
	sea $\{\lista b{m}\}$ el subconjunto de restos $\resto(nx)>p/2$
	con $x$ variando en el rango $1\leq x\leq\frac{p-1} 2$.
	Multiplicando,
	\begin{displaymath}
		\prod_{i=1}^l\,a_i\,\prod_{j=1}^m\,b_j\,\equiv\,
			\prod_{x=1}^{(p-1)/2}\,nx\,=\,
			n^{\frac{p-1} 2}\,\Big(\frac{p-1} 2\Big)!
		\text{ .}
	\end{displaymath}
	%
	Por otro lado, para cada $j$, $p-b_j$ pertenece al intervalo
	$0<p-b_j<p/2$, pues $p/2<b_j<p$; los $a_i$ tambi\'en pertenecen a
	dicho intervalo, pero deben ser distintos de los $p-b_j$
	\quedacomoejercicio%
	\hint{
		Una igualdad $a_i=p-b_j$ implicar\'{\i}a $nx=p-ny$ con
		$1\leq x,y\leq\frac{p-1} 2$, lo que es absurdo, pues
		ser\'{\i}a $n(x+y)\equiv 0$ y, por lo tanto, $x+y\equiv 0$
		pero $2\leq x+y\leq p-1$.
	}
	As\'{\i}, porque $l+m=\frac{p-1} 2$, vale que los conjuntos siguientes
	son iguales:
	\begin{displaymath}
		\big\{\lista a l\}\,\cup\,\big\{p-b_1,\,\dots,\,p-b_m\big\}
		\,=\,\Big\{1,\,2,\,\dots,\,\frac{p-1} 2\Big\}
		\text{ .}
	\end{displaymath}
	%
	Multiplicando,
	\begin{displaymath}
		\Big(\frac{p-1} 2\Big)!\,=\,
			\prod_{i=1}^l\,a_i\,\prod_{j=1}^m\,(p-b_j)\,\equiv\,
			(-1)^m\,\prod_{i=1}^l\,a_i\,\prod_{j=1}^m\,b_j
		\text{ .}
	\end{displaymath}
	%
	En definitiva,
	\begin{displaymath}
		\Big(\frac{p-1} 2\Big)!\,\equiv\,(-1)^m\,n^{\frac{p-1} 2}\,
			\Big(\frac{p-1} 2\Big)!
	\end{displaymath}
	%
	y, $1\equiv(-1)^mn^{\frac{p-1} 2}$, o, lo que es lo mismo,
	$(-1)^m\equiv n^{\frac{p-1} 2}$. Pero, por el Criterio de Euler
	(\lemaname~\ref{lema:residuos:euler}),
	$n^{\frac{p-1} 2}\equiv\tlegendre n p$.
\end{proof}

\begin{proof}[\proofname~del \teoname~\ref{teo:ecuacion:legendre}]
	S\'olo resta demostrar las partes \eqref{item:ecuacion:legendre:dos}
	y \eqref{item:ecuacion:legendre:reciprocidad}. Sea $q$ un primo
	positivo impar distinto de $p$, o bien $q=2$.
	Primero, probamos una congruencia v\'alida para ambos casos (ver~%
	\eqref{eq:residuos:legendre}), que luego usamos para deducir los
	resultados.

	Para cada $x\in\Enteros$ en el rango $1\leq x\leq \frac{p-1} 2$,
	como en la demostraci\'on del \lemaname~\ref{lema:residuos:gauss},
	con $n=q$, sea $\resto[p](xq)$ el resto de la divisi\'on por $p$.
	En particular, se cumple que
	\begin{displaymath}
		xq\,=\,\piso{\frac{xq} p}\,p\,+\,\resto[p](xq)
		\text{ ,}
	\end{displaymath}
	%
	donde $\piso{\cdot}$ denota la parte entera.%
	\footnote{
		$\piso z\in\Enteros$ y $\piso z\leq z<\piso z+1$.
	}
	Sabemos que estos restos son todos distintos y, como $q\neq p$,
	% $\mcd{q,p}=1$
	distintos de cero y que se separan en dos subconjuntos disjuntos
	$\{\lista a l\}$ y $\{\lista b m\}$, donde $a_i<p/2$, $b_j>p/2$ y
	$l+m=\frac{p-1} 2$. Adem\'as, sabemos que
	\begin{displaymath}
		\big\{\lista a l\big\}\,\cup\,\big\{\lista{p-b} m\big\}\,=\,
			\Big\{1,\,2,\,\dots,\,\frac{p-1} 2\Big\}
		\text{ .}
	\end{displaymath}
	%
	De esto, deducimos que
	\begin{displaymath}
		\sum_{x=1}^{(p-1)/2}\,x\,=\,\sum_{i=1}^l\,a_i\,+\,
			\sum_{j=1}^m\,(p-b_j)\,=\,a+mp-b
		\text{ ,}
	\end{displaymath}
	%
	donde $a:=\sum_{i=1}^l\,a_i$ y $b:=\sum_{j=1}^m\,b_j$.
	Por otra parte,
	\begin{displaymath}
		\sum_{x=1}^{(p-1)/2}\,\resto[p](xq)\,=\,a+b
		\quad\text{y tambi\'en}\quad
		\sum_{x=1}^{(p-1)/2}\,x\,=\,
			\frac{\frac{p-1} 2\frac{p+1} 2} 2
			\,=\,\frac{p^2-1} 8
		\text{ .}
	\end{displaymath}
	%
	Con esto,
	\begin{displaymath}
		\begin{aligned}
			\frac{p^2-1} 8\,(q-1) & \,=\,
				\sum_x\,(xq)\,-\,\sum_x\,x\,=\,
				\sum_x\,\piso{\frac{xq} p}\,p\,+\,
					\sum_x\,\resto[p](xq)
				\,-\,\sum_x\,x \\
			& \,=\,\Big(\sum_x\,\piso{\frac{xq} p}\Big)\,p+a+b
				\,-\,\big(a+mp-b\big) \\
			& \,=\,\Big(\sum_x\,\piso{\frac{xq} p}\,-\,m\Big)\,p
				\,+\,2b
		\end{aligned}
		%
	\end{displaymath}
	%
	y, en particular,
	\begin{equation}
		\label{eq:residuos:legendre}
		\frac{p^2-1} 8\,(q-1)\,\equiv\,\sum_{x=1}^{(p-1)/2}\,
			\piso{\frac{xq} p}\,+\,m
			\modulo[2]
		\text{ .}
	\end{equation}
	%

	Ahora bien, si $q=2$, del lado izquierdo de~%
	\eqref{eq:residuos:legendre} queda $\frac{p^2-1} 8$ y del lado
	derecho s\'olo $m$, pues $\piso{\frac{x2} p}=0$ para
	$1\leq x\leq\frac{p-1} 2$. Usando el \lemaname~%
	\ref{lema:residuos:gauss} y esta congruencia m\'odulo $2$,
	se deduce \eqref{item:ecuacion:legendre:dos}:
	\begin{displaymath}
		\legendre 2 p\,=\,(-1)^m\,=\,(-1)^{\frac{p^2-1} 8}
		\text{ .}
	\end{displaymath}
	%

	Si, en cambio, $q$ es un primo positivo impar distinto de $p$,
	el lado izquierdo de~\eqref{eq:residuos:legendre} es par y,
	de nuevo, usando el \lemaname~\ref{lema:residuos:gauss} y esta
	congruencia m\'odulo $2$,
	\begin{displaymath}
		\legendre q p\,=\,(-1)^m\,=\,(-1)^{\sum_x\,\piso{\frac{xq} p}}
		\text{ ,}
	\end{displaymath}
	%
	donde $1\leq x\leq\frac{p-1} 2$.
	An\'alogamente (porque $q$ es primo impar y $p\neq q$), se cumple que
	\begin{math}
		\tlegendre p q\,=\,(-1)^{\sum_y\,\piso{\frac{yp} q}}
	\end{math},
	donde $1\leq y\leq\frac{q-1} 2$. En particular,
	\begin{displaymath}
		\legendre q p\,\legendre p q\,=\,(-1)^{\sum_x+\sum_y}
		\text{ .}
	\end{displaymath}
	%
	Afirmamos que (y esto es suficiente para probar~%
	\eqref{item:ecuacion:legendre:reciprocidad})
	\begin{equation}
		\label{eq:residuos:legendre:reciprocidad}
		\sum_{x=1}^{(p-1)/2}\,\piso{\frac{xq} p}\,+\,
			\sum_{y=1}^{(q-1)/2}\,\piso{\frac{yp} q}\,\equiv\,
			\frac{p-1} 2\frac{q-1} 2\modulo[2]
		\text{ .}
	\end{equation}
	%
	De hecho, mostraremos que vale una igualdad.

	Para probar \eqref{eq:residuos:legendre:reciprocidad},
	recurrimos al siguiente artilugio. Sea $f$ la funci\'on%
	\footnote{
		Si $a,b\in\Enteros$, $\iIntervalo a b=\{a,\,a+1,\,\dots,\,b\}$.
	}
	\begin{displaymath}
		f\,:\,\iIntervalo 1{\tfrac{p-1} 2}\,\times\,
			\iIntervalo 1{\tfrac{q-1} 2}\,\rightarrow\,
			\Enteros
		\quad\text{dada por}\quad
		f(x,y)\,=\,-xq+yp
		\text{ ;}
	\end{displaymath}
	%
	como $q\neq p$ son primos, $f(x,y)\neq 0$.
	Ahora, el dominio de $f$ tiene cardinal $\frac{p-1} 2\frac{q-1} 2$.
	Vamos a expresar el dominio de $f$ como la uni\'on disjunta
	$\{f<0\}\cup\{f>0\}$ y determinar el cardinal de cada parte.
	% El truco de contar los elementos de un mismo conjunto de dos
	% maneras distintas.
	Notemos, primero, que $\frac{yp} q\not\in\Enteros$, si
	$1\leq y\leq\frac{q-1} 2$. Por otro lado, $f(x,y)>0$, si y s\'olo
	si $\frac{yp} q>x$, o sea, teniendo en cuenta la observaci\'on
	inmediata anterior, $f(x,y)>0$ equivale a
	$1\leq x\leq \piso{\frac{yp} q}$. Pero, fijado $y$ en el rango
	$q\leq y\leq\frac{q-1} 2$, todo $x$ en el rango $1\leq x<\frac{yp} q$
	cumple $1\leq x\leq\frac{p-1} 2$, con lo que $(x,y)$ pertenece al
	dominio de $f$. Entonces, fijado $y$, hay $\piso{\frac{yp} q}$ pares
	$(x,y)$ tales que $f(x,y)>0$. As\'{\i},
	\begin{displaymath}
		\cardinal{\big\{f>0\big\}}\,=\,\sum_{y=1}^{(q-1)/2}\,
			\piso{\frac{yp} q}
		\text{ .}
	\end{displaymath}
	%
	An\'alogamente,
	\begin{math}
		\cardinal{\{f<0\}}=\sum_{x=1}^{(p-1)/2}\,\piso{\frac{xq} p}
	\end{math}.
	En definitiva,
	\begin{displaymath}
		\frac{p-1} 2\frac{q-1} 2\,=\,
		% \cardinal{\Big(%
			% \iIntervalo 1{\tfrac{p-1} 2}\times%
			% \iIntervalo 1{\tfrac{q-1} 2}\Big)}\,=\,
		\cardinal{\big\{f<0\big\}}\,+\,
			\cardinal{\big\{f>0\big\}}\,=\,
			\sum_{x=1}^{(p-1)/2}\,\piso{\frac{xq} p}\,+\,
			\sum_{y=1}^{(q-1)/2}\,\piso{\frac{yp} q}
		\text{ .}
	\end{displaymath}
	%
\end{proof}

\subsection*{Ejercicios}
\theoremstyle{definition}
\newtheorem{ejerResiduos}{\ejername}[section]

%-------------

\begin{ejerResiduos}\label{ejer:residuos:potencia}
	Sean $N\in\Enteros$, $p$ un primo que no divide a $N$ y $l\geq 1$.
	La cantidad de soluciones a la ecuaci\'on de congruencia
	\begin{equation}
		\label{eq:ejer:residuos:potencia}
		x^2\,\equiv\,N\modulo[p^l]
	\end{equation}
	%
	es igual a:
	\begin{itemize}
		\item $1$, si $p=2$ y $l=1$,
		\item $0$, si $p=2$, $l=2$ y $N\equiv 3\tmodulo[4]$,
		\item $2$, si $p=2$, $l=2$ y $N\equiv 1\tmodulo[4]$,
		\item $0$, si $p=2$, $l\geq 3$ y $N\not\equiv 1\tmodulo[8]$,
		\item $4$, si $p=2$, $l\geq 3$ y $N\equiv 1\tmodulo[8]$ y
		\item $1+\tlegendre N{p}$, si $p$ es impar ($l$ arbitrario).
	\end{itemize}
	%
\end{ejerResiduos}

\begin{ejerResiduos}\label{ejer:residuos:compuesto}
	Sean $k,d\in\Enteros$, $k\geq 1$ y $\mcd{d,k}=1$. Entonces, la
	cantidad de soluciones a la ecuaci\'on de congruencia
	\begin{equation}
		\label{eq:residuos:compuesto}
		x^2\,\equiv\,d\modulo[4k]
	\end{equation}
	%
	es igual a
	\begin{displaymath}
		2\,\sum_{f\mid k}\,\jacobi d{f}
		\text{ ,}
	\end{displaymath}
	%
	donde $f$ recorre los divisores positivos libres de cuadrados
	($f=1$ inclusive) y $\tjacobi d{f}$ denota el s\'{\i}mbolo de
	Jacobi. Concluir de $(x+2k)^2\equiv x^2\tmodulo[4k]$ que la sumatoria
	$\sum_{f\mid k}\,\tjacobi d{f}$ es igual a la cantidad de enteros $x$
	en el rango $0\leq x<2k$ que satisfacen \eqref{eq:residuos:compuesto}.
\end{ejerResiduos}

\begin{ejerResiduos}
	Un entero $a$ se dice \emph{residuo bicuadr\'atico m\'odulo $p$},
	si existe $x\in\Enteros$ tal que $a\equiv x^4\tmodulo[p]$. Probar
	que $-4$ es residuo bicuadr\'atico m\'odulo $p$, si y s\'olo si
	$p\equiv 1\tmodulo[4]$.%
	\hint{
		$x^4+4=((x+1)^2+1)((x-1)^2+1)$.
	}
\end{ejerResiduos}

\begin{ejerResiduos}
	Probar que, si $m$ y $n$ son enteros coprimos, entonces
	\begin{displaymath}
		\frac{m-1} 2\frac{n-1} 2\,=\,
			\sum_{x=1}^{(m-1)/2}\,\piso{\frac{xn} m}\,+\,
			\sum_{y=1}^{(n-1)/2}\,\piso{\frac{ym} n}
		\text{ .}
	\end{displaymath}
	%
\end{ejerResiduos}

\begin{ejerResiduos}\label{ejer:residuos:generador}
	Sea $p$ un primo impar y sea $g\in\Enteros$ tal que
	$g^{p-1}\equiv 1\tmodulo[p]$, pero $g^k\not\equiv 1\tmodulo[p]$
	para $1\leq k\leq p-2$.
	Probar que $\tlegendre g p =-1$.
\end{ejerResiduos}

\begin{ejerResiduos}\label{ejer:residuos:minimo-noresiduo}
	Sea $p$ un primo impar. Probar que el m\'{\i}nimo
	entero positivo no residuo m\'odulo $p$ es un n\'umero primo.
\end{ejerResiduos}

\begin{ejerResiduos}\label{ejer:residuos:residuos-consecutivos}%
	\hint{
		Separar en los tres casos:
		\begin{itemize}
			\item $\tlegendre 2 p=1$,
			\item $\tlegendre 5 p=1$ y
			\item $\tlegendre 2 p=\tlegendre 5 p=-1$.
		\end{itemize}
		%
	}
	Probar que para todo primo impar $p\geq 7$, existe
	$n\in\Enteros$, $1\leq n\leq 9$ tal que
	\begin{displaymath}
		\legendre n p \,=\,\legendre{n+1} p\,=\,1
	\end{displaymath}
	%
\end{ejerResiduos}

\begin{ejerResiduos}%
	\hint{
		Notar que existen $a,u\in\Enteros$ tales que
		$2\mid a$, $1\leq a\leq p-1$ y $a^2=5+up$.
		Separar en casos $5\nmid u$ y $5\mid u$.
	}
	Sea $p\neq 5$ un primo (positivo) impar, tal que
	$\tlegendre 5 p=1$. Probar que, entonces,
	$p\equiv\pm 1\tmodulo[5]$.
\end{ejerResiduos}

\begin{ejerResiduos}\label{ejer:residuos:gauss}
	Dado un primo (positivo) impar $p$, definimos los conjuntos:
	\begin{displaymath}
		\begin{aligned}
			RN & \,=\,
				\cardinal{\Big\{ n\in\Enteros\,:\,1\leq n\leq p-2,\,\tlegendre n p=1,\,\tlegendre{n+1} p=-1\Big\}}
				\dispcomma \\
			RR & \,=\,
				\cardinal{\Big\{ n\in\Enteros\,:\,1\leq n\leq p-2,\,\tlegendre n p=1,\,\tlegendre{n+1} p=1\Big\}}
				\dispcomma \\
			NR & \,=\,
				\cardinal{\Big\{ n\in\Enteros\,:\,1\leq n\leq p-2,\,\tlegendre n p=-1,\,\tlegendre{n+1} p=1\Big\}}
				\dispand \\
			NN & \,=\,
				\cardinal{\Big\{ n\in\Enteros\,:\,1\leq n\leq p-2,\,\tlegendre n p=-1,\,\tlegendre{n+1} p=-1\Big\}}
				\dispstop
		\end{aligned}
		%
	\end{displaymath}
	%
	Probar las siguientes afirmaciones.
	\begin{enumerate}[(i)]
		\item\label{item:residuos:gauss:i}
			Se cumple que
			\begin{displaymath}
				RR+RN\,=\,\left\{
					\begin{cases}
						-1+(p-1)/2\dispcomma &
							\text{si }
							p\equiv 1\tmodulo[4]
							\dispand \\
						(p-1)/2\dispcomma &
							\text{si }
							p\equiv 3\tmodulo[4]
							\dispstop
					\end{cases}
					%
			\end{displaymath}
			%
			Demostrar f\'ormulas para
			$NR+NN$, $RR+NR$ y para $RN+NN$.
		\item\label{item:residuos:gauss:ii}
			Se cumple que
			$RR+NN-RN-NR=-1$.
		\item\label{item:residuos:gauss:iii}
			Se cumple que
			\begin{displaymath}
				(RR,RN,NR,NN)\,=\,
					\begin{cases}
						(x-1,x,x,x)\dispcomma &
							\text{si }
							p=4x+1\dispand \\
						(x,x+1,x,x)\dispcomma &
							\text{si }
							p=4x+3\dispstop
					\end{cases}
					%
			\end{displaymath}
			%
	\end{enumerate}
	%
\end{ejerResiduos}



