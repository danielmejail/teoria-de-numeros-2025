\theoremstyle{definition}
\newtheorem{ejerEcuacion}{\ejername}[section]

%-------------

\begin{ejerEcuacion}\label{ejer:ecuacion:kronecker}
	Sea $D\equiv 0,1\tmodulo[4]$, $D\neq 0$, y sea
	$\chi:\,\Enteros\rightarrow\cuadratico$ la funci\'on del
	\lemaname~\ref{lema:ecuacion:kronecker} (o, equivalentemente,
	el morfismo del \lemaname~\ref{lema:ecuacion:kronecker:bis}).
	\begin{enumerate}[(i)]
		\item\label{item:ejer:iecuacion:kronecker:menos-uno}
			Probar que
			\begin{displaymath}
				\chi(-1)\,=\,
					\begin{cases}
						1 \text{ ,} & \text{si }
							D>0\quad\text{y} \\
						-1\text{ ,} & \text{si }
							D<0\text{ .}
					\end{cases}
					%
			\end{displaymath}
			%
		\item\label{item:ejer:ecuacion:kronecker:dos}
			Probar que, si $D\equiv 1\tmodulo[4]$, entonces
			\begin{displaymath}
				\chi(2)\,=\,
					\begin{cases}
						1 \text{ ,} & \text{si }
							D\equiv 1\tmodulo[8]
							\quad\text{y} \\
						-1\text{ ,} & \text{si }
							D\equiv 5\tmodulo[8]
							\text{ .}
					\end{cases}
					%
			\end{displaymath}
	\end{enumerate}
	%
\end{ejerEcuacion}

\begin{ejerEcuacion}
	Sea $a\in\Enteros$ no un cuadrado perfecto. Probar que existen
	infinitos primos para los cuales $a$ no es un residuo cuadr\'atico:
	\begin{enumerate}[(i)]
		\item reducir al caso en que $a$ es libre de cuadrados, o
			sea $a=2^eq_1\cdots q_r$, donde $q_i$ son primos
			impares distintos, $r\geq 0$ y $e\in\{0,1\}$;
		\item suponer que $a$ es divisible por alg\'un primo impar
			(es decir, $a\neq 2$ y $r\geq 1$) y probar que,
			dado un conjunto finito de primos $\{\lista l{k}\}$
			que excluye a $2$ y los $q_i$, existe $b\in\Enteros$
			que satisface:
			\begin{displaymath}
				\begin{aligned}
					b & \,\equiv\,1\tmodulo[l_j]
						\text{ ,} \\
					b & \,\equiv\,1\tmodulo[8]
						\text{ ,} \\
					b & \,\equiv\,1\tmodulo[q_i]
					\quad\text{ , si }i\neq r
					\text{ y} \\
					b & \,\equiv\,s\tmodulo[q_r]
					\text{ ,}
				\end{aligned}
				%
			\end{displaymath}
			%
			donde $s$ es un no residuo cuadr\'atico m\'odulo
			$q_r$ y mostrar que $\tjacobi a{b}=-1$ y que,
			por lo tanto, existe un primo
			$p\not\in\{2,\,\lista q{r},\,\lista l{k}\}$
			tal que $\tlegendre a{p}=-1$;
		\item suponer que $a=2$ y, dado un conjunto finito de
			primos $\{\lista l{k}\}$ que excluye a $3$, definir
			$b:=8l_1\cdots l_k+3$ y probar que $\tjacobi 2{b}=-1$
			y que, por lo tanto, existe un primo
			$p\not\in\{2,\,3,\,\lista l{k}\}$ tal que
			$\tlegendre 2{p}=-1$.
	\end{enumerate}
	%
\end{ejerEcuacion}

\begin{ejerEcuacion}
	Calcular el s\'{\i}mbolo de Jacobi en los siguientes casos:
	\begin{itemize}
		\item $\tjacobi{113}{997}$
		\item $\tjacobi{215}{761}$
		\item $\tjacobi{514}{1093}$
		\item $\tjacobi{401}{757}$
	\end{itemize}
	%
\end{ejerEcuacion}

\begin{ejerEcuacion}
	Determinar las clases en $\Enterosmod[84]$ para las cuales
	$\tlegendre{-21} p=1$, usando reciprocidad cuadr\'atica.
\end{ejerEcuacion}

% \begin{solucion}
	% Sea $p$ un primo positivo, impar, $p\neq 3,7$. Entonces, usando
	% las propiedades del s\'{\i}mbolo de Legendre,
	% \begin{displaymath}
		% \legendre{-21} p\,=\,\legendre{-1} p\,\legendre 3 p\,
			% \legendre 7 p\,=\,\legendre{-1} p\,
				% (-1)^{\frac{p-1} 2}\,\legendre p 3\,
				% (-1)^{\frac{p-1} 2}\,\legendre p 7\,=\,
			% \legendre{-1} p\,\legendre p 3\,\legendre p 7
		% \text{ .}
	% \end{displaymath}
	% %
	% Sea $f(p)\in\cuadratico^3$ la funci\'on
	% \begin{displaymath}
		% f(p)\,=\,\bigg(\legendre{-1} p,\,\legendre p 3,\,\legendre p 7
			% \bigg)
		% \text{ .}
	% \end{displaymath}
	% %
	% Entonces, $\tlegendre{-21} p=1$, si y s\'olo si
	% \begin{displaymath}
		% f(p)\,\in\,\big\{(1,1,1),\,(1,-1,-1),\,(-1,1,-1),\,(-1,-1,1)
			% \big\}
		% \text{ .}
	% \end{displaymath}
	% %
	% Ahora, $f(p)=(1,1,1)$, si y s\'olo si
	% \begin{displaymath}
		% p\,\equiv\,1\tmodulo[4]\text{ ,}\quad
			% p\,\equiv\,1\tmodulo[3]\quad\text{y}\quad
			% p\,\equiv\,1,2,4\tmodulo[7]
		% \text{ ,}
	% \end{displaymath}
	% %
	% lo que da lugar a las clases
	% \begin{displaymath}
		% p\,\equiv\,1,25,37\modulo[84]
		% \text{ .}
	% \end{displaymath}
	% %
	% An\'alogamente, $(1,-1,-1)$, $(-1,1,-1)$ y $(-1,-1,1)$ dan lugar a
	% otras tres clases m\'odulo $84$ cada una. En total son $12$ clases;
	% notar que $\eulerphi(84)=\indice{\Unidadesmod[84]}=24$, o sea que
	% $\ker\,\chi$ es un subgrupo de \'{\i}ndice $2$ en $\Unidadesmod[84]$.
	% % \begin{displaymath}
		% % p\,\equiv\,1,25,37,17,5,41,31,19,55,59,23,11\modulo[84]
		% % \text{ .}
	% % \end{displaymath}
	% % %
% \end{solucion}

% \begin{ejerEcuacion}\label{ejer:ecuacion:reciprocidad:kronecker}
	% En la \S~\ref{sec:ecuacion}, \lemaname~\ref{lema:ecuacion:kronecker},
	% asumiendo las propiedades del s\'{\i}mbolo de Legendre enunciadas en
	% el \teoname~\ref{teo:ecuacion:legendre}, vimos c\'omo asociar a cada
	% $D\equiv 0,1\tmodulo[4]$, $D\neq 0$, una funci\'on $\chi(m)$,
	% $m\in\Enteros$, con ciertas propiedades que la determinan
	% un\'{\i}vocamente. En este ejercicio, veremos c\'omo, asumiendo la
	% existencia de estas funciones $\chi$, podemos demostrar las
	% propiedades del s\'{\i}mbolo de Legendre. En otras palabras, el
	% \lemaname~\ref{lema:ecuacion:kronecker} y el \teoname~%
	% \ref{teo:ecuacion:legendre} son equivalentes.
	% \begin{itemize}
		% \item
			% Sea $q$ un primo positivo e impar y sea
			% $D:=q^*=(-1)^{\frac{q-1} 2}\,q$; si bien $D$ puede ser
			% negativo, se cumple $D\equiv 1\tmodulo[4]$.
		% \item
			% Sea $D=-4$.
		% \item
			% Sea $D=8$.
	% \end{itemize}
	% %
% \end{ejerEcuacion}
% 
