\theoremstyle{plain}
\newtheorem{teoEcuacion}{\teoname}[section]
\newtheorem{lemaEcuacion}[teoEcuacion]{\lemaname}

\theoremstyle{definition}
\newtheorem{defEcuacion}[teoEcuacion]{\defname}
\newtheorem{obsEcuacion}[teoEcuacion]{\obsname}
\newtheorem{pregEcuacion}[teoEcuacion]{\pregname}

%-------------

% \subsection{Recapitulaci\'on}%
	% \label{subsec:recapitulacion}
El \teoname~\ref{teo:fermat} caracteriza los primos impares que se pueden
expresar como suma de dos cuadrados enteros. Nuestro objetivo ser\'a entender
en qu\'e medida es posible adaptar el m\'etodo que nos permiti\'o demostrar
este resultado para estudiar la ecuaci\'on $p=x^2+ny^2$.

El argumento de la demostraci\'on del \teoname~\ref{teo:fermat} fue dividido
en tres afirmaciones, agrupadas en dos partes: las dos primeras (\lemaname~%
\ref{lema:fermat:descenso:pre} y \lemaname~\ref{lema:fermat:descenso})
constituyen lo que llamamos ``paso de descenso''; a la tercera (\lemaname~%
\ref{lema:fermat:reciprocidad}) le dimos el nombre de ``paso de reciprocidad''.
El \lemaname~\ref{lema:fermat:descenso:pre} depende de la identidad
\eqref{eq:fermat:descenso:identidad}. Esta identidad se generaliza y, con ella,
el resultado del \lemaname~\ref{lema:fermat:descenso:pre}. Sin embargo, el
argumento de la demostraci\'on del \lemaname~\ref{lema:fermat:descenso} tiene
sus limitaciones: depende de una cota que se adapta casi sin cambios a los
casos $n=2$ y $n=3$, per que falla para $n\geq 4$. M\'as aun, la afirmaci\'on
misma del \lemaname~\ref{lema:fermat:descenso} puede ser falsa para $n\geq 4$,
si se traduce literalmente. Por ejemplo,
\begin{itemize}
	\item si $n=5$, $3\mid 21=1^2+5\cdot 2^2$, pero $3\neq x^2+5y^2$,
		pues $3<5$;
	\item si $n=6$, $5\mid 25=1^2+6\cdot 2^2$, pero $5\neq x^2+6y^2$,
		pues $5<6$.
\end{itemize}
%
?`Es cierto que si $p$ divide a un natural de la forma $a^2+nb^2$ con
$\mcd{a,b}=1$ entonces $p$ se puede expresar en la forma $x^2+ny^2$?

?`Qu\'e pasa con $n=4$ o $n=7$? Si bien es cierto que, por ejemplo,
$3\neq x^2+4y^2$ y que $5\neq x^2+7y^2$ ?`pueden $3$ ser un divisor de un
natural de la forma $a^2+4b^2$ o $5$ un divisor de un natural de la forma
$a^2+7b^2$ con $\mcd{a,b}=1$? Deliberadamente estamos mirando $p<n$, pues,
as\'{\i}, nos asegurar\'{\i}amos de que $p\neq x^2+ny^2$. Pero ?`si
busc\'aramos $p\geq n$, encontrar\'{\i}amos un $p$ tal que
$p\mid a^2+4b^2$, pero $p\neq x^2+4y^2$, o tal que $p\mid a^2+7b^2$, pero
$p\neq x^2+7y^2$? ?`Si $n\geq 8$? ?`Existen contraejemplos?
\begin{itemize}
	\item para $n=8$, se cumple $3\mid 12=2^2+8\cdot 1^2$,
		pero $3\neq x^2+8y^2$
		(ni $5$ ni $7$ funcionan);
		% $3\mid 33=1^2+8\cdot 2^2$
	\item para $n=9$, se cumple $5\mid 10=1^2+9\cdot 1^2$,
		pero $5\neq x^2+9y^2$
		($7$ no funciona);
	\item para $n=10$, se cumple $7\mid 14=2^2+10\cdot 1^2$,
		pero $7\neq x^2+10y^2$
		($3$ no funciona);
		% $7\mid 91=1^2+10\cdot 3^2$
	\item para $n=11$, se cumple $3,5\mid 15=2^2+11\cdot 1^2$,
		pero $3,5\neq x^2+11y^2$
		($7$ no funciona).
		% $3,5\mid 45=1^2+11\cdot 2^2$
\end{itemize}
%
Vamos a poder decir algo m\'as acerca de estas preguntas, una vez que hablemos
de formas cuadr\'aticas, reducci\'on, sus clases de equivalencia y los
n\'umeros de clases.

Ya vemos que nos encontramos con algunas dificultades al querer estudiar el
problema en general, en relaci\'on con el paso de descenso. Por otro lado,
con respecto a reciprocidad, el \lemaname~\ref{lema:fermat:reciprocidad}
nos muestra que una condici\'on de congruencia garantiza que un primo impar
divida una suma de cuadrados. Si bien la demostraci\'on de este resultado
puede parecer \emph{ad hoc}, veremos c\'omo rescatar la idea del enunciado.
Precisamente, vamos a ver que existen condiciones de congruencia sobre un
primo impar, $\mcd{p,n}=1$, que garantizan que $p\mid a^2+nb^2$, $\mcd{a,b}=1$.

% \subsection{El s\'{\i}mbolo de Legendre}%
	% \label{subsec:legendre}
\begin{defEcuacion}\label{def:ecuacion:legendre}
	Dados un entero $a\in\Enteros$ y un primo impar $p$ ($>0$), decimos
	que \emph{$a$ es un residuo cuadr\'atico m\'odulo $p$}, si
	$p\nmid a$ y si la ecuaci\'on $x^2\equiv a\tmodulo[p]$ tiene
	soluci\'on m\'odulo $p$. El \emph{s\'{\i}mbolo de Legendre}
	$\tlegendre a{p}$ es:
	\begin{displaymath}
		\legendre a{p}\,:=\,
			\begin{cases}
				0\text{ ,} & \quad\text{si }
					p\mid a \text{ ,} \\
				1\text{ ,} & \quad\text{si }
					p\nmid a\text{ y }a
					\text{ es residuo cuadr\'atico,} \\
				-1\text{ ,} & \quad\text{si }
					p\nmid a\text{ y }a
					\text{ no es residuo cuadr\'atico.}
			\end{cases}
			%
	\end{displaymath}
	%
\end{defEcuacion}

\begin{obsEcuacion}\label{obs:ecuacion:residuos}
	Los enteros $0$, $1$ y todos los cuadrados perfectos son residuos
	cuadr\'aticos m\'odulo cualquier entero. En particular,
	$\tlegendre{n^2} p=1$ para todo primo impar $p$ y todo entero $n$
	coprimo con $p$.
\end{obsEcuacion}

\begin{lemaEcuacion}\label{lema:ecuacion:legendre}
	Sea $n\in\Enteros$, $n\neq 0$, y sea $p$ un primo impar que no
	divide a $n$. Entonces, existen $a,b\in\Enteros$ tales que
	$p\mid a^2+nb^2$ y $\mcd{a,b}=1$, si y s\'olo si $\tlegendre{-n} p=1$.
\end{lemaEcuacion}

\begin{proof}
	Supongamos que $a,b\in\Enteros$ son tales que $p\mid a^2+nb^2$.
	Entonces, $x^2\equiv -ny^2\tmodulo[p]$. Si $p\nmid n$,
	debe ser $p\mid a$, si y s\'olo si $p\mid b$. Si, adem\'as,
	$\mcd{a,b}=1$, entonces $p\nmid a$ y $p\nmid b$. En particular,
	$(b^{-1}a)^2\equiv -n\tmodulo[p]$ y, as\'{\i},
	$\tlegendre{-n} p=1$. Completar la demostraci\'on \quedacomoejercicio.
\end{proof}

% \subsection{La ley de reciprocidad cuadr\'atica}%
	% \label{subsec:la-ley-de-reciprocidad-cuadratica}
\begin{pregEcuacion}\label{preg:ecuacion}
	?`Existen condiciones de congruencia sobre un primo impar $p$
	que garanticen que $\tlegendre{-n} p=1$?
\end{pregEcuacion}

Veamos con un poco m\'as de cuidado qu\'e podr\'{\i}a querer decir
``condiciones de congruencia''. En el caso $n=1$, suma de dos cuadrados,
la condici\'on es $p\equiv 1\tmodulo[4]$. Sea $f(x,y)=x^2+ny^2$. Podemos
tratar de entender la situaci\'on calculando distintos valores que la
funci\'on $f$ toma, con $x,y\in\Enteros$ tales que $\mcd{x,y}=1$, y luego
factorizando cada uno de estos valores obtenidos, para buscar sus factores
primos.
La \tablename~\ref{tab:ecuacion:legendre:positivos} y la \tablename~%
\ref{tab:ecuacion:legendre:negativos} muestran los primos obtenidos
de esta manera, para distintos valores de $n$, variando $x$ e $y$ en un
cierto rango, agrupados de acuerdo a su clase de congruencia m\'odulo
$4|n|$. Dicho de otra manera, la tabla muestra las distintas
\emph{clases de congruencia m\'odulo $4|n|$ representadas por primos que %
se obtienen como factores de enteros de la forma $a^2+nb^2$ con $a$ y $b$ %
coprimos}.

\begin{table}
	\begin{tabular}{r|l}
		$n$ & $\tlegendre{-n} p=1$ \\
		\hline
		$2$ & $p\equiv 1,3\tmodulo[8]$ \\
		$3$ & $p\equiv 1,7\tmodulo[12]$ \\
		$5$ & $p\equiv 1,3,7,9\tmodulo[20]$ \\
		$7$ & $p\equiv 1,9,11,15,23,25\tmodulo[28]$ \\
		$11$ & $p\equiv 1,3,5,9,15,23,25,27,31,37\tmodulo[44]$ \\
		$13$ & $p\equiv 1,7,9,11,15,17,19,25,29,31,47,49
			\tmodulo[52]$ \\
		$17$ & $p\equiv 1,3,7,9,11,13,21,23,25,27,31,33,39,49,53,63
			\tmodulo[68]$ \\
		\hline
		$6$ & $p\equiv 1,5,7,11\tmodulo[24]$ \\
		$10$ & $p\equiv 1,7,9,11,13,19,23,37\tmodulo[40]$ \\
		$14$ & $p\equiv 1,3,5,9,13,15,19,23,25,27,39,45\tmodulo[56]$ \\
		$15$ & $p\equiv 1,17,19,23,31,47,49,53\tmodulo[60]$ \\
		$21$ & $p\equiv 1,5,11,17,19,23,25,31,37,41,55,71
			\tmodulo[84]$ \\
		\hline
		$1$ & $p\equiv 1\tmodulo[4]$ \\
		$4$ & $p\equiv 1,5,9,13\tmodulo[16]$ 
	\end{tabular}
	\caption{
		Divisores primos de $a^2+nb^2$ con $\mcd{a,b}=1$ y
		$1\leq a,b\leq 50$.
	}\label{tab:ecuacion:legendre:positivos}
\end{table}

\begin{table}
	\begin{tabular}{r|l}
		$n$ & $\tlegendre{-n} p=1$ \\
		\hline
		$-2$ & $p\equiv 1,7\tmodulo[8]$ \\
		$-3$ & $p\equiv 1,11\tmodulo[12]$ \\
		$-5$ & $p\equiv 1,9,11,19\tmodulo[20]$ \\
		$-7$ & $p\equiv 1,3,9,19,25,27\tmodulo[28]$ \\
		$-11$ & $p\equiv 1,5,7,9,19,25,35,37,39,43\tmodulo[44]$ \\
		$-13$ & $p\equiv 1,3,9,17,23,25,27,29,35,43,49,51
			\tmodulo[52]$ \\
		$-17$ & $p\equiv 1,9,13,15,19,21,25,33,35,43,47,49,53,55,59,67
			\tmodulo[68]$ \\
		\hline
		$-6$ & $p\equiv 1,5,19,23\tmodulo[24]$ \\
		$-10$ & $p\equiv 1,3,9,13,27,31,37,39\tmodulo[40]$ \\
		$-14$ & $p\equiv 1,5,9,11,13,25,31,43,45,47,51,55
			\tmodulo[56]$ \\
		$-15$ & $p\equiv 1,7,11,17,43,49,53,59\tmodulo[60]$ \\
		$-21$ & $p\equiv 1,5,17,25,37,41,43,47,59,67,79,83
			\tmodulo[84]$ \\
		\hline
		$-1$ & $p\equiv 1,3\tmodulo[4]$ \\
		$-4$ & $p\equiv 1,3,5,7,9,11,13,15\tmodulo[16]$ 
	\end{tabular}
	\caption{
		Divisores primos de $a^2+nb^2$ con $\mcd{a,b}=1$ y
		$1\leq |a|,|b|\leq 50$.
	}\label{tab:ecuacion:legendre:negativos}
\end{table}

Asumiendo que los valores de $p$ que no aparecen son tales que
$\tlegendre{-n} p=-1$ (lo cual no hemos demostrado), se pueden
conjeturar algunas propiedades del s\'{\i}mbolo. Por ejemplo,
de la \tablename~\ref{tab:ecuacion:legendre:negativos}, mirando
las filas para $n=-2$, $n=-3$ y $n=-6$,
\begin{displaymath}
	\legendre 6{p}\,=\,1
	\quad\text{si y s\'olo si}\quad
	p\,\equiv\,1,5,19,23\tmodulo[24]
	\text{ ,}
\end{displaymath}
%
que equivale a
\begin{displaymath}
	\left\{
	\begin{array}{r@{\,\equiv\,}l}
		p & 1\tmodulo[8] \text{ y} \\[5pt]
		p & 1\tmodulo[3] \text{ ,}
	\end{array}\right.%\quad\text{o}\quad
	\left\{
	\begin{array}{r@{\,\equiv\,}l}
		p & 5\tmodulo[8] \text{ y} \\[5pt]
		p & 2\tmodulo[3] \text{ ,}
	\end{array}\right.%\quad\text{o}\quad
	\left\{
	\begin{array}{r@{\,\equiv\,}l}
		p & 3\tmodulo[8] \text{ y} \\[5pt]
		p & 1\tmodulo[3] \text{ ,}
	\end{array}\right.\quad\text{o}\quad
	\left\{
	\begin{array}{r@{\,\equiv\,}l}
		p & 7\tmodulo[8] \text{ y} \\[5pt]
		p & 2\tmodulo[3] \text{ .}
	\end{array}\right.
\end{displaymath}
%
Pero esto es lo mismo que
\begin{displaymath}
	\left\{
	\begin{array}{r@{\,\equiv\,}l}
		p & 1\tmodulo[8] \text{ y} \\[5pt]
		p & 1\tmodulo[12] \text{ ,}
	\end{array}\right.%\quad\text{o}\quad
	\left\{
	\begin{array}{r@{\,\equiv\,}l}
		p & 5\tmodulo[8] \text{ y} \\[5pt]
		p & 5\tmodulo[12] \text{ ,}
	\end{array}\right.%\quad\text{o}\quad
	\left\{
	\begin{array}{r@{\,\equiv\,}l}
		p & 3\tmodulo[8] \text{ y} \\[5pt]
		p & 7\tmodulo[12] \text{ ,}
	\end{array}\right.\quad\text{o}\quad
	\left\{
	\begin{array}{r@{\,\equiv\,}l}
		p & 7\tmodulo[8] \text{ y} \\[5pt]
		p & 11\tmodulo[12] \text{ ,}
	\end{array}\right.
\end{displaymath}
%
que equivale a
\begin{displaymath}
	\left\{
		\begin{array}{r@{\,=\,}l}
			\tlegendre 2{p} & 1 \quad\text{y} \\[5pt]
			\tlegendre 3{p} & 1 \text{ ,}
		\end{array}
	\right.
	\quad\text{o}\quad
	\left\{
		\begin{array}{r@{\,=\,}l}
			\tlegendre 2{p} & -1 \quad\text{y} \\[5pt]
			\tlegendre 3{p} & -1 \text{ .}
		\end{array}
	\right.
\end{displaymath}
%
Expresado de otra manera,
\begin{displaymath}
	\legendre 6{p}\,=\,\legendre 2{p}\,\legendre 3{p}
	\text{ .}
\end{displaymath}
%
Otra cosa que se puede observar es que, en la \tablename~%
\ref{tab:ecuacion:legendre:negativos}, en los casos en que $n=-q$ con
$q$ primo, las clases representadas son equivalentes a $\pm\beta^2$ con
$\beta$ impar, $\beta<q$. Sin embargo, esta observaci\'on deja de ser
v\'alida si $n$ es compuesto. Por ejemplo, si $n=-6$, las clases m\'odulo
$24$ representadas por factores primos $p$ tales que $\tlegendre 6{p}=1$
son, de acuerdo con la tabla, $1$, $5$, $19$ y $23$. Por otro lado, los
cuadrados m\'odulo $24$ son $1$, $4$, $9$, $12$ y $16$.
% , de los cuales s\'olo la clase de $1$ es coprima con $24$.
En particular, s\'olo $1$ y $23$ son de la forma $\pm\beta^2$ con
$\beta$ impar (y coprimo con $24$).

\begin{teoEcuacion}\label{teo:ecuacion:reciprocidad}
	Si $p,q>0$ son primos impares distintos, entonces
	\begin{displaymath}
		\legendre q{p}\,=\,1
		\text{ , si y s\'olo si }
		p\,\equiv\,\pm\beta^2\tmodulo[4q]
		\text{ ,}
	\end{displaymath}
	%
	para cierto $\beta\in\Enteros$ impar.
\end{teoEcuacion}

El \teoname~\ref{teo:ecuacion:reciprocidad} relaciona residuos cuadr\'aticos
m\'odulo $p$ con residuos m\'odulo $4q$, si $p$ y $q$ son primos positivos,
impares y distintos: $q$ es cuadrado m\'odulo $p$, si y s\'olo si $\pm p$ es
cuadrado m\'odulo $4q$
?`Qu\'e se puede decir si $n$ es compuesto? ?`Qu\'e pasa si $n>0$?
?`Hay relaci\'on entre $\tlegendre{-n} p$ y $\tlegendre n{p}$?
En la \S~\ref{sec:congruencias}, vimos que las ecuaciones de congruencia
$ax\equiv b\tmodulo[m]$ tienen soluci\'on exactamente cuando $\mcd{a,m}\mid b$
?`Hay alguna manera de saber si $x^2\equiv D\tmodulo[4k]$ tiene soluciones?
%
% Para dar respuesta a estas preguntas --y, en particular, para
% demostrar el \teoname~\ref{teo:ecuacion:reciprocidad}-- necesitaremos
% estudiar el s\'{\i}mbolo de Legendre $\tlegendre a{p}$ con m\'as cuidado.
Antes de pasar a estudiar el s\'{\i}mbolo de Legendre, veamos c\'omo podemos
usarlo para dar respuesta a la \pregname~\ref{preg:ecuacion}.

En primer lugar, con respecto al \teoname~\ref{teo:ecuacion:reciprocidad},
?`podemos precisar, en la congruencia $p\equiv\pm\beta^2$, en qu\'e casos
el signo deber\'{\i}a ser $+$ y en qu\'e casos deber\'{\i}a ser $-$?
De la \tablename~\ref{tab:ecuacion:legendre:negativos}, a partir de la
cual formulamos el \teoname~\ref{teo:ecuacion:reciprocidad}, vemos que
$p\equiv+\beta^2\tmodulo[4q]$, si $p\equiv 1\tmodulo[4]$, y que
$p\equiv-\beta^2\tmodulo[4q]$, si $p\equiv 3\tmodulo[4]$. Esto tiene
sentido: los \'unicos cuadrados m\'odulo $4$ son los $\equiv 1\tmodulo[4]$.
Pero, entonces, el signo deber\'{\i}a ser
\begin{displaymath}
	\pm\,=\,(-1)^{\frac{p-1} 2}
	\text{ .}
\end{displaymath}
%
En definitiva, $p\equiv\pm\beta^2\tmodulo[4q]$ quiere decir
\begin{displaymath}
	p\,\equiv\,(-1)^{\frac{p-1} 2}\,\beta^2\modulo[4q]
	\text{ ;}
\end{displaymath}
%
no hay ambig\"uedad en el signo. Si escribimos $p^*:=(-1)^{\frac{p-1} 2}p$,
entonces $p^*\equiv 1\tmodulo[4]$ y, en particular \quedacomoejercicio,
las siguientes afirmaciones son equivalentes:
\begin{itemize}
	\item $p\equiv\pm\beta^2\tmodulo[4q]$,
	\item $p^*\equiv\beta^2\tmodulo[4q]$,
	\item $p^*\equiv\beta^2\tmodulo[q]$,
	\item $\tlegendre{p^*} q=1$.
\end{itemize}
%

\begin{lemaEcuacion}\label{lema:ecuacion:reciprocidad}
	El \teoname~\ref{teo:ecuacion:reciprocidad} es equivalente a la
	afirmaci\'on: para todo par de primos positivos, impares y distintos,
	$p$ y $q$,
	\begin{displaymath}
		\legendre q{p}\,=\,\legendre{p^*} q
		\text{ .}
	\end{displaymath}
	%
\end{lemaEcuacion}

\begin{proof}
	De lo anterior, $p\equiv\pm\beta^2\tmodulo[4q]$, si y s\'olo si
	$\tlegendre{p^*} q=1$. Por lo tanto, el \teoname~%
	\ref{teo:ecuacion:reciprocidad} es equivalente a
	$\tlegendre q{p}=1$, si y s\'olo si $\tlegendre{p^*} q=1$,
	para todo par de primos positivos, impares y distintos.
	Pero $\tlegendre q{p}$ y $\tlegendre{p^*} q$ son $\pm1$, con lo que
	la afirmaci\'on
	``$\tlegendre q{p}=1$, si y s\'olo si $\tlegendre{p^*} q=1$''
	equivale a $\tlegendre q{p}=\tlegendre{p^*} q$.
\end{proof}

% \subsection{Propiedades del s\'{\i}mbolo de Legendre}%
	% \label{subsec:propiedades-del-simbolo-de-legendre}
El \teoname~\ref{teo:ecuacion:legendre} resume algunas de las propiedades del
s\'{\i}mbolo de Legendre; su demostraci\'on la dejamos para la
\S~\ref{sec:residuos}.

\begin{restatable}{teoEcuacion}{teoEcuacionLegendre}%
% \begin{teoEcuacion}%
	\label{teo:ecuacion:legendre}
	Sea $p>0$ un primo impar. Entonces, se cumple
	\begin{enumerate}[(i)]
		\item\label{item:ecuacion:legendre:clases}
			$\tlegendre n{p}=\tlegendre{n'} p$,
			si $n\equiv n'\tmodulo[p]$;
		\item\label{item:ecuacion:legendre:multiplicativo}
			$\tlegendre{ab} p=\tlegendre a{p}\tlegendre b{p}$,
			para todo par $a,b\in\Enteros$;
		\item\label{item:ecuacion:legendre:menos-uno}
			$\tlegendre{-1} p=(-1)^{\frac{p-1} 2}$ y,
			m\'as en general,
			$\tlegendre n{p}\equiv n^{\frac{p-1} 2}\tmodulo[p]$,
			para todo $n\in\Enteros$:
		\item\label{item:ecuacion:legendre:dos}
			$\tlegendre 2{p}=(-1)^{\frac{p^2-1} 8}$;
		\item\label{item:ecuacion:legendre:reciprocidad}
			$\tlegendre q{p}=\tlegendre{p^*} q$,
			si $q$ es un primo positivo, impar y distinto de $p$.
	\end{enumerate}
	%
% \end{teoEcuacion}
\end{restatable}

\begin{obsEcuacion}\label{obs:ecuacion:legendre}
	El \'{\i}tem~\eqref{item:ecuacion:legendre:menos-uno} es equivalente
	a
	\begin{displaymath}
		\legendre{-1} p\,=\,\begin{cases}
				1 \text{ ,} & \text{si }p\equiv 1\tmodulo[4]
					\quad\text{y} \\
				-1\text{ ,} & \text{si }p\equiv 3\tmodulo[4]
					\text{ .}
			\end{cases}
			%
	\end{displaymath}
	%
	El \'{\i}tem~\eqref{item:ecuacion:legendre:dos} equivale a
	\begin{displaymath}
		\legendre 2{p}\,=\,\begin{cases}
				1 \text{ ,} & \text{si }p\equiv 1,7\tmodulo[8]
					\quad\text{y} \\
				-1\text{ ,} & \text{si }p\equiv 3,5\tmodulo[8]
					\text{ .}
			\end{cases}
			%
	\end{displaymath}
	%
	El \'{\i}tem~\eqref{item:ecuacion:legendre:reciprocidad} lo podemos
	reescribir como
	\begin{displaymath}
		\legendre q{p}\,=\,(-1)^{\frac{p-1} 2\frac{q-1} 2}\,
			\legendre p{q}
		\text{ ,}
	\end{displaymath}
	%
	o bien como
	\begin{displaymath}
		\legendre q{p}\,\legendre p{q}\,=\,
			(-1)^{\frac{p-1} 2\frac{q-1} 2}
		\text{ .}
	\end{displaymath}
	%
	Este \'ultimo es, de acuerdo con el \lemaname~%
	\ref{lema:ecuacion:reciprocidad}, equivalente al
	\teoname~\ref{teo:ecuacion:reciprocidad}.
\end{obsEcuacion}

% \subsection{El s\'{\i}mbolo de Kronecker}%
	% \label{subsec:kronecker}
El \lemaname~\ref{lema:ecuacion:reciprocidad} nos permiti\'o reinterpretar el
problema de encontrar condiciones de congruencia que garanticen que un
primo impar divida a una expresi\'on de la forma $a^2+nb^2$ con $\mcd{a,b}=1$
en t\'erminos del s\'{\i}mbolo de Legendre. El \lemaname~%
\ref{lema:ecuacion:kronecker} nos permitir\'a reinterpretar el problema en
t\'erminos m\'as algebraicos (\teoname~\ref{teo:ecuacion:kronecker}).

\begin{lemaEcuacion}\label{lema:ecuacion:kronecker}
	Si $D\equiv 0,1\tmodulo[4]$, $D\neq 0$, existe una \'unica
	funci\'on $\chi(m)$, $m\in\Enteros$, que cumple que:
	\begin{enumerate}[(1)]
		\item\label{item:ecuacion:kronecker:clases}
			$\chi(m)=\chi(m')$, si $m\equiv m'\tmodulo[D]$,
		\item\label{item:ecuacion:kronecker:multiplicativo}
			$\chi(mn)=\chi(m)\chi(n)$, para todo par
			$m,n\in\Enteros$,
		\item\label{item:ecuacion:kronecker:legendre}
			$\chi(p)=\tlegendre D{p}$, si $p$ es un primo
			(positivo) impar que no divide a $D$ y
		\item\label{item:ecuacion:kronecker:cero}
			$\chi(m)=0$, si $\mcd{m,D}>1$.
			% \footnote{
				% Esta condici\'on se puede omitir, si,
				% en el \'{\i}tem
				% \eqref{item:ecuacion:kronecker:legendre}
				% no asumimos que $p\nmid D$.
			% }
	\end{enumerate}
	%
\end{lemaEcuacion}

Juntando el \lemaname~\ref{lema:ecuacion:kronecker} con el \lemaname~%
\ref{lema:ecuacion:legendre} obtenemos la siguiente consecuencia.

\begin{teoEcuacion}\label{teo:ecuacion:kronecker}
	Sean $n\in\Enteros$, $n\neq 0$, y $\chi$ la funci\'on del
	\lemaname~\ref{lema:ecuacion:kronecker} ($D=-4n$). Entonces,
	si $p$ es un primo positivo impar que no divide a $n$, las siguientes
	afirmaciones son equivalentes:
	\begin{enumerate}[(a)]
		\item\label{item:ecuacion:kronecker:equivalencias:divide}
			existen $a,b\in\Enteros$ tales que
			$p\mid a^2+nb^2$ y $\mcd{a,b}=1$;
		\item\label{item:ecuacion:kronecker:equivalencias:legendre}
			$\tlegendre{-n} p=1$;
		\item\label{item:ecuacion:kronecker:equivalencias:nucleo}
			$\chi(p)=1$.
	\end{enumerate}
	%
\end{teoEcuacion}

De esta manera, obtenemos lo que busc\'abamos (o, por lo menos, en
teor\'{\i}a). Si $p$ es un primo impar que no divide a $n$, existen
$a,b\in\Enteros$ tales que $p\mid a^2+nb^2$ y $\mcd{a,b}=1$, si y s\'olo si
$p$ pertenece a determinadas clases de congruencia: aquellas clases m\'odulo
$4|n|$ en las que $\chi$ toma el valor $1$.%
\footnote{
	Para hacer las cosas expl\'{\i}citas deber\'{\i}amos poder
	calcular esta funci\'on $\chi$.
}

A continuaci\'on, demostramos el \lemaname~\ref{lema:ecuacion:kronecker}
asumiendo las propiedades del s\'{\i}mbolo de Legendre enunciadas en el
\teoname~\ref{teo:ecuacion:legendre}.

\begin{proof}[\proofname~del \lemaname~\ref{lema:ecuacion:kronecker}]
	Esencialmente, tanto existencia, como unicidad de la funci\'on $\chi$,
	dependen del hecho de que todo n\'umero entero coprimo con $D$
	es congruente, m\'odulo $D$, a un entero positivo, impar y coprimo
	con $D$.%
	\footnote{
		M\'as aun, el teorema de Dirichlet sobre primos en
		progresiones aritm\'eticas garantiza que toda clase de
		congruencia m\'odulo $D$ coprima con $D$ contiene alg\'un
		n\'umero primo positivo impar.
	}

	Con respecto a la unicidad, supongamos que $\chi$ posee las
	propiedades deseadas. Entonces, cuando $m\in\Enteros$,
	\begin{itemize}
		\item si $\mcd{m,D}>1$, $\chi(m)=0$;
		\item si $\mcd{m,D}=1$,
			\begin{itemize}
				\item si $m<0$, se cumple que $m+kD>0$
					para alg\'un $k\in\Enteros$,
					$m+kD\equiv m\tmodulo[D]$ y
					$\chi(m+kD)=\chi(m)$;
				\item si $m$ es par, deb\'{\i}a ser
					$D\equiv 1\tmodulo[4]$ y, en
					consecuencia, $m+|D|$ es impar,
					$m+|D|\equiv m\tmodulo[D]$ y
					$\chi(m+|D|)=\chi(m)$;
				\item si $m$ es impar y positivo, se escribe
					como producto de primos positivos
					impares, $m=p_1\cdots p_r$
					(posiblemente, con repeticiones), con
					lo que
					\begin{displaymath}
						\chi(m)\,=\,\chi(p_1)\,\cdots\,
							\chi(p_r)\,=\,
							\legendre D{p_1}\,
							\cdots\,
							\legendre D{p_r}
						\text{ .}
					\end{displaymath}
					%
			\end{itemize}
			%
	\end{itemize}
	%
	En cualquier caso, al ser $\chi(mn)=\chi(m)\chi(n)$, $\chi$ queda
	determinada por su valor en los primos impares que no dividen a $D$
	y por la condici\'on $\chi(m)=0$, si $\mcd{m,D}>1$.

	Para probar la existencia, introducimos el s\'{\i}mbolo de Jacobi,
	que se puede interpretar como una extensi\'on del s\'{\i}mbolo de
	Legendre. Dados $M,m\in\Enteros$, $m>0$,
	% y $\mcd{M,m}=1$,
	si $m=p_1\cdots p_r$ es la factorizaci\'on de $m$ en producto de
	primos (posiblemente, con repeticiones), el \emph{s\'{\i}mbolo de %
	Jacobi} $\jacobi M{m}$ es
	\begin{displaymath}
		\jacobi M{m}\,=\,\legendre M{p_1}\,\cdots\,
			\legendre M{p_r}
		\text{ ,}
	\end{displaymath}
	%
	donde, del lado derecho, $\tlegendre M{p_i}$ denota el s\'{\i}mbolo
	de Legendre.
	% \footnote{
		% $\tjacobi M{m}=0$, si y s\'olo si $\tlegendre M{p_i}$ para
		% alg\'un $p_i$.
	% }
	El s\'{\i}mbolo de Jacobi tiene propiedades an\'alogas
	a las del s\'{\i}mbolo de Legendre % y que son consecuencia de ellas
	\quedacomoejercicio:
	\begin{enumerate}[(I)]
		\item\label{item:ecuacion:jacobi:clases}
			$\tjacobi N{m}=\tjacobi{N'} m$, si
			$N\equiv N'\tmodulo[m]$,
		\item\label{item:ecuacion:jacobi:multiplicativo}
			$\tjacobi{AB} m=\tjacobi A{m}\tjacobi B{m}$,
			para todo par $A,B\in\Enteros$,
		\item\label{item:ecuacion:jacobi:menos-uno}
			$\tjacobi{-1} m=(-1)^{\frac{m-1} 2}$,
		\item\label{item:ecuacion:jacobi:dos}
			$\tjacobi 2{m}=(-1)^{\frac{m^2-1} 8}$,
		\item\label{item:ecuacion:jacobi:reciprocidad}
			$\tjacobi M{m}=\tjacobi{m^*} M$, si $M$ es positivo,
			impar y $\mcd{M,m}=1$, donde
			$m^*=(-1)^{\frac{m-1} 2}m$.
	\end{enumerate}
	%
	En particular, en \eqref{item:ecuacion:jacobi:reciprocidad}, si
	$M\equiv 1\tmodulo[4]$, entonces $\tjacobi M{m}=\tjacobi{m} M$.

	El s\'{\i}mbolo de Jacobi tiene tambi\'en la propiedad multiplicativa:
	\begin{equation}
		\label{eq:ecuacion:kronecker:multiplicativa}
		\jacobi M{mn}\,=\,\jacobi M{m}\,\jacobi M{n}
		\text{ ,}
	\end{equation}
	%
	si $m,n>0$.
	% y $\mcd{M,mn}=1$.
	Pero, adem\'as, el s\'{\i}mbolo de Jacobi tiene la siguiente
	propiedad adicional: si $D\in\Enteros$, $D\neq 0$ y
	$D\equiv 0,1\tmodulo[4]$, entonces
	\begin{equation}
		\label{eq:ecuacion:kronecker:clases}
		\jacobi D{m}\,=\,\jacobi D{n}
		\text{ ,}
	\end{equation}
	%
	si $m,n\in\Enteros$, $m,n>0$, impares y $m\equiv n\tmodulo[D]$.
	Aceptando las propiedades \eqref{eq:ecuacion:kronecker:multiplicativa}
	y \eqref{eq:ecuacion:kronecker:clases}, definimos la siguiente
	funci\'on $\chi(m)$, dado $m\in\Enteros$:
	\begin{itemize}
		\item si $\mcd{m,D}>1$, $\chi(m):=0$;
		\item si $\mcd{m,D}=1$,
			\begin{itemize}
				\item si $m<0$, se cumple que $m+kD>0$
					para alg\'un $k\in\Enteros$,
					$m+kD\equiv m\tmodulo[D]$ y
					$\chi(m+kD):=\chi(m)$;
				\item si $m$ es par, deb\'{\i}a ser
					$D\equiv 1\tmodulo[4]$ y, en
					consecuencia, $m+|D|$ es impar,
					$m+|D|\equiv m\tmodulo[D]$ y
					$\chi(m+|D|):=\chi(m)$;
				\item si $m$ es impar y positivo,
					$\chi(m):=\tjacobi D{m}$.
			\end{itemize}
			%
	\end{itemize}
	%
	La propiedad \eqref{eq:ecuacion:kronecker:clases} muestra que $\chi$
	est\'a bien definida y, junto con
	\eqref{eq:ecuacion:kronecker:multiplicativa} y las propiedades del
	s\'{\i}mbolo de Jacobi, se deducen las propiedades mencionadas en el
	enunciado.
\end{proof}

\begin{lemaEcuacion}\label{lema:ecuacion:jacobi}
	El s\'{\i}mbolo de Jacobi tiene las propiedades
	\eqref{eq:ecuacion:kronecker:multiplicativa} y
	\eqref{eq:ecuacion:kronecker:clases}.
\end{lemaEcuacion}

\begin{proof}
	En cuanto a \eqref{eq:ecuacion:kronecker:multiplicativa}, si
	$m=p_1\cdots p_r$ y $n=q_1\cdots q_s$, entonces
	$mn=p_1\cdots p_rq_1\cdots q_s$ es la factorizaci\'on de $mn$ y
	\begin{displaymath}
		\jacobi M{mn}\,=\,
			\legendre M{p_1}\,\cdots\,\legendre M{p_r}\,
			\legendre M{q_1}\,\cdots\,\legendre M{q_s}\,=\,
			\jacobi M{m}\,\jacobi M{n}
		\text{ .}
	\end{displaymath}
	%
	En cuanto a \eqref{eq:ecuacion:kronecker:clases}, sean
	$D\equiv 0,1\tmodulo[4]$, $D\neq 0$ y sean $m,n>0$ impares tales
	que $m\equiv n\tmodulo[D]$. Queremos ver que
	$\tjacobi D{m}=\tjacobi D{n}$. Notemos que $\mcd{m,D}=\mcd{n,D}$,
	con lo cual $\tjacobi D{m}=0$, si y s\'olo si $\tjacobi D{n}=0$.
	En particular, si $m$ y $n$ no son coprimos con $D$,
	$\tjacobi D{m}=\tjacobi D{n}$ (ambos siendo iguales a $0$).
	En lo que queda, asumiremos que $\mcd{m,D}=\mcd{n,D}=1$.

	Supongamos, en primer lugar, que $D\equiv 1\tmodulo[4]$ y $D>0$.
	Entonces, por \eqref{item:ecuacion:jacobi:reciprocidad},
	\begin{displaymath}
		\jacobi D{m}\,=\,\jacobi{m^*} D\,=\,\jacobi{m} D\,=\,
		\jacobi{n} D\,=\,\jacobi{n^*} D\,=\,\jacobi D{n}
		\text{ .}
	\end{displaymath}
	%
	Manteniendo la hip\'otesis $D\equiv 1\tmodulo[4]$, supongamos, en
	segundo lugar, que $D<0$. Entonces, $|D|\equiv 3\tmodulo[4]$,
	con lo cual,
	\begin{displaymath}
		\jacobi D{m}\,=\,\jacobi{|D|} m\,\jacobi{-1} m\,=\,
		\jacobi{m^*}{|D|}\,\jacobi{-1}{|D|}\,=\,
		\jacobi m{|D|}
		\text{ .}
	\end{displaymath}
	%
	Como lo mismo es cierto para $n$, y $\tjacobi n{|D|}=\tjacobi m{|D|}$,
	se deduce que $\tjacobi D{m}=\jacobi D{n}$, en este caso.

	Supongamos, ahora, que $D\equiv 0\tmodulo[4]$ y que
	$D=2^{2k+\delta}D_0$, donde $k\geq 1$, $\delta\in\{0,1\}$ y $D_0$ es
	impar. Entonces, vale que
	\begin{displaymath}
		\jacobi D{m}\,=\,\jacobi 2{m}^\delta\,\jacobi{D_0} m
	\end{displaymath}
	%
	y lo mismo para $n$ en lugar de $m$. Adem\'as, como $4\mid D$,
	debe ser $m\equiv n\tmodulo[4]$ y, por lo tanto,
	\begin{displaymath}
		\jacobi{-1} m\,=\,(-1)^{\frac{m-1} 2}\,=\,
			(-1)^{\frac{n-1} 2}\,=\,\jacobi{-1} n
		\text{ .}
	\end{displaymath}
	%
	Para terminar, vamos a ver que $\tjacobi 2{m}=\tjacobi 2{n}$, si
	$\delta\neq 0$, y que $\tjacobi{D_0} m=\tjacobi{D_0} n$.
	Primero, si $\delta\neq 0$, en particular vale que $8\mid D$ y,
	en consecuencia, $m\equiv n\tmodulo[8]$. Pero, entonces,
	\begin{displaymath}
		\jacobi 2{m}\,=\,(-1)^{\frac{m^2-1} 8}\,=\,
			(-1)^{\frac{n^2-1} 8}\,=\,\jacobi 2{n}
		\text{ .}
	\end{displaymath}
	%
	Por \'ultimo, $D_0\equiv 1,3\tmodulo[4]$. En el caso en que
	$D_0\equiv 1\tmodulo[4]$ (sea positivo o negativo), por el caso
	anterior (el caso $D\equiv 1\tmodulo[4]$), sabemos que
	$\tjacobi{D_0} m=\tjacobi{D_0} n$. Si, en el otro caso,
	$D_0\equiv 3\tmodulo[4]$, entonces, para $D_0>0$,
	\begin{displaymath}
		\jacobi{D_0} m\,=\,\jacobi{m^*}{D_0}\,=\,
		\jacobi m{D_0}\,\jacobi{(-1)^{\frac{m-1} 2}}{D_0}\,=\,
		\jacobi n{D_0}\,\jacobi{(-1)^{\frac{n-1} 2}}{D_0}\,=\,
		\jacobi{n^*}{D_0}\,=\,\jacobi{D_0} n
		\text{ ,}
	\end{displaymath}
	%
	y, para $D_0<0$, $|D_0|\equiv 1\tmodulo[4]$, con lo que,
	\begin{displaymath}
		\jacobi{D_0} m\,=\,\jacobi{|D_0|} m\,\jacobi{-1} m\,=\,
		\jacobi{|D_0|} n\,\jacobi{-1} n\,=\,\jacobi{D_0} n
		\text{ .}
	\end{displaymath}
	%
\end{proof}

\begin{obsEcuacion}\label{obs:ecuacion:kronecker}
	Tanto en la demostraci\'on del \lemaname~\ref{lema:ecuacion:kronecker},
	como en la demostraci\'on del \lemaname~\ref{lema:ecuacion:jacobi},
	hicimos uso de las propiedades
	\eqref{item:ecuacion:jacobi:clases},
	\eqref{item:ecuacion:jacobi:multiplicativo},
	\eqref{item:ecuacion:jacobi:menos-uno},
	\eqref{item:ecuacion:jacobi:dos} y
	\eqref{item:ecuacion:jacobi:reciprocidad},
	que, a su vez, dependen de las propiedades an\'alogas del s\'{\i}mbolo
	de Legendre.
	Por otro lado, la funci\'on $\chi$ del \lemaname~%
	\ref{lema:ecuacion:kronecker} \emph{no es} el s\'{\i}mbolo de Jacobi:
	el s\'{\i}mbolo de Jacobi $\tjacobi M{m}$ s\'olo est\'a definido para
	valores impares y positivos de $m$. Sin embargo, $\chi$ est\'a
	definida en todo $\Enteros$.
\end{obsEcuacion}

\subsection*{\appendixname}
El \lemaname~\ref{lema:ecuacion:kronecker} tiene una versi\'on equivalente
expresada en el lenguaje de grupos:
\begin{lemaEcuacion}\label{lema:ecuacion:kronecker:bis}
	Si $D\equiv 0,1\tmodulo[4]$, $D\neq 0$, existe un \'unico morfismo
	de grupos $\chi:\,\Unidadesmod[D]\rightarrow\cuadratico$
	tal que $\chi\clase p=\tlegendre D{p}$, si $p$ es un primo (positivo)
	impar que no divide a $D$.
\end{lemaEcuacion}
\noindent
De la misma manera, el \'{\i}tem~%
\eqref{item:ecuacion:kronecker:equivalencias:nucleo} del
\teoname~\ref{teo:ecuacion:kronecker} se traduce en que la clase del primo
pertenezca al n\'ucleo del morfismo:
$\clase p\in\ker\,\chi\leq\Unidadesmod[4|n|]$.

\begin{obsEcuacion}\label{obs:ecuacion:kronecker:bis}
	Se puede probar, conociendo la estructura de los grupos abelianos
	finitos $\Unidadesmod[q]$ ($q$ primo impar), $\Unidadesmod[4]$ y
	$\Unidadesmod[8]$, que el \lemaname~\ref{lema:ecuacion:kronecker:bis}
	(o, equivalentemente, el \lemaname~\ref{lema:ecuacion:kronecker}) y
	el \teoname~\ref{teo:ecuacion:legendre} son equivalentes: asumiendo
	la existencia de los morfismos
	$\chi:\,\Unidadesmod[D]\rightarrow\cuadratico$, se pueden deducir
	las propiedades del s\'{\i}mbolo de Legendre.
\end{obsEcuacion}

\subsection*{Ejercicios}
\theoremstyle{definition}
\newtheorem{ejerEcuacion}{\ejername}[section]

%-------------

\begin{ejerEcuacion}\label{ejer:ecuacion:kronecker}
	Sea $D\equiv 0,1\tmodulo[4]$, $D\neq 0$, y sea
	$\chi:\,\Enteros\rightarrow\cuadratico$ la funci\'on del
	\lemaname~\ref{lema:ecuacion:kronecker} (o, equivalentemente,
	el morfismo del \lemaname~\ref{lema:ecuacion:kronecker:bis}).
	\begin{enumerate}[(i)]
		\item\label{item:ejer:iecuacion:kronecker:menos-uno}
			Probar que
			\begin{displaymath}
				\chi(-1)\,=\,
					\begin{cases}
						1 \text{ ,} & \text{si }
							D>0\quad\text{y} \\
						-1\text{ ,} & \text{si }
							D<0\text{ .}
					\end{cases}
					%
			\end{displaymath}
			%
		\item\label{item:ejer:ecuacion:kronecker:dos}
			Probar que, si $D\equiv 1\tmodulo[4]$, entonces
			\begin{displaymath}
				\chi(2)\,=\,
					\begin{cases}
						1 \text{ ,} & \text{si }
							D\equiv 1\tmodulo[8]
							\quad\text{y} \\
						-1\text{ ,} & \text{si }
							D\equiv 5\tmodulo[8]
							\text{ .}
					\end{cases}
					%
			\end{displaymath}
	\end{enumerate}
	%
\end{ejerEcuacion}

\begin{ejerEcuacion}
	Sea $a\in\Enteros$ no un cuadrado perfecto. Probar que existen
	infinitos primos para los cuales $a$ no es un residuo cuadr\'atico:
	\begin{enumerate}[(i)]
		\item reducir al caso en que $a$ es libre de cuadrados, o
			sea $a=2^eq_1\cdots q_r$, donde $q_i$ son primos
			impares distintos, $r\geq 0$ y $e\in\{0,1\}$;
		\item suponer que $a$ es divisible por alg\'un primo impar
			(es decir, $a\neq 2$ y $r\geq 1$) y probar que,
			dado un conjunto finito de primos $\{\lista l{k}\}$
			que excluye a $2$ y los $q_i$, existe $b\in\Enteros$
			que satisface:
			\begin{displaymath}
				\begin{aligned}
					b & \,\equiv\,1\tmodulo[l_j]
						\text{ ,} \\
					b & \,\equiv\,1\tmodulo[8]
						\text{ ,} \\
					b & \,\equiv\,1\tmodulo[q_i]
					\quad\text{ , si }i\neq r
					\text{ y} \\
					b & \,\equiv\,s\tmodulo[q_r]
					\text{ ,}
				\end{aligned}
				%
			\end{displaymath}
			%
			donde $s$ es un no residuo cuadr\'atico m\'odulo
			$q_r$ y mostrar que $\tjacobi a{b}=-1$ y que,
			por lo tanto, existe un primo
			$p\not\in\{2,\,\lista q{r},\,\lista l{k}\}$
			tal que $\tlegendre a{p}=-1$;
		\item suponer que $a=2$ y, dado un conjunto finito de
			primos $\{\lista l{k}\}$ que excluye a $3$, definir
			$b:=8l_1\cdots l_k+3$ y probar que $\tjacobi 2{b}=-1$
			y que, por lo tanto, existe un primo
			$p\not\in\{2,\,3,\,\lista l{k}\}$ tal que
			$\tlegendre 2{p}=-1$.
	\end{enumerate}
	%
\end{ejerEcuacion}

\begin{ejerEcuacion}
	Calcular el s\'{\i}mbolo de Jacobi en los siguientes casos:
	\begin{itemize}
		\item $\tjacobi{113}{997}$
		\item $\tjacobi{215}{761}$
		\item $\tjacobi{514}{1093}$
		\item $\tjacobi{401}{757}$
	\end{itemize}
	%
\end{ejerEcuacion}

\begin{ejerEcuacion}
	Determinar las clases en $\Enterosmod[84]$ para las cuales
	$\tlegendre{-21} p=1$, usando reciprocidad cuadr\'atica.
\end{ejerEcuacion}

% \begin{solucion}
	% Sea $p$ un primo positivo, impar, $p\neq 3,7$. Entonces, usando
	% las propiedades del s\'{\i}mbolo de Legendre,
	% \begin{displaymath}
		% \legendre{-21} p\,=\,\legendre{-1} p\,\legendre 3 p\,
			% \legendre 7 p\,=\,\legendre{-1} p\,
				% (-1)^{\frac{p-1} 2}\,\legendre p 3\,
				% (-1)^{\frac{p-1} 2}\,\legendre p 7\,=\,
			% \legendre{-1} p\,\legendre p 3\,\legendre p 7
		% \text{ .}
	% \end{displaymath}
	% %
	% Sea $f(p)\in\cuadratico^3$ la funci\'on
	% \begin{displaymath}
		% f(p)\,=\,\bigg(\legendre{-1} p,\,\legendre p 3,\,\legendre p 7
			% \bigg)
		% \text{ .}
	% \end{displaymath}
	% %
	% Entonces, $\tlegendre{-21} p=1$, si y s\'olo si
	% \begin{displaymath}
		% f(p)\,\in\,\big\{(1,1,1),\,(1,-1,-1),\,(-1,1,-1),\,(-1,-1,1)
			% \big\}
		% \text{ .}
	% \end{displaymath}
	% %
	% Ahora, $f(p)=(1,1,1)$, si y s\'olo si
	% \begin{displaymath}
		% p\,\equiv\,1\tmodulo[4]\text{ ,}\quad
			% p\,\equiv\,1\tmodulo[3]\quad\text{y}\quad
			% p\,\equiv\,1,2,4\tmodulo[7]
		% \text{ ,}
	% \end{displaymath}
	% %
	% lo que da lugar a las clases
	% \begin{displaymath}
		% p\,\equiv\,1,25,37\modulo[84]
		% \text{ .}
	% \end{displaymath}
	% %
	% An\'alogamente, $(1,-1,-1)$, $(-1,1,-1)$ y $(-1,-1,1)$ dan lugar a
	% otras tres clases m\'odulo $84$ cada una. En total son $12$ clases;
	% notar que $\eulerphi(84)=\indice{\Unidadesmod[84]}=24$, o sea que
	% $\ker\,\chi$ es un subgrupo de \'{\i}ndice $2$ en $\Unidadesmod[84]$.
	% % \begin{displaymath}
		% % p\,\equiv\,1,25,37,17,5,41,31,19,55,59,23,11\modulo[84]
		% % \text{ .}
	% % \end{displaymath}
	% % %
% \end{solucion}

% \begin{ejerEcuacion}\label{ejer:ecuacion:reciprocidad:kronecker}
	% En la \S~\ref{sec:ecuacion}, \lemaname~\ref{lema:ecuacion:kronecker},
	% asumiendo las propiedades del s\'{\i}mbolo de Legendre enunciadas en
	% el \teoname~\ref{teo:ecuacion:legendre}, vimos c\'omo asociar a cada
	% $D\equiv 0,1\tmodulo[4]$, $D\neq 0$, una funci\'on $\chi(m)$,
	% $m\in\Enteros$, con ciertas propiedades que la determinan
	% un\'{\i}vocamente. En este ejercicio, veremos c\'omo, asumiendo la
	% existencia de estas funciones $\chi$, podemos demostrar las
	% propiedades del s\'{\i}mbolo de Legendre. En otras palabras, el
	% \lemaname~\ref{lema:ecuacion:kronecker} y el \teoname~%
	% \ref{teo:ecuacion:legendre} son equivalentes.
	% \begin{itemize}
		% \item
			% Sea $q$ un primo positivo e impar y sea
			% $D:=q^*=(-1)^{\frac{q-1} 2}\,q$; si bien $D$ puede ser
			% negativo, se cumple $D\equiv 1\tmodulo[4]$.
		% \item
			% Sea $D=-4$.
		% \item
			% Sea $D=8$.
	% \end{itemize}
	% %
% \end{ejerEcuacion}
% 

\begin{ejerEcuacion}
	Sea $p\neq 5$ un n\'umero primo. Probar que
	$x^2\equiv 5\tmodulo[p]$ tiene soluci\'on, si y s\'olo si
	$p\equiv 1,4\tmodulo[5]$.
\end{ejerEcuacion}

\begin{ejerEcuacion}
	Describir el morfismo de grupos
	$\chi:\,\Unidadesmod[4a]\rightarrow\{\pm1\}$ en el caso $a=3$.
\end{ejerEcuacion}


