\theoremstyle{plain}

\theoremstyle{definition}

%-------------

Hasta el momento, contamos con un criterio para determinar si un
n\'umero primo se puede representar por una forma cuadr\'atica
de discriminante prescrito %.
% Incluso, en el caso de que lo sea, podemos determinar
% el g\'enero de las formas que lo representan.
?`Qu\'e pasa con n\'umeros compuestos?
?`Podemos representar $pq$ por una forma de discriminante $D$,
si podemos representar los enteros $p$ y $q$ por formas de discriminante $D$?
