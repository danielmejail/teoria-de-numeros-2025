A mediados del siglo XVII, Pierre de Fermat realizaba observaciones como
las siguientes: si $p$ es un primo impar que
\begin{itemize}
	\item excede en $1$ un m\'ultiplo de $4$, entonces existen enteros
		$x$ e $y$ tales que $p=x^2+y^2$;
	\item excede en $1$ o $3$ un m\'ultiplo de $8$, entonces existen
		enteros $x$ e $y$ tales que $p=x^2+2y^2$;
	\item excede en $1$ un m\'ultiplo de $3$, entonces existen enteros
		$x$ e $y$ tales que $p=x^2+3y^2$.
\end{itemize}
%
En un lenguaje un poco m\'as moderno, Fermat afirmaba que,
\begin{itemize}
	\item si $p$ es congruente con $1$ m\'odulo $4$,
		entonces $p$ es de la forma $x^2+y^2$;
	\item si $p$ es congruente con $1$ o $3$ m\'odulo $8$,
		entonces $p$ es de la forma $x^2+2y^2$;
	\item si $p$ es congruente con $1$ m\'odulo $3$,
		entonces $p$ es de la forma $x^2+3y^2$.
\end{itemize}
%
En general, se preguntaba, dado un entero $n$ ?`qu\'e primos $p$ se pueden
expresar en la forma $x^2+ny^2$, con $x$ e $y$ enteros?

La primera persona de la que queda registro que intent\'o dar demostraciones
de las observaciones de Fermat fue Leonhard Euler. Trabajando para alcanzar
dicho objetivo, Euler descubri\'o lo que conocemos por el nombre de Ley de
reciprocidad cuadr\'atica. Si bien no logr\'o demostrar este resultado,
consigui\'o demostrar algunas de las afirmaciones de Fermat, como tambi\'en
formular algunas conjeturas similares propias cuando $n>3$.
Euler afirmaba, por ejemplo, que, si $p$ es un primo impar, entonces
\begin{itemize}
	\item $p$ es de la forma $x^2+5y^2$, si y s\'olo si
		$p$ es congruente con $1$ o $9$ m\'odulo $20$;
	\item $p$ es de la forma $x^2+27y^2$, si y s\'olo si
		$p$ es congruente con $1$ m\'odulo $3$ y $2$ es congruente
		con un cubo m\'odulo $p$.
\end{itemize}
%

El siguiente avance vino de la mano de Joseph-Louis Lagrange, quien llev\'o
adelante el estudio de formas cuadr\'aticas definidas positivas. Lagrange
introdujo los conceptos de forma reducida, n\'umero de clases, g\'enero de
una forma cuadr\'atica. Estas ideas permiten dar respuesta a las conjeturas
de Euler para $n=5$ y otros casos similares. Pero la teor\'{\i}a de g\'eneros
tiene sus limitaciones.

Se le atribuye a Adrien-Marie Legendre la introducci\'on de ideas
rudimentarias acerca de la composici\'on de formas cuadr\'aticas. Pero es
Carl Friedrich Gauss quien establece la relaci\'on entre la teor\'{\i}a de
g\'eneros y composici\'on de formas cuadr\'aticas. De esta manera, Gauss
muestra que en el conjunto de formas cuadr\'aticas existe una estructura
algebraica subyacente. En relaci\'on con este trabajo, Gauss demuestra la
Ley de reciprocidad cuadr\'atica y formula las leyes de reciprocidad
c\'ubica y bicuadr\'atica. Esto \'ultimo permite dar respuesta a dos de los
casos del problema de Fermat que las ideas anteriores no lograban alcanzar:
los casos $n=27$ y $n=64$. Si bien este avance puede parecer insignificante,
result\'o ser de gran relevancia, pues abri\'o el camino al estudio de leyes
de reciprocidad de orden superior.

Para dar una idea del intervalo de tiempo en que ocurrieron estos avances,
las observaciones de Fermat se pueden fechar alrededor del a\~no 1640.
Las demostraciones por parte de Euler de algunas de dichas afirmaciones
consisten en dos partes: por un lado, profundizar una idea de Fermat,
\emph{descenso} y, por otro lado, \emph{reciprocidad}. El trabajo de Euler
abarca, aproximadamente, el per\'{\i}odo que va del a\~no 1730 al a\~no 1772;
pasaron ya m\'as de 130 a\~nos desde que Fermat primero estudi\'o el problema
de representar un primo como suma de dos cuadrados.
La primera publicaci\'on de \emph{Disquisitiones arithmeticae} de Gauss data
del a\~no 1801. En el interim, ocurri\'o el desarrollo de la teor\'{\i}a de
formas cuadr\'aticas por Lagrange y Legendre. En este sentido, mencionamos
los trabajos \emph{Recherches d'arithm\'etique}, de Lagrange (publicado
entre 1773 y 1775), y \emph{Th\'eorie des nombres}, de Legendre (publicado
en 1798 y cuya \'ultima versi\'on es de 1830). En este trabajo, Legendre
introduce el nombre de ``forma cuadr\'atica'' para referirse a expresiones
del tipo $ax^2+bxy+cy^2$ y para distingurlas de las ``formas lineales'', del
tipo $mx+a$.

Las fechas anteriores dan una aproximaci\'on del marco temporal en el que
ocurrieron estos avances. Para m\'as detalles, referimos a la introducci\'on y
la \S~1 del libro de Cox \cite{Cox}.

