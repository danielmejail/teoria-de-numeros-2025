A lo largo del curso, estudiaremos el problema de representar un n\'umero
primo de la forma $x^2+ny^2$ con $x$ e $y$ n\'umeros enteros.
Buscando darle respuesta a esta pregunta, nos encontraremos con dos temas que
nos ser\'an de ayuda: leyes de reciprocidad y formas cuadr\'aticas binarias.
% El curso estar\'a dividido en cinco partes.

Comenzaremos estudiando propiedades b\'asicas de los n\'umeros enteros,
como divisibilidad y la relaci\'on de congruencia que se puede derivar de
ella, e introduciremos algunas estructuras algebraicas, como anillos
conmutativos y grupos finitos; no estudiaremos estas estructuras en
profundidad, pero nos servir\'an para formular y enmarcar varios de los
resultados que se presentar\'an. Espec\'{\i}ficamente, usaremos las
definiciones como nombres, para poder hablar de los anillos de enteros
modulares $\Enterosmod[m]$, los anillos $\polinomios[\sqrt{-1}] \Enteros$ y
$\polinomios[\frac{-1+\sqrt{-3}} 2] \Enteros$ y sus grupos de unidades
(elementos inversibles).

Usando los conceptos introducidos, relacionaremos
la ecuaci\'on $p=x^2+ny^2$ con la Ley de reciprocidad cuadr\'atica,
uno de los resultados fundamentales de Teor\'{\i}a de n\'umeros.
Daremos algunas aplicaciones y veremos las limitaciones de esta herramienta.

Luego de demostrar este resultado, volveremos sobre el problema inicial y
lo pondremos en el contexto m\'as general de formas cuadr\'aticas binarias.
Estudiaremos, espec\'{\i}ficamente, formas cuadr\'aticas definidas positivas,
introduciendo los conceptos de equivalencia propia, clase y g\'enero de una
forma cuadr\'atica, composici\'on y grupo de clases. Esta teor\'{\i}a nos
permitir\'a dar una respuesta satisfactoria al problema en algunos casos
especiales, pero dejar\'a de manifiesto sus limitaciones.

La \'ultima parte del curso estar\'a dedicada a las leyes de reciprocidad
c\'ubica y bicuadr\'atica, las cuales nos permitir\'an dar un paso m\'as en
la direcci\'on de la resoluci\'on del problema.

Una respuesta completa a la pregunta de bajo qu\'e condiciones la ecuaci\'on
$p=x^2+ny^2$ tiene soluci\'on (en $\Enteros$) viene dada de la mano de la
Teor\'{\i}a de cuerpos de clases. No es el objetivo del curso entrar en ese
tema, sino presentar los conceptos b\'asicos para poder encarar temas m\'as
avanzados dentro del \'area.

Estas notas est\'an basadas, principalmente, en dos referencias: los libros
\emph{Primes of the form $x^2+ny^2$} de David A. Cox \cite{Cox} y
\emph{A Classical Introduction to Modern Number Theory} de Kenneth Ireland y
Michael Rosen \cite{IrelandRosen}.
El material que presentamos se puede encontrar en el cap\'{\i}tulo 1,
secciones 1 a 4 de \cite{Cox} y los cap\'{\i}tulos 1 a 9 de
\cite{IrelandRosen}. Hemos incluido una gran cantidad de ejercicios,
tomados de estas y otras referencias, que exploran en mayor profundidad
algunos de los temas que veremos, como tambi\'en otros temas que se pueden
abordar usando los conceptos desarrollados durante el curso.

Salvo talvez algunos puntos de la presentaci\'on, no hay, esencialmente,
nada novedoso en estas notas. Ojal\'a las fallas y omisiones de estas notas
no constituyan un obst\'aculo.

Les agradezco a Mariano y a Mat\'{\i}as por darle una le\'{\i}da a las
notas y por sus comentarios. Agradezco tambi\'en a Gonzalo y a Gustavo,
cuyas versiones del curso me han servido de gu\'{\i}a.
