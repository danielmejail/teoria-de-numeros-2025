\theoremstyle{plain}
\newtheorem{teoPell}{\teoname}[section]
\newtheorem{lemaPell}[teoPell]{\lemaname}

\theoremstyle{definition}
\newtheorem{obsPell}[teoPell]{\obsname}
\newtheorem{ejemPell}[teoPell]{\ejemname}

%-------------

\subsection{Un ejemplo}\label{subsec:pell:dos}
?`Qu\'e significa resolver la ecuaci\'on
\begin{equation}
	\label{eq:pell:dos:raiz}
	z^2\,=\,2
	\text{ ?}
\end{equation}
%
Desde un punto de vista geom\'etrico, la soluci\'on de
\eqref{eq:pell:dos:raiz} ``es'' la longitud de la diagonal de un
cuadrado de lados de longitud $1$.
La ecuaci\'on \eqref{eq:pell:dos:raiz} no tiene soluciones racionales.
El argumento se puede separar en dos partes.
Por un lado, el conjunto
\begin{displaymath}
	\big\{y\in\Enteros\,:\,y\geq 0,\,
		z=x/y\text{ cumple }\eqref{eq:pell:dos:raiz} \big\}
\end{displaymath}
%
tiene un primer elemento, o bien es vac\'{\i}o.
Por otro, si $z=x/y$ cumple \eqref{eq:pell:dos:raiz}, entonces,
por Factorizaci\'on \'unica, $2\mid x$, $4\mid 2y^2$ y $2\mid y$,
y, as\'{\i}, $z=x'/y'$, con $x'=x/2$ e $y'=y/2$. Pero, entonces,
$|y'|<|y|$.

Aunque \eqref{eq:pell:dos:raiz} no tenga soluciones racionales,
podemos buscar aproximaciones a una soluci\'on.
En primer lugar, la resolubilidad de la ecuaci\'on \eqref{eq:pell:dos:raiz}
con $z\in\Racionales$, equivale a la resolubilidad de
\begin{equation}
	\label{eq:pell:dos:raiz:bis}
	x^2\,=\,2y^2
\end{equation}
%
con $x,y\in\Enteros$ ($y\neq 0$). En el plano cartesiano real,
\eqref{eq:pell:dos:raiz:bis} describe dos rectas:
$x=y\sqrt 2$ y $x=-y\sqrt 2$, que se cortan en $(0,0)$.
La ecuaci\'on
\begin{equation}
	\label{eq:pell:dos}
	x^2-2y^2\,=\,1
\end{equation}
%
describe una hip\'erbola que tiene a la recta $x=y\sqrt 2$ como
as\'{\i}ntota. En particular, los pares $(x,y)$ que verifican
\eqref{eq:pell:dos} se pueden interpretar como aproximaciones a
$\sqrt 2$: si $x,y>0$ son soluciones a \eqref{eq:pell:dos}, entonces
\begin{displaymath}
	(x/y)^2\,-\,2\,=\,1/y^2
	\dispstop
\end{displaymath}
%
As\'{\i}, con $y\to\infty$, el valor $z=x/y$ tiende a $\sqrt 2$.
La ventaja de \eqref{eq:pell:dos} por sobre \eqref{eq:pell:dos:raiz:bis}
es que admite soluciones $x,y\in\Enteros$.

Una soluci\'on a \eqref{eq:pell:dos} es $(x,y)=(1,0)$.
Otra soluci\'on es $(x,y)=(3,2)$.
Toda otra soluci\'on se puede ``deducir'' de esta \'ultima.
Esto sugiere que deber\'{\i}a haber cierta estructura en el conjunto
de soluciones a \eqref{eq:pell:dos} que explique este fen\'omeno.
Se verifica la siguiente identidad:
\begin{equation}
	\label{eq:pell:dos:identidad}
	(uU+2vV)^2\,-\,2\,(uV+vU)^2\,=\,
		\big(u^2-2v^2\big)\,\big(U^2-2V^2\big)
	\dispcomma
\end{equation}
%
de lo que se deduce que,
si $(u,v)$ y $(U,V)$ resuelven \eqref{eq:pell:dos},%
\footnote{
	Notar que no estamos diciendo nada acerca de ``d\'onde''
	pueden tomar valores las inc\'ognitas $u$, $v$, $U$ y $V$.
}
entonces podemos construir una nueva soluci\'on:
\begin{equation}
	\label{eq:pell:dos:multiplicar}
	(u,v)\cdot (U,V)\,=\,\big(uU+2vV,uV+vU\big)
	\dispstop
\end{equation}
%
Es decir, $(x,y):=(u,v)\cdot (U,V)$ tambi\'en es soluci\'on
de \eqref{eq:pell:dos}.%
\footnote{
	Adem\'as, las coordenadas $x,y$ son funciones bilineales
	de las coordenadas $u,v$ y $U,V$.
}
La operaci\'on \eqref{eq:pell:dos:multiplicar} verifica:
\begin{displaymath}
	\begin{aligned}
		(u,v)\cdot (U,V) & \,=\,(U,V)\cdot (u,v)\dispcomma \\
		(1,0)\cdot (U,V) & \,=\,(U,V)\dispcomma \\
		(u,v)\cdot (1,0) & \,=\,(u,v)\dispcomma \\
		(x,y)\cdot\big((u,v)\cdot (U,V)\big) & \,=\,
			\big((x,y)\cdot (u,v)\big)\cdot (U,V)\dispand \\
		(3,2)\cdot (3,-2) & \,=\,(1,0)\dispstop
	\end{aligned}
	%
\end{displaymath}
%

\begin{teoPell}\label{teo:pell:dos}
	Sea $(x_1,y_1)=(3,2)$ y, para $n>1$, definimos
	$(x_n,y_n)=(3,2)\cdot (x_{n-1},y_{n-1})$.
	\begin{enumerate}[(i)]
		\item\label{item:pell:dos:soluciones}
			Dados enteros $u,v>0$ tales que
			$u^2-2v^2=1$, existe $n\geq 1$ tal que
			$(u,v)=(x_n,y_n)$.
		\item\label{item:pell:dos:matricial}
			La sucesi\'on $(x_n,y_n)$ verifica
			\begin{displaymath}
				\begin{bmatrix}
					x_n \\ y_n
				\end{bmatrix}\,=\,
				\begin{bmatrix}
					3 & 4 \\ 2 & 3
				\end{bmatrix}^n\,
				\begin{bmatrix} 1 \\ 0 \end{bmatrix}
				\dispstop
			\end{displaymath}
			%
		\item\label{item:pell:dos:cuadratico}
			La sucesi\'on $(x_n,y_n)$ verifica
			\begin{displaymath}
				x_n+y_n\sqrt 2\,=\,
					(3+2\sqrt 2)^n
				\dispstop
			\end{displaymath}
			%
		\item\label{item:pell:dos:aproxima}
			Dados enteros $x,y>0$ tales que $x^2-2y^2=1$,
			se cumple
			\begin{displaymath}
				\left|\frac x y\,-\,\sqrt 2\right|\,<\,
					\frac 1{2\sqrt 2 y^2}
				\dispstop
			\end{displaymath}
			%
		\item\label{item:pell:dos:aproxima:reciproca}
			Dados enteros $x,y>0$ tales que $|x/y-\sqrt 2|<1/2y^2$,
			se cumple $|x^2-2y^2|=1$. Rec\'{\i}procamente,
			si $|x^2-2y^2|=1$, entonces $|x/y-\sqrt 2|<1/2y^2$.
	\end{enumerate}
	%
\end{teoPell}

\begin{proof}
	En cuanto a \eqref{item:pell:dos:soluciones},
	notamos que no hay soluciones con $v=1$ y que, con $v=2$,
	la soluci\'on en enteros positivos es $(3,2)$. Si $(u,v)$ es
	soluci\'on con $u>0$ y $v>2$, entonces
	\begin{displaymath}
		(3,-2)\cdot (u,v)\,=\,(3u-4v,3v-2u)
		\dispstop
	\end{displaymath}
	%
	Pero,
	\begin{displaymath}
		3v-2u\,=\,\frac{9v^2-4u^2}{3v+2u}\,=\,
			\frac{v^2-4}{3v+2u}
	\end{displaymath}
	%
	muestra que $0<3v-2u<v$ y $u^2=2v^2+1$ implica
	$u>v\sqrt 2$, con lo que $3u-4v>(3\sqrt 2-4)\,v>0$.
	Inductivamente, $(3,-2)\cdot(u,v)=(3,2)^n$.
\end{proof}

El \'{\i}tem \teoname~\ref{teo:pell:dos}~%
\eqref{item:pell:dos:aproxima:reciproca}
sugiere estudiar tambi\'en la ecuaci\'on
\begin{equation}
	\label{eq:pell:dos:negativa}
	x^2\,-\,2y^2\,=\,-1
	\dispstop
\end{equation}
%
Se puede ver que $(1,1)$ es soluci\'on de \eqref{eq:pell:dos:negativa}.

\begin{teoPell}\label{teo:pell:dos:negativa}
	Dados enteros $u,v>0$ tales que $|u^2-2v^2|=1$, existe $n\geq 1$
	tal que $(u,v)=(1,1)^n$.
\end{teoPell}

\begin{proof}
	Se puede comprobar que $(3,2)=(1,1)^2$.
\end{proof}

\subsection{Existencia de soluciones}\label{subsec:pell:existencia}
