\theoremstyle{plain}
\newtheorem{teoFermat}{\teoname}%[section]
\newtheorem{lemaFermat}[teoFermat]{\lemaname}

\theoremstyle{definition}
\newtheorem{obsFermat}[teoFermat]{\obsname}
\newtheorem{ejemFermat}[teoFermat]{\ejemname}

%-------------

% Usando el lenguaje desarrollado en la \partname~%
% \ref{pt:divisibilidad-y-congruencia}, podemos expresar algunas de las
% conjeturas de Fermat: si $p$ es un primo impar,
% \begin{displaymath}
	% \begin{aligned}
		% p\,=\,x^2+y^2\text{ ,}\quad x,y\in\Enteros\text{ ,}
			% & \quad\text{si y s\'olo si}\quad
			% p\,\equiv\,1\tmodulo[4] \\
		% p\,=\,x^2+2y^2\text{ ,}\quad x,y\in\Enteros\text{ ,}
			% & \quad\text{si y s\'olo si}\quad
			% p\,\equiv\,1\text{ o }3\tmodulo[8] \\
		% p\,=\,x^2+3y^2\text{ ,}\quad x,y\in\Enteros\text{ ,}
			% & \quad\text{si y s\'olo si}\quad
			% p=3\text{ o }p\,\equiv\,1\tmodulo[3]
		% \text{ .}
	% \end{aligned}
	% %
% \end{displaymath}
% %
% Tambi\'en tenemos la siguiente conjetura, un poco distinta de las tres
% anteriores: si $p$ y $q$ son primos impares distintos,
% \begin{displaymath}
	% p\text{ y }q\,\equiv\,3\text{ o }7\tmodulo[20]
		% \quad\text{implica}\quad
		% pq\,=\,x^2+5y^2\text{ ,}\quad x,y\in\Enteros
	% \text{ .}
% \end{displaymath}
% %
% Todas estas afirmaciones est\'an relacionadas con el problema de representar
% un n\'umero en la forma $x^2+ny^2$. La primera demostraci\'on de las primeras
% tres afirmaciones parece haber sido encontrada por Euler alrededor de cien
% a\~nos despu\'es de que Fermat las enunciara. La cuarta fue demostrada m\'as
% tarde por Lagrange. Ver \cite[pp.~7--9]{Cox}.

\begin{ejemFermat}\label{ejem:fermat}
	Si $p=x^2+y^2$ con $x,y\in\Enteros$, entonces $p\equiv 1\tmodulo[4]$,
	pues $x^2\equiv 0$ o $x^2\equiv 1$, dependiendo de si $x$ es par o
	impar y, si $p$ es un primo impar, $x$ e $y$ deben tener distinta
	paridad.
	% An\'alogamente, si $p=x^2+2y^2$, entonces $p\equiv 1\tmodulo[8]$ o
	% bien $p\equiv 3\tmodulo[8]$. Y, si $p=x^2+3y^2$, entonces
	% $p=3$ o bien $p\equiv 1\tmodulo[3]$. \quedacomoejercicio.
\end{ejemFermat}

\begin{teoFermat}\label{teo:fermat}
	Un primo impar $p$ se escribe como $x^2+y^2$ con $x$ e $y$ enteros,
	si y s\'olo si $p\equiv 1\tmodulo[4]$.
\end{teoFermat}

\begin{proof}
	Vamos a probar que, si $p\equiv 1\tmodulo[4]$, entonces existen
	$x,y\in\Enteros$ tales que $p=x^2+y^2$. La demostraci\'on estar\'a
	dividida en dos pasos:
	\begin{itemize}
		\item[(Descenso)]
			si $p$ divide a una expresi\'on del tipo $a^2+b^2$
			donde $a,b\in\Enteros$ y $\mcd{a,b}=1$, entonces
			existen $x,y\in\Enteros$ tales que $p=x^2+y^2$;
		\item[(Reciprocidad)]
			si $p\equiv 1\tmodulo[4]$, entonces $p$ divide a
			alg\'un natural de la forma $a^2+b^2$ con
			$a,b\in\Enteros$ y $\mcd{a,b}=1$.
	\end{itemize}
	%
	Probamos cada una de estas afirmaciones, a continuaci\'on.
\end{proof}

La idea de Fermat era, aparentemente, la siguiente: si $p$ es primo,
$p\equiv 1\tmodulo[4]$ y $p$ \emph{no} se escribe como suma de dos cuadrados,
deber\'{\i}a existir un primo $p'<p$ que cumpla que es $p'\equiv 1\tmodulo[4]$
y que tampoco es suma de dos cuadrados. Eventualmente, llegar\'{\i}amos a $5$
que s\'{\i} es suma de dos cuadrados. De esta contradicci\'on
(de que $5$ es y no deber\'{\i}a ser suma de dos cuadrados) se deducir\'{\i}a
el resultado.

\begin{lemaFermat}\label{lema:fermat:descenso:pre}
	Sea $N$ un natural que se puede expresar como suma de dos cuadrados
	\emph{coprimos}. Si $q$ es un divisor primo de $N$ que se puede
	expresar como suma de dos cuadrados, entonces el cociente tambi\'en
	se puede expresar como suma de dos cuadrados coprimos.
\end{lemaFermat}

\begin{proof}
	Por hip\'otesis, existen $a,b\in\Enteros$ tales que $N=a^2+b^2$ y
	$\mcd{a,b}=1$ y existen, adem\'as, $x,y\in\Enteros$ tales que
	$q=x^2+y^2$ ($x$ e $y$ son, necesariamente, coprimos).
	Como $q\mid N$, se cumple que $q$ divide a $(bx-ay)(bx+ay)$, pues
	\begin{displaymath}
		Nx^2-a^2q\,=\,
		b^2x^2+a^2x^2-a^2x^2-a^2y^2\,=\,(bx)^2-(ay)^2\,=\,
		(bx-ay)\,(bx+ay)
		\text{ .}
	\end{displaymath}
	%
	Como $q$ es primo, $q\mid bx-ay$, o bien $q\mid bx+ay$.
	Cambiando el signo de $a$, podemos suponer que estamos en el primer
	caso. Esto quiere decir que existe $d\in\Enteros$ tal que
	\begin{math}
		bx-ay=dq
	\end{math}.
	Ahora bien,
	\begin{displaymath}
		bx-dx^2\,=\,
		bx-dq+dy^2\,=\,ay+dy^2\,=\,
		(a+dy)\,y
		\text{ .}
	\end{displaymath}
	%
	Como $x\mid bx-dx^2$, se cumple que $x\mid (a+dy)y$. Pero
	$\mcd{x,y}=1$, con lo que $x\mid a+dy$. Existe, entonces,
	$c\in\Enteros$ tal que
	\begin{math}
		a+dy=cx
	\end{math}.
	O sea,
	\begin{displaymath}
		a \,=\,cx-dy\quad\text{y}\quad b \,=\,dx+cy\text{ .}
		% bx\,=\,dq + ay\,=\,d\,(x^2+y^2)+(cx-dy)\,y\,=\,
		% dx^2+cxy\,=\,(dx+cy)\,x
	\end{displaymath}
	%
	Pero, entonces,
	\begin{displaymath}
		N\,=\,a^2+b^2\,=\,
		(cx-dy)^2+(dx+cy)^2\,=\,(c^2+d^2)\,(x^2+y^2)\,=\,
		(c^2+d^2)\,q
		\text{ .}
	\end{displaymath}
	%
	As\'{\i}, $N/q=c^2+d^2$, pero, adem\'as, $\mcd{a,b}=1$ implica
	que $\mcd{c,d}=1$
	\quedacomoejercicio.
\end{proof}

\begin{obsFermat}\label{obs:fermat:descenso}
	En la demostraci\'on del \lemaname~\ref{lema:fermat:descenso:pre}
	hicimos uso de la siguiente identidad \quedacomoejercicio:
	\begin{equation}
		\label{eq:fermat:descenso:identidad}
		(cx-dy)^2+(dx+cy)^2\,=\,(c^2+d^2)\,(x^2+y^2)
		\text{ .}
	\end{equation}
	%
\end{obsFermat}

Ahora s\'{\i}, probamos el paso de Descenso.
% La afirmaci\'on es la siguiente:
% todo primo que aparece como divisor de una suma de cuadrados coprimos es,
% a su vez, una suma de cuadrados.

\begin{lemaFermat}[Descenso]\label{lema:fermat:descenso}
	Sea $p$ un primo impar que divide a una expresi\'on del tipo $a^2+b^2$
	donde $a,b\in\Enteros$ y $\mcd{a,b}=1$. Entonces, existen
	$x,y\in\Enteros$ tales que $p=x^2+y^2$.
\end{lemaFermat}

\begin{proof}
	La hip\'otesis es que $p\mid N=a^2+b^2$ y $\mcd{a,b}=1$.
	Veamos, en primer lugar, que podemos asumir, adem\'as, que $|a|<p/2$
	y que $|b|<p/2$.
	Cambiando $a$ por $a-kp$ con alg\'un $k$ conveniente, conseguimos
	$|a-kp|<p/2$. Pero, $(a-kp)^2+b^2=a^2+b^2-2kp+k^2p^2$. Como
	$p\mid a^2+b^2$, deducimos que $p$ divide a $(a-kp)^2+b^2$. Cambiamos
	$a$ por $a-kp$ y ahora $p$ divide a una expresi\'on $a_1^2+b^2$, donde
	$|a_1|<p/2$.
	De la misma manera, conseguimos que $p$ divida una expresi\'on
	$N_1:=a_1^2+b_1^2$ con $|b_1|<p/2$.
	Pero, al hacer estos cambios, podemos haber introducido divisores
	comunes entre $a_1$ y $b_1$. Es decir, puede pasar que
	$\mcd{a_1,b_1}>1$.
	Sin embargo, dividiendo por $\mcd{a_1,b_1}$, volvemos al caso
	coprimo. Veamos esto. Sea $d:=\mcd{a_1,b_1}$. Por c\'omo fueron
	elegidos $a_1=a-kp$ y $b_1=b-lp$ con $k,l\in\Enteros$. Como
	$\mcd{a,b}=1$, en particular, $p$ no es un divisor com\'un de $a$ y
	de $b$. Pero, entonces, $p$ tampoco es un divisor com\'un de $a_1$ y
	de $b_1$. O sea, $\mcd{p,d}=1$. De esta manera, si $N_2:=N_1/d^2$,
	$a_2:=a_1/d$ y $b_2:=b_1/d$, entonces $a_2,b_2\in\Enteros$,
	$|a_2|\leq |a_1|<p/2$ y $|b_2|\leq |b_1|<p/2$, $\mcd{a_2,b_2}=1$ y,
	finalmente, $p\mid N_2$ (aqu\'{\i} usamos que $\mcd{p,d}=1$).

	Recapitulando, asumimos que $p\mid N$, donde $N=a^2+b^2$,
	$\mcd{a,b}=1$, $|a|<p/2$ y $|b|<p/2$. Bajo estas suposiciones
	adicionales, $N<p^2/2$. En particular, si $q\neq p$ es un divisor
	primo de $N$, debe ser $q<p$. Separamos dos casos: o bien todo tal
	$q$ es suma de dos cuadrados, o bien existe un divisor primo $q$ de
	$N$ que no es suma de cuadrados.
	En el primer caso, por el \lemaname~\ref{lema:fermat:descenso:pre},
	se deduce, eliminando todos los factores primos distintos de $p$,
	que $p$ tambi\'en debe ser suma de cuadrados (notar que $p^2\nmid N$).
	Supongamos, para llegar a una contradicci\'on que $p$ no es suma de
	cuadrados. Entonces, existir\'{\i}a un divisor primo de $N$, $q<p$,
	que tampoco es suma de cuadrados. Necesariamente, $q$ debe ser impar
	($2=1^2+1^2$) y, m\'as aun, $q\equiv 1\tmodulo[4]$.%
	\footnote{
		Si $q\mid a^2+b^2$, con $\mcd{a,b}=1$, entonces
		$a^2+b^2\equiv 0\tmodulo[q]$ y
		$-1\equiv (ab^{-1})^2\tmodulo[q]$.
		Es decir, existe $x\in\Enteros$ tal que
		$x^4\equiv 1\tmodulo[q]$, pero $x^2\not\equiv 1\tmodulo[q]$.
		Esto implica que $4\mid\eulerphi(q)=q-1$, o sea,
		$q\equiv 1\tmodulo[4]$.
	}
	% (ver el \ejemname~\ref{ejem:fermat}).
	Pero, entonces, estar\'{\i}amos en las
	mismas hip\'otesis del resultado que queremos probar: $q$ es primo
	impar que divide a $N=a^2+b^2$, con la diferencia de que $q$ es
	estrictamente m\'as chico que $p$. Eventualmente, deber\'{\i}amos
	llegar al menor primo con esta propiedad. Para terminar, notamos que
	el menor primo impar congruente con $1$ m\'odulo $4$ es $5$ que
	s\'{\i} es suma de cuadrados ($5=2^2+1^2$). Llegamos a la siguiente
	contradicci\'on $5$ es suma de cuadrados, pero, por el proceso por
	el que \emph{descendimos} hasta este primo, $5$ no deber\'{\i}a ser
	suma de cuadrados. Esta contradicci\'on viene de suponer que el primo
	$p$ del cual partimos no era suma de cuadrados.
\end{proof}

\begin{lemaFermat}[Reciprocidad]\label{lema:fermat:reciprocidad}
	Si $p$ es un n\'umero primo impar y $p\equiv 1\tmodulo[4]$,
	entonces $p$ divide a una expresi\'on del tipo $a^2+b^2$ donde
	$a,b\in\Enteros$ y $\mcd{a,b}=1$.
\end{lemaFermat}

\begin{proof}
	Por hip\'otesis, $p-1=4k$, $k\in\Enteros$,
	% y,
	entonces,
	% por el \teoname~\ref{teo:congruencias:fermatito},
	como $\eulerphi(p)=p-1$,
	$x^{4k}\equiv 1\tmodulo[p]$, para todo $x\in\Enteros$ coprimo con $p$
	(o sea, para todo $x\not\equiv 0\tmodulo[p]$), o, lo que es lo mismo,
	$p\mid x^{4k}-1$. Pero $x^{4k}-1=(x^{2k}-1)\,(x^{2k}+1)$. Como $p$ es
	primo, $p\mid x^{2k}-1$ o bien $p\mid x^{2k}+1$. En t\'erminos de
	congruencias, para todo $x\in\Enteros$, $\mcd{x,p}=1$,
	\begin{displaymath}
		x^{2k}-1\,\equiv\,0\tmodulo[p]
			\quad\text{o bien}\quad
			x^{2k}+1\,\equiv\,0\tmodulo[p]
			\text{ .}
	\end{displaymath}
	%
	Es decir, cada una de las $p-1$ clases de congruencia $\not\equiv 0$
	m\'odulo $p$ es soluci\'on de alguna de estas dos ecuaciones de
	congruencia. Pero, como $2k<p$, existe $x\in\Enteros$ tal que
	$x^{2k}-1\not\equiv 0\tmodulo[p]$ \quedacomoejercicio.%
	\footnote{
		La cantidad de ra\'{\i}ces m\'odulo $p$ es, a lo sumo,
		el grado del polinomio.
		% Para dar una prueba de esto, se puede usar el resultado del
		% \ejername~\ref{ejer:fermat:soluciones:euler}.
		% Usando los resultados de la \S~\ref{sec:polinomios}, se
		% puede dar una demostraci\'on m\'as algebraica.
		% M\'as adelante, veremos otra demostraci\'on de que la
		% cantidad de soluciones distintas a una ecuaci\'on de
		% congruencia $f(x)\equiv 0\tmodulo[p]$ est\'a acotada por
		% el grado del polinomio $f$ (\lemaname~%
		% \ref{lema:residuos:polinomios}).
	}
	Entonces, debe ser $x^{2k}+1\equiv 0\tmodulo[p]$.
	Obtenemos el resultado eligiendo $a=x^k$ y $b=1$.
\end{proof}

% \subsection*{Ejercicios}
% \theoremstyle{definition}
\newtheorem{ejerFermat}{\ejername}[section]

%-------------

\begin{ejerFermat}\label{ejer:fermat:identidades}
	Verificar las siguientes identidades:
	\begin{enumerate}[(i)]
		\item\label{item:ejer:fermat:identidades:cuadrados}
			\begin{math}
				(x^2+y^2)\,(z^2+w^2)=
					(xz\pm yw)^2+(xw\mp yz)^2
			\end{math};
		\item\label{item:ejer:fermat:identidades}
			\begin{math}
				(x^2+ny^2)\,(z^2+nw^2)=
					(xz\pm nyw)^2+n\,(xw\mp yz)^2
			\end{math}.
	\end{enumerate}
	%
	Generalizar y hallar una identidad del estilo
	\begin{displaymath}
		(ax^2+cy^2)\,(az^2+cw^2)\,=\,(\text{?`?})^2+ac\,(\text{?`?})^2
		\text{ .}
	\end{displaymath}
	%
\end{ejerFermat}

\begin{ejerFermat}\label{ejer:fermat:soluciones:euler}
	Sean $p$ un n\'umero primo y
	\begin{displaymath}
		f\,=\,a_0\,+\,a_1\,X\,+\,\cdots\,+\,a_{d-1}\,X^{d-1}\,+\,X^d
	\end{displaymath}
	%
	un polinomio m\'onico (coeficiente principal $=1$) de grado $d<p$.
	Una de las conclusiones de este ejercicio ser\'a que
	$f(x)\not\equiv 0\tmodulo[p]$ tiene soluci\'on, es decir, existe
	$x\in\Enteros$ tal que $f(x)\not\equiv 0\tmodulo[p]$. Dicho de otra
	manera, $f(x)\equiv 0\tmodulo[p]$, con $f$ de grado $d<p$, no puede
	tener $p$ soluciones disitintas m\'odulo $p$.
	Con este objetivo, definimos $\Delta f$ como el polinomio
	\begin{displaymath}
		\Delta f\,=\,f(X+1)\,-\,f(X)
		\text{ ;}
	\end{displaymath}
	%
	podemos iterar y definir $\Delta^{k+1}f=\Delta(\Delta^k f)$.
	Probar las siguientes afirmaciones:
	\begin{enumerate}[(i)]
		\item\label{item:ejer:fermat:soluciones:euler:i}
			si $k\geq 1$, entonces $\Delta^k f$ es una
			combinaci\'on lineal de $f(X)$, $f(X+1)$, \dots,
			$f(X+k)$ con coeficientes enteros;
		\item\label{item:ejer:fermat:soluciones:euler:ii}
			si $k=d$, el polinomio $\Delta^d f$ es constante:
			$\Delta^d f(X)=d!$;
		\item\label{item:ejer:fermat:soluciones:euler:iii}
			existe $x\in\Enteros$ tal que
			$f(x)\not\equiv 0\tmodulo[p]$.%
			\hint{
				si $f(x)\not\equiv 0\tmodulo[p]$
				\emph{no admitiera soluciones}, entonces
				por
				\eqref{item:ejer:fermat:soluciones:euler:i},
				$p\mid\Delta^d f$ y, por
				\eqref{item:ejer:fermat:soluciones:euler:ii}
				(y $d<p$) esto es imposible.
			}
	\end{enumerate}
	%
\end{ejerFermat}

\begin{ejerFermat}\label{ejer:fermat:descenso:pre}
	Probar la siguiente versi\'on an\'aloga del \lemaname~%
	\ref{lema:fermat:descenso:pre}:
	\begin{quote}
		\itshape
		Sea $n>0$ un n\'umero entero positivo y sea $N$ un natural
		que se puede expresar como $N=a^2+nb^2$ con $a,b\in\Enteros$
		y $\mcd{a,b}=1$. Si $q$ es un divisor primo de $N$ que no
		divide a $n$ y se puede expresar como $q=x^2+ny^2$,
		entonces el cociente $N/q$ tambi\'en se puede expresar como
		$c^2+nd^2$ con $c,d\in\Enteros$ y $\mcd{c,d}=1$.
	\end{quote}
	Probar, adem\'as, que esto es cierto tambi\'en si $n=3$ y $q=4$.%
	% \hint{
		% Emular el argumento del \lemaname~%
		% \ref{lema:fermat:descenso:pre}
	% }
\end{ejerFermat}

% \begin{solucion}
	% Si $N=a^2+nb^2$, $\mcd{a,b}=1$, y $q=x^2+ny^2$, entonces
	% \begin{displaymath}
		% Nx^2-a^2q\,=\,(a^2+nb^2)\,x^2-a^2\,(x^2+ny^2)\,=\,
			% n\,(bx-ay)\,(bx+ay)
		% \text{ .}
	% \end{displaymath}
	% %
	% Como $q$ es primo, $q\mid N$ y $q\nmid n$, se deduce que
	% $q\mid (bx-ay)$, o bien $q\mid (bx+ay)$. Cambiando el signo de $a$,
	% podemos asumir que estamos en el primer caso. Esto quiere decir que
	% existe $d\in\Enteros$ tal que $bx-ay=dq$. Ahora bien,
	% \begin{displaymath}
		% bx-dx^2\,=\,bx-dq+ndy^2\,=\,ay+ndy^2\,=\,(a+ndy)\,y
		% \text{ .}
	% \end{displaymath}
	% %
	% Como $x\mid bx-dx^2$, se cumple que $x\mid (a+ndy)\,y$. Pero
	% $\mcd{x,y}=1$, con lo que $x\mid a+ndy$. Existe, entonces,
	% $c\in\Enteros$ tal que $a+ndy=cx$. O sea,
	% \begin{displaymath}
		% a\,=\,cx-ndy\quad\text{y}\quad b\,=\,dx+cy
		% \text{ .}
		% % bx\,=\,dq + ay\,=\,d\,(x^2+ny^2)+(cx-ndy)\,y\,=\,
		% % dx^2+cxy\,=\,(dx+cy)\,x
	% \end{displaymath}
	% %
	% Pero, entonces,
	% \begin{displaymath}
		% N\,=\,a^2+nb^2\,=\,(cx-ndy)^2+(dx+cy)^2
		% \,=\,(c^2+nd^2)\,(x^2+ny^2)\,=\,(c^2+nd^2)\,q
		% \text{ .}
	% \end{displaymath}
	% %
	% As\'{\i}, $N/q=c^2+nd^2$, pero, adem\'as, $\mcd{a,b}=1$ implica
	% que $\mcd{c,d}=1$.
% 
	% Si $n=3$ y $q=4$, entonces la situaci\'on es la siguiente:
	% $x=y=1$, $N=a^2+3b^2$, $a,b\in\Enteros$, $\mcd{a,b}=1$ y $4\mid N$
	% (y $\mcd{q,n}=1$).
	% Si $4\mid b-a$, entonces $b=a+4d$. En ese caso, definimos
	% $c=a+3d$ y, as\'{\i},
	% \begin{displaymath}
		% a\,=\,c-3d\quad\text{y}\quad b\,=\,c+d
		% \text{ .}
	% \end{displaymath}
	% %
	% Pero, entonces,
	% \begin{displaymath}
		% N\,=\,a^2+3b^2\,=\,(c-3d)^2+3\,(d+c)^2\,=\,
			% 4\,(c^2+3d^2)
		% \text{ .}
	% \end{displaymath}
	% %
	% Si $4\nmid b+a$, entonces $2\mid b-a$; si, adem\'as, $4\nmid b-a$,
	% entonces
	% \begin{displaymath}
		% b\,\equiv\,a+2\tmodulo[4]\quad\text{y}\quad
			% b+a\,\equiv\,2\,(a+1)\tmodulo[4]
		% \text{ .}
	% \end{displaymath}
	% %
	% Como $2\mid a$ implica $2\mid b$, lo que contradir\'{\i}a
	% $\mcd{a,b}=1$, debe ser $2\nmid a$. Pero, entonces ser\'{\i}a
	% $a+b\equiv 0\tmodulo[4]$. Otra contradicci\'on. Debe ser entonces
	% $4\mid b+a$. Cambiando el signo de $a$, podemos asumir $4\mid b-a$.
	% En definitiva, si $N=a^2+3b^2$, $\mcd{a,b}=1$ y $4\mid N$, entonces
	% $N=4\,(c^2+3d^2)$ para ciertos $c,d\in\Enteros$, $\mcd{c,d}=1$.
% \end{solucion}
% 

\begin{ejerFermat}\label{ejer:fermat:descenso:caso}
	Sea $q$ un primo y sea $N=a^2+mqb^2$ un natural ($n=mq$) donde
	$a,b,m\in\Enteros$.
	Probar que, si $q$ divide a $N$, entonces $N/q=mc^2+qd^2$ donde
	$c,d\in\Enteros$.
	Probar, adem\'as, que, si $\mcd{a,b}=1$, entonces $\mcd{c,d}=1$.
\end{ejerFermat}

\begin{ejerFermat}\label{ejer:fermat:descenso}
	Probar las siguientes versiones an\'alogas del \lemaname~%
	\ref{lema:fermat:descenso} para $n=2,3$:%
	% \hint{
		% Emular el argumento del \lemaname~\ref{lema:fermat:descenso}
	% }
	\begin{quote}
		\itshape
		Sea $p$ un primo que divide a una expresi\'on
		del tipo $a^2+2b^2$ donde $a,b\in\Enteros$ y
		$\mcd{a,b}=1$. Entonces, existen
		$x,y\in\Enteros$ tales que $p=x^2+2y^2$.
	\end{quote}
	\begin{quote}
		\itshape
		Sea $p$ un primo impar que divide a una
		expresi\'on del tipo $a^2+3b^2$ donde
		$a,b\in\Enteros$ y $\mcd{a,b}=1$. Entonces,
		existen $x,y\in\Enteros$ tales que
		$p=x^2+3y^2$.
	\end{quote}
\end{ejerFermat}

\begin{ejerFermat}
	Probar la siguiente versi\'on an\'aloga del \lemaname~%
	\ref{lema:fermat:reciprocidad}:
	\begin{quote}
		\itshape
		Si $p$ es primo y $p\equiv 1\tmodulo[3]$, entonces $p$ divide
		a una expresi\'on del tipo $a^2+3b^2$ donde $a,b\in\Enteros$
		y $\mcd{a,b}=1$,
	\end{quote}
	usando, por ejemplo, la siguiente identidad:
	\begin{displaymath}
		4\,(x^{3k}-1)\,=\,4\,(x^{2k}+x^k+1)\,(x^k-1)\,=\,
			((2x^k+1)^2+3)\,(x^k-1)
		\text{ .}
	\end{displaymath}
	%
\end{ejerFermat}

\begin{ejerFermat}
	Probar que si $p\equiv 1\tmodulo[8]$ es primo, entonces $p$ divide a
	una expresi\'on del tipo $a^2+2b^2$ donde $a,b\in\Enteros$ y
	$\mcd{a,b}=1$, usando la siguiente identidad:
	\begin{displaymath}
		x^{8k}-1\,=\, ((x^{2k}-1)^2+2x^{2k})\,(x^{4k}-1)
		\text{ .}
	\end{displaymath}
	%
	Mostrar que existen primos $p\equiv 3\tmodulo[8]$ que dividen a
	n\'umeros de la forma $a^2+2b^2$.
	% con $a,b\in\Enteros$ y $\mcd{a,b}=1$.
\end{ejerFermat}

\begin{ejerFermat}
	Para cada primo $p\equiv 1\tmodulo[3]$, $p\leq 50$, buscar todos los
	valores $x,y\in\Enteros$ tales que $p=x^2+3y^2$.
\end{ejerFermat}

\begin{ejerFermat}\label{ejer:fermat:cinco}
	En este ejercicio investigaremos condiciones que sirvan para
	determinar si un primo divide a una expresi\'on de la forma
	$a^2+5b^2$ donde $a,b\in\Enteros$ y $\mcd{a,b}=1$.
	\begin{enumerate}[(i)]
		\item\label{item:ejer:fermat:cinco:i}
			Calcular que, si $p$ es un primo que divide a un
			natural de la forma $a^2+5b^2$ con $a,b\leq 40$
			y $\mcd{a,b}=1$, entonces $p$ puede ser
			$\equiv 1,3,2,4\tmodulo[5]$.
			O sea, en este caso, mirar congruencia m\'odulo $5$
			no parece dar informaci\'on acerca de si un primo
			divide o no a un natural de la forma $a^2+5b^2$.
		\item\label{item:ejer:fermat:cinco:ii}
			Hallar primos $p\equiv 2,4\tmodulo[5]$ que no
			dividen a ning\'un natural de la forma $a^2+5b^2$
			con $a,b\leq 40$ y $\mcd{a,b}=1$
			(dicho de otra manera, que no aparecen entre los
			divisores de ninguno de estos n\'umeros).
		\item\label{item:ejer:fermat:cinco:iii}
			Calcular que, si $p$ es un primo que divide a un
			natural de la forma $a^2+5b^2$ con $a,b\leq 40$
			y $\mcd{a,b}=1$, entonces
			$p\equiv 1,3,7,9\tmodulo[20]$.
	\end{enumerate}
	%
\end{ejerFermat}

\begin{ejerFermat}\label{ejer:fermat:otras}
	En este ejercicio investigamos otras formas de representar.
	En las siguientes afirmaciones, $a,b\in\Enteros$, $\mcd{a,b}=1$
	y $|a|,|b|\leq 40$.
	\begin{enumerate}[(i)]
		\item\label{item:ejer:fermat:siete}
			Determinar las posibles clases de congruencia
			m\'odulo $28$ de los primos $p$ que dividen a
			los enteros $a^2+7b^2$.
		\item\label{item:ejer:fermat:menos-tres}
			Determinar las posibles clases de congruencia
			m\'odulo $12$ de los primos $p$ que dividen a
			los enteros $a^2-3b^2$.
		\item\label{item:ejer:fermat:menos-cinco}
			Determinar las posibles clases de congruencia
			m\'odulo $20$ de los primos $p$ que dividen a
			los enteros $a^2-5b^2$.
		\item\label{item:ejer:fermat:menos-siete}
			Determinar las posibles clases de congruencia
			m\'odulo $28$ de los primos $p$ que dividen a
			los enteros $a^2-7b^2$.
	\end{enumerate}
	%
	Notar que en los casos \eqref{item:ejer:fermat:menos-tres},
	\eqref{item:ejer:fermat:menos-cinco} y
	\eqref{item:ejer:fermat:menos-siete} las clases obtenidas se pueden
	representar como $\pm\beta^2$ donde $\beta$ es un n\'umero impar.
\end{ejerFermat}


